\documentclass[12pt]{article}
\usepackage{amsmath}
\usepackage{amssymb}
\usepackage[letterpaper,margin=0.85in,centering]{geometry}
\usepackage{fancyhdr}
\usepackage{enumerate}
\usepackage{lastpage}
\usepackage{multicol}
\usepackage{graphicx}

\reversemarginpar

\pagestyle{fancy}
\cfoot{Page \thepage \ of \pageref{LastPage}}\rfoot{{\bf Total Points: 40}}
\chead{MATH 3500}\lhead{Test \# 2}\rhead{Monday, 10\textsuperscript{th} November, 2014}

\newcommand{\points}[1]{\marginpar{\hspace{24pt}[#1]}}
\newcommand{\skipline}{\vspace{12pt}}
%\renewcommand{\headrulewidth}{0in}
\headheight 30pt

\newcommand{\di}{\displaystyle}
\newcommand{\R}{\mathbb{R}}
\newcommand{\Q}{\mathbb{Q}}
\newcommand{\N}{\mathbb{N}}
\newcommand{\abs}[1]{\lvert #1\rvert}

\begin{document}

\author{Instructor: Sean Fitzpatrick}
\thispagestyle{plain}
\begin{center}
\emph{University of Lethbridge}\\
Department of Mathematics and Computer Science\\
10\textsuperscript{th} November 2014, 2:00-2:50 pm\\
{\bf MATH 3500 - Test \#2}\\
\end{center}
\skipline \skipline \skipline \noindent \skipline
Last Name:\underline{\hspace{350pt}}\\
\skipline
First Name:\underline{\hspace{348pt}}\\
\skipline
Student Number:\underline{\hspace{322pt}}\\
\skipline

\vspace{0.5in}


\begin{quote}

 
 {Record your answers below each question in the space provided.    Left-hand pages may be used as scrap paper for rough work.  If you want any work on the left-hand pages to be graded, please indicate so on the right-hand page.
 
 \bigskip
 
Partial credit will be awarded for partially correct work, so be sure to show your work, and include all necessary justifications needed to support your arguments. }

\end{quote}


\vspace{0.5in}

For grader's use only:

\begin{table}[hbt]
\begin{center}
\begin{tabular}{|l|r|} \hline
Page&Grade\\
\hline \hline
\cline{1-2} 2 & \enspace\enspace\enspace\enspace\enspace\enspace/8\\
\cline{1-2} 3 & \enspace\enspace\enspace\enspace\enspace\enspace/10\\
\cline{1-2} 4 & \enspace\enspace\enspace\enspace\enspace\enspace/10\\
\cline{1-2} 5 & \enspace\enspace\enspace\enspace\enspace\enspace/12\\
\cline{1-2} Total & \enspace\enspace\enspace\enspace\enspace\enspace/40\\
\hline
\end{tabular}

\skipline

\skipline

\skipline

\end{center}
\end{table}
\newpage


\begin{enumerate}
\item Let $(a_n)$ be the sequence defined by $a_n = \cos\left(\dfrac{n\pi}{3}\right)$ for $n=1,2,3,\ldots$.
\begin{enumerate}
 \item Recall that $\abs{\cos x}\leq 1$ for all $x\in\R$. Explain why this guarantees that $(a_n)$ has a covergent subsequence.\points{2}

\vspace{2in}

 \item Find the set of subsequential limits of $(a_n)$. (That is find the set of limits of convergent subsequences. It might help to recall that $\cos(\pi/3)=1/2$.) \points{4}

\vspace{3in}

 \item What are the values of $\limsup a_n$ and $\liminf a_n$?\points{2}
\end{enumerate}
\newpage

\item Use the $\epsilon-\delta$ definition of the limit to prove that $\displaystyle \lim_{x\to 2}\frac{x+1}{2x-1}=1$. \points{5}

\vspace{4.5in}

\item Use the $\epsilon-\delta$ definition of uniform continuity to prove that $f(x) = \dfrac{x}{x+1}$ is continuous on $[0,2]$.\points{5}
\newpage
\item Decide whether or not the given function $f$ is uniformly continuous on its domain $D$. Justify your answer in each case with a suitable theorem.
\begin{enumerate}
 \item $f(x)=x^2+2x$ on $D=[0,3]$\points{2}

\vspace{1in}


 \item $f(x)=1/x^2$ on $D=(0,1]$\points{2}

\vspace{1in}

 \item $f(x)=\dfrac{1}{x}\sin^2x$ on $D=(0,\pi)$.\points{2}

\vspace{1.5in}

\end{enumerate}
\item Suppose $f:[a,b]\to \R$ is a continuous function.
\begin{enumerate}
 \item Is it possible for the range of $f$ to equal $[0,1]\cup [2,3]$? Why or why not? \points{2}

\vspace{1.5in}

 \item Is it possible for the range of $f$ to equal either $(0,1)$ or $[1,\infty)$? Why or why not? \points{2}
\end{enumerate}
\newpage

\item Let $f(x)=\begin{cases} x^2 & \text{ if } x\in\Q\\ 0 & \text{ if } x\notin\Q\end{cases}$.
\begin{enumerate}
 \item Show that $f$ is discontinuous at every point $a\neq 0$.\points{4} (Hint: if $a$ is rational/irrational, consider a sequence of irrational/rational numbers converging to $a$.)

\vspace{2in}

 \item Prove that $f$ is differentiable (and hence continuous) at $x=0$. \points{4}

\vspace{2.5in}

\end{enumerate}
\item Prove that if $f'(x)\neq 0$ for all $x\in (a,b)$, then $f$ is either strictly increasing or strictly decreasing on $(a,b)$. \points{4} (Caution: $f'$ is not guaranteed to be continuous.)
\end{enumerate}



\end{document}