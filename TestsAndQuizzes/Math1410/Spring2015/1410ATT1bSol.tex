\documentclass[12pt]{article}
\usepackage{amsmath}
\usepackage{amssymb}
\usepackage[letterpaper,margin=0.85in,centering]{geometry}
\usepackage{fancyhdr}
\usepackage{enumerate}
\usepackage{lastpage}
\usepackage{multicol}
\usepackage{graphicx}

\newenvironment{amatrix}[1]{%
  \left[\begin{array}{@{}*{#1}{c}|c@{}}
}{%
  \end{array}\right]
}
\newenvironment{aamatrix}[1]{%
  \left[\begin{array}{@{}*{#1}{c}|cc}
}{%
  \end{array}\right]
}
\reversemarginpar

\pagestyle{fancy}
\cfoot{Page \thepage \ of \pageref{LastPage}}\rfoot{{\bf Total Points: 40}}
\chead{MATH 1410A}\lhead{Test \# 1}\rhead{Wednesday, 11\textsuperscript{th} February, 2015}

\newcommand{\points}[1]{\marginpar{\hspace{24pt}[#1]}}
\newcommand{\skipline}{\vspace{12pt}}
%\renewcommand{\headrulewidth}{0in}
\headheight 30pt

\newcommand{\di}{\displaystyle}
\newcommand{\R}{\mathbb{R}}
\newcommand{\aaa}{\mathbf{a}}
\newcommand{\bbb}{\mathbf{b}}
\newcommand{\ccc}{\mathbf{c}}
\newcommand{\dotp}{\boldsymbol{\cdot}}
\newcommand{\abs}[1]{\lvert #1\rvert}
\newcommand{\len}[1]{\lVert #1\rVert}
\newcommand{\ivec}{\,\boldsymbol{\hat{\imath}}}
\newcommand{\jvec}{\,\boldsymbol{\hat{\jmath}}}
\newcommand{\kvec}{\,\boldsymbol{\hat{k}}}
\DeclareMathOperator{\comp}{comp}

\begin{document}

\author{Instructor: Sean Fitzpatrick}
\thispagestyle{plain}
\begin{center}
\emph{University of Lethbridge}\\
Department of Mathematics and Computer Science\\
11\textsuperscript{th} February, 2015, 10:00 - 10:50 am\\
{\bf MATH 1410A - Test \#1}\\
\end{center}
\skipline \skipline \skipline \noindent \skipline
Last Name:\underline{\hspace{50pt}{\bf Solutions}\hspace{248pt}}\\
\skipline
First Name:\underline{\hspace{50pt}{\bf The}\hspace{275pt}}\\
\skipline
Student Number:\underline{\hspace{323pt}}\\
\skipline
Tutorial Section: \underline{\hspace{320pt}}\\


\vspace{0.5in}


\begin{quote}
 {\bf Record your answers below each question in the space provided.    Left-hand pages may be used as scrap paper for rough work.  If you want any work on the left-hand pages to be graded, please indicate so on the right-hand page.
 
 \bigskip
 
Partial credit will be awarded for partially correct work, so be sure to show your work, and include all necessary justifications needed to support your arguments.

\bigskip

No external aids are allowed, with the exception of a 5-function calculator.}
\end{quote}


\vspace{0.5in}

For grader's use only:

\begin{table}[hbt]
\begin{center}
\begin{tabular}{|l|r|} \hline
Page&Grade\\
\hline \hline
\cline{1-2} 2 & \enspace\enspace\enspace\enspace\enspace\enspace/10\\
\cline{1-2} 3 & \enspace\enspace\enspace\enspace\enspace\enspace/10\\
\cline{1-2} 4 & \enspace\enspace\enspace\enspace\enspace\enspace/10\\
\cline{1-2} 5 & \enspace\enspace\enspace\enspace\enspace\enspace/10\\
\cline{1-2} Total & \enspace\enspace\enspace\enspace\enspace\enspace/40\\
\hline
\end{tabular}

\skipline

\skipline

\skipline

B
\end{center}
\end{table}
\newpage


\begin{enumerate}
\item SHORT ANSWER: For each of the questions below, please provide a short (one line) answer.
 \begin{enumerate}
\item Do the values $x=3$, $y=-2$, $z=4$ provide a solution to the system of equations below? Why or why not?\points{2}
\[
\begin{array}{ccccccc}
x&+&y&+&z&=&5\\
2x&+&4y&-&z&=&-6\\
-3x&-&5y&+&z&=&4
\end{array}
\]

\bigskip

\noindent {\bf Solution:} No, $-3(3)-5(-2)+4=5\neq 4$.

\bigskip

\item If $A$ is an $m\times n$ matrix and $B$ is a $k\times l$ matrix, what condition on the numbers $k,l,m,n$ is needed for the product $AB$ to be defined? \points{2}

\bigskip

\noindent {\bf Solution:} You need to have $n=k$.

\bigskip

\item The matrix $E=\begin{bmatrix}
1&0&0\\2&1&0\\0&0&1
\end{bmatrix}$ is an elementary matrix. If $A$ is any other $3\times 3$ matrix, what elementary row operation would let us obtain $EA$ from $A$? \points{2}


\bigskip

\noindent {\bf Solution:} Adding 2 times Row 1 to Row 2.

\bigskip

\item Identify the matrices below as symmetric, antisymmetric, or neither: \points{2}
\[
\begin{bmatrix}
1&2\\2&-3
\end{bmatrix}\hspace{1in} \begin{bmatrix}
1&3\\2&4
\end{bmatrix}\hspace{1in} \begin{bmatrix}
0&4\\-4&0
\end{bmatrix}
\]

\bigskip

\noindent {\bf Solution:} The matrices, from left to right, are:

Symmetric, Neither, and Anti-symmetric.

\bigskip

\item What does it mean to say that an $n\times n$ matrix $A$ is invertible? \points{2}


\bigskip

\noindent {\bf Solution:} There exists another $n\times n$ matrix $B$ such that $AB=BA=I_n$, where $I_n$ is the $n\times n$ identity matrix.




\end{enumerate}
\newpage

\item Find the general solution to the following system of linear equations: \points{10}
\[
\begin{array}{ccccccccc}
x&-&y& &  &+&2w&=&1\\
 & &y&+&3z&-&w&=&1\\
x&&&+&3z&+ &w &=&2 
\end{array}
\]

\bigskip

\noindent {\bf Solution:} We set-up and reduce the augmented matrix of the system as follows:
\begin{align*}
 \begin{amatrix}{4}
  1&-1&0&2&1\\0&1&3&-1&1\\1&0&3&-1&2
 \end{amatrix}&\xrightarrow[]{R_3\to R_3-R_1}\begin{amatrix}{4}1&-1&0&2&1\\0&1&3&-1&1\\0&1&3&-1&1\end{amatrix}\\
&\xrightarrow[]{R_3\to R_3-R_2}\begin{amatrix}{4}
                             1&-1&0&2&1\\0&1&3&-1&1\\0&0&0&0&0
                            \end{amatrix}\\
&\xrightarrow[]{R_1\to R_1+R_2}\begin{amatrix}{4}
                                 1&0&3&1&2\\0&1&3&-1&1\\0&0&0&0&0
                                \end{amatrix}.
\end{align*}
From the reduced row-echelon form of the augmented matrix, we see that $x$ and $y$ are leading variables, and we assign the non-leading variables to parameters. With $z=s$ and $w=t$, where $s$ and $t$ can be any real numbers, we have the general solution
\begin{align*}
 x& = 2-3s-t\\
 y& = 1-3s+t\\
 z&=s\\
 w&=t.
\end{align*}


\bigskip

\newpage

\item Suppose $A$, $B$, and $X$ are $2\times 2$ matrices.
\begin{enumerate}
\item Given that $X^T-2A=B$, solve for $X$ in terms of $A$ and $B$.\points{3}


\bigskip

\noindent {\bf Solution:} By adding $2A$ to both sides and then taking the transpose, we have
\[
 X = (B+2A)^T = B^T+2A^T.
\]


\bigskip


\item If $A=\begin{bmatrix}3&-1\\2&4\end{bmatrix}$ and $B=\begin{bmatrix}2&0\\0&5\end{bmatrix}$ and $X$ is as in part (a), determine the entries of $X$. \points{3}


\bigskip

\noindent {\bf Solution:} Plugging the given values for $A$ and $B$ into the expression above, we have
\[
 X = \begin{bmatrix}2&0\\0&5\end{bmatrix}+2\begin{bmatrix}3&2\\-1&4\end{bmatrix} = \begin{bmatrix}8&4\\-2&13\end{bmatrix}
\]


\bigskip


\end{enumerate}
\item Suppose that $A,B,C,$ and $D$ are $n\times n$ matrices, with $A,B,$ and $C$ {\em invertible}. Given that \points{4}
\[
BA^{-1}XAC=BD^T,
\]
solve for $X$ in terms of $A,B,C,$ and $D$.


\bigskip

\noindent {\bf Solution:} We have
\[
 X = AB^{-1}(BA^{-1}XAC)(C^{-1}A^{-1}) = AB^{-1}(BD^T)C^{-1}A^{-1} = AD^TC^{-1}A^{-1}.
\]


\bigskip

\newpage

\item Let $A=\begin{bmatrix}3&6\\-3&-5\end{bmatrix}$.
\begin{enumerate}
\item Find $A^{-1}$. \points{5}


\bigskip

\noindent {\bf Solution:} Using the augmented matrix algorithm for the inverse, we have
\begin{align*}
 \begin{aamatrix}{2}
  3&6&1&0\\-3&-5&0&1
 \end{aamatrix} & \xrightarrow[]{R_2\to R_2+R_1}\begin{aamatrix}{2}
                                                3&6&1&0\\0&1&1&1
                                               \end{aamatrix}\\
&\xrightarrow[]{R_1\to\frac{1}{3}R_1}\begin{aamatrix}{2}
                                      1&2&\frac{1}{3}&0\\0&1&1&1
                                     \end{aamatrix}\\
&\xrightarrow[]{R_1\to R_1-2R_2}\begin{aamatrix}{2}
                                 1&0&-\frac{5}{3}&-2\\0&1&1&1
                                \end{aamatrix},
\end{align*}
so $A^{-1} = \begin{bmatrix}-\frac{5}{3}&-2\\1&1\end{bmatrix}$.

\bigskip


\item Write $A^{-1}$ as a product of elementary matrices. \points{3}


\bigskip

\noindent {\bf Solution:} We know that $A^{-1}=E_3E_2E_1$, where $E_1,E_2,E_3$ are the elementary matrices corresponding to the three row operations above, in the order they were performed. Therefore,
\[
 A^{-1}=\begin{bmatrix}1&-2\\0&1\end{bmatrix}\begin{bmatrix}\frac{1}{3}&0\\0&1\end{bmatrix}\begin{bmatrix}1&0\\1&1\end{bmatrix}.
\]


\bigskip


\item Write $A$ as a product of elementary matrices. \points{2}


\bigskip

\noindent {\bf Solution:} Since $A^{-1}=E_3E_2E_1$, we have
\[
 A = (E_3E_2E_1)^{-1} = E_1^{-1}E_2^{-1}E_3^{-1} = \begin{bmatrix}1&0\\-1&1\end{bmatrix}\begin{bmatrix}3&0\\0&1\end{bmatrix}\begin{bmatrix}1&2\\0&1\end{bmatrix}.
\]


\bigskip

\end{enumerate}
\end{enumerate}
\end{document}