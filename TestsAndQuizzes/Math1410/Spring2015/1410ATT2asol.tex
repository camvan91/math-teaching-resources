\documentclass[12pt]{article}
\usepackage{amsmath}
\usepackage{amssymb}
\usepackage[letterpaper,margin=0.85in,centering]{geometry}
\usepackage{fancyhdr}
\usepackage{enumerate}
\usepackage{lastpage}
\usepackage{multicol}
\usepackage{graphicx}

\reversemarginpar

\pagestyle{fancy}
\cfoot{Page \thepage \ of \pageref{LastPage}}\rfoot{{\bf Total Points: 40}}
\chead{MATH 1410A}\lhead{Test \# 2}\rhead{Friday, 20\textsuperscript{th} March, 2015}

\newcommand{\points}[1]{\marginpar{\hspace{24pt}[#1]}}
\newcommand{\skipline}{\vspace{12pt}}
%\renewcommand{\headrulewidth}{0in}
\headheight 30pt

\newcommand{\di}{\displaystyle}
\newcommand{\R}{\mathbb{R}}
\newcommand{\aaa}{\mathbf{a}}
\newcommand{\bbb}{\mathbf{b}}
\newcommand{\ccc}{\mathbf{c}}
\newcommand{\dotp}{\boldsymbol{\cdot}}
\newcommand{\abs}[1]{\lvert #1\rvert}
\newcommand{\len}[1]{\lVert #1\rVert}
\newcommand{\ivec}{\,\boldsymbol{\hat{\imath}}}
\newcommand{\jvec}{\,\boldsymbol{\hat{\jmath}}}
\newcommand{\kvec}{\,\boldsymbol{\hat{k}}}
\DeclareMathOperator{\proj}{proj}

\begin{document}

\author{Instructor: Sean Fitzpatrick}
\thispagestyle{plain}
\begin{center}
\emph{University of Lethbridge}\\
Department of Mathematics and Computer Science\\
20\textsuperscript{th} March, 2015, 10:00 - 10:50 am\\
{\bf MATH 1410A - Test \#2}\\
\end{center}
\skipline \skipline \skipline \noindent \skipline
Last Name:\underline{\hspace{353pt}}\\
\skipline
First Name:\underline{\hspace{350pt}}\\
\skipline
Student Number:\underline{\hspace{323pt}}\\
\skipline
Tutorial Section: \underline{\hspace{320pt}}\\


\vspace{0.5in}


\begin{quote}
 {\bf Record your answers below each question in the space provided.    Left-hand pages may be used as scrap paper for rough work.  If you want any work on the left-hand pages to be graded, please indicate so on the right-hand page.
 
 \bigskip
 
Partial credit will be awarded for partially correct work, so be sure to show your work, and include all necessary justifications needed to support your arguments.

\bigskip

No external aids are allowed, with the exception of a 5-function calculator.}
\end{quote}


\vspace{0.5in}

For grader's use only:

\begin{table}[hbt]
\begin{center}
\begin{tabular}{|l|r|} \hline
Page&Grade\\
\hline \hline
\cline{1-2} 2 & \enspace\enspace\enspace\enspace\enspace\enspace/10\\
\cline{1-2} 3 & \enspace\enspace\enspace\enspace\enspace\enspace/10\\
\cline{1-2} 4 & \enspace\enspace\enspace\enspace\enspace\enspace/10\\
\cline{1-2} 5 & \enspace\enspace\enspace\enspace\enspace\enspace/5\\
\cline{1-2} 6 & \enspace\enspace\enspace\enspace\enspace\enspace/5\\
\cline{1-2} Total & \enspace\enspace\enspace\enspace\enspace\enspace/40\\
\hline
\end{tabular}

\skipline

\skipline

\skipline

A
\end{center}
\end{table}
\newpage


\begin{enumerate}
\item SHORT ANSWER: For each of the questions below, please provide a short (one line) answer.
 \begin{enumerate}
\item Suppose $\det A = 4$ and the matrix $B$ is obtained from $A$ by  \points{2} first multiplying the first row of $A$ by 5, and then exchanging rows 1 and 3. What is $\det B$?

\bigskip

$\det B = -5\det A = -20$.

\bigskip

\item Let $A$ and $B$ be $3\times 3$ matrices. If $\det A=2$ and $\det B = -3$, what is the value of \points{2} $\det(2A^2B^TA^{-1})$?

\bigskip

$\det(2A^2B^TA^{-1}) = 2^3\abs{A}^2\abs{B}\left(\dfrac{1}{\abs{A}}\right) = 8\abs{A}\abs{B} = -48.$

\bigskip

\item Calculate the dot product of the vectors $\vec{u}=\begin{bmatrix}1&-2&4\end{bmatrix}^T$ and $\vec{v} = \begin{bmatrix}4&-2&3\end{bmatrix}^T$ \points{2}

\bigskip

$\vec{u}\dotp\vec{v} = 1(4)-2(-2)+4(3) = 20$.

\bigskip


\item Let $\vec{v}$ be a vector in $\R^n$. If $\len{\vec{v}}=3$ and $\vec{w} = -4\vec{v}$, what is $\len{\vec{w}}$? \points{2}

\bigskip

$\len{\vec{w}}=\len{-4\vec{v}} = \abs{-4}\len{\vec{v}} = 4(3)=12.$

\bigskip

\item Find $\vec{x}$, given that $\vec{u} = \begin{bmatrix}2&-1&3\end{bmatrix}^T$ and $\vec{v} = \begin{bmatrix}-4&7&5\end{bmatrix}^T$, and $3\vec{u}-2\vec{x}=\vec{v}$. \points{2}

\bigskip

$\vec{x} = \frac{1}{2}(3\vec{u}-\vec{v}) = \frac{3}{2}\vec{u}-\frac{1}{2}\vec{v} =  \begin{bmatrix} 5&-5&2\end{bmatrix}$

\end{enumerate}
\newpage

\item Let $A = \begin{bmatrix}2&-1&0\\0&3&4\\-3&0&1\end{bmatrix}$.
\begin{enumerate}
 \item Compute $\det A$. \points{5}

\bigskip

By adding 2 times the second column to the first and then using cofactor expansion along the first row, we get

\[
 \det{A} = \begin{vmatrix}2&-1&0\\0&3&4\\-3&0&1\end{vmatrix} = \begin{vmatrix}0&-1&0\\6&3&4\\-3&0&1\end{vmatrix} = (-1)(-1)\begin{vmatrix}6&4\\-3&1\end{vmatrix} = 6+12=18.
\]



 \item Given the system of equations $\begin{bmatrix}2&-1&0\\0&3&4\\-3&0&1\end{bmatrix}\begin{bmatrix}x\\y\\z\end{bmatrix} =\begin{bmatrix}2\\-1\\3\end{bmatrix}$, use Cramer's rule to find the value of $y$, if possible. \points{5}

\bigskip

Cramer's rule tells us that $y = \dfrac{\abs{A_2}}{\abs{A}}$, where $A_2$ is obtained from $A$ by replacing the second column of $A$ by the right-hand side of the system. This gives us
\[
 \det{A_2} = \begin{vmatrix}2&2&0\\0&-1&4\\-3&3&1\end{vmatrix} = \begin{vmatrix}2&0&0\\0&-1&4\\3&6&1\end{vmatrix} = 2\begin{vmatrix}-1&4\\6&1\end{vmatrix} = 2(-1-24)=-50.
\]
Thus, $y=\dfrac{-50}{18} = -\dfrac{25}{9}$.
\end{enumerate}


\newpage

\item \begin{enumerate}
       \item Find a vector equation of the line in $\R^3$ that passes through the points $P=(2,-3,1)$ and $Q=(4,1,-2)$. \points{4}

\bigskip

A direction vector is $\vec{d} = \overrightarrow{PQ} = \begin{bmatrix} 2\\4\\-3\end{bmatrix}$, so an equation of the line is
\[
 \begin{bmatrix}x\\y\\z\end{bmatrix} = \begin{bmatrix}2\\-3\\1\end{bmatrix}+t\begin{bmatrix}2\\4\\-3\end{bmatrix}.
\]

\bigskip

       \item Let $L_1$ be the line through $P_1=(2,0,-1)$ with direction vector $\vec{d}_1 = \begin{bmatrix}-1&3&2\end{bmatrix}^T$, and let $L_2$ be the line through $P_2=(8,6,7)$ with direction vector $\vec{d}_2=\begin{bmatrix}-4&0&-2\end{bmatrix}^T$. Determine the point of intersection of $L_1$ and $L_2$, if any.\points{6}

\bigskip

For $L_1$ we have $x=2-s, y=3s, z=-1+2s$, and for $L_2$ we have $x=8-4t, y=6, z=7-2t$. If the lines intersect, we must have $2-s = 8-4t$, $3s=6$, and $-1+2s=7-2t$.

The second of these equations gives us $s=2$. Putting this into the first equation gives $t=2$ as well. In the third equation, we verify that $-1+2(2)=3-7-2(2)$, so the two lines intersect.

Putting $s=2$ into the parametric equations for $L_1$ gives $x=0, y=6, z=3$, so the point of intersection is $(0,6,3)$.
      \end{enumerate}




\newpage
\item Find the shortest distance from the point $P=(3,2,-1)$ to the line $L$ given by the vector equation \points{5}
\[
\begin{bmatrix}x\\y\\z\end{bmatrix} = \begin{bmatrix}2\\1\\3\end{bmatrix} + t\begin{bmatrix}3\\-1\\-2\end{bmatrix}
\]

\bigskip

We have the point $P_0=(2,1,3)$ on the line, and the direction vector $\vec{d}=\langle 3,-1,-2\rangle$. If $Q$ is the point on the line closest to $P$, then we have
\[
 \proj_{\vec{d}}\overrightarrow{P_0P}=\overrightarrow{P_0Q},
\]
and the shortest distance is given by $\len{\overrightarrow{QP}}=\len{\overrightarrow{P_0P}-\overrightarrow{P_0Q}}$. We find
\[
 \overrightarrow{P_0P} = \langle 1,1,-4\rangle,
\]
so
\[
 \overrightarrow{P_0Q} = \frac{\vec{d}\dotp\overrightarrow{P_0P}}{\len{\vec{d}}^2}\vec{d} = \frac{10}{14}\begin{bmatrix}3\\-1\\-2\end{bmatrix}=\begin{bmatrix}15/7\\-5/7\\-10/7\end{bmatrix}.
\]
Thus $\overrightarrow{QP} = \overrightarrow{P_0P}-\overrightarrow{P_0Q} = \begin{bmatrix}-8/7\\12/7\\18/7\end{bmatrix}$, which gives the distance
\[
 \len{\overrightarrow{QP}} = \sqrt{(-8/7)^2+(12/7)^2+(18/7)^2}.
\]


\newpage

\item Consider the triangle in $\R^3$ with vertices $P=(2,0,-3)$, $Q=(5,-2,1)$, and $R=(7,5,3)$.
\begin{enumerate}
 \item Show that the triangle is a right-angled triangle. \points{3}

\bigskip

The vectors $\overrightarrow{QP}=\langle -3,2,-4\rangle$ and $\overrightarrow{QR} = \langle 2,7,2\rangle$ make up two of the three sides of the triangle, and
\[
 \overrightarrow{QP}\dotp\overrightarrow{QR} = -6+14-8 = 0
\]
which shows that these two sides are perpendicular, and thus the triangle is a right-angled triangle.

\bigskip



 \item Compute the lengths of the three sides of the triangle and verify that the Pythagorean theorem ($a^2+b^2=c^2$) holds. \points{2}

 We have $\len{QP}^2 = (-3)^2+2^2+(-4)^2 = 29$ and $\len{QR}^2 = 2^2+7^2+2^2 = 57$, so $\len{QP}^2+\len{QR}^2 = 86$.

 The remaining side is given by $\overrightarrow{PR}=\langle 5,5,6\rangle$, and we see that
\[
 \len{\overrightarrow{PR}}^2 = 5^2+5^2+6^2 = 86
\]
as well, so the Pythagorean theorem holds.


\end{enumerate}
 
\end{enumerate}
\end{document}