\documentclass[12pt]{article}
\usepackage{amsmath}
\usepackage{amssymb}
\usepackage[letterpaper,margin=0.85in,centering]{geometry}
\usepackage{fancyhdr}
\usepackage{enumerate}
\usepackage{lastpage}
\usepackage{multicol}
\usepackage{graphicx}

\reversemarginpar

\pagestyle{fancy}
\cfoot{Page \thepage \ of \pageref{LastPage}}\rfoot{{\bf Total Points: 40}}
\chead{MATH 1410A}\lhead{Test \# 2}\rhead{Friday, 20\textsuperscript{th} March, 2015}

\newcommand{\points}[1]{\marginpar{\hspace{24pt}[#1]}}
\newcommand{\skipline}{\vspace{12pt}}
%\renewcommand{\headrulewidth}{0in}
\headheight 30pt

\newcommand{\di}{\displaystyle}
\newcommand{\R}{\mathbb{R}}
\newcommand{\aaa}{\mathbf{a}}
\newcommand{\bbb}{\mathbf{b}}
\newcommand{\ccc}{\mathbf{c}}
\newcommand{\dotp}{\boldsymbol{\cdot}}
\newcommand{\abs}[1]{\lvert #1\rvert}
\newcommand{\len}[1]{\lVert #1\rVert}
\newcommand{\ivec}{\,\boldsymbol{\hat{\imath}}}
\newcommand{\jvec}{\,\boldsymbol{\hat{\jmath}}}
\newcommand{\kvec}{\,\boldsymbol{\hat{k}}}
\DeclareMathOperator{\comp}{comp}

\begin{document}

\author{Instructor: Sean Fitzpatrick}
\thispagestyle{plain}
\begin{center}
\emph{University of Lethbridge}\\
Department of Mathematics and Computer Science\\
20\textsuperscript{th} March, 2015, 10:00 - 10:50 am\\
{\bf MATH 1410A - Test \#2}\\
\end{center}
\skipline \skipline \skipline \noindent \skipline
Last Name:\underline{\hspace{353pt}}\\
\skipline
First Name:\underline{\hspace{350pt}}\\
\skipline
Student Number:\underline{\hspace{323pt}}\\
\skipline
Tutorial Section: \underline{\hspace{320pt}}\\


\vspace{0.5in}


\begin{quote}
 {\bf Record your answers below each question in the space provided.    Left-hand pages may be used as scrap paper for rough work.  If you want any work on the left-hand pages to be graded, please indicate so on the right-hand page.
 
 \bigskip
 
Partial credit will be awarded for partially correct work, so be sure to show your work, and include all necessary justifications needed to support your arguments.

\bigskip

No external aids are allowed, with the exception of a 5-function calculator.}
\end{quote}


\vspace{0.5in}

For grader's use only:

\begin{table}[hbt]
\begin{center}
\begin{tabular}{|l|r|} \hline
Page&Grade\\
\hline \hline
\cline{1-2} 2 & \enspace\enspace\enspace\enspace\enspace\enspace/10\\
\cline{1-2} 3 & \enspace\enspace\enspace\enspace\enspace\enspace/10\\
\cline{1-2} 4 & \enspace\enspace\enspace\enspace\enspace\enspace/10\\
\cline{1-2} 5 & \enspace\enspace\enspace\enspace\enspace\enspace/5\\
\cline{1-2} 6 & \enspace\enspace\enspace\enspace\enspace\enspace/5\\
\cline{1-2} Total & \enspace\enspace\enspace\enspace\enspace\enspace/40\\
\hline
\end{tabular}

\skipline

\skipline

\skipline

B
\end{center}
\end{table}
\newpage


\begin{enumerate}
\item SHORT ANSWER: For each of the questions below, please provide a short (one line) answer.
 \begin{enumerate}


\item Calculate the dot product of the vectors $\vec{u}=\begin{bmatrix}3&-5&2\end{bmatrix}^T$ and $\vec{v} = \begin{bmatrix}4&3&-2\end{bmatrix}^T$ \points{2}

\vspace{1.2in}

\item Let $\vec{v}$ be a vector in $\R^n$. If $\len{\vec{v}}=5$ and $\vec{w} = -3\vec{v}$, what is $\len{\vec{w}}$? \points{2}

\vspace{1.2in}

\item Find $\vec{x}$, given that $\vec{u} = \begin{bmatrix}-3&1&2\end{bmatrix}^T$ and $\vec{v} = \begin{bmatrix}4&-7&2\end{bmatrix}^T$, and $4\vec{u}-3\vec{x}=\vec{v}$. \points{2}

\vspace{1.4in}

\item Suppose $\det A = -3$ and the matrix $B$ is obtained from $A$ by  \points{2} first multiplying the first row of $A$ by 6, and then exchanging rows 2 and 3. What is $\det B$?

\vspace{1.2in}

\item Let $A$ and $B$ be $4\times 4$ matrices. If $\det A=3$ and $\det B = -2$, what is the value of \points{2} $\det(2A^2B^TA^{-1})$?


\end{enumerate}
\newpage

\item Let $A = \begin{bmatrix}3&0&-2\\1&-3&0\\0&-2&1\end{bmatrix}$.
\begin{enumerate}
 \item Compute $\det A$. \points{5}

\vspace{3.5in}

 \item Given the system of equations $\begin{bmatrix}3&0&-2\\1&-3&0\\0&-2&1\end{bmatrix}\begin{bmatrix}x\\y\\z\end{bmatrix} =\begin{bmatrix}-2\\1\\4\end{bmatrix}$, use Cramer's rule to find the value of $z$, if possible. \points{5}
\end{enumerate}


\newpage

\item \begin{enumerate}
       \item Find a vector equation of the line in $\R^3$ that passes through the points $P=(-3,2,-1)$ and $Q=(-2,4,2)$. \points{4}

\vspace{2.5in}

       \item Let $L_1$ be the line through $P_1=(1,2,0)$ with direction vector $\vec{d}_1 = \begin{bmatrix}3&-2&2\end{bmatrix}^T$, and let $L_2$ be the line through $P_2=(4,5,6)$ with direction vector $\vec{d}_2=\begin{bmatrix}-2&3&0\end{bmatrix}^T$. Determine the point of intersection of $L_1$ and $L_2$, if any.\points{6}
      \end{enumerate}




\newpage
\item Find the shortest distance from the point $P=(1,-1,3)$ to the line $L$ given by the vector equation \points{5}
\[
\begin{bmatrix}x\\y\\z\end{bmatrix} = \begin{bmatrix}1\\0\\-1\end{bmatrix} + t\begin{bmatrix}2\\1\\4\end{bmatrix}
\]

\newpage

\item Consider the triangle in $\R^3$ with vertices $P=(3,1,-2)$, $Q=(6,-1,2)$, and $R=(8,6,4)$.
\begin{enumerate}
 \item Show that the triangle is a right-angled triangle. \points{3}

{\em Hint:} Recall that if the dot product of two non-zero vectors is zero, then those vectors meet at a right angle.

\vspace{4in}

 \item Compute the lengths of the three sides of the triangle and verify that the Pythagorean theorem ($a^2+b^2=c^2$) holds. \points{2}




\end{enumerate}
 
\end{enumerate}
\end{document}