\documentclass[12pt]{article}
\usepackage{amsmath}
\usepackage{amssymb}
\usepackage[letterpaper,margin=0.85in,centering]{geometry}
\usepackage{fancyhdr}
\usepackage{enumerate}
\usepackage{lastpage}
\usepackage{multicol}
\usepackage{graphicx}

\reversemarginpar

\pagestyle{fancy}
\cfoot{Page \thepage \ of \pageref{LastPage}}\rfoot{{\bf Total Points: 50}}
\chead{MATH 1410A}\lhead{Test \# 1}\rhead{Monday, 3\textsuperscript{rd} October, 2016}

\newcommand{\points}[1]{\marginpar{\hspace{24pt}[#1]}}
\newcommand{\skipline}{\vspace{12pt}}
%\renewcommand{\headrulewidth}{0in}
\headheight 30pt

\newcommand{\di}{\displaystyle}
\newcommand{\R}{\mathbb{R}}
\newcommand{\aaa}{\mathbf{a}}
\newcommand{\bbb}{\mathbf{b}}
\newcommand{\ccc}{\mathbf{c}}
\newcommand{\dotp}{\boldsymbol{\cdot}}
\newcommand{\abs}[1]{\lvert #1\rvert}
\newcommand{\len}[1]{\lVert #1\rVert}
\newcommand{\ivec}{\,\boldsymbol{\hat{\imath}}}
\newcommand{\jvec}{\,\boldsymbol{\hat{\jmath}}}
\newcommand{\kvec}{\,\boldsymbol{\hat{k}}}
\newcommand{\bvm}{\begin{vmatrix}}
\newcommand{\evm}{\end{vmatrix}}

\DeclareMathOperator{\comp}{comp}

\begin{document}

\author{Instructor: Sean Fitzpatrick}
\thispagestyle{plain}
\begin{center}
\emph{University of Lethbridge}\\
Department of Mathematics and Computer Science\\
3\textsuperscript{rd} October, 2016, 9:00 - 9:50 am\\
{\bf MATH 1410A - Test \#1}\\
\end{center}
\skipline \skipline \skipline \noindent \skipline
Last Name:\underline{\hspace{353pt}}\\
\skipline
First Name:\underline{\hspace{350pt}}\\
\skipline
Student Number:\underline{\hspace{323pt}}\\
\skipline
Tutorial Section: \underline{\hspace{320pt}}\\


\vspace{0.5in}


\begin{quote}
 {\bf Record your answers below each question in the space provided.    Left-hand pages may be used as scrap paper for rough work.  If you want any work on the left-hand pages to be graded, please indicate so on the right-hand page.
 
 \bigskip
 
Partial credit will be awarded for partially correct work, including intermediate steps. (You should show your work if you want to earn part marks.) Unless otherwise indicated, failure to justify your work may result in loss of marks, even for a correct answer. 

\bigskip

No external aids are allowed, with the exception of a 5-function calculator.}
\end{quote}


\vspace{0.5in}

For grader's use only:

\begin{table}[hbt]
\begin{center}
\begin{tabular}{|l|r|} \hline
Page&Grade\\
\hline \hline
\cline{1-2} 2 & \enspace\enspace\enspace\enspace\enspace\enspace/14\\
\cline{1-2} 3 & \enspace\enspace\enspace\enspace\enspace\enspace/14\\
\cline{1-2} 4 & \enspace\enspace\enspace\enspace\enspace\enspace/12\\
\cline{1-2} 5 & \enspace\enspace\enspace\enspace\enspace\enspace/10\\
\cline{1-2} Total & \enspace\enspace\enspace\enspace\enspace\enspace/50\\
\hline
\end{tabular}

\skipline

\skipline

\skipline

A
\end{center}
\end{table}
\newpage


\begin{enumerate}
\item Given the complex numbers $z=2-3i$ and $w=1+i$, compute the following. You do not need to explain your work.
 \begin{enumerate}
\item $z+w$ \points{2}

\vspace{1in}

\item $\overline{z}$ \points{2}

\vspace{0.75in}

\item $\abs{w}$ \points{2}

\vspace{1in}

\item $zw$ \points{2}

\vspace{1.25in}

\item $\dfrac{z}{w}$ \points{3}

\vspace{1.5in}

\item The polar form of $w$. \points{3}
\end{enumerate}

\newpage

\item Given the vectors $\vec{v} = \langle 2, -1, 3\rangle$ and $\vec{w} = \langle 0, -4, 2\rangle$, compute the following. You do not need to explain your work.
\begin{enumerate}
 \item $\vec{v}-3\vec{w}$  \points{2}

\vspace{1in}
  
 \item $\len{\vec{v}}$ \points{2}

\vspace{1in}

 \item $\vec{v}\dotp\vec{w}$ \points{2}

\vspace{1in}

 \item $\vec{v}\times \vec{w}$ \points{4}

\vspace{2in}

 \item $\operatorname{proj}_{\vec{v}}\vec{w}$ \points{4}
\end{enumerate}
\newpage

\item 
\begin{enumerate}
 \item Verify that $z=-\sqrt{3}+i$ can be written in the polar form $z=2e^{i(5\pi/6)}$. \points{3}

\vspace{2in}

 \item Compute the power $(-\sqrt{3}+i)^5$. Express your answer in the form $x+iy$. \points{5}
\end{enumerate}

\vspace{3in}

\item Find the point of intersection of the line $\langle x,y,z\rangle = \langle 2,-1,3\rangle+t\langle 0,1,-2\rangle$ and the plane $2x-y+3z=-7$. \points{4}

\newpage

\item Find the vector equation of the line $\ell$ that passes through the points $P=(1,0,-3)$ and $Q=(3,1,-1)$. \points{4}

\vspace{3in}

\item Compute \textbf{one} of the following two distances. \points{6} To earn full marks, your solution must include a clearly labelled diagram.
\begin{itemize}
\item From the point $P=(2,5,6)$ to the line $\langle x,y,z\rangle = \langle -1,1,1\rangle + t\langle 1,0,1\rangle$, or  
\item From the point $P=(5, 6, 2)$ to the plane $x-2y+2z=3$.
\end{itemize}





\end{enumerate}
\end{document}