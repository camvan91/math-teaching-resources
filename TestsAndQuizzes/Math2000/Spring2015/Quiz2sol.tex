\documentclass[12pt]{article}
\usepackage{amsmath}
\usepackage{amssymb}
\usepackage[letterpaper,margin=0.85in,centering]{geometry}
\usepackage{fancyhdr}
\usepackage{enumerate}
\usepackage{lastpage}
\usepackage{multicol}
\usepackage{graphicx}

\reversemarginpar

\pagestyle{fancy}
\cfoot{}
\lhead{Math 2000B}\chead{Quiz \# 2}\rhead{Thursday, 22\textsuperscript{nd} January, 2015}
\rfoot{Total: 10 points}
%\chead{{\bf Name:}}
\newcommand{\points}[1]{\marginpar{\hspace{24pt}[#1]}}
\newcommand{\skipline}{\vspace{12pt}}
%\renewcommand{\headrulewidth}{0in}
\headheight 30pt

\newcommand{\di}{\displaystyle}
\newcommand{\R}{\mathbb{R}}
\newcommand{\Z}{\mathbb{Z}}
\newcommand{\aaa}{\mathbf{a}}
\newcommand{\bbb}{\mathbf{b}}
\newcommand{\ccc}{\mathbf{c}}
\newcommand{\dotp}{\boldsymbol{\cdot}}
\begin{document}
{\bf Name: Solutions}
%\author{Instructor: Sean Fitzpatrick}
\thispagestyle{fancy}
%\noindent{{\bf Name and student number:}}

 \begin{enumerate}
 \item  Use previously established logical equivalences to prove the following:\points{6}
\[
 P\to (Q\wedge R)\equiv (P\to Q)\wedge (P\to R)
\]

\bigskip

Starting from the left-hand side, we have
\begin{align*}
P \to (Q\wedge R)& \equiv \neg P\vee(Q\wedge R)\\
&\equiv (\neg P\vee Q)\wedge (\neg P\vee R)\\
&\equiv (P\to Q)\wedge (P\to R).
\end{align*}

Note: the first and last equivalences used the equivalence $A\to B\equiv \neg A\vee B$. The middle equivalence is one of the two distributive laws.

\bigskip

\item For each of the following sets, describe the set in English, and then list the elements of the set using the ``roster method'':
\begin{align*}
 A &= \{x\in \mathbb{Z} : -3\leq x\leq 5\}  & B = \{x\in\R : x^2=4\}\\
 C &= \{2k+1 : k\in \Z\} & D = \{k\in\mathbb{Z} : k \text{ is even}\}
\end{align*}

\bigskip

$A = \{-3,-2,-1,0,1,2,3,4,5\}$ is the set of all integers from $-3$ to 5.

$B = \{-2,2\}$ is the set of all real numbers whose square is equal to 4.

$C = \{\ldots -5,-3,-1,1,3,5\ldots\}$ is the set of all odd integers.

$D = \{\ldots, -4,-2,0,2,4,6\ldots\}$ is the set of all even integers.
 \end{enumerate}
\end{document}