\documentclass[12pt]{article}
\usepackage{amsmath}
\usepackage{amssymb}
\usepackage[letterpaper,margin=0.85in,centering]{geometry}
\usepackage{fancyhdr}
\usepackage{enumerate}
\usepackage{lastpage}
\usepackage{multicol}
\usepackage{graphicx}

\reversemarginpar

\pagestyle{fancy}
\cfoot{}
\lhead{Math 2000B}\chead{Quiz \# 1}\lfoot{Thursday, 15\textsuperscript{th} January, 2015}
\rfoot{Total: 10 points}
\chead{{\bf Name:} Solutions}
\newcommand{\points}[1]{\marginpar{\hspace{24pt}[#1]}}
\newcommand{\skipline}{\vspace{12pt}}
%\renewcommand{\headrulewidth}{0in}
\headheight 30pt

\newcommand{\di}{\displaystyle}
\newcommand{\R}{\mathbb{R}}
\newcommand{\aaa}{\mathbf{a}}
\newcommand{\bbb}{\mathbf{b}}
\newcommand{\ccc}{\mathbf{c}}
\newcommand{\dotp}{\boldsymbol{\cdot}}
\begin{document}

%\author{Instructor: Sean Fitzpatrick}
\thispagestyle{fancy}
%\noindent{{\bf Name and student number:}}

 \begin{enumerate}
 \item  For the conditional statements below,
\begin{enumerate}
 \item Identify the hypothesis and conclusion. (Underlining and labelling each is fine.)
 \item Indicate whether the statement is true or false. 
\begin{enumerate}
 \item If $7>10$, then Stephen Harper will be Prime Minister For Life. \points{2}

\bigskip

Hypothesis: $7>10$. Conclusion: Stephen Harper will be Prime Minister For Life.\\
This is a {\bf true} statement, since the hypothesis is false.

\bigskip
 
 
 \item If $4$ is even, then 10 is odd. \points{2}

\bigskip

Hypothesis: 4 is even. Conclusion: 10 is odd.
This is a {\bf false} statement, since the hypothesis is true, but the conclusion is false.

\bigskip

 \item If $2+2=4$, then $6-5 = 1$. \points{2}
 
 \bigskip
 
 Hypothesis: $2+2=4$. Conclusion: $6-5=1$.
 This is a{\bf true} statement, since both the hypothesis and the conclusion are true.
\end{enumerate}

\end{enumerate}

\bigskip

\item Prove the following statement:
\begin{quotation}
 If $n$ is an even integer and $m$ is any integer, then $nm$ is an even integer.\points{4}
\end{quotation}
(You can use either a two-column proof or paragraph form.)

\bigskip


{\bf Proof}: Suppose that $n$ and $m$ are integers and that $n$ is even. Since $n$ is even, there is some integer $k$ such that $n=2k$. It follows that $nm = (2k)m = 2(km)$ is even, since it can be written as 2 times an integer.
 \end{enumerate}
\end{document}