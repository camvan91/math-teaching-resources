\documentclass[12pt]{article}
\usepackage{amsmath}
\usepackage{amssymb}
\usepackage[letterpaper,margin=0.85in,centering]{geometry}
\usepackage{fancyhdr}
\usepackage{enumerate}
\usepackage{lastpage}
\usepackage{multicol}
\usepackage{graphicx}

\reversemarginpar

\pagestyle{fancy}
\cfoot{}
\lhead{Math 2000B}\chead{Quiz \# 6}\rhead{Thursday, 26\textsuperscript{th} February, 2015}
\rfoot{Total: 10 points}
%\chead{{\bf Name:}}
\newcommand{\points}[1]{\marginpar{\hspace{24pt}[#1]}}
\newcommand{\skipline}{\vspace{12pt}}
%\renewcommand{\headrulewidth}{0in}
\headheight 30pt

\newcommand{\di}{\displaystyle}
\newcommand{\R}{\mathbb{R}}
\newcommand{\Z}{\mathbb{Z}}
\newcommand{\aaa}{\mathbf{a}}
\newcommand{\bbb}{\mathbf{b}}
\newcommand{\ccc}{\mathbf{c}}
\newcommand{\dotp}{\boldsymbol{\cdot}}
\renewcommand{\div}[2]{#1\, |\,#2}

\begin{document}
{\bf Name: Solutions}
%\author{Instructor: Sean Fitzpatrick}
\thispagestyle{fancy}
%\noindent{{\bf Name and student number:}}

\bigskip


Use mathematical induction to prove that for each natural number $n$,
\[
 2+5+8+\cdots + (3n-1) = \frac{n(3n+1)}{2}.
\]

{\bf Hint:} Before you begin, you might want some rough work on the side where you plug both $n=k$ and $n=k+1$ into the equation above. That way, when you get to the induction step, you'll know (a) what to assume, and (b) what you need to prove.

\bigskip

{\bf Solution:} Let $P(n)$ represent the predicate
\begin{equation}\label{1}
 2+5+\cdots + (3n-1) = \frac{n(3n+1)}{2}.
\end{equation}
When $n=1$, we see that $\dfrac{1(3(1)+1)}{2} = 2$, which shows that $P(1)$ (the base case) is true.

Assume that for some $k\geq 1$, we know that $P(k)$ is true; that is, that
\begin{equation}\label{2}
 2+5+\cdots + (3k-1) = \frac{k(3k+1)}{2}.
\end{equation}
We wish to show that $P(k)\to P(k+1)$. Note that if we set $n=k+1$ in \eqref{1}, we obtain
\begin{equation}\label{3}
 2+5+\cdots + (3k+2) = \frac{(k+1)(3k+4)}{2}.
\end{equation}
Thus, we need to show that \eqref{3} can be obtained from \eqref{2}. To see this, note that if we add $3k+2 = 3(k+1)-1$ to both sides of \eqref{2}, then we obtained
\begin{align*}
2+5+\cdots + (3k-1) + (3k+2) &= \frac{k(3k+1)}{2}+(3k+2)\\
& = \frac{3k^2+k+2(3k+2)}{2}\\
& = \frac{3k^2+7k+4}{2}\\
& = \frac{(k+1)(3k+4)}{2},
\end{align*}
which is what we needed to show. Thus, since $P(1)$ is true and $P(k)\to P(k+1)$ for all $k\geq 1$, it follows that $P(n)$ is true for all $n\in\mathbb{N}$, by induction.


\end{document}