\documentclass[12pt]{article}
\usepackage{amsmath}
\usepackage{amssymb}
\usepackage[letterpaper,margin=0.85in,centering]{geometry}
\usepackage{fancyhdr}
\usepackage{enumerate}
\usepackage{lastpage}
\usepackage{multicol}
\usepackage{graphicx}

\reversemarginpar

\pagestyle{fancy}
\cfoot{}
\lhead{Math 2000B}\chead{Quiz \# 3}\rhead{Thursday, 29\textsuperscript{th} January, 2015}
\rfoot{Total: 10 points}
%\chead{{\bf Name:}}
\newcommand{\points}[1]{\marginpar{\hspace{24pt}[#1]}}
\newcommand{\skipline}{\vspace{12pt}}
%\renewcommand{\headrulewidth}{0in}
\headheight 30pt

\newcommand{\di}{\displaystyle}
\newcommand{\R}{\mathbb{R}}
\newcommand{\Z}{\mathbb{Z}}
\newcommand{\aaa}{\mathbf{a}}
\newcommand{\bbb}{\mathbf{b}}
\newcommand{\ccc}{\mathbf{c}}
\newcommand{\dotp}{\boldsymbol{\cdot}}
\renewcommand{\div}[2]{#1\, |\,#2}

\begin{document}
{\bf Name: Solutions}
%\author{Instructor: Sean Fitzpatrick}
\thispagestyle{fancy}
%\noindent{{\bf Name and student number:}}

\bigskip


Prove any {\bf two} of the following three statements. (5 points each)
\begin{enumerate}
 \item For all integers $a, b,$ and $c$, with $a\neq 0$, if $\div{a}{b}$ and $\div{a}{c}$, then $\div{a}{(b-c)}$.

\bigskip

{\bf Solution:} First, as a reminder/clarification: the notation $\div{a}{b}$ represents a {\em statement}, not a number. If we write $\div{a}{b}$ we are saying that $a$ {\bf divides} $b$, which means that we can write $b=ak$ for some integer $k$. You can't replace it with expressions such as $\dfrac{b}{a}$ (which would represent a number; namely, a fraction). If you're ever unsure, try putting some numbers in to see if the sentence makes sense. \\
(We probably would not make a statement like ``If $\frac{2}{3}$ and $\frac{7}{3}$, then $\frac{-5}{3}$.'')

The proof is as follows: suppose $\div{a}{b}$ and $\div{a}{c}$. Then there exist integers $k$ and $l$ such that $b=ak$ and $c=al$. It follows that
\[
 b-c = ak-al = a(k-l).
\]
Since $k-l\in\Z$, it follows that $\div{a}{b-c}$.


 \item For any integer $n$, if $n$ is an odd integer, then $n^3$ is an odd integer.

\bigskip

There are two methods here: the ``brute force'' method, and the ``rely on previous knowledge'' method. First, the brute force method:

Suppose $n$ is an odd integer. Then $n=2k+1$ for some $k\in \Z$. It follows that 
\[
 n^3 = (2k+1)^3 = 8k^3+12k^2+6k+1 = 2(4k^3+6k+3k)+1,
\]
which is of the form $n^3=2l+1$, where $l$ is the integer $4k^3+6k+3k)$, so $n^3$ is odd.

{\bf Note:} Be careful with the expression $(2k+1)^3$. I noticed a few people with the result $8k^3+1$ as the quizzes were handed in. Don't forget that there are cross terms:
\[
 (2k+1)^3 = (2k+1)(2k+1)(2k+1).
\]
(I saw at least 5 or 6 like this, so you're not alone if you made this mistake.) If you've encountered the binomial formula before, you can raise a binomial to the third (or higher) power reasonably quickly:
\[
 (a+b)^n = a^n + na^{n-1}b + \binom{n}{2} a^{n-2}b^2+\cdots + \binom{n}{n-2}a^2b^{n-2}+nab^{n-1}+b^n,
\]
where $\binom{n}{k} = \dfrac{n!}{k!(n-k)!}$ are the ``binomial coefficients''. If you haven't seen these before (and/or have no idea what $n!$ means), don't worry about it. If you've ever encountered Pascal's Triangle, the binomial coefficients are the numbers you see there.

\newpage

Now, here is the ``rely on previous knowledge'' method:

From the textbook, we know that the product of two odd integers is odd. In particular, this means that if $n$ is odd, then $n^2 = n\cdot n$ is odd. Since $n$ is odd and $n^2$ is odd, it follows that
\[
 n^3 = n(n^2)
\]
is odd.

\medskip

If you don't want to quote the textbook, you could also write something like the following:

Lemma: the product of two odd integers is an odd integer.

Proof: suppose $x$ and $y$ are odd integers. Then there exist integers $k,l\in\Z$ such that $x=2k+1$ and $y=2l+1$, so
\[
 xy = (2k+1)(2l+1) = 4kl+2k+2l+1 = 2(2kl+k+l)+1
\]
is odd.

From here, you could proceed as above to conclude that $n$ and $n^2$ are both odd, so $n^3$ must be odd as well.

(A ``lemma'' is a small theorem needed to prove a bigger theorem.)

\bigskip

 \item For each integer $a$, if $\div{4}{(a-1)}$, then $\div{4}{(a^2-1)}$.

\bigskip

Let $a$ be an integer, and suppose that $\div{4}{(a-1)}$. Then there exists some $k\in \Z$ such that $a-1 = 4k$. It follows that
\[
 a^2-1 = (a-1)(a+1) = 4k(a+1) = 4[k(a+1)].
\]
Thus, $\div{4}{(a^2-1)}$.
\end{enumerate}


\end{document}