\documentclass[12pt]{article}
\usepackage{amsmath}
\usepackage{amssymb}
\usepackage[letterpaper,margin=0.85in,centering]{geometry}
\usepackage{fancyhdr}
\usepackage{enumerate}
\usepackage{lastpage}
\usepackage{multicol}
\usepackage{graphicx}

\reversemarginpar

\pagestyle{fancy}
\cfoot{}
\lhead{Math 2000B}\chead{Quiz \# 9}\rhead{Thursday, 19\textsuperscript{th} March, 2015}
\rfoot{Total: 10 points}
%\chead{{\bf Name:}}
\newcommand{\points}[1]{\marginpar{\hspace{24pt}[#1]}}
\newcommand{\skipline}{\vspace{12pt}}
%\renewcommand{\headrulewidth}{0in}
\headheight 30pt

\newcommand{\di}{\displaystyle}
\newcommand{\R}{\mathbb{R}}
\newcommand{\Z}{\mathbb{Z}}
\newcommand{\aaa}{\mathbf{a}}
\newcommand{\bbb}{\mathbf{b}}
\newcommand{\ccc}{\mathbf{c}}
\newcommand{\dotp}{\boldsymbol{\cdot}}
\renewcommand{\div}[2]{#1\, |\,#2}

\begin{document}
{\bf Name:}
%\author{Instructor: Sean Fitzpatrick}
\thispagestyle{fancy}
%\noindent{{\bf Name and student number:}}

\bigskip

{\bf Note:} There are questions on both sides of the page.

Let $\Z_5=\{0,1,2,3,4\}$ and $\Z_6 = \{0,1,2,3,4,5\}$. For each function below, determine whether it is an injection or a surjection (or neither, or both):
\begin{enumerate}
 \item $f:\Z_5\to\Z_5$, given by $f(x)=3x+2 \pmod{5}$. \points{3}

\bigskip

We compute the values of $f$ via the following table:
\[
 \begin{array}{c|c|c}
  x&3x+2&f(x)\\
\hline
  0&2&2\\
 1&5&0\\
 2&8&3\\
 3&11&1\\
 4&14&4
 \end{array}
\]
We see that $f$ attains every value in $\Z_5$, and never takes the same value twice; therefore, $f$ is both an injection and a surjection.

\bigskip

 \item $g:\Z_6\to \Z_6$, given by $g(x) = 3x+2 \pmod{6}$. \points{3}

\bigskip

We again compute the values of $g$ according to the table below:

\[
 \begin{array}{c|c|c}
  x&3x+2&g(x)\\
\hline
 0&2&2\\
 1&5&5\\
 2&8&2\\
 3&11&5\\
 4&14&2\\
 5&17&5
 \end{array}
\]
Since $\operatorname{range} g = \{2,5\}\neq \Z_6$, $g$ is not sujective, and since $g(0)=g(2)=2$, $g$ is not injective.

\newpage

 \item $h:\Z_5\to \Z_5$, given by $h(x) = x^3+4 \pmod{5}$. \points{3}

\bigskip

(Note: I've corrected the typo -- $h$ is defined mod 5, not mod 6.) The table of values for $h$ is given by
\[
 \begin{array}{c|c|c}
  x&x^3+4&h(x)\\
\hline
 0&4&4\\
 1&5&0\\
 2&12&2\\
 3&31&1\\
 4&68&3
 \end{array}
\]
We see that every element of $\Z_5$ appears as $h(x)$ for some $x$, so $h$ is a surjection, and no value of $h(x)$ appears twice, so $h$ is an injection.

\bigskip


 \item $H:\Z_5 \to \Z_6$, given by $H(x) = x^3+4 \pmod{6}$. \points{1}

\bigskip

This problem proceeds as above, except that we compute remainders modulo 6. The table of values is
\[
 \begin{array}{c|c|c}
  x&x^3+4&H(x)\\
\hline
 0&4&4\\
 1&5&5\\
 2&12&0\\
 3&31&1\\
 4&68&2
 \end{array}
\]
From the table, we see that $H$ is injective, since no value appears twice, but $H$ is not surjective, since $3\notin \operatorname{range}H$.
\end{enumerate}


\end{document}