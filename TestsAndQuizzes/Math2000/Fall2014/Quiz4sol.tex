\documentclass[12pt]{article}
\usepackage{amsmath}
\usepackage{amssymb}
\usepackage[letterpaper,margin=0.85in,centering]{geometry}
\usepackage{fancyhdr}
\usepackage{enumerate}
\usepackage{lastpage}
\usepackage{multicol}
\usepackage{graphicx}

\reversemarginpar

\pagestyle{fancy}
\cfoot{}
\lhead{Math 2000}\chead{Quiz \# 4}\rhead{Thursday, 2\textsuperscript{nd} October, 2014}
\rfoot{Total: 10 points}

\newcommand{\points}[1]{\marginpar{\hspace{24pt}[#1]}}
\newcommand{\skipline}{\vspace{12pt}}
%\renewcommand{\headrulewidth}{0in}
\headheight 30pt

\newcommand{\di}{\displaystyle}
\newcommand{\R}{\mathbb{R}}
\newcommand{\aaa}{\mathbf{a}}
\newcommand{\bbb}{\mathbf{b}}
\newcommand{\ccc}{\mathbf{c}}
\newcommand{\dotp}{\boldsymbol{\cdot}}
\begin{document}

%\author{Instructor: Sean Fitzpatrick}
\thispagestyle{fancy}
\noindent{{\bf Name and student number: Solutions}}

 \begin{enumerate}
 \item  Consider the assertion $\exists x\in\R: (\forall y\in\R, 2x-y=3)$.
\begin{enumerate}
 \item Is the statement true or false? Explain your answer. \points{3}

\bigskip

{\bf Solution}: The statement is false. Suppose such an $x$ exists; let's say $x=a$. Then it would have to be true that $2a-y=3$ for every $y\in\R$, but we must have $y=2a-3$, which is a unique number. (For example, if $a=2$, then $y=1$.) Since there is more than one real number, this is impossible.

\bigskip

 \item Write the negation of this assertion in symbolic form. \points{3}

\bigskip

{\bf Solution}: Using the rules for negation with quantifiers, we have
\begin{align*}
 \neg[\exists x\in\R: (\forall y\in \R, 2x-y=3)] &\equiv \forall x\in \R, \neg(\forall y\in \R, 2x-y=3)\\
&\equiv \forall x\in\R, (\exists y\in \R: \neg(2x-y=3))\\
&\equiv \forall x\in\R, (\exists y\in \R: 2x-y\neq 3).
\end{align*}
From this, we can also see that the statement in (a) is false, for its negation is certainly true: given any real number $x$, we can choose $y$ such that $2x-y\neq 3$; indeed, any $y\neq 2x-3$ will do the job.
\end{enumerate}
\newpage

\item Consider the assertion
\points{4}
\[
 \text{For all natural numbers } n, n^2+1 \text{ is prime.}
\]
(Note: a number $n\in\mathbb{N}$ is prime if it cannot be written in the form $n=a\cdot b$, where $a$ and $b$ are natural numbers other than 1 and $n$. For example, 5 is prime but 6 is not, since $6=2\cdot 3$.)

Calculate $n^2+1$ for $n=1,2,3,4.$ Based on this evidence, is the above assertion true or false? Explain your answer.

\bigskip

{\bf Solution}: We compute $n^2+1$ as follows:
\begin{align*}
 1^2 +1 & = 1\\
 2^2 +1 & = 5\\
 3^2 +1 & = 10\\
 4^2 +1 & = 17
\end{align*}
From the above, we see that when $n=3$, we have $3^2+1 = 10 = 2\cdot 5$, so $3^2+1$ is not prime, and thus it cannot be true that $n^2+1$ is prime for all natural numbers $n$, since 3 is a natural number, and $3^2+1$ is not prime.

 \end{enumerate}
\end{document}