\documentclass[12pt]{article}
\usepackage{amsmath}
\usepackage{amssymb}
\usepackage[letterpaper,margin=0.85in,centering]{geometry}
\usepackage{fancyhdr}
\usepackage{enumerate}
\usepackage{lastpage}
\usepackage{multicol}
\usepackage{graphicx}

\reversemarginpar

\pagestyle{fancy}
\cfoot{}
\lhead{Math 2000}\chead{Quiz \# 9}\rhead{Thursday, 13\textsuperscript{th} November, 2014}
\rfoot{Total: 10 points}

\newcommand{\points}[1]{\marginpar{\hspace{24pt}[#1]}}
\newcommand{\skipline}{\vspace{12pt}}
%\renewcommand{\headrulewidth}{0in}
\headheight 30pt

\newcommand{\di}{\displaystyle}
\newcommand{\R}{\mathbb{R}}
\newcommand{\Z}{\mathbb{Z}}
\newcommand{\N}{\mathbb{N}}
\newcommand{\modd}[3]{#1\equiv #2 \pmod{#3}}

\begin{document}

%\author{Instructor: Sean Fitzpatrick}
\thispagestyle{fancy}
\noindent{{\bf Name and student number:}}

 \begin{enumerate}
 \item Let $f:\R\to\R$ be defined by $f(x)=x^2$. If $A=[0,2]$ and $B=[-1,1]$, compute:
\begin{enumerate}
 \item $f(A\cap B)$\points{2}

\vspace{2in}

 \item $f(A)\cap f(B)$\points{2}

\vspace{2in}

 \item $f^{-1}(f(A))$\points{2}
\end{enumerate}
\newpage

\item Let $f:A\to B$ be a given function. We know that for any subset $C\subseteq A$, $C\subseteq f^{-1}(f(C))$, since if $x\in C$, then $f(x)\in f(C)=\{f(c)|c\in C\}$, and thus 
\[
x\in f^{-1}(f(C))=\{x\in A | f(x)\in f(C)\}. 
\]
Prove that if $f$ is one-to-one, then the reverse inclusion holds; that is, $f^{-1}(f(C))\subseteq C$.\points{4}

Hint: Suppose $x\in f^{-1}(f(C))$, and let $y= f(x)$. If $y\in f(C)$, that doesn't immediately guarantee that $x\in C$, but it does guarantee that there is some $c\in C$ such that $f(c)=y$.

 \end{enumerate}
\end{document}