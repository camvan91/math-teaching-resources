\documentclass[12pt]{article}
\usepackage{amsmath}
\usepackage{amssymb}
\usepackage[letterpaper,margin=0.85in,centering]{geometry}
\usepackage{fancyhdr}
\usepackage{enumerate}
\usepackage{lastpage}
\usepackage{multicol}
\usepackage{graphicx}

\reversemarginpar

\pagestyle{fancy}
\cfoot{}
\lhead{Math 2000}\chead{Quiz \# 5}\rhead{Thursday, 9\textsuperscript{th} October, 2014}
\rfoot{Total: 10 points}

\newcommand{\points}[1]{\marginpar{\hspace{24pt}[#1]}}
\newcommand{\skipline}{\vspace{12pt}}
%\renewcommand{\headrulewidth}{0in}
\headheight 30pt

\newcommand{\di}{\displaystyle}
\newcommand{\R}{\mathbb{R}}
\newcommand{\aaa}{\mathbf{a}}
\newcommand{\bbb}{\mathbf{b}}
\newcommand{\ccc}{\mathbf{c}}
\newcommand{\dotp}{\boldsymbol{\cdot}}
\begin{document}

%\author{Instructor: Sean Fitzpatrick}
\thispagestyle{fancy}
\noindent{{\bf Name and student number:} {\bf Solutions}}

\bigskip

 \begin{enumerate}
 \item  Let $A$ and $B$ be subsets of some universal set $U$. \points{5} Prove that if $A\subseteq B$, then $B^c\subseteq A^c$.

\bigskip

\noindent {\bf Solution}: By definition, $A\subseteq B$ if and only if for all $x\in U$, $x\in A \to x\in B$. But this is if and only if for all all $x\in U$, $x\notin B\to x\notin A$, by taking the contrapositive. This in turn tells us that for all $x\in U$, $x\in B^c\to x\in A^c$ by definition of the complement of a set, and therefore $B^c\subseteq A^c$, by definition of the subset relation.

\newpage

\item  Prove the following assertion, or give a counterexample to show that it is false:\points{5}

For any subsets $A$, $B$, $C$, and $D$ of some universal set $U$, if $A\subseteq C$ and $B\subseteq D$, and $A\cap B=\emptyset$, then $C\cap D=\emptyset$.
(Here, $\emptyset$ denotes the empty set.)

\bigskip

\noindent {\bf Solution}: Consider the case where $A=\{1\}$, $B=\{2\}$, and $C=D=\{1,2\}$. Then $A\subseteq C$, $B\subseteq D$, and $A\cap B = \emptyset$, but $C\cap D = \{1,2\}\neq \emptyset$.
\end{enumerate}
\end{document}