\documentclass[12pt]{article}
\usepackage{amsmath}
\usepackage{amssymb}
\usepackage[letterpaper,margin=0.85in,centering]{geometry}
\usepackage{fancyhdr}
\usepackage{enumerate}
\usepackage{lastpage}
\usepackage{multicol}
\usepackage{graphicx}

\reversemarginpar

\pagestyle{fancy}
\cfoot{}
\lhead{Math 2000}\chead{Quiz \# 4}\rhead{Thursday, 2\textsuperscript{nd} October, 2014}
\rfoot{Total: 10 points}

\newcommand{\points}[1]{\marginpar{\hspace{24pt}[#1]}}
\newcommand{\skipline}{\vspace{12pt}}
%\renewcommand{\headrulewidth}{0in}
\headheight 30pt

\newcommand{\di}{\displaystyle}
\newcommand{\R}{\mathbb{R}}
\newcommand{\aaa}{\mathbf{a}}
\newcommand{\bbb}{\mathbf{b}}
\newcommand{\ccc}{\mathbf{c}}
\newcommand{\dotp}{\boldsymbol{\cdot}}
\begin{document}

%\author{Instructor: Sean Fitzpatrick}
\thispagestyle{fancy}
\noindent{{\bf Name and student number:}}

 \begin{enumerate}
 \item  Consider the assertion $\exists x\in\R: (\forall y\in\R, 2x-y=3)$.
\begin{enumerate}
 \item Is the statement true or false? Explain your answer. \points{3}

\vspace{3.5in}

 \item Write the negation of this assertion in symbolic form. \points{3}
\end{enumerate}
\newpage

\item Consider the assertion
\points{4}
\[
 \text{For all natural numbers } n, n^2+1 \text{ is prime.}
\]
(Note: a number $n\in\mathbb{N}$ is prime if it cannot be written in the form $n=a\cdot b$, where $a$ and $b$ are natural numbers other than 1 and $n$. For example, 5 is prime but 6 is not, since $6=2\cdot 3$.)

Calculate $n^2+1$ for $n=1,2,3,4.$ Based on this evidence, is the above assertion true or false? Explain your answer.
 \end{enumerate}
\end{document}