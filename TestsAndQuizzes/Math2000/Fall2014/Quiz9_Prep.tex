\documentclass[letterpaper,12pt]{article}

\usepackage{ucs}
\usepackage[utf8x]{inputenc}
\usepackage{amsmath}
\usepackage{amsfonts}
\usepackage{amssymb}
\usepackage[margin=1in]{geometry}
\usepackage{multicol}

\newcommand{\N}{\mathbb{N}}
\newcommand{\Z}{\mathbb{Z}}
\newcommand{\R}{\mathbb{R}}
\newcommand{\Q}{\mathbb{Q}}
\newcommand{\modd}[3]{#1\equiv #2 \pmod{#3}}


\title{Practice Problems for Quiz 9\\Math 2000A}
\date{}
\begin{document}
 \maketitle
\vspace{-0.5in}

Quiz \#9 will take place in class on Thursday, November 13th. There's no class on the 11th (Remembrance Day) so this quiz covers less material than usual. 

As usual, solving the problems on this sheet will significantly improve your chances of getting a high score on the quiz.

{\bf Note to help session tutors}: It's 100\% OK for you to help my students solve these questions.
\begin{enumerate}
\item A function $f:\R\to\R$ is called {\em periodic} if there exists some $T\in\R$, with $T>0$, such that $f(x+T)=f(x)$ for all $x\in\R$. The smallest such $T$ is called the {\em period} of $f$. Prove that no periodic function can be one-to-one.

\item Let $f:S\to T$ be a function, and let $A$ and $B$ be subsets of $S$, and $C$ and $D$ subsets of $T$. For $x\in S$ and $y\in T$, carefully explain what it means to say that
\begin{multicols}{2}
 \begin{enumerate}
  \item $y\in f(A\cap B)$
  \item $y\in f(A\cup B)$
  \item $y\in f(A)\cap f(B)$
  \item $y\in f(A)\cup f(B)$
  \item $x\in f^{-1}(C\cap D)$
  \item $x\in f^{-1}(C\cup D)$
  \item $x\in f^{-1}(C)\cap f^{-1}(D)$
  \item $x\in f^{-1}(C)\cup f^{-1}(D)$
 \end{enumerate}
\end{multicols}
\item Let $f:\R\to\R$ be given by $f(x)=-2x+1$. Let
\[
 A = [2,5]\quad B = [-1,3]\quad C = [-2,3] \quad D =[1,4].
\]
Find each of the following:
\begin{multicols}{2}
 \begin{enumerate}
  \item $f(A)$
  \item $f^{-1}(f(A))$
  \item $f^{-1}(C)$
  \item $f(f^{-1}(C))$
  \item $f(A\cap B)$
  \item $f(A)\cap f(B)$
  \item $f^{-1}(C\cap D)$
  \item $f^{-1}(C)\cap f^{-1}(D)$
 \end{enumerate}
\end{multicols}
\item Repeat Question 3 for $f(x) = x^2+3$.
\item Let $f:A\to B$ be a given function, and let $g:A\to f(A)$ be defined by $g(x)=f(x)$ for all $x\in A$. Prove that $g$ is onto.
\newpage
\item Let $f:A\to B$ be a function, and let $C\subseteq A$.
\begin{enumerate}
 \item Prove that $C\subseteq f^{-1}(f(C))$.
 \item Give an example where $C$ is a {\em proper} subset of $f^{-1}(f(C))$.
 \item Prove that if $f$ is one-to-one, then $C=f^{-1}(f(C))$. (This fact might suggest an example for part (b))
\end{enumerate}
\item Let $g:A\to B$ be a function, and let $D\subseteq B$.
\begin{enumerate}
 \item Prove that $f(f^{-1}(D))\subseteq D$.
 \item Give an example where $f(f^{-1}(D))$ is a {\em proper} subset of $D$.
 \item Prove that if $f$ is onto, then $f(f^{-1}(D))=D$.
\end{enumerate}
\item Let $f:A\to B$ and let $C$ and $D$ be subsets of $B$. Prove the following, or give a counterexample to show it is false:
\begin{enumerate}
 \item If $C\subseteq D$, then $f^{-1}(C)\subseteq f^{-1}(D)$.
 \item If $f^{-1}(C)\subseteq f^{-1}(D)$, then $C\subseteq D$. 
\end{enumerate}
\item Let $f:A\to B$ and let $U$ and $V$ be subsets of $A$. Prove the following, or give a counterexample to show it is false:\label{a}
\begin{enumerate}
 \item $f(U\cap V)\subseteq f(U)\cap f(V)$
 \item $f(U)\cap f(V) \subseteq f(U\cap V)$
\end{enumerate}
\item For Question \ref{a}, you should have found that (a) was true, but (b) was false. Does it affect your answer if you know that $f$ is one-to-one?

\end{enumerate}



\end{document}
 
