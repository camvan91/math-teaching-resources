\documentclass[12pt]{article}
\usepackage{amsmath}
\usepackage{amssymb}
\usepackage[letterpaper,margin=0.85in,centering]{geometry}
\usepackage{fancyhdr}
\usepackage{enumerate}
\usepackage{lastpage}
\usepackage{multicol}
\usepackage{graphicx}

\reversemarginpar

\pagestyle{fancy}
\cfoot{}
\lhead{Math 2000}\chead{Quiz \# 12}\rhead{Thursday, 4\textsuperscript{th} December, 2014}
\rfoot{Total: 10 points}

\newcommand{\points}[1]{\marginpar{\hspace{24pt}[#1]}}
\newcommand{\skipline}{\vspace{12pt}}
%\renewcommand{\headrulewidth}{0in}
\headheight 30pt

\newcommand{\abs}[1]{\lvert #1\rvert}
\newcommand{\di}{\displaystyle}
\newcommand{\R}[2]{#1\,R\,#2}
\newcommand{\Z}{\mathbb{Z}}
\newcommand{\N}{\mathbb{N}}
\newcommand{\modd}[3]{#1\equiv #2 \pmod{#3}}

\begin{document}

%\author{Instructor: Sean Fitzpatrick}
\thispagestyle{fancy}
\noindent{{\bf Name and student number:}}

\bigskip


 \begin{enumerate}
\item Let $A = \{a,b,c\}$ and let $R=\{(a,a), (a,c), (b,b), (b,c), (c,a), (c,b)\}$ define a relation on $A$. Determine whether the following statements are true or false. Explain your answer.
\begin{enumerate}
 \item For each $x\in A$, $\R{x}{x}$. \points{1}

\vspace{1.5in}

 \item For every $x,y\in A$, if $\R{x}{y}$, then $\R{y}{x}$. \points{2}

\vspace{1.5in}

 \item For every $x,y,z\in A$, if $\R{x}{y}$ and $\R{y}{z}$, then $\R{x}{z}$.\points{2}

\vspace{1.5in}

 \item The relation $R$ defines a function from $A$ to $A$. \points{1}

\end{enumerate}
\newpage

\item Let $A=\{a,b\}$, and consider the relations $R_1=\{(a,a),(b,b)\}$ and $R_2=\{(a,a),(a,b)\}$. Show that $R_1$ is an equivalence relation but $R_2$ is not. Is $R_2$ transitive? \points{4}
 \end{enumerate}
\end{document}