\documentclass[12pt]{article}
\usepackage{amsmath}
\usepackage{amssymb}
\usepackage[letterpaper,top=1.35in,bottom=0.75in,left=0.75in,right=0.75in,centering]{geometry}
\usepackage{fancyhdr}
\usepackage{enumerate}
\usepackage{lastpage}
\usepackage{multicol}
\usepackage{graphicx}
\usepackage{vwcol}
\reversemarginpar

\pagestyle{fancy}
\cfoot{Page \thepage \ of \pageref{LastPage}}\rfoot{{\bf Total Points: 20}}
\lhead{\hspace*{2.2in}\underline{MATH 1560: Test 2}}

\newcommand{\points}[1]{\marginpar{\hspace{24pt}[#1]}}
\newcommand{\skipline}{\vspace{12pt}}
\renewcommand{\headrulewidth}{0in}
\headheight 20pt

\newcommand{\di}{\displaystyle}
\newcommand{\abs}[1]{\lvert #1\rvert}
\newcommand{\R}{\mathbb{R}}
\newcommand{\C}{\mathbb{C}}
\renewcommand{\P}{\mathcal{P}}
\DeclareMathOperator{\nul}{null}
\DeclareMathOperator{\range}{range}
\DeclareMathOperator{\spn}{span}
\newcommand{\len}[1]{\lVert #1\rVert}
\newcommand{\Q}{\mathbb{Q}}
\newcommand{\N}{\mathbb{N}}
\renewcommand{\L}{\mathcal{L}}
\newcommand{\dotp}{\boldsymbol{\cdot}}
\newenvironment{amatrix}[1]{%
  \left[\begin{array}{@{}*{#1}{c}|c@{}}
}{%
  \end{array}\right]
}
\newcommand{\bam}{\begin{amatrix}}
\newcommand{\eam}{\end{amatrix}}
\newcommand{\bbm}{\begin{bmatrix}}
\newcommand{\ebm}{\end{bmatrix}}

\begin{document}


\author{Instructor: Sean Fitzpatrick}
\thispagestyle{plain}
\begin{center}
\emph{University of Lethbridge}\\
Department of Mathematics and Computer Science\\
3 October, 2017\\
{\bf MATH 1560 - Test \#2 -- Group Stage}\\
Examiner: Sean Fitzpatrick
\end{center}

\skipline \skipline \skipline \noindent \skipline

Record the names of your group members below. Groups must contain between 3 and 5 members. \\

\textbf{Please print clearly.}

\skipline

\begin{enumerate}
\item Last Name:\underline{\hspace{200pt}} \quad First Name:\underline{\hspace{140pt}}

\skipline\skipline

\item Last Name:\underline{\hspace{200pt}} \quad First Name:\underline{\hspace{140pt}}

\skipline\skipline

\item Last Name:\underline{\hspace{200pt}} \quad First Name:\underline{\hspace{140pt}}

\skipline\skipline

\item Last Name:\underline{\hspace{200pt}} \quad First Name:\underline{\hspace{140pt}}

\skipline\skipline

\item Last Name:\underline{\hspace{200pt}} \quad First Name:\underline{\hspace{140pt}}

\end{enumerate}
%\skipline
%
%\skipline
%Student Number:\underline{\hspace{322pt}}\\
%\skipline



\vspace{0.1in}

\vspace*{\fill}

\begin{quote}
Print your name and student number clearly in the space above. You may remove this cover page, and use the back for scrap paper. If you want any work on the back of this page to be graded, you must clearly indicate this on the page containing the corresponding question.

\medskip

Answer the questions in the space provided. Show all work and necessary justification. Partial credit may be awarded for partially correct work.
 
\medskip

No outside aids are permitted, with the exception of a basic calculator. 
\end{quote}




\newpage

 \begin{enumerate}
 \item  Evaluate the limit: \points{3}\\
 \\
$\di  \lim_{x\to \infty}\,\frac{5+2x-3x^3}{5x^3-4x^2+7}=$
 
 
 \vspace{2in}
 
 \item Is the function \points{3}
 \[
 f(x) = \begin{cases} 5x-x^2, & \text{ if } x<2\\ 4x-2, & \text{ if } x\geq 2\end{cases}
 \]
 continuous at $x=2$? Why or why not? 
 
 \vspace{3in}
 
 \item Let $f(x) = \sqrt{x^2+1}$. Write down, but do not evaluate, a limit that computes $f'(0)$ \textit{according to the definition of the derivative}. \points{2}
 
\newpage

\item Compute $f'(x)$ for each function $f(x)$ below. You do \textbf{not} need to simplify your answers.
\begin{enumerate}
\item $f(x) = 4x^5-2x^3+\sqrt{2}x-3^4$\points{2}

\vspace{1.5in}

\item $f(x) = x^3\sin(x)$ \points{2}

\vspace{1.75in}

\item $f(x) = \dfrac{x^3-\sqrt{x}}{x^2}$ \points{3}

\vspace{2in}

\item $\di f(x) = e^{\sqrt{x^2+1}}$ \points{3}
\end{enumerate}

\newpage

\item (Extra group question!) Suppose $f$ and $g$ are continuous functions on an interval $[a,b]$, and you know that $f(a)<g(a)$, and $f(b)>g(b)$. \points{2}

Show that there must be some number $c\in (a,b)$ such that $f(c)=g(c)$.

\medskip

\textit{Hint:} Apply the Intermediate Value Theorem to $h(x)=f(x)-g(x)$. Be sure to justify your work.
\end{enumerate}
\end{document}