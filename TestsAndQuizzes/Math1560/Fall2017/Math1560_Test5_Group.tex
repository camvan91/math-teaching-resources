\documentclass[12pt]{article}
\usepackage{amsmath}
\usepackage{amssymb}
\usepackage[letterpaper,top=1.35in,bottom=0.75in,left=0.75in,right=0.75in,centering]{geometry}
\usepackage{fancyhdr}
\usepackage{enumerate}
\usepackage{lastpage}
\usepackage{multicol}
\usepackage{graphicx}
\usepackage{vwcol}
\reversemarginpar

\pagestyle{fancy}
\cfoot{Page \thepage \ of \pageref{LastPage}}\rfoot{{\bf Total Points: 22}}
\lhead{\hspace*{2.2in}\underline{MATH 1560: Test 5}}

\newcommand{\points}[1]{\marginpar{\hspace{24pt}[#1]}}
\newcommand{\skipline}{\vspace{12pt}}
\renewcommand{\headrulewidth}{0in}
\headheight 20pt

\newcommand{\di}{\displaystyle}
\newcommand{\abs}[1]{\lvert #1\rvert}
\newcommand{\R}{\mathbb{R}}
\newcommand{\C}{\mathbb{C}}
\renewcommand{\P}{\mathcal{P}}
\DeclareMathOperator{\nul}{null}
\DeclareMathOperator{\range}{range}
\DeclareMathOperator{\spn}{span}
\newcommand{\len}[1]{\lVert #1\rVert}
\newcommand{\Q}{\mathbb{Q}}
\newcommand{\N}{\mathbb{N}}
\renewcommand{\L}{\mathcal{L}}
\newcommand{\dotp}{\boldsymbol{\cdot}}
\newenvironment{amatrix}[1]{%
  \left[\begin{array}{@{}*{#1}{c}|c@{}}
}{%
  \end{array}\right]
}
\newcommand{\bam}{\begin{amatrix}}
\newcommand{\eam}{\end{amatrix}}
\newcommand{\bbm}{\begin{bmatrix}}
\newcommand{\ebm}{\end{bmatrix}}

\begin{document}


\author{Instructor: Sean Fitzpatrick}
\thispagestyle{plain}
\begin{center}
\emph{University of Lethbridge}\\
Department of Mathematics and Computer Science\\
21 November, 2017\\
{\bf MATH 1560 - Test \#5 -- Group Stage}\\
Examiner: Sean Fitzpatrick
\end{center}

\skipline \skipline \skipline \noindent \skipline

Record the names of your group members below. Groups must contain between 3 and 5 members. \\

\textbf{Please print clearly.}

\skipline

\begin{enumerate}
\item Last Name:\underline{\hspace{200pt}} \quad First Name:\underline{\hspace{140pt}}

\skipline\skipline

\item Last Name:\underline{\hspace{200pt}} \quad First Name:\underline{\hspace{140pt}}

\skipline\skipline

\item Last Name:\underline{\hspace{200pt}} \quad First Name:\underline{\hspace{140pt}}

\skipline\skipline

\item Last Name:\underline{\hspace{200pt}} \quad First Name:\underline{\hspace{140pt}}

\skipline\skipline

\item Last Name:\underline{\hspace{200pt}} \quad First Name:\underline{\hspace{140pt}}

\end{enumerate}
%\skipline
%
%\skipline
%Student Number:\underline{\hspace{322pt}}\\
%\skipline



\vspace{0.1in}

\vspace*{\fill}

\begin{quote}
Print your name and student number clearly in the space above. You may remove this cover page, and use the back for scrap paper. If you want any work on the back of this page to be graded, you must clearly indicate this on the page containing the corresponding question.

\medskip

Answer the questions in the space provided. Show all work and necessary justification. Partial credit may be awarded for partially correct work.
 
\medskip

No outside aids are permitted, with the exception of a basic calculator. 
\end{quote}





\newpage

 \begin{enumerate}
 \item Complete \textbf{one} of the following two problems. If you attempt parts of both, clearly indicate which one you wish to have graded. \points{8}
 \begin{enumerate}
 \item A box with a square base and no lid is required to hold a volume of $500 \text{ cm}^3$. What are the dimensions of the box that uses the least amount of material? Be sure to confirm that your answer is the minimum.
 \item A farmer has 60 metres of fencing. She plans to construct a rectangular pen, using a barn for one of the four sides. Given that the barn is 40 metres long, what is the largest area she can fence? Be sure to confirm that your answer is the maximum.
 \end{enumerate}
 
 \newpage
 
 
 \item Determine the linear approximation to $f(x)=x^6$ at $a=1$. Use your approximation to estimate the value of $(1.02)^6$. \points{4}
 
 \vspace{2.5in}
 
 \item Calculate the indicated Taylor polynomials. 
 \begin{enumerate}
 \item For $f(x)= \sin(2x)$, degree 5, about $a=0$. \points{3}
 
 \vspace{2.5in}
 
 \item For $g(x) = \dfrac{1}{x^2}$, degree 3, about $a=1$. \points{3}
 \end{enumerate}
 \newpage
 
 \item Extra group question! \points{4}
 
 Complete whichever one of the problems from Question 1 you did not already complete.
\end{enumerate}
\end{document}