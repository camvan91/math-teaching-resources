\documentclass[12pt]{article}
\usepackage{amsmath}
\usepackage{amssymb}
\usepackage[letterpaper,top=1.35in,bottom=0.75in,left=0.75in,right=0.75in,centering]{geometry}
\usepackage{fancyhdr}
\usepackage{enumerate}
\usepackage{lastpage}
\usepackage{multicol}
\usepackage{graphicx}
\usepackage{vwcol}
\reversemarginpar

\pagestyle{fancy}
\cfoot{Page \thepage \ of \pageref{LastPage}}\rfoot{{\bf Total Points: 24}}
\lhead{\hspace*{2.2in}\underline{MATH 1560: Test 6}}

\newcommand{\points}[1]{\marginpar{\hspace{24pt}[#1]}}
\newcommand{\skipline}{\vspace{12pt}}
\renewcommand{\headrulewidth}{0in}
\headheight 20pt

\newcommand{\di}{\displaystyle}
\newcommand{\abs}[1]{\lvert #1\rvert}
\newcommand{\R}{\mathbb{R}}
\newcommand{\C}{\mathbb{C}}
\renewcommand{\P}{\mathcal{P}}
\DeclareMathOperator{\nul}{null}
\DeclareMathOperator{\range}{range}
\DeclareMathOperator{\spn}{span}
\newcommand{\len}[1]{\lVert #1\rVert}
\newcommand{\Q}{\mathbb{Q}}
\newcommand{\N}{\mathbb{N}}
\renewcommand{\L}{\mathcal{L}}
\newcommand{\dotp}{\boldsymbol{\cdot}}
\newenvironment{amatrix}[1]{%
  \left[\begin{array}{@{}*{#1}{c}|c@{}}
}{%
  \end{array}\right]
}
\newcommand{\bam}{\begin{amatrix}}
\newcommand{\eam}{\end{amatrix}}
\newcommand{\bbm}{\begin{bmatrix}}
\newcommand{\ebm}{\end{bmatrix}}

\begin{document}


\author{Instructor: Sean Fitzpatrick}
\thispagestyle{plain}
\begin{center}
\emph{University of Lethbridge}\\
Department of Mathematics and Computer Science\\
5 December, 2017\\
{\bf MATH 1560 - Test \#6 -- Individual Stage}\\
Examiner: Sean Fitzpatrick
\end{center}
%\skipline \skipline \skipline \noindent \skipline
%Last Name:\underline{\hspace{350pt}}\\
%\skipline
%First Name:\underline{\hspace{348pt}}\\
%\skipline
%Student Number:\underline{\hspace{322pt}}\\
%\skipline



\vspace{0.1in}

\vspace*{\fill}

\begin{quote}
Print your name and student number clearly in the space above. You may remove this cover page, and use the back for scrap paper. If you want any work on the back of this page to be graded, you must clearly indicate this on the page containing the corresponding question.

\medskip

Answer the questions in the space provided. Show all work and necessary justification. Partial credit may be awarded for partially correct work.
 
\medskip

No outside aids are permitted, with the exception of a basic calculator. 
\end{quote}



\newpage

This page left intentionally blank. You may use this page for rough work.
\newpage

 \begin{enumerate}
 \item  Compute the following antiderivatives:
 \begin{enumerate}
 \item The antiderivative $F$ of $f(x) = 2x+\sec^2(x)$ such that $F(0)=4$. \points{3}
 
 \vspace{4.5cm}
 
 \item $\di \int (3x^2+2\sqrt{x}-5)\,dx$\points{3}
 
 \vspace{4.5cm}
 
 \item $\di \int \left(\cos(x)-\frac{1}{\sqrt{1-x^2}}\right)\,dx$ \points{3}
 
 \vspace{4.5cm}
 
 \item $\di \int x^3 e^{x^4+2}\,dx$ \points{3}
 \end{enumerate}
 \newpage
 
 \item Use Part I of the Fundamental Theorem of Calculus to compute the derivatives of the following functions:
 \begin{enumerate}
 \item $\di F(x) = \int_2^x \sin(t^2+3t)\,dt$ \points{2}
 
 \vspace{3cm}
 
 \item $\di G(x) = \int_x^{x^2}\sqrt{t^4+1}\,dt$ \points{3}
 
 \vspace{4cm}
  \end{enumerate}
 \item Use Part II of the Fundamental Theorem of Calculus to evaluate the following definite integrals:
 \begin{enumerate}
 \item $\di \int_0^1\left(3x^2-2x+4\right)\,dx$ \points{3}
 
 \vspace{4cm}
 
 \item $\di \int_0^{\pi/2}\cos(x)\sin^3(x)\,dx$ \points{4}
 \end{enumerate}
\end{enumerate}
\end{document}