\documentclass[12pt]{article}
\usepackage{amsmath}
\usepackage{amssymb}
\usepackage{fullpage}
\usepackage{fancyhdr}
\usepackage{lastpage}
%\input quizstyle.tex

\reversemarginpar

\pagestyle{fancy}
\addtolength{\headheight}{\baselineskip}

\lhead{{\bf Date:} Thursday, May 31st }
\chead{{\bf Time:} 7:10 pm}
\rhead{MAT232HF}
\cfoot{Page \thepage \ of \pageref{LastPage}}
\rfoot{{\bf Total Marks:} 60}

%Redefine the plain page style in fancyhdr package
\fancypagestyle{plain}{%  
\fancyhead{}
\fancyfoot{}
\fancyfoot[C]{Page \thepage \ of \pageref{LastPage}}
\renewcommand{\headrulewidth}{0pt}}

\newcounter{probnum}
\newcounter{subprobnum}

\DeclareMathOperator{\im}{im}
\DeclareMathOperator{\spn}{span}
\DeclareMathOperator{\col}{col}
\DeclareMathOperator{\rank}{rank}
\DeclareMathOperator{\diag}{diag}
\DeclareMathOperator{\adj}{adj}
\DeclareMathOperator{\nll}{null}
\DeclareMathOperator{\tr}{tr}

\newenvironment{problems}{
\begin{list}{\arabic{probnum}.}{\usecounter{probnum}}
}{
\end{list}
}

\newenvironment{subproblem}{ % start for subprob
\begin{list}{ % first arg for list
(\alph{subprobnum})
}{ % second arg for list
\usecounter{subprobnum}
\setlength{\topsep}{0in}
} % end of list def
}{ % end for subprob
\end{list}
}

\newcommand{\skipline}{\vspace{12pt}}
%\input local.tex
\renewcommand{\labelenumi}{(\roman{enumi})}
\newcommand{\sol}{\noindent {\bf Solution:  }}

\begin{document}
\thispagestyle{plain}
%Supresses the headers on the front page

\centerline {\bf University of Toronto at Mississauga}
\medskip
\centerline {\bf Mid-Term Exam}
\medskip
\centerline {\bf MAT232HF}
\centerline {\bf Calculus of Several Variables}
\medskip
\centerline {Instructor: Sean Fitzpatrick}
\centerline {Duration: 110 minutes}
\bigskip
\bigskip

\noindent {\bf NO AIDS ALLOWED.} \hfill {\bf Total: 60 marks}
\vglue .25truein
\begin{tabular}{ll}
Family Name: &\underline{SOLUTIONS \hspace{3.48in}} \\
   &{\hskip 2truein } {\footnotesize (Please Print)}\\
[15pt]
Given Name(s): &\underline{THE \hspace{4.1in}} \\
    &{\hskip 2truein } {\footnotesize (Please Print)}\\
[15pt]
Please sign here: &\underbar {\hskip 4.5in}\\
[25pt]
Student ID Number: &\underbar {\hskip 4.5in}\\
\end{tabular}
\bigskip


%\vspace{1in}
\begin{quote}
{\large \bf You may not use calculators, cell phones, or PDAs during
the exam.  Partial credit will be given for partially correct work.
Please read through the entire test before starting, and take note of
how many points each question is worth.  Please put a box around your
solutions so that the grader may find them easily.  }
\end{quote}

\vspace{.25in}
\begin{center}
\begin{tabular}{|l|r|}
\hline \hline
\multicolumn{2}{|c|}
{\rule[-3mm]{0mm}{8mm}
FOR MARKER'S USE ONLY} \\
\hline
Problem 1: & \hspace{.5in}  /11 \\ [3pt]
\hline
Problem 2: & \hspace{.5in}  /10 \\ [3pt]
\hline
Problem 3: & \hspace{.5in}  /8 \\ [3pt]
\hline
Problem 4: & \hspace{.5in}  /15 \\ [3pt]
\hline
Problem 5: & \hspace{.5in}  /10 \\ [3pt]
\hline
Problem 6: & \hspace{.5in}  /6 \\ [3pt]
\hline
\hline 
{\rule[-3mm]{0mm}{8mm} TOTAL:}  & /60  \\
\hline
\end{tabular}
\end{center}

\newpage
\vglue1pt
%vglue adds space between headers and text

\begin{problems}
\item   \begin{subproblem}
	\item Sketch the conic section $4y^2 - 9x^2 - 18x -8y = 41$ 
\marginpar{[3]}

\skipline
\sol We first complete the square in both $x$ and $y$, obtaining
\begin{equation*}
4(y-1)^2 - 9(x+1)^2 = 36.
\end{equation*}
Dividing through by 36, we obtain the standard form of a hyperbola,
\begin{equation*}
\frac{-(x-1)^2}{2^2} + \frac{(y-1)^2}{3^2}=1.
\end{equation*}
We see that the hyperbola has centre $(-1,1)$, and asymptotes $y-1 = \pm 
3/2 (x+1)$, and that $y=1$ is impossible, while $x=-1$ gives $y-1=\pm 3$, 
so the 
hyperbola opens vertically, with vertices $(-1,4)$ and $(-1,-2)$.

Your sketch should convey this information, and look roughly like a 
hyperbola.
\vspace{.25in}
	\item Sketch the parametric curve $x(t) =
3+2\cos t$, $y(t) = 5-3\sin t$, $t\in [0,2\pi]$. \marginpar{[4]}

\skipline
\sol We sketch the curve by first eliminating the parameter $t$.  Since we 
know that $\sin^2 t + \cos^2 t = 1$, we solve for $\sin t$ and $\cos t$, 
getting
\begin{equation*}
\cos t = \frac{x-3}{2},\quad \sin t = \frac{5-y}{3}.
\end{equation*}
Thus, we get that
\begin{equation*}
1 = \sin^2 t + \cos^2 t = \left(\frac{y-5}{3}\right)^2 + 
\left(\frac{x-3}{2}\right)^2,
\end{equation*}
or $\frac{(x-3)^2}{2^2} + \frac{(y-5)^2)}{3^2} = 1$, which is the equation 
of an ellipse, with centre $(3,5)$, and vertices $(3,2), (3, 8), (5,5), 
(1,5)$.

Since $\sin t$ and $\cos t$ both have period $2\pi$, we see that we get 
the entire ellipse for $t\in [0,2\pi]$.

\newpage
\vglue1pt
	\item Identify the traces of the surface $9x^2 + 4z^2
- 36y^2-36=0$ in each of the co-ordinate planes.  Then, sketch the 
surface. \marginpar{[4]}

\skipline
\sol The trace in the $yz$-plane ($x=0$) is the curve $4z^2-36y^2-36 = 0$, 
or $\frac{z^2}{3^2}-y^2 = 1$, which is a hyperbola.

Similarly, the trace in the $xz$-plane ($y=0$) is the ellipse 
$\frac{x^2}{2^2}+\frac{z^2}{3^2}=1$, and the trace in the $xy$-plane 
($z=0$) is the hyperbola $\frac{x^2}{3^2}-y^2 = 1$.

The surface is the hyperboloid of one sheet given by 
$\frac{x^2}{4}-y^2+\frac{z^2}{9}=1$ which has its axis in the $y$ 
direction (that is, it opens outward along the $y$-axis) and centre at the 
origin.  Your sketch should look roughly like figure 12.7.17 of the 
textbook, but turned on its side to line up with the $y$-axis.

	\end{subproblem}

\newpage
\vglue1pt

\item Let $\vec{u}$ and $\vec{v}$ be non-zero vectors such that $|\vec{u}
+ \vec{v}| = |\vec{u} - \vec{v}|$.
	\begin{subproblem}
	\item What can you conclude about the parallelogram spanned by
$\vec{u}$ and $\vec{v}$?\marginpar{[2]}

\skipline
\sol If we sketch the parallelogram in question, we see that its diagonals 
are the vectors $\vec{u}+\vec{v}$ and $\vec{u}-\vec{v}$.  Since the 
lengths 
of these two vectors are equal, we have a parallelogram whose diagonals 
are of equal length; that is, a rectangle.
\vspace{.6in}
	\item Show that $\vec{u}\cdot\vec{v} = 0$.\marginpar{[3]}

\skipline
\sol Recall that $\vec{u}\cdot\vec{v} = |\vec{u}||\vec{v}|\cos\theta$, 
where $\theta$ is the angle between the two vectors.  Since these vectors 
form the sides of a rectangle, we must have $\theta = \pi/2$, so that 
$\cos\theta = 0$.  Thus, $\vec{u}\cdot\vec{v}=0$.

Alternatively, this can be shown algebraically, using techniques similar 
to those used in part (c) below.
\vspace{.7in}
	\item Prove the paralleogram law: $|\vec{u}-\vec{v}|^2 +
|\vec{u}+\vec{v}|^2 = 2|\vec{u}|^2 + 2|\vec{v}|^2$.\marginpar{[5]}

\skipline
\sol Recall that for any vector $\vec{a}$, we have that $|\vec{a}|^2 = 
\vec{a}\cdot\vec{a}$.  Thus, we have:
\begin{align*}
|\vec{u}-\vec{v}|^2 + |\vec{u}+\vec{v}|^2 &= 
(\vec{u}-\vec{v})\cdot(\vec{u}-\vec{v}) + 
(\vec{u}+\vec{v})\cdot(\vec{u}+\vec{v})\\
&=(\vec{u}\cdot\vec{u} - \vec{u}\cdot\vec{v} - \vec{v}\cdot\vec{u} + 
\vec{v}\cdot\vec{v}) + (\vec{u}\cdot\vec{u} 
+ \vec{u}\cdot\vec{v} + \vec{v}\cdot\vec{u} + \vec{v}\cdot\vec{v})\\
&= |\vec{u}|^2 - 2\vec{u}\cdot\vec{v} + |\vec{v}|^2 + |\vec{u}|^2 + 
2\vec{u}\cdot\vec{v} + |\vec{v}|^2\\
&= 2|\vec{u}|^2 + 2|\vec{v}|^2.
\end{align*}

	\end{subproblem}
\newpage
\vglue1pt
\item Determine whether or not the following sets of points lie on the
same line.  If they do, give the vector equation of the line.  If not,
give the equation of the plane containing them.
	\begin{subproblem}
	\item $P = (0, -2, 4)$, $Q = (1, -3, 5)$, $R = (4, -6, 8)$
\marginpar{[4]}

\skipline
\sol We see that $\overrightarrow{PQ} = <1-0, -3-(-2), 5-4> = <1,-1,1>$, 
while 
similarly $\overrightarrow{QR} =  <3, -3, 3> = 3\overrightarrow{PQ}$.  
Thus the two vectors are 
parallel, and so the points $P,Q,R$ must all lie on the same line.

To get the equation of the line we need a point and a vector in the 
direction of the line.  The point $P$ and the vector $\overrightarrow{PQ}$ 
fit that 
description, so the equation of the line is
\begin{equation*}
\vec{r} = <x,y,z> = <0,-2,4> + t<1, -1, 1>.
\end{equation*}
\vspace{.5in}
	\item $P = (1, 1, 1)$, $Q = (3, -2, 3)$, $R = (3, 4, 6)$.
\marginpar{[4]}

\skipline
\sol In this case, we find $\overrightarrow{PQ} = <2, -3, 2>$, and 
$\overrightarrow{QR} = <0, 6, 
3>$, and these two vectors are clearly not parallel.  Thus the points do 
not lie on the same line, and therefore define a plane.

We can give the equation of the plane using a point on the plane ($P$ will 
do) and a normal vector.  The obvious candidate for normal vector is
\begin{align*}
\vec{n} = \overrightarrow{PQ}\times\overrightarrow{QR} &= \begin{vmatrix} 
\hat{\imath} & 
\hat{\jmath} & \hat{k}\\ 2&-3&2\\0&6&3\end{vmatrix}\\
&=\hat{\imath}(-3\cdot 3-2\cdot 6)-\hat{\jmath}(2\cdot 3-2\cdot 0) 
+\hat{k}(2\cdot 6 - (-3)\cdot 0)\\
&= -21\hat{\imath}-6\hat{\jmath}+12\hat{k}.
\end{align*}
Thus, the equation of the plane is
\begin{equation*}
-21(x-1)-6(y-1)+12(z-1) = 0.
\end{equation*}

	\end{subproblem}
\newpage
\vglue1pt
\item \begin{subproblem}
	\item Find all first-order partial derivatives 
of the following functions:\marginpar{[8]}
	\begin{enumerate}
		\item $f(x,y) = e^2e^{xy}$.

\skipline
\sol We have 
\begin{equation*}
f_x(x,y) = \frac{\partial}{\partial x}(e^2e^{xy}) = 
ye^2e^{xy},
\end{equation*}
and
\begin{equation*}
f_y(x,y) = \frac{\partial}{\partial y}(e^2e^{xy}) = 
xe^2e^{xy}.
\end{equation*}
\vspace{.3in}
                \item $h(x,y,z) = x^2y^3z^4$.

\skipline
\sol We have
\begin{equation*}
h_x(x,y,z) = 2xy^3z^4,
\end{equation*}

\begin{equation*}
h_y(x,y,z) = 3x^2y^2z^4,
\end{equation*}
and
\begin{equation*}
h_z(x,y,z) = 4x^2y^3z^3.
\end{equation*}
\vspace{.3in}
                \item $k(x,y,z) = z\sin(x-y)$.
\skipline
\sol We have
\begin{equation*}
k_x(x,y,z) = z\cos(x-y),
\end{equation*}

\begin{equation*}
k_y(x,y,z) = -z\cos(x-y),
\end{equation*}
and
\begin{equation*}
k_z(x,y,z) = \sin(x-y).
\end{equation*}
	\end{enumerate}
\newpage
\vglue1pt
	\item Verify that $f_{xy} = f_{yx}$ for $f(x,y) = xye^{-xy}$.
\marginpar{[4]}

\skipline
\sol The first partial derivatives are
\begin{equation*}
f_x(x,y) = ye^{-xy}-xy^2e^{-xy},
\end{equation*}
and
\begin{equation*}
f_y(x,y) = xe^{-xy}-x^2ye^{-xy}.
\end{equation*}
The mixed partials are therefore
\begin{align*}
f_{xy}(x,y) &= \frac{\partial}{\partial y}f_x(x,y)\\
&= e^{-xy}-xye^{-xy}-2xye^{-xy}+x^2y^2e^{-xy},
\end{align*}
while
\begin{align*}
f_{yx}(x,y) &= \frac{\partial}{\partial x}f_y(x,y)\\
&=  e^{-xy} - xye^{-xy}-2xye^{-xy} + x^2y^2e^{-xy}\\
&= f_{xy}(x,y).
\end{align*}

	\item Can there exist a continuous function $f(x,y)$
such that $f_x(x,y) = \cos^2(xy)$ and $f_y(x,y) = \sin^2(xy)$?  Why? 
\marginpar{[3]}

\skipline
\sol We see that both first-order partial derivatives of $f$ are 
continuous, and all of the second-order partial derivatives would be as 
well.  However, we can check that we would have $f_{xy}(x,y) = 
-2y\cos(xy)\sin(xy)$, while $f_{yx}(x,y) = 2x\sin(xy)\cos(xy)$.

Therefore, no such continuous function $f$ can exist, for if $f$ were a 
continuous function with continuous first and second order partial 
derivatives, then its mixed partial derivatives $f_{xy}$ and $f_{yx}$ 
would have to be the same.

\end{subproblem}
\newpage
\vglue1pt

\item Find and classify the critical points of the function 
\marginpar{[6]}
\begin{equation*} f(x,y) = 2x^3 + y^3 - 3x^2 -12x-3y \end{equation*}

\skipline
\sol The first partial derivatives of $f(x,y)$ are 
\begin{equation*}
f_x(x,y) = 6x^2 - 6x -12\quad \mbox{and}\quad f_y(x,y) = 3y^2-3,
\end{equation*}
and since these are continuous everywhere, the critical points of $f$ will 
be those points where both derivatives are zero:

If $f_x(x,y) = 6(x^2-x-2) = 6(x-2)(x+1) = 0$, then $x=2$, or $x=-1$, and 
if 
$f_y(x,y) = 3(y^2-1) = 3(y+1)(y-1) = 0$, then $y=1$ or $y=-1$, so the 
critical points of $f(x,y)$ are $(2,1)$, $(2,-1)$, $(-1,1)$ and $(-1,-1)$.

The second order partial derivatives of $f(x,y)$ are
\begin{equation*}
f_{xx}(x,y) = 12x-6,\quad f_{xy}(x,y) = f_{yx}(x,y) = 
0\quad\mbox{and}\quad f_{yy}(x,y) = 6y.
\end{equation*}

At the point $(2,1)$ we have $A=f_{xx}(2,1) = 18$, $B=f_{xy}(2,1)=0$, 
$C=f_{yy}(x,y)(2,1) = 6$, and $\Delta = AC-B^2 = 108$

Since $A>0$ and $\Delta>0$, $f(x,y)$ has a local minimum at $(2,1)$.

\skipline

Similarly, at the point $(2,-1)$ we have $A=18$, $B=0$, $C=-6$, and 
$\Delta = -108$.  Since $\Delta<0$ the critical point is neither a local 
maximum nor a local minimum.

\skipline

At $(-1,1)$ we have $A=-18$, $B=0$, $C=6$ and $\Delta = -108$.  Since 
$\Delta<0$, the critical point is neither a local maximum nor a local 
minimum.

\skipline

Finally, at $(-1,-1)$ we have $A=-18$, $B=0$, $C=-6$ and $\Delta = 108$.  
Since $A<0$ and $\Delta>0$, $f(x,y)$ has a local maximum at the point 
$(-1,-1)$.


\newpage
\vglue1pt

\item Evaluate the limit
\begin{equation*} \lim_{(x,y)\rightarrow
(0,0)}\frac{xy^2}{(x^2+y^2)^{3/2}} \end{equation*}
as $(x,y)$ approaches the origin along:
        \begin{subproblem}
        \item the $x$-axis.\marginpar{[1]}

\skipline
\sol Along the $x$-axis we have $y=0$, so the limit becomes
\begin{equation*}
\lim_{x\rightarrow 0} \frac{x\cdot 0}{(0^2+y^2)^{3/2}} = 0.
\end{equation*}
\vspace{.75in}
        \item the line $y=mx$. \marginpar{[2]}

\skipline
\sol Substituting $y=mx$, and noting that if $x\rightarrow 0$, so does 
$y$, the limit becomes
\begin{equation*}
\lim_{x\rightarrow 0}\frac{x(mx)^2}{(x^2+(mx)^2)^{3/2}} = 
\lim_{x\rightarrow 0}\frac{m^2x^3}{((1+m^2)x^2)^{3/2}} = 
\frac{m^2}{(1+m^2)^{3/2}}.
\end{equation*}
\vspace{.5in}
        \item the path $\vec{r}(t) = \frac{1}{t}\hat{\imath} + \frac{\sin
t}{t}\hat{\jmath}$, $t>0$. \marginpar{[3]}
        
\skipline
\sol We need the path defined by $\vec{r}(t)$ to approach the origin 
$(0,0)$; that is, for $x(t)\rightarrow 0$ and $y(t)\rightarrow 0$.  
For the given functions, this happens as $t\rightarrow \infty$.  

Thus, along the given path our limit becomes
\begin{equation*}
\lim_{t\rightarrow\infty}\frac{(\frac{1}{t})(\frac{\sin t}{t})^2}{\left( 
(\frac{1}{t})^2 + (\frac{\sin t}{t})^2\right)^{3/2}} = 
\lim_{t\rightarrow\infty}\frac{\sin^2 t}{(1+\sin^2 t)^{3/2}},
\end{equation*}
and this limit does not exist, since there arbitrarily large values of $t$ 
at which the expression is equal to either $0$ or $\frac{1}{2\sqrt{2}}$.

\end{subproblem}
\newpage
\vglue1pt
Extra space for rough work. Do {\bf not} tear out this page.
\end{problems}
\end{document}
\newpage

