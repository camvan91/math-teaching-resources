\documentclass[12pt]{article}
\usepackage{amsmath}
\usepackage{amssymb}
\usepackage[letterpaper,margin=0.85in,centering]{geometry}
\usepackage{fancyhdr}
\usepackage{enumerate}
\usepackage{lastpage}
\usepackage{multicol}
\usepackage{graphicx}

\reversemarginpar

\pagestyle{fancy}
\cfoot{Page \thepage \ of \pageref{LastPage}}\rfoot{{\bf Total Marks: 80}}
\chead{MATH 53}\lhead{Final Exam}\rhead{Friday, 10\textsuperscript{th} August, 2012}

\newcommand{\points}[1]{\marginpar{\hspace{24pt}[#1]}}
\newcommand{\skipline}{\vspace{12pt}}
\newcommand{\R}{\mathbb{R}}
%\renewcommand{\headrulewidth}{0in}
\headheight 30pt

\newcommand{\di}{\displaystyle}
\begin{document}

\author{Instructor: Sean Fitzpatrick}
\thispagestyle{plain}
\begin{center}
\emph{University of California, Berkeley}\\
Department of Mathematics\\
10\textsuperscript{th} August, 2012, 8:00-10:00 am\\
{\bf MATH 53 - Final Exam}\\
\end{center}
\skipline \skipline \skipline \noindent \skipline
Last Name:\underline{\hspace{350pt}}\\
\skipline
First Name:\underline{\hspace{348pt}}\\
\skipline
Student Number:\underline{\hspace{322pt}}\\
\skipline


\vspace{0.5in}


\begin{quote}
 {\bf Record your answers below each question in the space provided.    Left-hand pages may be used as scrap paper for rough work.  If you want any work on the left-hand pages to be graded, please indicate so on the right-hand page.
 
 \bigskip
 
Partial credit will be awarded for partially correct work, so be sure to show your work, and include all necessary justifications needed to support your arguments.}
\end{quote}


\vspace{0.5in}

For grader's use only:

\begin{table}[hbt]
\begin{center}
\begin{tabular}{|l|l|} \hline
Problem&Score\\
\hline \hline
\cline{1-2} 1 & \enspace\enspace\enspace\enspace\enspace\enspace/12\\
\cline{1-2} 2 & \enspace\enspace\enspace\enspace\enspace\enspace/10\\
\cline{1-2} 3 & \enspace\enspace\enspace\enspace\enspace\enspace/14\\
\cline{1-2} 4 & \enspace\enspace\enspace\enspace\enspace\enspace/16\\
\cline{1-2} 5 & \enspace\enspace\enspace\enspace\enspace\enspace/16\\
\cline{1-2} 6 & \enspace\enspace\enspace\enspace\enspace\enspace/12\\
%\cline{1-2} 6 & \enspace\enspace\enspace\enspace\enspace\enspace/10\\
\cline{1-2} Total & \enspace\enspace\enspace\enspace\enspace\enspace/80\\
\hline
\end{tabular}
\end{center}
\end{table}
\newpage


\begin{enumerate}
\item Let $S$ be the oriented surface given by the vector-valued function
\[
\vec{r}(u,v) = \langle v^2,-uv,u^2\rangle,\quad u\in [0,3],\, v\in [-3,3].
\]
\begin{enumerate}
\item Find the equation of tangent plane to $S$ at the point $(4,-2,1)$.\points{6}

\vspace{3in}

\item Set up, but do not evaluate, the integral that will compute the surface area of $S$. \points{4}

\vspace{3in}

\item At what (if any) points is the tangent plane to $S$ horizontal? \points{2}
\end{enumerate}

\newpage

\item Let $D$ be the region in the first quadrant bounded by the circles $x^2+y^2=4$ and $x^2+y^2=9$, and the hyperbolas $x^2-y^2=1$ and $x^2-y^2=4$.
\begin{enumerate}
\item Sketch the region $D$. \points{3}

\vspace{1.5in}

\item Evaluate the integral $\di \iint_D xy \, dA.$ \points{7}

\noindent Hint: the boundary curves for $D$ given above should suggest how to define the inverse transformation $T^{-1}$ that gives $u$ and $v$ in terms of $x$ and $y$. It won't be necessary to solve for $x$ and $y$ in terms of $u$ and $v$ in order to compute the Jacobian (although you can). Instead, use the relationship
\[
J_{T}(u,v) = \frac{1}{J_{T^{-1}}(x(u,v),y(u,v))}.
\]

\end{enumerate}
\newpage

\item Let $E$ be the solid region bounded by the surfaces $z=\sqrt{x^2+y^2}$ and $z=\sqrt{2-x^2-y^2}$. (You may want to sketch the region.)
\begin{enumerate}
\item Using {\bf cylindrical coordinates}, find the mass of the solid occupying the region $E$ if its mass density is given by $\delta(x,y,z) = \lambda z$, where $\lambda$ is a positive constant.\points{7}

\vspace{3.5in}

\item Using {\bf spherical coordinates}, find the volume of the solid bounded by $E$. \points{7}
\end{enumerate}

\newpage

\item Consider the vector field $\vec{F}(x,y) = (3x^2+2y^2)\hat{\imath} + 
(4xy+6y^2)\hat{\jmath}$.
	\begin{enumerate}
	
	\item Show that $\vec{F}(x,y)$ is conservative. \points{2}
\vspace{3in}
	\item Find a function $f(x,y)$ such that ${\bf \nabla} 
f(x,y) = \vec{F}(x,y)$.\points{5}
\newpage
%\vglue1pt

	\item If $C$ is given by the parabolic arc $x=2y^2$ from $(0,0)$ 
to $(2,1)$, followed by the line segment from $(2,1)$ to $(-1,3)$, compute the line integral ${\displaystyle \int_C \vec{F}\cdot 
d\vec{r}}$:
	\begin{enumerate}[(i)]
	\item Directly. \points{7}

\vspace{5in}

	\item Using the Fundamental Theorem of Calculus for line 
integrals.\points{2}
\end{enumerate}
	\end{enumerate}
	
\newpage

\item Let $S$ be the surface given by $z=4-x^2-y^2$, for $0\leq z\leq 3$, oriented with outward-pointing normal vector field, and let $\vec{F} = \langle yz, -xz, z^3\rangle$.
\begin{enumerate}
\item Sketch the surface, and indicate the direction of its positively-oriented boundary curve(s) $C$. \points{3}


\vspace{3in}

\item Compute $\nabla\times\vec{F}$. \points{2}


\vspace{1.5in}

\item Explain why Green's Theorem is a special case of Stokes' Theorem. \points{4}

\newpage

\item Compute $\di \iint_S (\nabla\times \vec{F})\cdot d\vec{S}$. \points{7}

\noindent (Hint: use Stokes' Theorem twice.)

\end{enumerate}
\newpage

\item Verify that the Divergence Theorem is true for the vector field $\vec{F} = x\hat{\imath}+y\hat{\jmath}+z\hat{k}$ and the region $E$ given by the solid ball $x^2+y^2+z^2\leq 9$. \points{12}

\end{enumerate}


\end{document}