\documentclass[12pt]{article}
\usepackage{amsmath}
\usepackage{amssymb}
\usepackage[margin=1in, letterpaper]{geometry}
\usepackage{fancyhdr}
\usepackage{lastpage}
\usepackage{enumerate}
%\input quizstyle.tex

\reversemarginpar

\pagestyle{fancy}
\addtolength{\headheight}{\baselineskip}

\lhead{{\bf Date:} 14$^{\text{th}}$ December, 2012 }
\chead{{\bf Time:} 11:30-1:30 pm}
\rhead{MATH 53}
\cfoot{Page \thepage \ of \pageref{LastPage}}
\rfoot{{\bf Total Marks:} 60}

%Redefine the plain page style in fancyhdr package
\fancypagestyle{plain}{%  
\fancyhead{}
\fancyfoot{}
\fancyfoot[C]{Page \thepage \ of \pageref{LastPage}}
\renewcommand{\headrulewidth}{0pt}}

\newcounter{probnum}
\newcounter{subprobnum}

\newcommand{\points}[1]{\marginpar{\hspace{24pt}[#1]}}
\newcommand{\R}{\mathbb{R}}
\renewcommand{\S}{\mathbf{S}}
\renewcommand{\r}{\mathbf{r}}
\newcommand{\dotp}{\,\boldsymbol{\cdot}\,}
\renewcommand*{\thefootnote}{\fnsymbol{footnote}}
\newcommand{\di}{\displaystyle}
\newcommand{\F}{\mathbf{F}}

\DeclareMathOperator{\im}{im}
\DeclareMathOperator{\spn}{span}
\DeclareMathOperator{\col}{col}
\DeclareMathOperator{\rank}{rank}
\DeclareMathOperator{\diag}{diag}
\DeclareMathOperator{\adj}{adj}
\DeclareMathOperator{\nll}{null}
\DeclareMathOperator{\tr}{tr}
\DeclareMathOperator{\curl}{curl}
\DeclareMathOperator{\Div}{div}


\newenvironment{problems}{
\begin{list}{\arabic{probnum}.}{\usecounter{probnum}}
}{
\end{list}
}

\newenvironment{subproblem}{ % start for subprob
\begin{list}{ % first arg for list
(\alph{subprobnum})
}{ % second arg for list
\usecounter{subprobnum}
\setlength{\topsep}{0in}
} % end of list def
}{ % end for subprob
\end{list}
}

\newcommand{\skipline}{\vspace{12pt}}
%\input local.tex
%\renewcommand{\labelenumi}{(\roman{enumi})}

\begin{document}
\thispagestyle{plain}
%Supresses the headers on the front page

\centerline {\bf University of California, Berkeley}

\bigskip

\centerline {FINAL EXAMINATION, Spring 2012}
\centerline {DURATION: $3$ hours}

\medskip

\centerline {Department of Mathematics}

\medskip

\centerline {{\bf MATH 49} Multivariable Calculus Equivalency}
 
\medskip

\centerline {Examiner: Sean Fitzpatrick}

\bigskip

\noindent {\bf Total: 60 points}
\vglue .25truein
\begin{tabular}{ll}
Family Name: &\underbar {\hskip 4.2in} \\
   &{\hskip 2truein } {\footnotesize (Please Print)}\\
[15pt]
Given Name(s): &\underbar {\hskip 4.2in} \\
    &{\hskip 2truein } {\footnotesize (Please Print)}\\
[15pt]
Please sign here: &\underbar {\hskip 4.2in}\\
[25pt]
Student ID Number: &\underbar {\hskip 4.2in}\\
\end{tabular}
\bigskip


\vspace{.15in}
\begin{quote}
{\large  No aids, electronic or otherwise, are permitted, with the exception of the formula sheet provided with your exam.  
Partial credit will be given for partially correct work. 
Please read through the entire exam before beginning, and take note of
how many points each question is worth.}
\end{quote}
\begin{center}
{\bf Good Luck!}
\end{center}
\vspace{.25in}

\begin{center}
\begin{tabular}{|l|r|}
\hline
\multicolumn{2}{|c|}
{\rule[-3mm]{0mm}{8mm}
FOR GRADER'S USE ONLY} \\
\hline
Problem 1: & \hspace{.5in}  /8 \\ [3pt]
\hline
Problem 2: & \hspace{.5in}  /10 \\ [3pt]
\hline
Problem 3: & \hspace{.5in}  /12 \\ [3pt]
\hline
Problem 4: & \hspace{.5in}  /8 \\ [3pt]
\hline
Problem 5: & \hspace{.5in}  /12 \\ [3pt]
\hline
Problem 6: & \hspace{.5in}  /10 \\ [3pt]
\hline

\hline 
  {\rule[-3mm]{0mm}{8mm} TOTAL:}  & /60  \\
\hline
\end{tabular}
\end{center}


\newpage
%\vglue1pt
%vglue adds space between headers and text

\begin{enumerate}
\item Evaluate the line integral $\di \int_C \F\dotp d\r$, where $\F(x,y,z) = \langle 2x, x+y, y+3z\rangle$, and $C$ is the straight line segment from $(1,0,2)$ to $(3,-4,4)$. \points{8}

\newpage


\item Consider the vector field $\F(x,y,z) = \langle y^2,axy+z^3,byz^2\rangle$, where $a$ and $b$ are constants.
\begin{enumerate}
\item Find values of $a$ and $b$ such that $\F$ is a conservative vector field, and then find a potential function $f(x,y,z)$ such that $\nabla f(x,y,z) = \F(x,y,z)$. \points{6}

\vspace{4in}

\item Compute $\di \int_C \F\dotp d\r$, where $C$ is the line segment between the points $(1,2,0)$ and $(2,-1,3)$
using the values of $a$ and $b$ found in part (a). \points{4}
\end{enumerate}
\newpage

\item Let $S$ be the parametric surface given by $\r(u,v)=\langle u\cos v, u\sin v, v\rangle$, for $u\in [0,2]$ and $v\in [-\pi,\pi]$. ($S$ is a ``spiral ramp''.)
\begin{enumerate}
\item Find an equation for the tangent plane to $S$ at the point $(0,1,\pi/2)$. \points{6}

\vspace{3.5in}

\item Find the surface area of $S$. \points{6}
\end{enumerate}
\newpage

\item Use Stokes' Theorem to evaluate $\di \iint_S \curl\F\dotp d\S$, where $\F(x,y,z) = \langle-y,x,-2\rangle$, and $S$ is the surface of the cone $z=\sqrt{x^2+y^2}$, for $1\leq z\leq 4$, oriented towards the $xy$-plane. \points{8}

\newpage

\item Verify that the Divergence Theorem is true for the vector field $\F(x,y,z) = \langle x^2, xy, z\rangle$ and the solid $E$ bounded by the paraboloid $z=4-x^2-y^2$ and the $xy$-plane. \points{12}

\newpage

\item  Prove {\em Gauss' Law}: Let $\F$ be the vector field $\F = \dfrac{\r}{\lVert\r\rVert^3}$, where $\r(x,y,z) = \langle x,y,z\rangle$. Let $E$ be any closed, bounded region in $\R^3$ with piecewise-smooth boundary $S$, oriented by the outward-pointing unit normal vector, such that $S$ does not pass through the origin. Then\points{10}
\[
\iint_S \F\dotp \,d\S = \begin{cases} 4\pi, & \text{ if } 0\in E\\ 0, & \text{ if } 0\neq E\end{cases},
\]
Hint: Calculate $\iint_S \F\dotp d\S=\iint_S \F\dotp \mathbf{n}\,dS$ directly for the sphere $x^2+y^2+z^2=a^2$. Then use the Divergence Theorem.
\end{enumerate}


\end{document}


