\documentclass[12pt]{article}
\usepackage{amsmath}
\usepackage{amssymb}
\usepackage{fullpage}
\usepackage{fancyhdr}
\usepackage{lastpage}
%\input quizstyle.tex

\reversemarginpar

\pagestyle{fancy}
\addtolength{\headheight}{\baselineskip}

\lhead{MAT232HF}
\rhead{Page \thepage \ of \pageref{LastPage}}
\cfoot{}
\rfoot{{\bf Total Marks:} 100}


%Redefine the plain page style in fancyhdr package
\fancypagestyle{plain}{%  
\fancyhead{}
\fancyfoot{}
\fancyfoot[C]{Page \thepage \ of \pageref{LastPage}}
\renewcommand{\headrulewidth}{0pt}}

\newcounter{probnum}
\newcounter{subprobnum}

\DeclareMathOperator{\im}{im}
\DeclareMathOperator{\spn}{span}
\DeclareMathOperator{\col}{col}
\DeclareMathOperator{\rank}{rank}
\DeclareMathOperator{\diag}{diag}
\DeclareMathOperator{\adj}{adj}
\DeclareMathOperator{\nll}{null}
\DeclareMathOperator{\tr}{tr}

\newenvironment{problems}{
\begin{list}{\arabic{probnum}.}{\usecounter{probnum}}
}{
\end{list}
}

\newenvironment{subproblem}{ % start for subprob
\begin{list}{ % first arg for list
(\alph{subprobnum})
}{ % second arg for list
\usecounter{subprobnum}
\setlength{\topsep}{0in}
} % end of list def
}{ % end for subprob
\end{list}
}

\newcommand{\skipline}{\vspace{12pt}}
%\input local.tex
\renewcommand{\labelenumi}{(\roman{enumi})}

\begin{document}
\thispagestyle{plain}
%Supresses the headers on the front page

\centerline {\bf UNIVERSITY OF TORONTO AT MISSISSAUGA}
\medskip
\centerline {\bf June 2007 Examination}
\medskip
\centerline {\bf MAT232HF}
\centerline {\bf Instructor: Sean Fitzpatrick}
\medskip
\centerline {\bf Duration: 3 Hours}
\bigskip
\bigskip

\noindent {\bf NO AIDS ALLOWED.} \hfill {\bf Total: 100 marks}
\vglue .25truein
\begin{tabular}{ll}
Family Name: &\underbar {\hskip 4.5in} \\
   &{\hskip 2truein } {\footnotesize (Please Print)}\\
[15pt]
Given Name(s): &\underbar {\hskip 4.5in} \\
    &{\hskip 2truein } {\footnotesize (Please Print)}\\
[15pt]
Please sign here: &\underbar {\hskip 4.5in}\\
[25pt]
Student ID Number: &\underbar {\hskip 4.5in}\\
\end{tabular}
\bigskip


%\vspace{1in}
\begin{quote}
{\em Students may be charged with an academic offence for 
possessing the following items during the writing of an exam:  any 
unauthorized aid, cell phones, pagers, wristwatch computers, personal 
digital assistants (PDAs), IPODS, MP3 players, or any other electronic 
device.  If any of these items are in your possession, please turn them 
off and put them with your belongings at the front of the room before the 
examination begins and no penalty will be imposed.  A penalty MAY BE 
imposed if any of these items are kept with you during the writing of your 
exam. 

\skipline

\noindent Please note, students are {\bf NOT} allowed to petition to 
RE-WRITE a final examination.} 
\end{quote}

\vspace{.25in}
\begin{center}
\begin{tabular}{|l|r|l|r|}
\hline \hline
\multicolumn{4}{|c|}
{\rule[-3mm]{0mm}{8mm}
FOR MARKER'S USE ONLY} \\
\hline
Problem 1: & \hspace{.5in}  /22 & Problem 4: & \hspace{.5in} /15 \\ [3pt]
\hline
Problem 2: & \hspace{.5in}  /16 & Problem 5: & \hspace{.5in} /15 \\ [3pt]
\hline
Problem 3: & \hspace{.5in}  /12 & Problem 6: & \hspace{.5in} /20 \\ [3pt]
\hline
\hline
 & & {\rule[-3mm]{0mm}{8mm} TOTAL:}  & /100  \\
\hline
\end{tabular}
\end{center}

\newpage
\vglue1pt

\noindent {\bf Instructions:} There are 6 problems in this exam, sorted by 
topic (not necessarily by difficulty).  Please read over the entire exam 
before beginning, and make sure that there are no missing pages.

Each of the six problems has several sub-parts.  The value of each 
sub-part is indicated in the left-hand margin in square brackets.
In the case that one sub-part depends on a part preceeding it, the correct 
use of an incorrect previous answer will receive full credit.

Please write your solutions in the space provided.  If there is not enough 
space you may continue your solution on the back of the {\em previous} 
page.  There are also two pages at the end of the exam for rough work.

Try to solve as many problems as possible.  Partial credit will be given 
for partially correct work.

\skipline

To aid in evaluating some of the integrals which you may encounter, a list 
of indefinite integrals is given below:

\begin{align*}
\int u^a d u &= \frac{u^{a+1}}{a+1}, \mbox{ for } a\neq -1\\
\int \sin\theta d \theta &= -\cos \theta\\
\int \cos\theta d \theta &= \sin \theta\\
\int \sin^2\theta d \theta &= \frac{\theta}{2} - \frac{1}{4}\sin 2\theta\\
\int \cos^2\theta d \theta & = \frac{\theta}{2} + \frac{1}{4} \sin 
2\theta\\
\int \sin^3\theta d \theta & = \frac{1}{3}\cos^3 \theta -\cos \theta\\
\int \sin^4\theta d \theta &= \frac{3\theta}{8} - \frac{1}{4} \sin 2\theta 
+\frac{1}{32} \sin 4\theta
\end{align*}


\newpage
\vglue1pt
%vglue adds space between headers and text

\begin{problems}
\item   \begin{subproblem}
	\item For the function $f(x,y) = \sin(x^2\ln y)$, find 
$f_x(x,y)$ and $f_y(x,y)$. 
\marginpar{[4]}
\vspace{3.5in}
	\item Let $w=\sqrt{u^2 + v^2 + z^2}$, where $u=3e^t\sin s$, $v= 
3e^t\cos s$, and $z=4e^t$.
	\begin{enumerate}
	\item Write out the chain rule formulas for ${\displaystyle 
\frac{\partial w}{\partial s}}$ and ${\displaystyle
\frac{\partial w}{\partial t}}$.\marginpar{[2]}
\newpage
\vglue1pt

	\item Calculate ${\displaystyle
\frac{\partial w}{\partial s}}$ and ${\displaystyle
\frac{\partial w}{\partial t}}$ using the formulas in (i). \marginpar{[4]}
\vspace{3.6in}
	\item Calculate ${\displaystyle
\frac{\partial w}{\partial s}}$ and ${\displaystyle
\frac{\partial w}{\partial t}}$ by explicitly making the given 
substitutions.  Confirm that your answers agree with those in (ii). 
\marginpar{[4]}
	\end{enumerate}
\newpage
\vglue1pt

	\item Let $h(x,y,z) = x^3 + y^3 + z^3 - 5xyz$.
	\begin{enumerate}
	\item Calculuate ${\bf \nabla} 
h(2,1,1)$.\marginpar{[3]}
\vspace{2.4in}
	\item Calculuate the directional derivative of $h(x,y,z)$ at the 
point $(2,1,1)$ in the direction of $\vec{a} = <1,0,-1>$. \marginpar{[2]}
\vspace{2.2in}
	\item Find the equation for the tangent plane to the surface 
$h(x,y,z) = 0$ at the point $(2,1,1)$.\marginpar{[3]}
	\end{enumerate}	
	\end{subproblem}

\newpage
\vglue1pt
\item Let $D$ be the region contained within the curve given in 
polar coordinates by $r=2\sin \theta$.
	\begin{subproblem}
	\item Sketch the region $D$, and find its area. \marginpar{[4]}
\vspace{3.6in}
	\item Suppose that $D$ bounds a plane lamina of mass density 
$\delta(x,y) = \sqrt{x^2 + y^2}$.  Find the mass of the lamina. 
\marginpar{[3]}
\newpage
\vglue1pt

	\item Find the centroid of the region $D$, with respect to the 
density $\delta(r,\theta) = 1$. \marginpar{[3]}
\vspace{4.5in}
	\item Find the volume of the solid of revolution obtained by 
revolving $D$ about the $x$-axis.\marginpar{[2]}

\noindent {\bf Hint:} $V= A\cdot d$, by the First Theorem of Pappus.
\newpage
\vglue1pt

	\item Find the volume of the solid which lies above $D$ and below 
the surface $z=4-\sqrt{(x^2+y^2)}$.\marginpar{[4]}
	\end{subproblem}
\newpage
\vglue1pt

\item Consider the parametric surface $S$ given by the equations
\begin{equation*}
x = a\sin u\cos v,\: y = b\sin u\sin v,\: z = c\cos u,
\end{equation*}
for $0\leq u\leq \pi$ and $0\leq v\leq 2\pi$.
	\begin{subproblem}
	\item Use trigonometric identities (such as $\sin^2 t + \cos^2 t = 
1$) to eliminiate the parameters $u$ and $v$, and obtain the equation of a 
quadric surface. \marginpar{[4]}
\vspace{3in}

	\item Identfiy and sketch this surface, if $a=1$, $b=1$, and 
$c=2$.\marginpar{[3]}
\newpage
\vglue1pt

	\item Set up the integral ${\displaystyle A(S) = \iint\limits_R 
\left|\vec{N}(u,v)\right| d u d v}$, in terms of $a$, $b$ and $c$,
which gives the surface area of $S$ \marginpar{[5]}
(where $\vec{N}(u,v)=\vec{r}_u\times\vec{r}_v$ is the normal to the 
surface $S$ at $(u,v)\in R$). 

Evaluate the integral in the case $a=b=c$. (Your answer should look 
familiar.)
	\end{subproblem}
\newpage
\vglue1pt
\item Let $D$ be the region in the $xy$-plane bounded by the curves $y=x$, 
$y=2x$, $xy=1$ and $xy=2$.
	\begin{subproblem}
	\item Sketch the region, noting all points of intersection of the 
given curves. \marginpar{[3]}
\vspace{2.5in}
	\item Find a change of variables $u=f(x,y)$, $v=g(x,y)$ that 
transforms the region $D$ into the rectangle $1\leq u\leq 2$, $1\leq v\leq 
2$.\marginpar{[2]}
\vspace{2in}
	\item Solve for $x$ and $y$ in terms of $u$ and 
$v$.\marginpar{[2]}
\newpage
\vglue1pt
	\item Compute the Jacobian $J={\displaystyle \frac{\partial 
(x,y)}{\partial (u,v)}}$. \marginpar{[3]}
\vspace{3.5in}
	\item Evaluate ${\displaystyle \iint\limits_D \left( 
\frac{(x-y)^2}{x^2} 
-1\right) d A}$.\marginpar{[5]}
	\end{subproblem}
\newpage
\vglue1pt

\item Consider the function $f(x,y) = 4xy-2x^4-y^2$.
	\begin{subproblem}
	\item Find the critical points of $f(x,y)$.\marginpar{[4]}
\vspace{3in}
	\item Classify all critical points found in (a). \marginpar{[6]}
\newpage
\vglue1pt

	\item Find the absolute maximum and minimum (if any) of $f(x,y)$ 
subject to the constraint $2x-y=1$.\marginpar{[5]}
	\end{subproblem}
\newpage
\vglue1pt

\item Consider the vector field $\vec{F}(x,y) = (3x^2+2y^2)\hat{\imath} + 
(4xy+6y^2)\hat{\jmath}$.
	\begin{subproblem}
	\item Show that $\vec{F}(x,y)$ is conservative. \marginpar{[3]}
\vspace{3in}
	\item Find a function $f(x,y)$ such that ${\bf \nabla} 
f(x,y) = \vec{F}(x,y)$.\marginpar{[5]}
\newpage
\vglue1pt

	\item If $C$ is the segment of the parabola $x=2y^2$ from $(0,0)$ 
to $(2,1)$, compute the line integral ${\displaystyle \int_C \vec{F}\cdot 
d\vec{r}}$:
	\begin{enumerate}
	\item Directly. \marginpar{[5]}
\newpage
\vglue1pt
	\item Using the Fundamental Theorem of Calculus for line 
integrals.\marginpar{[3]}
\vspace{3in}
	\end{enumerate}
	\item State Green's Theorem in the plane. \marginpar{[4]}
	\end{subproblem}
\newpage
\vglue1pt
Extra space for rough work. Do {\bf not} tear out this page.

\newpage
\vglue1pt
Extra space for rough work. Do {\bf not} tear out this page.

\end{problems}
\end{document}
\newpage

