\documentclass[12pt]{article}
\usepackage{amsmath}
\usepackage{amssymb}
\usepackage[margin=1in, letterpaper]{geometry}
\usepackage{fancyhdr}
\usepackage{lastpage}
\usepackage{enumerate}
%\input quizstyle.tex

\reversemarginpar

\pagestyle{fancy}
\addtolength{\headheight}{\baselineskip}

\lhead{{\bf Date:} 14$^{\text{th}}$ December, 2012 }
\chead{{\bf Time:} 11:30-2:30 pm}
\rhead{MATH 53}
\cfoot{Page \thepage \ of \pageref{LastPage}}
\rfoot{{\bf Total Marks:} 100}

%Redefine the plain page style in fancyhdr package
\fancypagestyle{plain}{%  
\fancyhead{}
\fancyfoot{}
\fancyfoot[C]{Page \thepage \ of \pageref{LastPage}}
\renewcommand{\headrulewidth}{0pt}}

\newcounter{probnum}
\newcounter{subprobnum}

\newcommand{\points}[1]{\marginpar{\hspace{24pt}[#1]}}
\newcommand{\R}{\mathbb{R}}
\renewcommand{\S}{\mathbf{S}}
\renewcommand{\r}{\mathbf{r}}
\newcommand{\dotp}{\,\boldsymbol{\cdot}\,}
\renewcommand*{\thefootnote}{\fnsymbol{footnote}}
\newcommand{\di}{\displaystyle}
\newcommand{\F}{\mathbf{F}}
\newcommand{\G}{\mathbf{G}}

\DeclareMathOperator{\im}{im}
\DeclareMathOperator{\spn}{span}
\DeclareMathOperator{\col}{col}
\DeclareMathOperator{\rank}{rank}
\DeclareMathOperator{\diag}{diag}
\DeclareMathOperator{\adj}{adj}
\DeclareMathOperator{\nll}{null}
\DeclareMathOperator{\tr}{tr}
\DeclareMathOperator{\curl}{curl}
\DeclareMathOperator{\Div}{div}


\newenvironment{problems}{
\begin{list}{\arabic{probnum}.}{\usecounter{probnum}}
}{
\end{list}
}

\newenvironment{subproblem}{ % start for subprob
\begin{list}{ % first arg for list
(\alph{subprobnum})
}{ % second arg for list
\usecounter{subprobnum}
\setlength{\topsep}{0in}
} % end of list def
}{ % end for subprob
\end{list}
}

\newcommand{\skipline}{\vspace{12pt}}
%\input local.tex
%\renewcommand{\labelenumi}{(\roman{enumi})}

\begin{document}
\thispagestyle{plain}
%Supresses the headers on the front page

\centerline {\bf University of California, Berkeley}

\bigskip

\centerline {FINAL EXAMINATION, Fall 2012}
\centerline {DURATION: $3$ hours}

\medskip

\centerline {Department of Mathematics}

\medskip

\centerline {{\bf MATH 53} Multivariable Calculus}
 
\medskip

\centerline {Examiner: Sean Fitzpatrick}

\bigskip

\noindent {\bf Total: 100 points}
\vglue .25truein
\begin{tabular}{ll}
Family Name: &\underbar {\hskip 4.2in} \\
   &{\hskip 2truein } {\footnotesize (Please Print)}\\
[15pt]
Given Name(s): &\underbar {\hskip 4.2in} \\
    &{\hskip 2truein } {\footnotesize (Please Print)}\\
[15pt]
Please sign here: &\underbar {\hskip 4.2in}\\
[25pt]
Student ID Number: &\underbar {\hskip 4.2in}\\
\end{tabular}
\bigskip


\vspace{.15in}
\begin{quote}
{\large  No aids, electronic or otherwise, are permitted, with the exception of the formula sheet provided with your exam.  
Partial credit will be given for partially correct work. 
Please read through the entire exam before beginning, and take note of
how many points each question is worth.}
\end{quote}
\begin{center}
{\bf Good Luck!}
\end{center}
\vspace{.25in}

\begin{center}
\begin{tabular}{|l|r|}
\hline
\multicolumn{2}{|c|}
{\rule[-3mm]{0mm}{8mm}
FOR GRADER'S USE ONLY} \\
\hline
Problem 1: & \hspace{.5in}  /14 \\ [3pt]
\hline
Problem 2: & \hspace{.5in}  /12 \\ [3pt]
\hline
Problem 3: & \hspace{.5in}  /12 \\ [3pt]
\hline
Problem 4: & \hspace{.5in}  /12 \\ [3pt]
\hline
Problem 5: & \hspace{.5in}  /10 \\ [3pt]
\hline
Problem 6: & \hspace{.5in}  /12 \\ [3pt]
\hline
Problem 7: & \hspace{.5in}  /16 \\ [3pt]
\hline
Problem 8: & \hspace{.5in}  /12 \\ [3pt]
\hline
\hline 
  {\rule[-3mm]{0mm}{8mm} TOTAL:}  & /100  \\
\hline
\end{tabular}
\end{center}


%\newpage
%\vglue1pt
%vglue adds space between headers and text

\begin{enumerate}
\item Consider the curve $C$ in $\R^2$ given by the vector-valued function $\r(t)=\langle t\sin t,t\cos t\rangle$, for $-\infty<t<\infty$.
\begin{enumerate}
\item Find the equations of the tangent lines to $C$ when $t=0,\, \pi/2$ and $-\pi/2$.  \points{5}

\vspace{4in}

\item If we restrict to $t\in [-\pi/2,\pi/2]$ we obtain a simple, closed curve. Sketch the curve using your results from part (a). Indicate the orientation of the curve.  \points{3} 

\newpage

\item For the simple, closed curve in part (b), set up, but do not evaluate\footnote{You don't need to evaluate the integrals, but you should attempt to simplify the integrand.}, integrals for
\begin{enumerate}[(i)]
\item The area enclosed by the curve. \points{3}

\vspace{3in}

\item The arc length of the curve. \points{3}

\end{enumerate}
%\vspace{2.5in}

%\item {\bf Bonus}: Sketch the curve for $t\in [-\pi,\pi]$. What about $t\in [-2\pi,2\pi]$? \points{4}
\end{enumerate}
\newpage

\item Let $f(x,y) = x^3+y^3+3xy-27$.
\begin{enumerate}
\item Compute $\nabla f(x,y)$. \points{2}

\vspace{1in}

\item Find and classify all critical points of $f$. \points{6}

\vspace{4in}

\item Compute the derivative of $f$ at the point $(2,4)$ in the direction of the {\em curve} given by $\r(t) = \langle 2t^2, 3t+1\rangle$. \points{4}

\end{enumerate}

\newpage

\item
\begin{enumerate}
\item Find the equation of the tangent plane to the surface $xyz^2=6$ at the point $(3,2,1)$. \points{5}

\vspace{3in}

\item If $\mathbf{c}(t) = \langle x(t),y(t)\rangle$ is a smooth curve in the $xy$-plane, and $z=f(x,y)$ is the graph of a continuously differentiable function, then $\r(t) = \langle x(t),y(t),f(x(t),y(t))\rangle$ is a smooth curve that lies on the graph. Find the equation of its tangent line at the point $(x_0,y_0,z_0)$ given by $t=t_0$. \points{7}

\end{enumerate}


\newpage

\item The integral $\di \int_{-2}^2\int_{0}^{\sqrt{4-x^2}}\int_{\sqrt{x^2+y^2}}^{\sqrt{8-x^2-y^2}}\,dz\,dy\,dx$ represents the volume of a solid.
\begin{enumerate}
\item Sketch the solid. \points{2}

\vspace{2in}

\item Re-write the integral using both cylindrical and spherical coordinates. \points{6}

\vspace{3in}

\item Find the volume using whichever one of the above integrals you prefer. \points{4}
\end{enumerate}
\newpage

\item Evaluate the integral $\di \iint_D xy\,dA$, where $D$ is the region in the first quadrant bounded by the curves $y=x$, $y=3x$, $xy=1$, and $xy=4$, using an appropriate change of variables. \points{10}

Hint: Let $u=y/x$ and $v=xy$, and then solve for $x$ and $y$ in terms of $u$ and $v$.

\newpage

\item Consider the vector field $\F(x,y,z) = \langle y^2,axy+z^3,byz^2\rangle$, where $a$ and $b$ are constants.
\begin{enumerate}
\item Find values of $a$ and $b$ such that $\F$ is a conservative vector field, and then find a potential function $f(x,y,z)$ such that $\nabla f(x,y,z) = \F(x,y,z)$. \points{8}

\vspace{4in}

\item Compute $\di \int_C \F\dotp d\r$, where $C$ is the line segment from the point $(1,2,0)$ to the point $(2,-1,3)$
using the values of $a$ and $b$ found in part (a). \points{4}
\end{enumerate}
\newpage

\item 
\begin{enumerate}
\item Use Stokes' Theorem to evaluate $\di \iint_S \curl\F\dotp d\S$, where $\F(x,y,z) = \langle z,-x,-y\rangle$, and $S$ is the surface of the paraboloid $z=x^2+y^2$, for $1\leq z\leq 4$, oriented towards the $xy$-plane. \points{8}

\vspace{3.5in}

\item Verify Stokes' Theorem in this case by computing the original surface integral directly. \points{8}

%\item Explain why we can conclude that $\F(x,y,z) = (x,-y,3)$ is the curl of some other vector field $\G(x,y,z)$ (that is, $\F = \nabla\times \G$), and --- without trying to find $\G$ --- evaluate $\di \iint_S \F\dotp d\S$, where $S$ is the hemisphere $z=\sqrt{4-x^2-y^2}$. \points{6}

\end{enumerate}
\newpage

%\item 
%\begin{enumerate}
%\item Let $C$ be any simple, closed, piecewise-smooth curve in $\R^2$ that does not pass through the point $(0,0)$. What are the possible values of the line integral 
%\[
%\int_C \frac{-y}{x^2+y^2}\,dx+\frac{x}{x^2+y^2}\,dy?
%\]
%Justify your answer. \points{6}
%
%\newpage
%
\item Prove {\em Gauss' Law}: Let $\F$ be the vector field $\F = \dfrac{\r}{\lVert\r\rVert^3}$, where $\r(x,y,z) = \langle x,y,z\rangle$. Let $E$ be any closed, bounded region in $\R^3$ with piecewise-smooth boundary $S$, oriented by the outward-pointing unit normal vector, such that $S$ does not pass through the origin. Then\points{12}
\[
\iint_S \F\dotp \,d\S = \begin{cases} 4\pi, & \text{ if } (0,0,0)\in E\\ 0, & \text{ if } (0,0,0)\notin E\end{cases},
\]
Hint: Calculate $\iint_S \F\dotp d\S = \iint_S \F\dotp \mathbf{n}\,dS$ directly for the sphere $x^2+y^2+z^2=a^2$. Then use the Divergence Theorem.
%\end{enumerate}
\end{enumerate}


\end{document}


