\documentclass[12pt]{article}
\usepackage{amsmath}
\usepackage{amssymb}
\usepackage{fullpage}
\usepackage{fancyhdr}
\usepackage{lastpage}
%\input quizstyle.tex

\reversemarginpar

\pagestyle{fancy}
\addtolength{\headheight}{\baselineskip}

\lhead{{\bf Date:} 6th November, 2008 }
\chead{{\bf Time:} 4:10 pm}
\rhead{MAT294H1Y}
\cfoot{Page \thepage \ of \pageref{LastPage}}
\rfoot{{\bf Total Marks:} 50}

%Redefine the plain page style in fancyhdr package
\fancypagestyle{plain}{%  
\fancyhead{}
\fancyfoot{}
\fancyfoot[C]{Page \thepage \ of \pageref{LastPage}}
\renewcommand{\headrulewidth}{0pt}}

\newcounter{probnum}
\newcounter{subprobnum}

\DeclareMathOperator{\im}{im}
\DeclareMathOperator{\spn}{span}
\DeclareMathOperator{\col}{col}
\DeclareMathOperator{\rank}{rank}
\DeclareMathOperator{\diag}{diag}
\DeclareMathOperator{\adj}{adj}
\DeclareMathOperator{\nll}{null}
\DeclareMathOperator{\tr}{tr}

\newenvironment{problems}{
\begin{list}{\arabic{probnum}.}{\usecounter{probnum}}
}{
\end{list}
}

\newenvironment{subproblem}{ % start for subprob
\begin{list}{ % first arg for list
(\alph{subprobnum})
}{ % second arg for list
\usecounter{subprobnum}
\setlength{\topsep}{0in}
} % end of list def
}{ % end for subprob
\end{list}
}

\newcommand{\skipline}{\vspace{12pt}}
%\input local.tex
\renewcommand{\labelenumi}{(\roman{enumi})}

\begin{document}
\thispagestyle{plain}
%Supresses the headers on the front page

\centerline {\bf FACULTY OF APPLIED SCIENCE AND ENGINEERING}
\centerline {\bf University of Toronto}
\medskip
\centerline {\bf MAT294H1Y}
\centerline {\bf Calculus and Differential Equations}
\medskip
\centerline {Term Test \#2}
\centerline {Duration: 110 minutes}
\bigskip
\bigskip

\noindent {\bf NO AIDS ALLOWED.} \hfill {\bf Total: 50 marks}
\vglue .25truein
\begin{tabular}{ll}
Family Name: &\underbar{SOLUTIONS {\hskip 3.5in}} \\
   &{\hskip 2truein } {\footnotesize (Please Print)}\\
[15pt]
Given Name(s): &\underbar{ THE {\hskip 4.05in}} \\
    &{\hskip 2truein } {\footnotesize (Please Print)}\\
[15pt]
Please sign here: &\underbar {\hskip 4.5in}\\
[25pt]
Student ID Number: &\underbar {\hskip 4.5in}\\
\end{tabular}
\bigskip


%\vspace{1in}
\begin{quote}
{\large \bf You may not use calculators, cell phones, or PDAs during
the test.  Partial credit will be given for partially correct work.
Please read through the entire test before starting, and take note of
how many points each question is worth.}
\end{quote}

\vspace{.25in}
\begin{center}
\begin{tabular}{|l|r|}
\hline
\multicolumn{2}{|c|}
{\rule[-3mm]{0mm}{8mm}
FOR MARKER'S USE ONLY} \\
\hline
Problem 1: & \hspace{.5in}  /10 \\ [3pt]
\hline
Problem 2: & \hspace{.5in}  /10 \\ [3pt]
\hline
Problem 3: & \hspace{.5in}  /10 \\ [3pt]
\hline
Problem 4: & \hspace{.5in}  /10 \\ [3pt]
\hline
Problem 5: & \hspace{.5in}  /10 \\ [3pt]
\hline
\hline 
  {\rule[-3mm]{0mm}{8mm} TOTAL:}  & /50  \\
\hline
\end{tabular}
\end{center}

\newpage
\vglue1pt
%vglue adds space between headers and text

\begin{problems}
\item
\begin{subproblem}
\item Evaluate the double integral \marginpar{[5]}
\begin{equation*}
\int^1_0\int^{\sqrt{y}}_y(x+y)\,dx\,dy.
\end{equation*}
\noindent {\bf Solution:}
\begin{align*}
\int^1_0\int^{\sqrt{y}}_y (x+y)\,dx\,dy & = \int^1_0 \left(\left.\frac{x^2}{2}+xy\right|_y^{\sqrt{y}}\right)\,dy\\
&=\int^1_0 (\frac{1}{2}y + y^{\frac{3}{2}}-\frac{1}{2}y^2 - y^2)\,dy\\
&=\left.\frac{1}{4}y^2+\frac{2}{5}y^{\frac{5}{2}}-\frac{1}{2}y^3 \right|_0^1\\
&=\frac{3}{20}
\end{align*}

\newpage
\vglue1pt

\item Evaluate the double integral \marginpar{[5]}
\begin{equation*}
\int_0^1\int^1_y e^{y/x}\,dx\,dy.
\end{equation*}
\noindent {\bf Hint:} Change the order of integration.

\noindent {\bf Solution:}
The region $y\leq x \leq 1$ and $0 \leq y \leq 1$ is the same as the region $0\leq x\leq 1$ and $0\leq y \leq x$ (Sketch it!)  Therefore, we have
\begin{align*}
\int_0^1\int^1_y e^{y/x}\,dx\,dy & = \int_0^1\int^x_0 e^{y/x}\,dy\,dx\\
& = \int^1_0 \left(\left.\frac{e^{y/x}}{1/x}\right|_0^x\right)\,dx\\
& = \int^1_0 x(e^1-e^0)\,dx\\
& = \left.(e-1)\frac{x^2}{2}\right|_0^1\\
& = \frac{e-1}{2}.
\end{align*}

\end{subproblem}
\newpage
\vglue1pt

\item \begin{subproblem}
\item Find the volume of the solid bounded by the coordinate planes, and the plane \marginpar{[5]}
\begin{equation*}
\frac{x}{a} + \frac{y}{b} + \frac{z}{c} = 1,
\end{equation*}
where $a, b, c >0$.

\noindent {\bf Solution:} We can express this as the volume under the surface $\displaystyle z = c - \frac{c}{a}x - \frac{c}{b}y$, and above the region $R$ bounded by the $x$-axis, the $y$-axis, and the line $\displaystyle y = b-\frac{b}{a}x$. (Set $z=0$ and solve for $y$.)  The volume is therefore given by the integral
\begin{align*}
 V &= \int_0^a\int_0^{b-\frac{b}{a}x}\left(c - \frac{c}{a}x - \frac{c}{b}y\right)\,dy\,dx\\
&= c\int_0^a \left(\left.y - \frac{x}{a}y - \frac{1}{2b}y^2 \right|^{b-\frac{b}{a}x}_0\right)\,dx\\
& =c\int^a_0 \left(b-\frac{b}{a}x - \frac{x}{a}\left(b-\frac{b}{a}x\right) - \frac{1}{2b}\left(b-\frac{b}{a}x\right)^2\right)\,dx\\
& = bc\int_0^a\left(\frac{1}{2}-\frac{x}{a}-\frac{x^2}{2a^2}\right)\,dx\\
& = \frac{bc}{2}\left(\left.x - \frac{x^2}{a} + \frac{x^3}{3a^2}\right|_0^a\right)\\
& = \frac{abc}{3}
\end{align*}
\newpage
\vglue1pt

\item Find the volume of the solid bounded by the cone $z^2 = x^2 + y^2$ and the cylinder $x^2+y^2 = 2y$. \marginpar{[5]}

\noindent {\bf Hint \#1:} Use polar coordinates.

\noindent {\bf Hint \#2:} $\displaystyle \frac{d}{d\theta}\left(\frac{\cos^3\theta}{3}-\cos\theta\right) = \sin^3\theta$.

\noindent {\bf Solution:} In polar coordinates, the two surfaces have the equations $z^2 = r^2$ and $r=2\sin\theta$.  The first equation gives $z=r$ (the upper half of the cone), and $z=-r$ (the lower half of the cone).  So we are trying to find the volume between these two surfaces over the region given by $0\leq r \leq 2\sin\theta$, and $0\leq \theta \leq \pi$, since the circle sits above the $x$-axis.  The volume is therefore:
\begin{align*}
V &= \int_0^\pi\int_0^{2\sin\theta}(r-(-r))r\,dr\,d\theta\\
&=\int_0^\pi\int_0^{2\sin\theta}2r^2\,dr\,d\theta\\
&=\int_0^\pi\frac{2}{3}(2\sin\theta)^3\,d\theta\\
&=\frac{16}{3}\int_0^\pi \sin^3\theta\,d\theta\\
&=\frac{16}{3}\left(\left.\frac{\cos^3\theta}{3}-\cos\theta\right|_0^\pi\right)\\
&=\frac{64}{9}
\end{align*}
\end{subproblem}


\newpage
\vglue1pt
\item A plane lamina has mass density $\delta(x,y) = xy$, and occupies the region $R$ given by the inequalities
\begin{equation*}
0\leq x \leq 1 \quad \text{ and }\quad x^2\leq y \leq 1.
\end{equation*}
\begin{subproblem}
\item Find the mass of the lamina. \marginpar{[4]}

\noindent {\bf Solution:} The mass of the lamina is given by
\begin{align*}m &= \iint_R \delta(x,y)\,dA\\
&=\int_0^1\int_{x^2}^1 xy \,dy\,dx\\
&=\int_0^1 \left(\left.\frac{1}{2}xy^2\right|_{x^2}^1\right)\,dx\\
&=\frac{1}{2}\int_0^1 x(1-x^4)\,dx\\
&=\frac{1}{2}\left(\left.\frac{x^2}{2}-\frac{x^6}{6}\right|_0^1\right)\\
&=\frac{1}{6}.
\end{align*}
\newpage
\vglue1pt
\item Find the centre of mass of the lamina.\marginpar{[6]}

\noindent {\bf Solution:} The centre of mass coordinates $(\overline{x},\overline{y})$ are given by
\begin{align*}
\overline{x} &= \frac{1}{m}\iint_R x\delta(x,y)\,dA \\
&= 6\int^1_0\int_{x^2}^1 x^2 y\,dy\,dx\\
&= 6\int^1_0 x^2 \left(\frac{1}{2}(1-x^4)\right)\,dx\\
&= 3\left(\left.\frac{x^3}{3} - \frac{x^7}{7}\right|_0^1\right)\\
&= \frac{4}{7}.
\end{align*}
\begin{align*}
\overline{y} & =\frac{1}{m}\iint_R y\delta(x,y)\,dA\\
&= 6\int^1_0\int_{x^2}^{1}xy^2\,dy\,dx\\
&= 6\int^1_0 \frac{1}{3}x(1^3 - (x^2)^3)\,dx\\
&= 2\frac{1}{3}\left(\left.\frac{x^2}{2}-\frac{x^8}{8}\right|^1_0\right)\\
&= \frac{3}{4}
\end{align*}


\end{subproblem}
\newpage
\vglue1pt

\item Let $T$ be the region in space bounded by the surfaces $z=\sqrt{x^2+y^2}$ and $z=\sqrt{2-x^2-y^2}$. (You may want to sketch the region.)
\begin{subproblem}
\item Using {\bf cylindrical coordinates}, find the mass of the solid bounded by $T$ whose mass density is given by $\delta(x,y,z) = \lambda z$, where $\lambda$ is a positive constant.\marginpar{[5]}

\noindent {\bf Solution:} $T$ is the $z$-simple region bounded above by the sphere $z=\sqrt{2-r^2}$ and below by the cone $z=r$.  The two surfaces intersect when $r=\sqrt{2-r^2}$, or $r=1$, so we have $0 \leq r \leq 1$, and $0 \leq \theta \leq 2\pi$.  Therefore, the mass of the solid is given by
\begin{align*}
m &= \iiint\limits_T \delta(x,y,z)\,dV\\
&=\int^{2\pi}_0\int^1_0\int^{\sqrt{2-r^2}}_r \lambda z\, r\,dz\,dr\,d\theta\\
&=\int^{2\pi}_0\int^1_0 \left(\left.\frac{\lambda}{2}z^2\right|_r^{\sqrt{2-r^2}}\right)r\,dr\,d\theta\\
&=\frac{\lambda}{2}\int^{2\pi}_0\int^1_0 (2r-2r^3)\,dr\,d\theta\\
&=\frac{\lambda}{2}\int^{2\pi}_0\left(\left.r^2-\frac{r^4}{2}\right|_0^1\right) \,d\theta\\
&=\frac{\lambda}{2}(2\pi)(1-\frac{1}{2})\\
&= \frac{\pi\lambda}{2}.
\end{align*}

\newpage
\vglue1pt
\item Using {\bf spherical coordinates}, find the volume of the solid bounded by $T$. \marginpar{[5]}

\noindent {\bf Solution:} In spherical coordinates, the sphere $z=\sqrt{2-x^2-y^2}$ has the equation $\rho = \sqrt{2}$ (the radius of the sphere).  Since $z=\rho\cos\phi$ and $r = \rho\sin\phi$, the equation of the cone gives
\[\rho\sin\phi = \rho\cos\phi,\]
which simplifies to $\tan\phi = 1$, which gives $\phi = \pi/4$.  The volume of the solid is therefore
\begin{align*}
V &= \iiint\limits_T \,dV\\
&= \int^{2\pi}_0\int^{\frac{\pi}{4}}_0\int^{\sqrt{2}}_0 \rho^2\sin\phi\,d\rho\,d\phi\,d\theta\\
&= 2\pi\int^{\frac{\pi}{4}}_0 \left(\left.\frac{\rho^3}{3}\right|^{\sqrt{2}}_0\right)\sin\phi \,d\phi\\
&= \frac{4\sqrt{2}\pi}{3}(\left.-\cos\phi)\right|^{\frac{\pi}{4}}_0\\
&= \frac{4\pi}{3}(\sqrt{2}-1).
\end{align*}


\end{subproblem}

\newpage
\vglue1pt
\item Evaluate the integral
\begin{equation*}
 \iint_R xy\,dx\,dy,
\end{equation*}
where $R$ is the region in the $1^{\text{st}}$ quadrant bounded by the curves $x^2+y^2 = 4$, $x^2+y^2=9$, $x^2-y^2 = 1$ and $x^2-y^2 =4$.

\noindent {\bf Solution:} If we set $u=x^2+y^2$ and $v=x^2-y^2$, then $R$ is the image of the rectangle $4\leq u \leq 9$, $1\leq v \leq 4$ under the transformation $T$ whose inverse is given by 
\[T^{-1}(x,y) = (x^2+y^2,x^2-y^2).\]
According to the change of variables formula,
\[\iint_R xy\,dx\,dy = \int^9_4\int_1^4 x(u,v)y(u,v)|J_T(u,v)|\,dv\,du.\]
Solving for $x$ and $y$ in terms of $u$ and $v$ is a bit of work, so we use the formula
\[J_T(u,v) = \frac{1}{J_{T^{-1}}(T(u,v))}\]
instead.

We have
\[J_{T^{-1}}(x,y) = \det\begin{pmatrix}\partial u/\partial x & \partial u/\partial y\\\partial v/\partial x & \partial v/\partial y\end{pmatrix}
 = \det\begin{pmatrix} 2x & 2y \\ 2x & -2y \end{pmatrix} = -8xy.\]
Therefore, we have
\begin{align*}
\iint_R xy\,dx\,dy &= \int^9_4\int_1^4 x(u,v)y(u,v)|J_T(u,v)|\,dv\,du\\
&= \int^9_4\int^4_1 x(u,v)y(u,v)\left|\frac{1}{8x(u,v)y(u,v)}\right|\,dv\,du\\
&= \int^9_4\int^4_1 \frac{1}{8} \,dv\,du\\
&= \frac{1}{8}(9-4)(4-1) = \frac{15}{8}.
\end{align*}

\end{problems}
\end{document}


