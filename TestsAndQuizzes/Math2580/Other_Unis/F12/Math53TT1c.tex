\documentclass[12pt]{article}
\usepackage{amsmath}
\usepackage{amssymb}
\usepackage[letterpaper,margin=0.85in,centering]{geometry}
\usepackage{fancyhdr}
\usepackage{enumerate}
\usepackage{lastpage}
\usepackage{multicol}
\usepackage{graphicx}

\reversemarginpar

\pagestyle{fancy}
\cfoot{Page \thepage \ of \pageref{LastPage}}\rfoot{{\bf Total Points: 36}}
\chead{MATH 53}\lhead{Test \# 1}\rhead{Friday, 5\textsuperscript{th} October, 2012}

\newcommand{\points}[1]{\marginpar{\hspace{24pt}[#1]}}
\newcommand{\skipline}{\vspace{12pt}}
%\renewcommand{\headrulewidth}{0in}
\headheight 30pt

\newcommand{\di}{\displaystyle}
\newcommand{\R}{\mathbb{R}}
\begin{document}

\author{Instructor: Sean Fitzpatrick}
\thispagestyle{plain}
\begin{center}
\emph{University of California, Berkeley}\\
Department of Mathematics\\
5\textsuperscript{th} October, 2012, 12:10-12:55 pm\\
{\bf MATH 53 - Test \#1}\\
\end{center}
\skipline \skipline \skipline \noindent \skipline
Last Name:\underline{\hspace{350pt}}\\
\skipline
First Name:\underline{\hspace{348pt}}\\
\skipline
Discussion Section: \underline{\hspace{307pt}}\\
\skipline
Name of GSI: \underline{\hspace{336pt}}\\

\vspace{0.5in}


\begin{quote}
 {\bf Record your answers below each question in the space provided.    Left-hand pages may be used as scrap paper for rough work.  If you want any work on the left-hand pages to be graded, please indicate so on the right-hand page.
 
 \bigskip
 
Partial credit will be awarded for partially correct work, so be sure to show your work, and include all necessary justifications needed to support your arguments.}
\end{quote}


\vspace{0.5in}

For grader's use only:

\begin{table}[hbt]
\begin{center}
\begin{tabular}{|l|r|} \hline
Page&Grade\\
\hline \hline
\cline{1-2} 1 & \enspace\enspace\enspace\enspace\enspace\enspace/12\\
\cline{1-2} 2 & \enspace\enspace\enspace\enspace\enspace\enspace/12\\
\cline{1-2} 3 & \enspace\enspace\enspace\enspace\enspace\enspace/12\\
\cline{1-2} Total & \enspace\enspace\enspace\enspace\enspace\enspace/36\\
\hline
\end{tabular}

\skipline

\skipline

\skipline

\skipline

C
\end{center}
\end{table}
\newpage


\begin{enumerate}
\item Find the equation of the tangent line to the curve $C$ represented by the vector-valued function $\mathbf{r}(t) = \langle t^3, \sin(\pi t), 2t+1\rangle$ at the point $(1,0,3)$.\points{4}

\vspace{2.5in}

\item Find the area of the cardioid $r=1+\cos\theta$. \points{5}

\vspace{3in}

\item Evaluate $\di \lim_{(x,y)\to (1,1)}\frac{x^3-y^3}{x-y}$, or explain why it does not exist. \points{3}
\newpage

\item Consider the two lines in $\R^3$ given by
\begin{align*}
\mathbf{r}_1(t) & = \langle 1,6,1\rangle + t\langle 0,4,-2\rangle\\
\mathbf{r}_2(s) & = \langle 0,3,0\rangle +s\langle 1,-1,3\rangle.
\end{align*}
\begin{enumerate}
\item Verify that the two lines intersect at the point $(1,2,3)$. \points{2}

\vspace{1.2in}

\item Find the cosine of the angle between the two lines. \points{3}

\vspace{1.3in}

\item Find the equation of the plane that contains the two lines. \points{4}

\vspace{2.2in}

\item Find the distance between the point $P(3,4,-2)$ and the plane from part (c).\points{3}
\end{enumerate}
\newpage


\item \begin{enumerate}
\item Find the linear approximation of the function $f(x,y,z)=\sqrt{x^2+2y^2+z^2}$ at the point $(4,5,6)$. (Note: $4^2+2(5^2)+6^2 = 100 = 10^2$.) \points{5}

\vspace{2.75in}

\item Use your result from part (a) to approximate the value of $\sqrt{(4.1)^2+2(4.95)^2+(6.03)^2}$ \points{2}

\vspace{1.7in}
\end{enumerate} 



\item Use the chain rule to compute $\dfrac{\partial f}{\partial u}$ and $\dfrac{\partial f}{\partial v}$ if $\di f(x,y,z) = xe^{y^2z}$, where $x = 2uv$, $y=u^2-v^2$, and $z=u$. \points{5}
\end{enumerate}



\end{document}