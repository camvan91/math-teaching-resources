\documentclass[12pt]{article}
\usepackage{amsmath}
\usepackage{amssymb}
\usepackage[letterpaper,margin=0.85in,centering]{geometry}
\usepackage{fancyhdr}
\usepackage{enumerate}
\usepackage{lastpage}
\usepackage{multicol}
\usepackage{graphicx}

\reversemarginpar

\pagestyle{fancy}
\cfoot{Page \thepage \ of \pageref{LastPage}}\rfoot{{\bf Total Points: 36}}
\chead{MATH 53}\lhead{Test \# 1}\rhead{Friday, 5\textsuperscript{th} October, 2012}

\newcommand{\points}[1]{\marginpar{\hspace{24pt}[#1]}}
\newcommand{\skipline}{\vspace{12pt}}
%\renewcommand{\headrulewidth}{0in}
\headheight 30pt

\newcommand{\di}{\displaystyle}
\newcommand{\R}{\mathbb{R}}
\newcommand{\vv}{\mathbf{v}}
\newcommand{\dotp}{\boldsymbol{\cdot}}
\newcommand{\len}[1]{\lVert #1\rVert}
\renewcommand{\r}{\mathbf{r}}

\begin{document}

\author{Instructor: Sean Fitzpatrick}
\thispagestyle{plain}
\begin{center}
\emph{University of California, Berkeley}\\
Department of Mathematics\\
5\textsuperscript{th} October, 2012, 12:10-12:55 pm\\
{\bf MATH 53 - Test \#1}\\
\end{center}
\skipline \skipline \skipline \noindent \skipline
Last Name:\underline{\hspace{350pt}}\\
\skipline
First Name:\underline{\hspace{348pt}}\\
\skipline
Discussion Section: \underline{\hspace{307pt}}\\
\skipline
Name of GSI: \underline{\hspace{336pt}}\\

\vspace{0.5in}


\begin{quote}
 {\bf Record your answers below each question in the space provided.    Left-hand pages may be used as scrap paper for rough work.  If you want any work on the left-hand pages to be graded, please indicate so on the right-hand page.
 
 \bigskip
 
Partial credit will be awarded for partially correct work, so be sure to show your work, and include all necessary justifications needed to support your arguments.}
\end{quote}


\vspace{0.5in}

For grader's use only:

\begin{table}[hbt]
\begin{center}
\begin{tabular}{|l|r|} \hline
Page&Grade\\
\hline \hline
\cline{1-2} 1 & \enspace\enspace\enspace\enspace\enspace\enspace/12\\
\cline{1-2} 2 & \enspace\enspace\enspace\enspace\enspace\enspace/12\\
\cline{1-2} 3 & \enspace\enspace\enspace\enspace\enspace\enspace/12\\
\cline{1-2} Total & \enspace\enspace\enspace\enspace\enspace\enspace/36\\
\hline
\end{tabular}

\skipline

\skipline

\skipline

\skipline

C
\end{center}
\end{table}
\newpage


\begin{enumerate}
\item Find the equation of the tangent line to the curve $C$ represented by the vector-valued function $\mathbf{r}(t) = \langle t^3, \sin(\pi t), 2t+1\rangle$ at the point $(1,0,3)$.\points{4}

\bigskip

The point $(1,0,3)$ corresponds to $t=1$, and we have $\r'(t) = \langle 3t^2, \pi\cos \pi t, 2\rangle$, so the tangent vector to the curve at $(1,0,3)$ is $\r'(1) = \langle 3,-\pi,2\rangle$. The equation of the tangent line is therefore
\[
 \langle x,y,z\rangle = \langle 1,0,3\rangle +t\langle 3, -\pi,2\rangle.
\]

\bigskip

\bigskip

\item Find the area of the cardioid $r=1+\cos\theta$. \points{5}

\bigskip

The cardioid is traced out for $0\leq \theta\leq 2\pi$, so its area is given by
\begin{align*}
 A &= \int_0^{2\pi} \frac{1}{2}\left(1+\cos\theta\right)^2d\theta\\
& = \frac{1}{2}\int_0^{2\pi} \left(1+2\cos\theta+\cos^2\theta\right)d\theta\\
& = \frac{1}{2}\int_0^{2\pi}\left(1+2\cos\theta+\frac{1}{2}+\frac{1}{2}\cos 2\theta\right)d\theta\\
& = \int_0^{2\pi}\left(\frac{3}{4}+\cos\theta+\frac{1}{4}\cos 2\theta\right)d\theta\\
& = \left[\frac{3}{4}\theta+\sin\theta+\frac{1}{8}\sin 2\theta\right]_0^{2\pi}\\
& = \frac{3\pi}{2}.
\end{align*}

\bigskip

\item Evaluate $\di \lim_{(x,y)\to (1,1)}\frac{x^3-y^3}{x-y}$, or explain why it does not exist. \points{3}

\bigskip

Since $x^3-y^3 = (x-y)(x^2+xy+y^2)$, we have
\[
 \lim_{(x,y)\to (1,1)}\frac{x^3-y^3}{x-y} = \lim_{(x,y)\to (1,1)}(x^2+xy+y^2) = 3
\]
by direct substitution, since polynomial functions are continuous.

\newpage

\item Consider the two lines in $\R^3$ given by
\begin{align*}
\mathbf{r}_1(t) & = \langle 1,6,1\rangle + t\langle 0,4,-2\rangle\\
\mathbf{r}_2(s) & = \langle 0,3,0\rangle +s\langle 1,-1,3\rangle.
\end{align*}
\begin{enumerate}
\item Verify that the two lines intersect at the point $(1,2,3)$. \points{2}

\bigskip

Since $\r_1(-1) = \langle 1,2,3\rangle$ and $\r_2(1) - \langle 1,2,3\rangle$, the two curves intersect at $(1,2,3)$ by direct computation.

\bigskip

\bigskip


\item Find the cosine of the angle between the two lines. \points{3}

\bigskip

The direction vectors for the two lines are $\vv_1 = \langle 0,4,-2\rangle$ and $\vv_2 = \langle 1,-1,3\rangle$, so the angle between the two lines at their point of intersection is given by 
\[
\cos\theta = \frac{\vv_1\dotp\vv_2}{\len{\vv_1}\len{\vv_2}}=\frac{0(1)+4(-1)-2(3)}{\sqrt{16+4}\sqrt{1+1+9}} =\frac{10}{\sqrt{20}\sqrt{11}}.
\]

\bigskip

\item Find the equation of the plane that contains the two lines. \points{4}

\bigskip

We know that the point $(1,2,3)$ lies on the plane, and a normal vector is given by
\[
\mathbf{n} = \vv_1\times\vv_2 = \langle 4(3)-(-2)(-1), -2(1)-0(3), 0(-1)-4(1)\rangle = \langle 10, -2, -4\rangle.
\]
The equation of the plane is therefore $10(x-1)-2(y-2)-4(z-3)=0$, or $10x-2y-4z+6=0$.

\bigskip



\item Find the distance between the point $P(3,4,-2)$ and the plane from part (c).\points{3}


\bigskip

The distance between a point $P(x_1,y_1,z_1)$ and the plane $ax+by+cz+d=0$ is given by $D=\dfrac{\lvert ax_1+by_1+cz_1+d\rvert}{\sqrt{a^2+b^2+c^2}}$, so the distance is equal to
\[
D = \frac{\lvert 10(3)-2(4)-4(-2)+6\rvert}{\sqrt{100+4+16}} = \frac{36}{\sqrt{120}}.
\]

\end{enumerate}
\newpage


\item \begin{enumerate}
\item Find the linear approximation of the function $f(x,y,z)=\sqrt{x^2+2y^2+z^2}$ at the point $(4,5,6)$. (Note: $4^2+2(5^2)+6^2 = 100 = 10^2$.) \points{5}

\bigskip

The partial derivatives of $f$ are given by
\[
\hspace{-24pt} f_x(x,y,z) = \frac{x}{\sqrt{x^2+2y^2+z^2}}, \, f_y(x,y,z) = \frac{2y}{\sqrt{x^2+2y^2+z^2}}, \,f_z(x,y,z) = \frac{z}{\sqrt{x^2+2y^2+z^2}},
\]
so at the point $(4,5,6)$ we have $f_x(4,5,6) = \dfrac{4}{10}=\dfrac{2}{5}$, $f_y(4,5,6) = \dfrac{10}{10} = 1$, and $f_z(4,5,6) = \dfrac{6}{10} = \dfrac{3}{5}$, so the function $L(x,y,z)$ that gives the desired linear approximation is
\begin{align*}
L(x,y,z) &= f(4,5,6) + f_x(4,5,6)(x-4)+f_y(4,5,6)(y-5)+f_z(4,5,6)(z-6)\\
 & = 10+\frac{2}{5}(x-4)+(y-5)+\frac{3}{5}(z-6).
\end{align*}

\bigskip


\item Use your result from part (a) to approximate the value of $\sqrt{(4.1)^2+2(4.95)^2+(6.03)^2}$ \points{2}

\bigskip

Since $(4.1, 4.95, 6.03)$ is close to $(4,5,6)$, we have $f(4.1, 4.95, 6.03)\approx L(4.1, 4.95, 6.03)$, and thus,
\begin{align*}
\sqrt{(4.1)^2+2(4.95)^2+(6.03)^2} &\approx 10 + \frac{2}{5}(0.1)+1(-0.05)+\frac{3}{5}(0.03)\\
 &= 10+0.04-0.05+0.018 = 10.008.
\end{align*}

\bigskip


\end{enumerate} 



\item Use the chain rule to compute $\dfrac{\partial f}{\partial u}$ and $\dfrac{\partial f}{\partial v}$ if $\di f(x,y,z) = xe^{y^2z}$, where $x = 2uv$, $y=u^2-v^2$, and $z=u$. \points{5}

\bigskip

We have
\begin{align*}
\frac{\partial f}{\partial u} & = \frac{\partial f}{\partial x}\frac{\partial x}{\partial u} + \frac{\partial f}{\partial y}\frac{\partial y}{\partial u} + \frac{\partial f}{\partial z}\frac{\partial z}{\partial u}\\
& = e^{y^2z}(2v)+2xyze^{y^2z}(2u)+xy^2e^{y^2z}(1)\\
& = 2ve^{(u^2-v^2)^2u}+8u^3v(u^2-v^2)e^{(u^2-v^2)^2u}+2uv(u^2-v^2)^2e^{(u^2-v^2)^2u},
\end{align*}
and
\begin{align*}
\frac{\partial f}{\partial v} & = \frac{\partial f}{\partial x}\frac{\partial x}{\partial v} + \frac{\partial f}{\partial y}\frac{\partial y}{\partial v} + \frac{\partial f}{\partial z}\frac{\partial z}{\partial v}\\
& = e^{y^2z}(2u)+2xyze^{y^2z}(-2v)+xy^2e^{y^2z}(0)\\
& = 2ue^{(u^2-v^2)^2u}-8u^2v^2(u^2-v^2)e^{(u^2-v^2)^2u}.
\end{align*}

\end{enumerate}



\end{document}