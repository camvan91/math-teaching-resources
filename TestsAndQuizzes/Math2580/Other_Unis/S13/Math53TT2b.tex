\documentclass[12pt]{article}
\usepackage{amsmath}
\usepackage{amssymb}
\usepackage[letterpaper,margin=0.85in,centering]{geometry}
\usepackage{fancyhdr}
\usepackage{enumerate}
\usepackage{lastpage}
\usepackage{multicol}
\usepackage{graphicx}

\reversemarginpar

\pagestyle{fancy}
\cfoot{Page \thepage \ of \pageref{LastPage}}\rfoot{{\bf Total Points: 40}}
\chead{MATH 53}\lhead{Test \# 2}\rhead{Friday, 15\textsuperscript{th} March, 2013}

\newcommand{\points}[1]{\marginpar{\hspace{24pt}[#1]}}
\newcommand{\skipline}{\vspace{12pt}}
%\renewcommand{\headrulewidth}{0in}
\headheight 30pt

\newcommand{\di}{\displaystyle}
\newcommand{\R}{\mathbb{R}}
\newcommand{\aaa}{\mathbf{a}}
\newcommand{\bbb}{\mathbf{b}}
\newcommand{\ccc}{\mathbf{c}}
\newcommand{\dotp}{\boldsymbol{\cdot}}
\newcommand{\pd}[2]{\frac{\partial #1}{\partial #2}}
\newcommand{\rd}[2]{\frac{d #1}{d #2}}
\begin{document}

\author{Instructor: Sean Fitzpatrick}
\thispagestyle{plain}
\begin{center}
\emph{University of California, Berkeley}\\
Department of Mathematics\\
15\textsuperscript{th} March, 2013, 12:10-12:55 pm\\
{\bf MATH 53 - Test \#2}\\
\end{center}
\skipline \skipline \skipline \noindent \skipline
Last Name:\underline{\hspace{350pt}}\\
\skipline
First Name:\underline{\hspace{348pt}}\\
\skipline
Student Number:\underline{\hspace{322pt}}\\
\skipline
Discussion Section: \underline{\hspace{307pt}}\\
\skipline
Name of GSI: \underline{\hspace{336pt}}\\

\vspace{0.5in}


\begin{quote}
 {\bf Record your answers below each question in the space provided.    Left-hand pages may be used as scrap paper for rough work.  If you want any work on the left-hand pages to be graded, please indicate so on the right-hand page.
 
 \bigskip
 
Partial credit will be awarded for partially correct work, so be sure to show your work, and include all necessary justifications needed to support your arguments. 

There is a list of potentially useful formulas available on the last page of the exam.}
\end{quote}


\vspace{0.5in}

For grader's use only:

\begin{table}[hbt]
\begin{center}
\begin{tabular}{|l|r|} \hline
Page&Grade\\
\hline \hline
\cline{1-2} 1 & \enspace\enspace\enspace\enspace\enspace\enspace/15\\
\cline{1-2} 2 & \enspace\enspace\enspace\enspace\enspace\enspace/15\\
\cline{1-2} 3 & \enspace\enspace\enspace\enspace\enspace\enspace/10\\
\cline{1-2} Total & \enspace\enspace\enspace\enspace\enspace\enspace/40\\
\hline
\end{tabular}

\skipline

\skipline

\skipline

B
\end{center}
\end{table}
\newpage


\begin{enumerate}
\item Let $f(x,y) = y^2e^{xy}$.
\begin{enumerate}
\item Find the linearization of $f$ at the point $(0,1)$. \points{4}

\vspace{1.6in}


\item Find the derivative of $f$ in the direction of $\mathbf{v} = \langle 3,-4\rangle$ at the point $(0,1)$. \points{3}

\vspace{1.5in}

\item If $x(t)=2-2t$ and $y(t) = t^2$, use the chain rule to find the tangent vector to the curve $\mathbf{r}(t) = \langle x(t),y(t),f(x(t),y(t))\rangle$ when $t=1$. \points{5}

\vspace{2.3in}

\item Verify that the tangent vector found in part (c) is tangent to the surface $z=f(x,y)$ at the point $(0,1,1)$. \points{3}
\end{enumerate}
\newpage

\item Let $f(x,y) = 8x^3+12xy-y^3$.
\begin{enumerate}
\item Find and classify the critical points of $f$. \points{8}

\vspace{4in}

\item Find the absolute maximum and minimum of $f$ on the set $D$ given by the triangular region with vertices at $(0,0)$, $(1,0)$, and $(1,-2)$. \points{7}
\end{enumerate}
\newpage

\item \begin{enumerate}
\item Define what it means for a function $f(x,y,z)$ to be {\em continuous} at a point $(a,b,c)$ in its domain. \points{2}

\vspace{1.5in}

\item Define what it means for a function $f(x,y,z)$ to be {\em differentiable} at a point $(a,b,c)$ in its domain. \points{3}

\vspace{2in}

\item Show that if $f$ is differentiable at a point $(a,b,c)$, then it is continuous at $(a,b,c)$.\points{5}

{\em Hint:} You can show this using only the above two definitions and the limit laws.
\end{enumerate}
\end{enumerate}
\newpage

\begin{center}
List of potentially useful formulas and facts:
\end{center}


\begin{itemize}
\item The limit of $f(x,y)$ as $(x,y)\to (a,b)$ is $L$ if for every $\epsilon>0$ there exists a $\delta>0$ such that
\[
0<\sqrt{(x-a)^2+(y-b)^2}<\delta \quad\Rightarrow\quad \lvert f(x,y)-L\rvert<\epsilon.
\]
\item Clairaut's Theorem: If $f$ is defined on a disk $D$ and $f_{xy},\, f_{yx}$ are continuous on $D$, then $f_{xy}=f_{yx}$ on $D$.
\item The linearization of a function $f(x,y)$ at a point $(a,b)$ is given by
\[
L(x,y) = f(a,b)+f_x(a,b)(x-a)+f_y(a,b)(y-b),
\]
and the equation of the tangent plane to $z=f(x,y)$ at the point $(a,b,f(a,b))$ is $z=L(x,y)$.
\item If the first order partial derivatives of a function $f$ exist in a neighborhood of a point, and are continuous at that point, then $f$ is differentiable at that point.
\item If $w=f(x,y,z)$ and $x=x(t),\, y=y(t),\, z=z(t)$, then
\[
\frac{dw}{dt} = \pd{f}{x}\rd{x}{t}+\pd{f}{y}\rd{y}{t}+\pd{f}{z}\rd{z}{t}.
\]
If $x=x(u,v), \, y=(u,v),\, z=z(u,v)$, then
\[
\pd{w}{u} = \pd{f}{x}\pd{x}{u}+\pd{f}{y}\pd{y}{u}+\pd{f}{z}\pd{z}{u},
\]
with a similar formula for $\partial w/\partial v$, and variations on the above Chain Rule formulas for other numbers of variables.
\item The derivative of $f$ in the direction of a vector $\mathbf{v}$ at a point $\mathbf{x}_0$ can be computed according to $\di D_{\mathbf{v}}f(\mathbf{x}_0) = \frac{\nabla f(\mathbf{x}_0)\dotp \mathbf{v}}{\lVert\mathbf{v}\rVert}$.
\item The tangent plane to a level surface $f(x,y,z)=k$ at a point $(a,b,c)$ is given by $\nabla f(a,b,c)\dotp\langle x-a, y-b, z-c\rangle = 0$.
\item The tangent vector to a curve $\mathbf{r}(t) = \langle x(t),y(t),z(t)\rangle$ is $\mathbf{r}'(t) = \langle x'(t), y'(t), z'(t)\rangle$.
\item Suppose $f$ has a critical point at $(a,b)$, and the second derivatives of $f$ are continuous on a neighborhood of $(a,b)$. Let $D = f_{xx}(a,b)f_{yy}(a,b)-f_{xy}(a,b)^2$. If
\begin{enumerate}[(a)]
\item $D>0$ and $f_{xx}(a,b)>0$, then $f$ has a local minimum at $(a,b)$.
\item $D>0$ and $f_{xx}(a,b)<0$, then $f$ has a local maximum at $(a,b)$.
\item $D<0$, then $f$ has a saddle point at $(a,b)$.
\end{enumerate}
\item If $f(x,y)$ has a maximum or minimum subject to the constraint $g(x,y)=c$ at $(a,b)$, then $\nabla f(a,b) = \lambda \nabla g(a,b)$.
\end{itemize}



\end{document}