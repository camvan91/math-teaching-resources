\documentclass[letterpaper,12pt]{article}

%\usepackage{ucs}
%\usepackage[utf8x]{inputenc}
\usepackage{amsmath}
\usepackage{amsfonts}
\usepackage{amssymb}
%\usepackage[canadian]{babel}
\usepackage[margin=1in]{geometry}
\newcommand{\R}{\mathbb{R}}
\renewcommand{\i}{\mathbf{i}}
\renewcommand{\j}{\mathbf{j}}
\renewcommand{\k}{\mathbf{k}}
\newcommand{\bbm}{\begin{bmatrix}}
\newcommand{\ebm}{\end{bmatrix}}    
\newcommand{\len}[1]{\lVert #1\rVert}
\newcommand{\dotp}{\boldsymbol{\cdot}}
                  
\title{Solutions for Quiz 1 Practice\\Math 2580\\Spring 2016}
\author{Sean Fitzpatrick}
\date{January 12th, 2016}

\begin{document}
 \maketitle

\begin{enumerate}
 \item At what point does the line through the point $(1,0,3)$ in the direction of the vector $\mathbf{v} = \i+2\j+\k$ cross the $xy$-plane?

\bigskip

The vector equation of the line is $\bbm x\\y\\z\ebm = \bbm 1\\0\\3\ebm+t\bbm 1\\2\\1\ebm$, and the $xy$-plane is given by $z=0$. Since $z=3+t$ for points on the line, setting $z=0$ gives $t=-3$, and thus $x=1+(-3)(1) = -2$ and $y=0+(-3)(2) = -6$, so the point is $(-2,-6,0)$.

\medskip

{\bf Quiz version:} The line is through the point $(4,-2,3)$ in the direction of $\mathbf{v} = 2\i+3\j-4\k$, so the vector equation of the line is
\[
 \bbm x\\y\\z\ebm = \bbm 4\\-2\\3\ebm+t\bbm 2\\3\\-4\ebm,
\]
and the $yz$-plane is given by $x=0$. At the point of intersection we have both $x=0$ and $x=4+2t$, so we must have $t=-2$, which gives us $y=-2-2(3)=-8$ and $z=3-2(-4) = 11$. The point is therefore $(0,-8,11)$.

 \item Find the distance from the point $(1,2,0)$ to the plane $x-2y+z=4$.

\bigskip

The given plane has normal vector $\mathbf{n} = \i-2\j+\k$, and the point $(4,0,0)$ lies on the plane. If we let
\[
 \mathbf{v} = \bbm 1-4\\2-0\\0-0\ebm = \bbm -3\\2\\0\ebm
\]
denote the vector from the point $(4,0,0)$ to the given point $(1,2,0)$, then the distance is given by the length of the projection of $\mathbf{v}$ onto $\mathbf{n}$:
\[
 d = \len{\operatorname{proj}_{\mathbf{n}}\mathbf{v}} = \left\lVert\frac{\mathbf{v}\dotp \mathbf{n}}{\len{\mathbf{n}}^2}\mathbf{n}\right\rVert = \left\lvert \frac{\mathbf{v}\dotp \mathbf{n}}{\len{\mathbf{n}}} \right\rvert.
\]
We compute $\len{\mathbf{n}} = \sqrt{1^2+(-2)^2+1^2} = \sqrt{6}$ and $\mathbf{v}\dotp \mathbf{n} = -3(1)+2(-2)+0(1) = -7$, so the distance is $d = \dfrac{7}{\sqrt{6}}$.

\bigskip

{\bf Alternative solution:} The line through $(1,2,0)$ in the direction of $\mathbf{n} = \i-2\j+\k$ is given by $x=1+t$, $y=2-2t$, and $z=t$. At the point of intersection of this line and the plane $x-2y+z=4$ we must have 
\[
 (1+t)-2(2-2t)+t=4,
\]
which gives $6t=7$, so $t=7/6$. The point $Q$ on the plane closest to the the point $P=(1,2,0)$ is therefore $Q=(1+7/6,2-2(7/6),7/6) = (13/6, -1/3, 7/6)$, and the distance from $P$ to $Q$ is
\begin{align*}
 d &= \sqrt{(13/6-1)^2+(-1/3-2)^2+(7/6-0)^2}\\& = \sqrt{(7/6)^2+(-7/3)^2+(7/6)^2}\\& = \sqrt{(7/6)^2(1+4+1)}\\& = \frac{7}{6}(\sqrt{6})= \frac{7}{\sqrt{6}}.
\end{align*}




 \item Find the area of the triangle whose vertices are $(0,1,2)$, $(1,1,1)$, and $(2,1,0)$.

\bigskip

As it turns out, the given three points are colinear (oops!), so the ``triangle'' is in fact a line segment, and therefore has zero area. If I hadn't messed up and given you three points all on the same line, the right approach to this problem would be to label the points as $P,Q,R$ and compute the vectors $\mathbf{v} = \overrightarrow{PQ}$ and $\mathbb{w} = \overrightarrow{PR}$ that make up two of the three sides of the triangle. The area of the triangle is then given by the formula
\[
 A = \len{\mathbf{v}\times\mathbf{w}}.
\]

 \item Determine the domain of the function $f(x,y) = \dfrac{x+y}{x^2+y^2-1}$ and find the value $f(1,2)$.

\bigskip

The function is given by a rational expression in $x$ and $y$, so it's defined as long as the denominator is nonzero, and this is the case as long as $x^2+y^2\neq 1$. The domain is therefore the set $D = \{(x,y)\in\R^2 | x^2+y^2\neq 1\}$, and we have
\[
 f(1,2) = \frac{1+2}{1^2+2^2-1} = \frac{3}{4}.
\]

 \item For a given function $f(x,y)$ of two variables and a value $c$ in the range of $f$, what is the difference between the {\em level curve} $f(x,y)=c$ and the {\em section}\footnote{Sections are also known as {\em traces}} of the graph $z=f(x,y)$ corresponding to $z=c$? How are the two related?

\bigskip

The level curves $f(x,y)=c$ are defined to be subsets of the {\em plane} $\R^2$ given by $\{(x,y) | f(x,y)=c\}$. On the other hand the section of a graph $z=f(x,y)$ in the plane $z=c$ is a subset of $\R^3$: it is the set of points $\{(x,y,c) | f(x,y)=c\}$. Since the two sets are subsets of different spaces, they are not the same. However, we see that there is a bijection between the two sets given by $f(x,y) = (x,y,c)$ for each point $(x,y)$ such that $f(x,y)=c$. If we view the $xy$-plane ($z=0$) as the standard copy of $\R^2$ sitting inside of $\R^3$, then the section of the graph is given by lifting the level curve from the plane $z=0$ to the plane $z=c$.

\bigskip

 \item The subset of $\R^2$ defined by the equation $x^2+y^2=1$ is the unit circle. What does this equation define as a subset of $\R^3$?

\bigskip

As a subset of $\R^3$, we have the set of points $(x,y,z)$ such that $x^2+y^2=1$. For any fixed value $z=c$, we get a copy of the unit circle sitting inside the plane $z=c$. Taking all of the circles together, we get an infinite cylinder parallel to the $z$-axis that intersects the $xy$-plane in the unit circle.
\end{enumerate}

\end{document}
 
