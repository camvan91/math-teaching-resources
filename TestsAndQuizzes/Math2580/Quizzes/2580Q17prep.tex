\documentclass[letterpaper,12pt]{article}

%\usepackage{ucs}
%\usepackage[utf8x]{inputenc}
\usepackage{amsmath}
\usepackage{amsfonts}
\usepackage{amssymb}
\usepackage{graphicx}
%\usepackage[canadian]{babel}
\usepackage[margin=1in]{geometry}
\usepackage{multicol}
\newcommand{\R}{\mathbb{R}}
\renewcommand{\i}{\mathbf{i}}
\renewcommand{\j}{\mathbf{j}}
\renewcommand{\k}{\mathbf{k}}
\newcommand{\pd}[2]{\dfrac{\partial #1}{\partial #2}}
\newcommand{\di}{\displaystyle}
\renewcommand{\r}{\mathbf{r}}
\newcommand{\len}[1]{\left\lVert #1\right\rVert}
\newcommand{\dotp}{\boldsymbol{\cdot}}
\newcommand{\F}{\mathbf{F}}

\title{Practice for Quiz 17\\Math 2580\\Spring 2016}
\author{Sean Fitzpatrick}
\date{March 22nd, 2016}

\begin{document}
 \maketitle

Problems from Quiz 17 will be a review of material from Math 2570, and some basic problems from 18.1.

If you can answer the following problems, you should be well-prepared for Quiz 17:



\begin{enumerate}
 \item Calculate the derivative of the following vector-valued functions:
\begin{enumerate}
 \item $\r(t) = \langle t^2, t^3, t^4\rangle$
 \item $\r(t) = \langle \sin(t), e^{3t}, \cos(2t)\rangle$.
 \item $\r(t) = \langle \sin(t^2), \ln(t^2+1)\rangle$.
\end{enumerate}
 \item Calculate $\len{\r'(t)}$ for the vector-valued functions in problem 1. (Note that if $\r(t)$ is interpreted as position with respect to time, then $\r'(t)$ is velocity, and $\len{\r'(t)}$ is speed.
 \item Show that $\dfrac{d}{dt}\len{\r(t)}^2 = 2\r(t)\dotp r'(t)$. 
 \item Determine a vector-valued function $\r(t)$ and an interval $[a,b]$ that parameterize the line segment from $(1,2,0)$ to $(4,-3,2)$.
 \item Evaluate $\di \int_a^b \F(\r(t))\dotp r'(t)\,dt$ for the vector field $\F$ and curve $\r$ given by
\begin{enumerate}
 \item $\F(x,y) = x^2\i -xy\j$, and $\r(t) = \sin(t)\i+\cos(t)\j$, $a=0$, $b=\pi$.
 \item $\F(x,y,z) = \langle xy^2, xyz, yz^2\rangle$, $\r(t) = \langle t, t^2, 4t\rangle$, $a=0$, $b=1$.
\end{enumerate}


 \end{enumerate}

\end{document}
 
