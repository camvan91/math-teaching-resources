\documentclass[letterpaper,12pt]{article}

%\usepackage{ucs}
%\usepackage[utf8x]{inputenc}
\usepackage{amsmath}
\usepackage{amsfonts}
\usepackage{amssymb}
%\usepackage[canadian]{babel}
\usepackage[margin=1in]{geometry}
\newcommand{\R}{\mathbb{R}}
\renewcommand{\i}{\mathbf{i}}
\renewcommand{\j}{\mathbf{j}}
\renewcommand{\k}{\mathbf{k}}
\renewcommand{\r}{\mathbf{r}}
\newcommand{\bbm}{\begin{bmatrix}}
\newcommand{\ebm}{\end{bmatrix}}

\title{Solutions to Quiz 4 Practice Problems\\Math 2580\\Spring 2016}
\author{Sean Fitzpatrick}
\date{January 21st, 2016}

\begin{document}
 \maketitle



\begin{enumerate}
 \item Find the equation of the tangent plane to the graph of $f$ at the point $(2,-1,f(2,-1))$, if $f(x,y) = x^2+4y^2$.

\bigskip

We have $f_x(x,y) = 2x$, so $f_x(2,-1) = 4$, and $f_y(x,y) = 8y$, so $f_y(2,-1) = -8$. We note that $f(2,-1) = 4+4=8$, so the equation of the tangent plane is
\begin{align*}
 z &= f(2,-1)+f_x(2,-1)(x-2)+f_y(2,-1)(y+1)\\
&= 8+4(x-2)-8(y+1) = 4x-8y-8.
\end{align*}

 \item Let $f(x,y)=x^2y+xy^3$. Find a normal vector to the graph $z=f(x,y)$ at the point $(1,1,2)$.

\bigskip

A normal vector to $z=f(x,y)$ at $(a,b)$ is given by $\mathbf{n} = \bbm f_x(a,b)&f_y(a,b)&-1\ebm^T$. We have $f_x(x,y) = 2xy+y^3$ and $f_y(x,y) = x^2+3xy^2$, so $f_x(1,1) = 3$ and $f_y(1,1) = 4$. One possible normal vector is therefore $\mathbf{n}=\bbm 3\\4\\-1\ebm$. (Any nonzero scalar multiple of this vector would also do.)

\bigskip

 \item Use a linear approximation to the function $f(x,y) = x^3+y^3-6xy$ to give an approximate value for
\[
 (0.99)^3+(2.01)^3-6(0.99)(2.01).
\]

\bigskip

The point $(0.99,2.01)$ is close to the point $(1,2)$, and we know that $f(x,y)\approx L_{(1,2)}(x,y)$, where $f(x,y) = x^3+y^3-6xy$ and $L_{(1,2)}$ is the linear approximation of $f$ near the point $(1,2)$, given by
\[
 L_{(1,2)} = f(1,2)+f_x(1,2)(x-1)+f_y(1,2)(y-2).
\]
We have $f(1,2) = 1^3+2^3-6(1)(2) = -3$, $f_x(x,y) = 3x^2-6y$, so $f_x(1,2) = -9$, and $f_y(x,y) = 3y^2-6x$, so $f_y(1,2) = 6$. Thus,
\[
 f(0.99,1.01) \approx -3-9(0.99-1)+6(2.01-2) = -3+0.09+0.06 = -2.85.
\]
(The actual value is $-2.8485$.)

 \item Verify\footnote{That is, calculate $\dfrac{d}{dt}(f(\mathbf{r}(t))$ first by using the chain rule, and then by explicitly substituting in the parameterization and differentiating with respect to $t$, and verify that the two answers are the same} the chain rule for the function $f(x,y,z) = x+y^2+z^3$ and curve $\mathbf{r}(t)=(\cos t,\sin t, t)$.

\bigskip

According to the chain rule, 
\begin{align*}
 \frac{d}{dt}f(\mathbf{r}(t)) &= \frac{\partial f}{\partial x}\frac{dx}{dt}+\frac{\partial f}{\partial y}\frac{dy}{dt}+\frac{\partial f}{\partial z}\frac{dz}{dt}\\
& = (1)(-\sin t)+2y(\cos(t))+3z^2(1)\\
& = -\sin t + 2\sin(t)\cos(t) + 3t^2,
\end{align*}
where in the last line we've substituted in the values of $x,y,z$ in terms of $t$. On the other hand, if we substitute first, we have
\[
 f(\mathbf{r}(t)) = \cos(t)+(\sin(t))^2+(t)^3,
\]
so
\[
 \frac{d}{dt}(f(\mathbf{r}(t)) = \frac{d}{dt}(\cos (t) + \sin^2(t) + t^3) = -\sin (t) +2\sin(t)\cos(t) + 3t^2,
\]
which agrees with our answer from above.

 \item Express your chain rule formula from the previous problem as a product of two derivative matrices. (One will be a row vector, and one will be a column vector.)

\bigskip

The derivative matrix of $f$ at a point $(x,y,z)$ is given by
\[
 D_{(x,y,z)}f = \bbm f_x(x,y,z) & f_y(x,y,z) & f_z(x,y,z)\ebm = \bbm 1 & 2y & 3z^2\ebm,
\]
so if $(x,y,z) = \mathbf{r}(t) = (\cos t, \sin t, t)$, we have
\[
 D_{\mathbf{r}(t)}f = \bbm 1 & 2\sin t & 3t^2\ebm
\]
The derivative matrix of $\r(t)$ at a point $t$ is given by
\[
 D_t\r = \bbm x'(t)\\y'(t)\\z'(t)\ebm = \bbm -\sin t\\ \cos t\\1\ebm.
\]
Thus, according to the general chain rule,
\[
 D_t(f\circ \r) = D_{\r(t)}f\, D_t\r = \bbm 1 & 2\sin t & 3t^2\ebm \bbm -\sin t\\ \cos t\\ 1\ebm = -\sin t + 2\sin t\cos t +3t^2,
\]
as before.

\bigskip

 \item Find the derivative matrix for the function $f:\R^2\to \R^2$ defined by $f(u,v) = (u\sin v, e^{uv})$, and evaluate it at the point $(0,1)$.

(That is, compute $\dfrac{\partial(x,y)}{\partial (u,v)}$ if $x=u\sin v$ and $y=e^{uv}$.)

 \bigskip

The derivative matrix for $f$ is given by $D_{(u,v)} = \bbm x_u&x_v\\y_u&y_v\ebm$, where 
\begin{align*}
 x_u & = \frac{\partial x}{\partial u} = \frac{\partial }{\partial u}(u\sin v) = \sin v\\
 x_v & = \frac{\partial x}{\partial v} = \frac{\partial }{\partial v}(u\sin v) = u\cos v\\
 y_u & = \frac{\partial y}{\partial u} = \frac{\partial }{\partial u}(e^{uv}) = ve^{uv}\\
 y_v & = \frac{\partial y}{\partial v} = \frac{\partial }{\partial v}(e^{uv}) = ue^{uv}\\
\end{align*}
Evaluating everything at the point $(0,1)$ gives us
\[
 D_{(0,1)}f = \bbm \sin(1) & 0\\1&0\ebm.
\]

\end{enumerate}

\end{document}
 
