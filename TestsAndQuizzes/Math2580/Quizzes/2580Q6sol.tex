\documentclass[letterpaper,12pt]{article}

%\usepackage{ucs}
%\usepackage[utf8x]{inputenc}
\usepackage{amsmath}
\usepackage{amsfonts}
\usepackage{amssymb}
%\usepackage[canadian]{babel}
\usepackage[margin=1in]{geometry}
\newcommand{\R}{\mathbb{R}}
\renewcommand{\i}{\mathbf{i}}
\renewcommand{\j}{\mathbf{j}}
\renewcommand{\k}{\mathbf{k}}
\newcommand{\pd}[2]{\dfrac{\partial #1}{\partial #2}}
\newcommand{\dotp}{\boldsymbol{\cdot}}

\title{Solutions to Quiz 6 Practice Problems\\Math 2580\\Spring 2016}
\author{Sean Fitzpatrick}
\date{January 28th, 2016}

\begin{document}
 \maketitle


\begin{enumerate}
 \item Show that $\nabla(1/r^2) = -2\mathbf{r}/r^4$ for $r\neq 0$, where $\mathbf{r}=\langle x,y,z\rangle$ is the position vector for the point $(x,y,z)$, and $r=\lVert \mathbf{r}\rVert = \sqrt{x^2+y^2+z^2}$.

\bigskip

Our function is $f(x,y,z) = r^{-2} = (x^2+y^2+z^2)^{-1}$. The gradient vector is thus
\begin{align*}
 \nabla (1/r^2) &= \left\langle \pd{f}{x}, \pd{f}{y}, \pd{f}{z}\right\rangle\\
& = \left\langle -2x(x^2+y^2+z^2)^{-2}, -2y(x^2+y^2+z^2)^{-2}, -2z(x^2+y^2+z^2)^{-2}\right\rangle\\
& = \frac{-2}{(x^2+y^2+z^2)^2}\langle x,y,z\rangle\\
& = \frac{-2}{r^4}\mathbf{r}.
\end{align*}

 \item Verify the chain rule for the function $f(x,y,z) = e^{xyz}$ and curve $\mathbf{r}(t)=(6t,3t^2,t^3)$.

\bigskip

Using the chain rule, we have
\begin{align*}
 \frac{d}{dt}f(\mathbf{r}(t)) & = \nabla f(\mathbf{r}(t))\dotp \mathbf{r}'(t)\\
& = \left.\langle yze^{xyz}, xze^{xyz}, xye^{xyz}\rangle\right|_{(x,y,z)=\mathbf{r}(t)}\dotp \langle 6, 6t, 3t^2\rangle\\
& = (3t^2)(t^3)e^{6t(3t^2)(t^3)}(6)+6t(t^3)e^{6t(3t^2)(t^3)}(6t)+6t(3t^2)e^{6t(3t^2)(t^3)}(3t^2)\\
& = e^{18t^6}(18t^5+36t^5+54t^5) = 108t^5e^{18t^6}.
\end{align*}

If instead we first evaluate $f(x,y,z)$ at $\mathbf{r}(t)$, we get
\[
 f(\mathbf{r}(t))=f(6t,3t^3,t^3) = e^{6t(3t^2)(t^3)} = e^{18t^6},
\]
so $\dfrac{d}{dt}f(\mathbf{r}(t)) = \dfrac{d}{dt}(e^{18t^6}) = e^{18t^6}(6(18t^5)) = 108t^5e^{18t^6}$, as before.

 \bigskip

 
 \item Calculate the derivative of the function $f(x,y)=e^{x^2\cos y}$ at the point $(1,\pi/2)$ in the direction of the vector $\mathbf{v} = \frac{1}{5}\langle 3,4\rangle$.
 

\bigskip

The gradient of $f$ is given by
\[
 \nabla f(x,y) = \left\langle \pd{}{x}e^{x^2\cos y}, \pd{}{y}e^{x^2\cos y}\right\rangle = \langle 2x\cos ye^{x^2\cos y}, -x^2\sin y e^{x^2\cos y}\rangle,
\]
so 
\[
 \nabla f(1,\pi/2) = \langle 2\cos(\pi/2)e^{\cos (\pi/2)}, -1^2\sin(\pi/2)e^{\cos(\pi/2)}\rangle = \langle 0,-1\rangle.
\]
Thus,
\[
 d_{\mathbf{v}}f(1,\pi/2) = \nabla f(1,\pi/2)\dotp \mathbf{v} = \langle 0, -1\rangle \dotp \langle 3/5, 4/5\rangle = -\frac{4}{5}.
\]

 \item Determine the direction in which the function $f(x,y)=e^x\sin y$ is increasing fastest at the point $(1,1)$.

\bigskip

Since a function always increases fastest in the direction of its gradient vector, the desired direction is $\nabla f(1,1)$. We compute
\[
 \nabla f(x,y) = \langle e^x\sin y, e^x\cos y\rangle, \text{ so } \nabla f(1,1) = \langle e\sin 1, e\cos 1\rangle.
\]

 \item Find a unit normal vector to the surface $xyz=8$ at the point $(2,2,2)$.

\bigskip

Letting $f(x,y,z)=xyz$, a normal vector to $xyz=8$ is given by $\nabla f(2,2,2)$. We have
\[
 \nabla f(x,y,z) = \langle yz, xz, xy\rangle, \text{ so } \nabla f(2,2,2) = \langle 4,4,4\rangle.
\]
A unit vector in this direction is then $\mathbf{n} = \dfrac{1}{\sqrt{3}}\langle 1,1,1\rangle$.

\bigskip

 \item Find the equation of the tangent plane to the ellipsoid $x^2+2y^2+3z^2=9$ at the point $(2,1,1)$.

\bigskip

Letting $f(x,y,z) = x^2+2y^2+3z^2$, we have $\nabla f(x,y,z) = \langle 2x, 4y, 6z\rangle$, so a normal vector to the tangent plane at $(2,1,1)$ is $\nabla f(2,1,1) = \langle 4,4,6\rangle$. The equation of the tangent plane is therefore
\[
 4(x-2)+4(y-1)+6(z-1)=0.
\]

\end{enumerate}

\end{document}
 
