\documentclass[letterpaper,12pt]{article}

%\usepackage{ucs}
%\usepackage[utf8x]{inputenc}
\usepackage{amsmath}
\usepackage{amsfonts}
\usepackage{amssymb}
\usepackage{graphicx}
%\usepackage[canadian]{babel}
\usepackage[margin=1in]{geometry}
\usepackage{multicol}
\newcommand{\R}{\mathbb{R}}
\renewcommand{\i}{\mathbf{i}}
\renewcommand{\j}{\mathbf{j}}
\renewcommand{\k}{\mathbf{k}}
\newcommand{\pd}[2]{\dfrac{\partial #1}{\partial #2}}
\newcommand{\di}{\displaystyle}
\renewcommand{\r}{\mathbf{r}}
\newcommand{\len}[1]{\left\lVert #1\right\rVert}
\newcommand{\dotp}{\boldsymbol{\cdot}}
\newcommand{\F}{\mathbf{F}}

\title{Solutions to Quiz 19 Practice Problems\\Math 2580\\Spring 2016}
\author{Sean Fitzpatrick}
\date{March 29th, 2016}

\begin{document}
 \maketitle

\begin{enumerate}
 \item Find a function $f$ such that $\nabla f = \F$, and use this to evaluate $\int_C \F\dotp d\r$ for the given curve:
\begin{enumerate}
 \item $\F(x,y) = \dfrac{y^2}{1+x^2}\i + 2y\arctan x\j$, $C$ parameterized by $\r(t) = t^2\i+2t\j$, $0\leq t\leq 1$.

\bigskip

Let $\F=P\i+Q\j$ and suppose $\F = \nabla f$ for some function $f$. If $f_x(x,y) = \dfrac{y^2}{1+x^2} = P(x,y)$, then we must have $f(x,y) = y^2\arctan x + g(y)$ for some function $g(y)$ of $y$ only. Then on the one hand we have $f_y(x,y) = 2y\arctan x +g'(y)$, and on the other hand, $f_y(x,y) = 2y\arctan x = Q(x,y)$, so we can take $g(y)=0$, and $f(x,y)=y^2\arctan x$.

The endpoints of the curve are given by $\r(0) = \langle 0,0\rangle$ and $\r(1) = \langle 1, 2\rangle$, so by the Fundamental Theorem of Calculus we have
\[
 \int_C \F\dotp d\r = f(\r(1))-f(\r(0)) = f(1,2)-f(0,0) = 4\arctan 1 - 0 = \pi.
\]



 \item $\F(x,y,z) = (2xz+y^2)\i+2xy\j + (x^2+3z^2)\k$, $C$ parameterized by $x=t^2, y=t+1, z=2t-1$, $0\leq t\leq 2$.

\bigskip

Let $\F = P\i+Q\j+R\k$ and suppose $\F = \nabla f$ for some function $f$. Then we must have $f_x(x,y,z) = P(x,y,z) = 2xz+y^2$, which gives $f(x,y,z) = x^2z+xy^2+g(y,z)$ for some function $g$ that does not depend on $x$. Now we must have
\[
 Q(x,y,z) = 2xy = f_y(x,y,z) = 2xy+g_y(y,z),
\]
which tells us that $g_y(y,z)=0$, so $g(y,z)=g(z)$ depends only on $z$. Comparing $z$-components, we have
\[
 R(x,y,z) = x^2+3z^2 = f_z(x,y,z) = x^2+g'(z),
\]
from which we find $g'(z) = 3z^2$, so $g(z)=z^3$ and $f(x,y,z) = x^2z+xy^2+z^3$. The endpoints of our curve are given by $(0^2, 0+1, 2(0)-1) = (0,1,-1)$ and $(2^2, 2+1, 2(2)-1) = (4,3,3)$ so we have
\[
 \int_C \F\dotp d\r = f(4,3,3)-f(0,1,-1) = 16(3)+4(9)+27-(-1)^3 = 110.
\]

 \item $\F(x,y,z) = \langle e^y, xe^y, (z+1)e^z\rangle$, $C$ parameterized by $\r(t) = \langle t, t^2, t^3\rangle$, $t\in [0,1]$.

\bigskip

Let $\F = P\i+Q\j+R\k$ and suppose $\F = \nabla f$ for some function $f$. Then we must have $f_x(x,y,z) = P(x,y,z) = e^y$, so $f(x,y,z) = xe^y+g(y,z)$ for some function $g$ of $y$ and $z$. Comparing $y$ components, we have
\[
 Q(x,y,z) = xe^y = f_y(x,y,z) = xe^y+g_y(y,z),
\]
so $g_y(y,z) = 0$, and $g(y,z) = g(z)$ depends on $z$ only. Comparing $z$ components, 
\[
 R(x,y,z) = (z+1)e^z = g'(z),
\]
and since $(z+1)e^z = \dfrac{d}{dz}(ze^z)$, we have $g(z)=ze^z$ and $f(x,y,z) = xe^y+ze^z$. Thus,
\[
 \int_C\F\dotp d\r = f(1,1,1)-f(0,0,0) = 2e.
\]

\end{enumerate}
 \item Calculate the curl of the given vector field:
\begin{enumerate}
 \item $\F(x,y,z) = \langle 2xy, xz, y^2z\rangle$

\bigskip

\[
 \nabla \F(x,y,z) = \begin{vmatrix} \i&\j&\k\\ \pd{}{x} & \pd{}{y} & \pd{}{z}\\ 2xy & xz & y^2z\end{vmatrix} = (2yz-x)\i+(z-2x)\k.
\]

 \item $\F(x,y,z) = \langle y\cos xy, x\cos xy, -\sin z\rangle$

\bigskip

\[
 \nabla \F(x,y,z) = \begin{vmatrix} \i&\j&\k\\ \pd{}{x} & \pd{}{y} & \pd{}{z}\\y\cos xy & x\cos xy & -\sin z\end{vmatrix} = \langle 0,0,0\rangle.
\]

\end{enumerate}
 \item Verify that Green's Theorem holds for the following line integrals in the plane:
\begin{enumerate}
 \item $\int_C xy^2\,dx+x^3\,dy$, where $C$ is the rectangle with corners at $(0,0), (2,0), (2,3)$, and $(0,3)$.

\bigskip

Let $C=C_1+C_2+C_3+C_4$, as follows:

$C_1$ is the bottom of the rectangle, with $y=0$ and $x=2t$, $t\in [0,1]$, so along $C_1$, $y=0$ and $dy=0$, which means $xy^2\,dx+x^3\,dy = 0$.

$C_2$ is the right side of the rectangle, with $x=2$ and $y=3t$, $t\in [0,1]$. Along $C_1$, $dx=0$ since $x$ is constant, so
\[
 xy^2\,dx + x^3\,dy = 0 + (2^3)(3\,dt) = 24\,dt.
\]

$C_3$ is the top of the rectangle, with $y=3$, $dy=0$, and $x=2-2t$, $t\in [0,1]$, so $dx= -2\,dt$ and
\[
 xy^2\,dx+x^3\,dy = (2-2t)(3)^2(-2\,dt) = (36t-36)\,dt.
\]

$C_4$ is the left side of the rectangle, where $x=0$ and $dx=0$, so $xy^2\,dx+x^3\,dy = 0$. We thus have
\[
 \int_C xy^2\,dx+x^3\,dy = 0+\int_0^1 24\,dt + \int_0^1 (36t-36)\,dt + 0 = 6.
\]
On the other hand, we have
\[
 \pd{Q}{x}-\pd{P}{y} = \pd{x^3}{x}-\pd{xy^2}{y} = 3x^2-2xy,
\]
and
\[
 \int_0^2\int_0^3 (3x^2-2xy)\,dy\,dx = \int_0^2 (9x^2-9x)\,dx 3(2^3)-\frac{9}{2}(2^2) = 24-18=6.
\]

 \item $\int_C y\,dx-x\,dy$, where $C$ is the unit circle.

\bigskip

To compute the integral directly, we parameterize $C$ using $x=\cos t$, $y=\sin t$, $t\in [0,2\pi]$. Then
\[
 \int_C y\,dx-x\,dy = \int_0^{2\pi}\sin t(-\sin t\,dt)-(\cos t)(\cos t\,dt) = \int_0^{2\pi}(- 1)\,dt = -2\pi.
\]
Using Green's Theorem, we have $Q_x(x,y)-P_y(x,y) = -1-1=-2$, so
\[
 \iint_D\left(\pd{Q}{x}-\pd{P}{y}\right)\,dA = \iint_D(-2)\,dA = -2A(D) = -2\pi.
\]

 \item $\int_C x\,dx+y\,dy$, where $C$ consists of the line segments from $(0,1)$ to $(0,0)$, and from $(0,0)$ to $(1,0)$, and the portion of the parabola $y=1-x^2$ from $(1,0)$ to $(0,1)$.

\bigskip

First, we compute the line integral directly: we have $C=C_1+C_2+C_3$, where $C_1$ is the portion of the curve along the $y$-axis, $C_2$ is along the $x$-axis, and $C_3$ is the parabolic arc. Along $C_1$ we have $x=0$ and $dx=0$, while $y=1-t$, $dy = -dt$, with $t\in [0,1]$, so
\[
 \int_C{_1}x\,dx+y\,dy = \int_0^1 (0+(1-t)(-dt)) = \int_0^1 (t-1)\,dt = -\frac{1}{2}.
\]
Along $C_2$ we have $x=t$, $dx=dt$, with $t\in [0,1]$, and $y=0$, $dy=0$, so
\[
 \int_{C_2}x\,dx+y\,dy = \int_0^1 t\,dt = \frac{1}{2}.
\]
Finally, along $C_3$, we take $x=1-t$, $y=1-(1-t)^2 = 2t-t^2$, so $dx = -dt$ and $dy = (2-2t)\,dt$. This gives us
\[
 \int_{C_3} x\,dx+y\,dy = \int_0^1 (1-t)(-dt)+(2t-t^2)(2-2t)\,dt = \int_0^1 (2t^3-6t^2+5t-1)\,dt = \frac{1}{2}-2+\frac{5}{2}-1 = 0.
\]
Thus, $\int_C x\,dx+y\,dy = -\frac{1}{2}+\frac{1}{2}+0 = 0.$

(Note, for $C_3$, you can take $x=t$ and $y=1-t^2$, with $t\in [0,1]$, as long as you're aware that this parameterization sends you along the parabola in the wrong direction, requiring you to change the sign of your answer at the end.)

Now, to compute the integral using Green's Theorem, we simply note that $\pd{Q}{x}-\pd{P}{x} = 0-0 = 0$, so the double integral is automatically zero, in agreement with the line integral above.
\end{enumerate}

\end{enumerate}




\end{document}
 
