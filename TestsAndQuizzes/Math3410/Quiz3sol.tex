\documentclass[12pt]{article}
\usepackage{amsmath}
\usepackage{amssymb}
\usepackage[letterpaper,margin=0.85in,centering]{geometry}
\usepackage{fancyhdr}
\usepackage{enumerate}
\usepackage{lastpage}
\usepackage{multicol}
\usepackage{graphicx}

\reversemarginpar

\pagestyle{fancy}
\cfoot{}
\lhead{Math 3410A}\chead{Quiz \# 3}\rhead{Friday, 30\textsuperscript{th} January, 2015}
\rfoot{Total: 10 points}
%\chead{{\bf Name:}}
\newcommand{\points}[1]{\marginpar{\hspace{24pt}[#1]}}
\newcommand{\skipline}{\vspace{12pt}}
%\renewcommand{\headrulewidth}{0in}
\headheight 30pt

\newcommand{\di}{\displaystyle}
\newcommand{\R}{\mathbb{R}}
\newcommand{\C}{\mathbb{C}}
\newcommand{\vv}{\mathbf{v}}
\newcommand{\aaa}{\mathbf{a}}
\newcommand{\bbb}{\mathbf{b}}
\newcommand{\ccc}{\mathbf{c}}
\newcommand{\dotp}{\boldsymbol{\cdot}}
\DeclareMathOperator{\spn}{span}

\begin{document}

%\author{Instructor: Sean Fitzpatrick}
\thispagestyle{fancy}
%\noindent{{\bf Name and student number:}}
{\bf Name: Solutions}

\bigskip

Solve {\bf one} of the following two questions:
 \begin{enumerate}
 \item  Let $U = \{p\in \mathcal{P}_4(\R) : p(6)=0\}$.
\begin{enumerate}
 \item Find a basis for $U$. \points{2} (Hint: note that $p(x)=x-6$ is an element of $U$.)

\bigskip

Consider the polynomials $p_1(x)=x-6,\, p_2(x) = x^2-6x,\, p_3(x) = x^3-6x^2,\, p_4(x)=x^4-6x^3$. Since $x-6$ is a factor of each polynomial, we see that they are all elements of $U$. Moreover, if
\begin{align*}
 0 & = c_1p_1(x)+c_2p_2(x)+c_3p_3(x)+c_4p_4(x)\\
& = c_1(x-6)+c_2(x^2-6x)+c_3(x^3-6x^2)+c_4(x^4-6x^3)\\
& = -6c_1+(c_1-6c_2)x+(c_2-6c_3)x^2+(c_3-6c_4)x^3+c_4x^4,
\end{align*}
then we must have $c_1=c_2=c_3=c_4=0$, so the set $B=\{p_1,p_2,p_3,p_4\}$ is linearly independent. Since $U$ is a proper subspace of $\mathcal{P}_4(\R)$ (since, for example, the polynomial $q(x)=1$ does not belong to $U$), we must have $\dim U\leq 4$, and since the set $B$ contains four linearly independent vectors, it must be a basis for $U$.

\bigskip


 \item Extend the basis in part (a) to a basis of $\mathcal{P}_4(\R)$.\points{4}

\bigskip

As noted above, the polynomial $q(x)=1$ does not belong to $U$, since $q(6)=1\neq 0$. It follows that $q$ is not in the span of $B$, and therefore the set $B' = \{p_1,p_2,p_3,p_4,q\}$ is linearly independent. Since $\dim \mathcal{P}_4(\R)=5$, $B'$ must be a basis.

\bigskip

 \item Find a subspace $W$ of $\mathcal{P}_4(\R)$ such that $U\oplus W = \mathcal{P}_4(\R)$. \points{4}

\bigskip

Let $W = \spn\{q\}$, where $q(x)=1$ as above. We must have $\mathcal{P}_4(\R) = U+W$, since any polynomial $p(x)\in \mathcal{P}_4(\R)$ can be written as
\[
 p(x) = c_1p_1(x)+c_2p_2(x)+c_3p_3(x)+c_4p_4(x)+c_5q(x) = u(x)+w(x),
\]
where $u(x)=c_1p_1(x)+c_2p_2(x)+c_3p_3(x)+c_4p_4(x)\in U$ and $w(x)=c_5q(x)\in W$, since $B'$ is a basis. Since $W$ is the subspace of constant polynomials, and the only constant polynomial that is equal to zero when $x=6$ is the zero polynomial, we must have $U\cap W=\{0\}$, and the result follows.
\end{enumerate}

\newpage

\item Suppose $U$ and $W$ are subspaces of $V$ such that $V=U\oplus W$. Show that if $\{u_1,\ldots, u_m\}$ is a basis for $U$, and $\{w_1,\ldots, w_k\}$ is a basis for $W$, then \points{10}
\[
 \{u_1,\ldots, u_m,w_1,\ldots, w_k\}
\]
is a basis for $V$.

\bigskip

Let $v\in V$ be any vector. Since $V=U\oplus W$, we have in particular that $V=U+W$, so there exist vectors $u\in U$ and $w\in W$ such that
\[
 v=u+w.
\]
Since $\{u_1,\ldots, u_m\}$ is a basis for $U$, there exist scalars $a_1,\ldots, a_m$ such that
\[
 u=a_1u_1+\cdots +a_mu_m,
\]
and similarly, there exist scalars $b_1,\ldots, b_k$ such that
\[
 w = b_1w_1+\cdots +b_kw_k.
\]
It follows that $v$ can be written as
\[
 v= a_1u_1+\cdots + a_mu_m + b_1w_1+\cdots + b_kw_k.
\]
Since $v\in V$ was arbitary, we can conclude that the set $\{u_1,\ldots, u_m, w_1,\ldots, w_k\}$ spans $V$. It remains to show that the set is independent. Suppose we have
\[
 a_1u_1+\cdots + a_mu_m + b_1w_1+\cdots + b_kw_k=0
\]
for some scalars $a_1,\ldots, a_m,b_1,\ldots,b_k$. Then $0=u+w$, where $u=a_1u_1+\cdots +a_mu_m\in U$ and $w = b_1w_1+\cdots +b_kw_k\in W$. Since $V=U\oplus W$ is a direct sum, we know that the only way to write $0=u+w$ with $u\in U$ and $w\in W$ is if $u=w=0$.

But if $u=0$, then $a_1=\cdots =a_m=0$, since $\{u_1,\ldots, u_m\}$ is a basis for $U$. Similarly, since $w=0$, we must have $b_1=\cdots=b_k=0$. Thus, the only linear combination equal to zero is the trivial combination, so the set is linearly independent, and therefore a basis for $V$.
 \end{enumerate}
\end{document}