\documentclass[12pt]{article}
\usepackage{amsmath}
\usepackage{amssymb}
\usepackage[letterpaper,margin=0.85in,centering]{geometry}
\usepackage{fancyhdr}
\usepackage{enumerate}
\usepackage{lastpage}
\usepackage{multicol}
\usepackage{graphicx}

\reversemarginpar

\pagestyle{fancy}
\cfoot{}
\lhead{Math 3410A}\chead{Quiz \# 7}\rhead{Friday, 13\textsuperscript{th} March, 2015}
\rfoot{Total: 10 points}
%\chead{{\bf Name:}}
\newcommand{\points}[1]{\marginpar{\hspace{24pt}[#1]}}
\newcommand{\skipline}{\vspace{12pt}}
%\renewcommand{\headrulewidth}{0in}
\headheight 30pt

\newcommand{\di}{\displaystyle}
\newcommand{\R}{\mathbb{R}}
\newcommand{\C}{\mathbb{C}}
\newcommand{\vv}{\mathbf{v}}
\newcommand{\aaa}{\mathbf{a}}
\newcommand{\bbb}{\mathbf{b}}
\newcommand{\ccc}{\mathbf{c}}
\newcommand{\dotp}{\boldsymbol{\cdot}}
\DeclareMathOperator{\range}{range}
\DeclareMathOperator{\nul}{null}

\begin{document}

%\author{Instructor: Sean Fitzpatrick}
\thispagestyle{fancy}
%\noindent{{\bf Name and student number:}}
{\bf Name:}

\bigskip

Solve the following {\bf two} questions. (Question \#2 is on the back of the page.)
 \begin{enumerate}
 \item Suppose $T\in\mathcal{L}(V)$ and $(T-2I)(T-3I)(T-4I)=0$. Suppose $\lambda$ is an eigenvalue of $T$. Prove that $\lambda =2$ or $\lambda =3$ or $\lambda=4$.\points{5}

{\em Hint:} Compute $(T-2I)(T-3I)(T-4I)v$, where $v$ is an eigenvector with eigenvalue $\lambda$.

\newpage

 \item Suppose $T\in\mathcal{L}(V)$ is invertible. Prove that $E(\lambda, T)=E(\frac{1}{\lambda},T^{-1})$ for every $\lambda\in\mathbb{F}$ with $\lambda\neq 0$.

{\em Reminder:} the eigenspace $E(\lambda, T)$ is defined to be $\nul(T-\lambda I)$.
 \end{enumerate}
\end{document}