\documentclass[12pt]{article}
\usepackage{amsmath}
\usepackage{amssymb}
\usepackage[letterpaper,margin=0.85in,centering]{geometry}
\usepackage{fancyhdr}
\usepackage{enumerate}
\usepackage{lastpage}
\usepackage{multicol}
\usepackage{graphicx}

\reversemarginpar

\pagestyle{fancy}
\cfoot{}
\lhead{Math 3410A}\chead{Quiz \# 7 Solutions}\rhead{Friday, 13\textsuperscript{th} March, 2015}
\rfoot{Total: 10 points}
%\chead{{\bf Name:}}
\newcommand{\points}[1]{\marginpar{\hspace{24pt}[#1]}}
\newcommand{\skipline}{\vspace{12pt}}
%\renewcommand{\headrulewidth}{0in}
\headheight 30pt

\newcommand{\di}{\displaystyle}
\newcommand{\R}{\mathbb{R}}
\newcommand{\C}{\mathbb{C}}
\newcommand{\vv}{\mathbf{v}}
\newcommand{\aaa}{\mathbf{a}}
\newcommand{\bbb}{\mathbf{b}}
\newcommand{\ccc}{\mathbf{c}}
\newcommand{\dotp}{\boldsymbol{\cdot}}
\DeclareMathOperator{\range}{range}
\DeclareMathOperator{\nul}{null}

\begin{document}

%\author{Instructor: Sean Fitzpatrick}
\thispagestyle{fancy}
%\noindent{{\bf Name and student number:}}
{\bf Name: Solutions}

\bigskip


 \begin{enumerate}
 \item Suppose $T\in\mathcal{L}(V)$ and $(T-2I)(T-3I)(T-4I)=0$. Suppose $\lambda$ is an eigenvalue of $T$. Prove that $\lambda =2$ or $\lambda =3$ or $\lambda=4$.\points{5}

{\em Hint:} Compute $(T-2I)(T-3I)(T-4I)v$, where $v$ is an eigenvector with eigenvalue $\lambda$.

\bigskip

Suppose that $\lambda$ is an eigenvalue of $T$. Thus, $Tv=\lambda v$ for some $v\neq 0$. Then we have that
\[
 (T-4I)v = Tv-4v = \lambda v-4v = (\lambda-4)v.
\]
Thus, 
\[
 (T-3I)(T-4I)v = (T-3I)[(\lambda-4)v] = (\lambda-4)(T-3I)v = (\lambda -4)(\lambda -3)v,
\]
and repeating this process one more time, we get
\[
 0 = (T-2I)(T-3I)(T-4I)v = (\lambda-4)(\lambda-3)(\lambda-2)v,
\]
and since $v\neq 0$, we must have $(\lambda-4)(\lambda-3)(\lambda-2)=0$. Since this is a product of scalars equal to zero, one of the terms in the product must be zero, and it follows that $\lambda$ must equal one of 2, 3, or 4.

\bigskip

 \item Suppose $T\in\mathcal{L}(V)$ is invertible. Prove that $E(\lambda, T)=E(\frac{1}{\lambda},T^{-1})$ for every $\lambda\in\mathbb{F}$ with $\lambda\neq 0$.\points{5}

{\em Reminder:} the eigenspace $E(\lambda, T)$ is defined to be $\nul(T-\lambda I)$.

\bigskip

Suppose that $v\in E(\lambda, T)$ for some $\lambda\neq 0$. Then $(T-\lambda I)v=0$, so we must have $Tv=\lambda v$. Since $T$ is invertible and $\lambda\neq 0$, we have that
\[
 Tv=\lambda v \Rightarrow v=T^{-1}(\lambda v) = \lambda T^{-1}v \Rightarrow T^{-1}v = \frac{1}{\lambda}v.
\]
Thus, $\left(T^{-1}-\dfrac{1}{\lambda}I\right)v = 0$, which shows that $v\in E(\frac{1}{\lambda}, T^{-1})$, and therefore $E(\lambda, T)\subseteq E(\frac{1}{\lambda},T^{-1})$.

The proof that $E(\frac{1}{\lambda},T^{-1})\subseteq E(\lambda, T)$ is identical, and obtained by reversing the steps above, so the result follows.
 \end{enumerate}
\end{document}