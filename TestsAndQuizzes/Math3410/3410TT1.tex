\documentclass[12pt]{article}
\usepackage{amsmath}
\usepackage{amssymb}
\usepackage[letterpaper,margin=0.85in,centering]{geometry}
\usepackage{fancyhdr}
\usepackage{enumerate}
\usepackage{lastpage}
\usepackage{multicol}
\usepackage{graphicx}

\reversemarginpar

\pagestyle{fancy}
\cfoot{Page \thepage \ of \pageref{LastPage}}\rfoot{{\bf Total Points: 30}}
\chead{MATH 3410}\lhead{Test \# 1}\rhead{Friday, 13\textsuperscript{th} February, 2015}

\newcommand{\points}[1]{\marginpar{\hspace{24pt}[#1]}}
\newcommand{\skipline}{\vspace{12pt}}
%\renewcommand{\headrulewidth}{0in}
\headheight 30pt

\newcommand{\di}{\displaystyle}
\newcommand{\R}{\mathbb{R}}
\newcommand{\aaa}{\mathbf{a}}
\newcommand{\bbb}{\mathbf{b}}
\newcommand{\ccc}{\mathbf{c}}
\newcommand{\dotp}{\boldsymbol{\cdot}}
\newcommand{\abs}[1]{\lvert #1\rvert}
\newcommand{\len}[1]{\lVert #1\rVert}
\newcommand{\ivec}{\,\boldsymbol{\hat{\imath}}}
\newcommand{\jvec}{\,\boldsymbol{\hat{\jmath}}}
\newcommand{\kvec}{\,\boldsymbol{\hat{k}}}
\DeclareMathOperator{\comp}{comp}

\begin{document}

\author{Instructor: Sean Fitzpatrick}
\thispagestyle{plain}
\begin{center}
\emph{University of Lethbridge}\\
Department of Mathematics and Computer Science\\
13\textsuperscript{th} February, 2015, 3:00 - 3:50 pm\\
{\bf MATH 3410 - Test \#1}\\
\end{center}
\skipline \skipline \skipline \noindent \skipline
Last Name:\underline{\hspace{353pt}}\\
\skipline
First Name:\underline{\hspace{350pt}}\\
\skipline
Student Number:\underline{\hspace{323pt}}\\
\skipline



\vspace{0.5in}


\begin{quote}
 {\bf Record your answers below each question in the space provided.    Left-hand pages may be used as scrap paper for rough work.  If you want any work on the left-hand pages to be graded, please indicate so on the right-hand page.
 
 \bigskip
 
Partial credit will be awarded for partially correct work, so be sure to show your work, and include all necessary justifications needed to support your arguments.}
\end{quote}


\vspace{0.5in}

For grader's use only:

\begin{table}[hbt]
\begin{center}
\begin{tabular}{|l|r|} \hline
Page&Grade\\
\hline \hline
\cline{1-2} 2 & \enspace\enspace\enspace\enspace\enspace\enspace/12\\
\cline{1-2} 3 & \enspace\enspace\enspace\enspace\enspace\enspace/8\\
\cline{1-2} 4 & \enspace\enspace\enspace\enspace\enspace\enspace/10\\
\cline{1-2} Total & \enspace\enspace\enspace\enspace\enspace\enspace/30\\
\hline
\end{tabular}




\end{center}
\end{table}
\newpage


\begin{enumerate}
\item True/False: For each of the statements below, state whether it is true or false, and give a {\bf brief} explanation supporting your choice.
 \begin{enumerate}
\item The set $U=\{(x,y,xy)\,|\,x,y\in\R\}$ is a subspace of $\R^3$.\points{3}

\vspace{1.6in}

\item If a vector space $V$ can be written as a direct sum $V=U\oplus W$, and for some $v\in V$ we have $v\notin U$, then $v\in W$.\points{3}

\vspace{1.6in}

\item For any subspace $U\subseteq V$, where $V$ is finite-dimensional, there exists a subspace $W\subseteq V$ such that $V=U\oplus W$.\points{3}

\vspace{1.6in}

\item If $T:V\to W$ is a linear transformation, and we know $\dim V=4$ and $\dim W=3$, then $T$ cannot be one-to-one.\points{3}

\end{enumerate}
\newpage
Please provide a solution to {\bf one} of the two problems on this page: 
\item Suppose that the vectors $v_1,v_2,v_3,v_4$ form a basis for $V$. Prove that the vectors \points{8}
\[
 v_1+v_2,v_2+v_3,v_3+v_4,v_4
\]
also form a basis for $V$.
\item Determine whether or not the vector $v=(1,3,-4)$ belongs to the span of the vectors $(2,0,1), (0,3,-4)$, and  $(4,-3,9)$. \points{8}

\newpage
Please provide a solution to {\bf one} of the two problems on this page:
\item Suppose $T:V\to W$ is injective, and the vectors $v_1,\ldots, v_n$ are linearly independent in $V$. Prove that the vectors $Tv_1,\ldots, Tv_n$ are linearly independent in $W$. \points{10}
\item Let $V=\R^{3,1} = \left\{\begin{bmatrix}x\\y\\z\end{bmatrix} : x,y,z\in\R\right\}$, and let $T:V\to V$ be the linear transformation given by\points{10}
\[
T\left(\begin{bmatrix}x\\y\\z\end{bmatrix}\right) = \begin{bmatrix}2&-1&3\\-1&0&4\\4&-1&-5\end{bmatrix}\begin{bmatrix}
x\\y\\z
\end{bmatrix}=\begin{bmatrix}
2x-y+3z\\-x+4z\\4x-y-5z
\end{bmatrix}.
\]
Determine the null space and range of $T$.
\end{enumerate}
\newpage

Extra space for rough work or to complete a problem, as needed. Please do not remove this page. If there is work to be graded on this page, please indicate this next to the corresponding question.
\end{document}