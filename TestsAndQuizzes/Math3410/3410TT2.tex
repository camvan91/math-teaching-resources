\documentclass[12pt]{article}
\usepackage{amsmath}
\usepackage{amssymb}
\usepackage[letterpaper,margin=0.85in,centering]{geometry}
\usepackage{fancyhdr}
\usepackage{enumerate}
\usepackage{lastpage}
\usepackage{multicol}
\usepackage{graphicx}

\reversemarginpar

\pagestyle{fancy}
\cfoot{Page \thepage \ of \pageref{LastPage}}\rfoot{{\bf Total Points: 40}}
\chead{MATH 3410}\lhead{Test \# 2}\rhead{Friday, 20\textsuperscript{th} March, 2015}

\newcommand{\points}[1]{\marginpar{\hspace{24pt}[#1]}}
\newcommand{\skipline}{\vspace{12pt}}
%\renewcommand{\headrulewidth}{0in}
\headheight 30pt

\newcommand{\di}{\displaystyle}
\newcommand{\R}{\mathbb{R}}
\newcommand{\aaa}{\mathbf{a}}
\newcommand{\bbb}{\mathbf{b}}
\newcommand{\ccc}{\mathbf{c}}
\newcommand{\dotp}{\boldsymbol{\cdot}}
\newcommand{\abs}[1]{\lvert #1\rvert}
\newcommand{\len}[1]{\lVert #1\rVert}
\newcommand{\ivec}{\,\boldsymbol{\hat{\imath}}}
\newcommand{\jvec}{\,\boldsymbol{\hat{\jmath}}}
\newcommand{\kvec}{\,\boldsymbol{\hat{k}}}
\DeclareMathOperator{\comp}{comp}
\DeclareMathOperator{\nul}{null}
\DeclareMathOperator{\range}{range}

\begin{document}

\author{Instructor: Sean Fitzpatrick}
\thispagestyle{plain}
\begin{center}
\emph{University of Lethbridge}\\
Department of Mathematics and Computer Science\\
20\textsuperscript{th} March, 2015, 3:00 - 3:50 pm\\
{\bf MATH 3410 - Test \#2}\\
\end{center}
\skipline \skipline \skipline \noindent \skipline
Last Name:\underline{\hspace{353pt}}\\
\skipline
First Name:\underline{\hspace{350pt}}\\
\skipline
Student Number:\underline{\hspace{323pt}}\\
\skipline



\vspace{0.5in}


\begin{quote}
 {Record your answers below each question in the space provided.    Left-hand pages may be used as scrap paper for rough work.  If you want any work on the left-hand pages to be graded, please indicate so on the right-hand page.
 
 \bigskip
 
Partial credit will be awarded for partially correct work, so be sure to show your work, and include all necessary justifications needed to support your arguments.}

 \bigskip

You must solve all problems on pages 2, 3, and 4, but you only need to do either page 5 or page 6. {\bf Do not complete both page 5 and page 6.}
\end{quote}


\vspace{0.5in}

For grader's use only:

\begin{table}[hbt]
\begin{center}
\begin{tabular}{|l|r|} \hline
Page&Grade\\
\hline \hline
\cline{1-2} 2 & \enspace\enspace\enspace\enspace\enspace\enspace/8\\
\cline{1-2} 3 & \enspace\enspace\enspace\enspace\enspace\enspace/8\\
\cline{1-2} 4 & \enspace\enspace\enspace\enspace\enspace\enspace/12\\
\cline{1-2} 5/6 & \enspace\enspace\enspace\enspace\enspace\enspace/12\\
\cline{1-2} Total & \enspace\enspace\enspace\enspace\enspace\enspace/40\\
\hline
\end{tabular}




\end{center}
\end{table}
\newpage


\begin{enumerate}
\item Provide definitions for the following terms:
\begin{enumerate}
 \item What it means for a linear map $T:V\to W$ to be {\bf invertible}. \points{2}

\vspace{2in}
 
 \item An {\bf invariant subspace} for an operator $T:V\to V$. \points{2}

\vspace{2in}

 \item What it means for a linear operator $T:V\to V$ to be {\bf diagonalizable}.\points{2}

\vspace{2in}

 \item The {\bf eigenspace} $E(\lambda, T)$ of an operator $T:V\to V$ and scalar $\lambda$.\points{2}
\end{enumerate}
\newpage

\item Short answer: provide a brief answer to the questions below. You do not have to explain your answers.
\begin{enumerate}
 \item If $V$ and $W$ are finite-dimensional vector spaces, what is $\dim\mathcal{L}(V,W)$?\points{1}

\vspace{1in}

 \item What is the matrix (with respect to the standard bases) of the linear map $T:\R^3\to\R^2$ given by \points{3}
\[
 T(x,y,z) = (2x-3y+z,-x+2y+4z)?
\]

\vspace{2in}

\item If $T$ is the operator on $\R^{2,1}$ given by
\[
 T\left(\begin{bmatrix}x\\y\end{bmatrix}\right) = \begin{bmatrix}-3&-2\\2&5\end{bmatrix}\begin{bmatrix}x\\y\end{bmatrix}
\]
 and $p(x) = 2x^2-3x+5$, determine the operator $p(T)$. \points{4}

\end{enumerate}
\newpage

Please solve {\bf both} problems on this page.

\bigskip

\item Let $S,T\in\mathcal{L}(V)$, where $V$ is finite-dimensional. Prove that the operator $ST$ is invertible if and only if $S$ and $T$ are invertible. \points{6}

\vspace{4in}

\item Suppose that $S,T\in \mathcal{L}(V)$ satisfy $ST=TS$. Prove that $\operatorname{null}S$ is invariant under $T$. \points{6}

\newpage

You may either solve both problems on this page, or leave it blank, and move on to the next page.

\item Let $V$ be finite-dimensional, and let $P\in\mathcal{L}(V)$. Prove that if $P^2=P$, then\\ $V=\nul P\oplus\range P$.\points{6}

{\em Hint:} $\dim V = \dim \nul P+\dim \range P$, so it suffices to show that $\nul P\cap \range P=\{0\}$.

\vspace{3.5in}

\item Suppose that $\dim V = n$, $T\in\mathcal{L}(V)$ has $n$ distinct eigenvalues, and $S\in\mathcal{L}(V)$ has the same eigenvectors as $T$ (but not necessarily the same eigenvalues). Prove that $ST=TS$. \points{6}





\newpage

If you solved the two problems on the previous page, then leave this page blank. If you skipped the last page, then please solve the following:

\item Let $T:\R^2\to\R^2$ be the operator $T(x,y) = (5x-2y,7x-4y)$.
\begin{enumerate}
\item Compute the matrix $\mathcal{M}(T)$ of $T$ with respect to the standard basis of $\R^2$. \points{2}

\vspace{1in}

\item Find the eigenvalues of $T$. \points{4}

\vspace{2in}

\item Find a basis of $\R^2$ consisting of eigenvectors of $\R^2$. \points{4}

\vspace{2.75in}

\item Is the operator $T$ diagonalizable? Why or why not? If it is, give a matrix $P$ such that $P^{-1}\mathcal{M}(T)P$ is diagonal. (You don't have to verify it's diagonal.) \points{2}
\end{enumerate}
\end{enumerate}
\newpage

Extra space for rough work or to complete a problem, as needed. Please do not remove this page. If there is work to be graded on this page, please indicate this next to the corresponding question.
\end{document}