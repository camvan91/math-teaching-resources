\documentclass[12pt]{article}
\usepackage{amsmath}
\usepackage{amssymb}
\usepackage[letterpaper,margin=0.85in,centering]{geometry}
\usepackage{fancyhdr}
\usepackage{enumerate}
\usepackage{lastpage}
\usepackage{multicol}
\usepackage{graphicx}

\reversemarginpar

\pagestyle{fancy}
\cfoot{}
\lhead{Math 3410A}\chead{Quiz \# 3}\rhead{Friday, 30\textsuperscript{th} January, 2015}
\rfoot{Total: 10 points}
%\chead{{\bf Name:}}
\newcommand{\points}[1]{\marginpar{\hspace{24pt}[#1]}}
\newcommand{\skipline}{\vspace{12pt}}
%\renewcommand{\headrulewidth}{0in}
\headheight 30pt

\newcommand{\di}{\displaystyle}
\newcommand{\R}{\mathbb{R}}
\newcommand{\C}{\mathbb{C}}
\newcommand{\vv}{\mathbf{v}}
\newcommand{\aaa}{\mathbf{a}}
\newcommand{\bbb}{\mathbf{b}}
\newcommand{\ccc}{\mathbf{c}}
\newcommand{\dotp}{\boldsymbol{\cdot}}
\begin{document}

%\author{Instructor: Sean Fitzpatrick}
\thispagestyle{fancy}
%\noindent{{\bf Name and student number:}}
{\bf Name:}

\bigskip

Solve {\bf one} of the following two questions:
 \begin{enumerate}
 \item  Let $U = \{p\in \mathcal{P}_4(\R) : p(6)=0\}$.
\begin{enumerate}
 \item Find a basis for $U$. \points{2} (Hint: note that $p(x)=x-6$ is an element of $U$.)
 \item Extend the basis in part (a) to a basis of $\mathcal{P}_4(\R)$.\points{4}
 \item Find a subspace $W$ of $\mathcal{P}_4(\R)$ such that $U\oplus W = \mathcal{P}_4(\R)$. \points{4}
\end{enumerate}
\item Suppose $U$ and $W$ are subspaces of $V$ such that $V=U\oplus W$. Show that if $\{u_1,\ldots, u_m\}$ is a basis for $U$, and $\{w_1,\ldots, w_k\}$ is a basis for $W$, then \points{10}
\[
 \{u_1,\ldots, u_m,w_1,\ldots, w_k\}
\]
is a basis for $V$.
 \end{enumerate}
\end{document}