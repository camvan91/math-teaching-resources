\documentclass[11pt,letterpaper]{amsart}
\usepackage[utf8]{inputenc}
%\usepackage{amsmath}
%\usepackage{amsfonts}
%\usepackage{amssymb}
\linespread{1.6}
\usepackage[left=2cm,right=2cm,top=2cm,bottom=2cm]{geometry}
\author{Math 1560, Fall 2018}
\title{Standards Checklist}
\begin{document}
\maketitle


Chapter 1: Limits and continuity %
\begin{enumerate}
\item Explain the concept of a limit using graphical and numerical information.\hspace*{\fill} Grade: \framebox{1} \quad \framebox{2} \quad \framebox{3}
\item Apply limit laws in an abstract setting (explicit functions not given).\hspace*{\fill} Grade: \framebox{1} \quad \framebox{2} \quad \framebox{3}
\item Use algebraic manipulation to evaluate limits.\hspace*{\fill} Grade: \framebox{1} \quad \framebox{2} \quad \framebox{3}
\item Evaluate limits involving trigonometric functions.\hspace*{\fill} Grade: \framebox{1} \quad \framebox{2} \quad \framebox{3}
\item Algebraically and graphically determine one-sided limits.\hspace*{\fill} Grade: \framebox{1} \quad \framebox{2} \quad \framebox{3}
\item Evaluate limits involving infinity.\hspace*{\fill} Grade: \framebox{1} \quad \framebox{2} \quad \framebox{3}
\item Demonstrate continuity of a function using the definition.\hspace*{\fill} Grade: \framebox{1} \quad \framebox{2} \quad \framebox{3}
\item Understand and apply the \emph{Intermediate Value Theorem}.\hspace*{\fill} Grade: \framebox{1} \quad \framebox{2} \quad \framebox{3}
\end{enumerate}
%
\bigskip

Chapter 2: Derivatives %
\begin{enumerate} 
\item Understand and apply the limit definition of the derivative.\hspace*{\fill} Grade: \framebox{1} \quad \framebox{2} \quad \framebox{3}
\item Understand and apply basic derivative rules (sum, constant, power).\hspace*{\fill} Grade: \framebox{1} \quad \framebox{2} \quad \framebox{3}
\item Calculate derivatives using the product rule.\hspace*{\fill} Grade: \framebox{1} \quad \framebox{2} \quad \framebox{3}
\item Calculate derivatives using the quotient rule.\hspace*{\fill} Grade: \framebox{1} \quad \framebox{2} \quad \framebox{3}
\item Calculate derivatives using the chain rule.\hspace*{\fill} Grade: \framebox{1} \quad \framebox{2} \quad \framebox{3}
\item Symbolically apply derivative rules in an abstract setting.\hspace*{\fill} Grade: \framebox{1} \quad \framebox{2} \quad \framebox{3}
\item Use implicit differentiation to compute the equation of a tangent line.\hspace*{\fill} Grade: \framebox{1} \quad \framebox{2} \quad \framebox{3}
\item Compute derivatives using logarithmic differentiation.\hspace*{\fill} Grade: \framebox{1} \quad \framebox{2} \quad \framebox{3}
\item Compute derivatives of trigonometric and inverse trigonometric functions.\hspace*{\fill} Grade: \framebox{1} \quad \framebox{2} \quad \framebox{3}
\end{enumerate}
%
\bigskip

Chapter 3: Graphical behaviour of functions %
\begin{enumerate} 
\item Determine extreme values of a continuous function on a closed interval.\hspace*{\fill} Grade: \framebox{1} \quad \framebox{2} \quad \framebox{3}
\item State the \emph{Mean Value Theorem} and apply it to theoretical problems.\hspace*{\fill} Grade: \framebox{1} \quad \framebox{2} \quad \framebox{3}
\item Determine intervals of increase/decrease; classify critical points.\hspace*{\fill} Grade: \framebox{1} \quad \framebox{2} \quad \framebox{3}
\item Use the second derivative to determine concavity.\hspace*{\fill} Grade: \framebox{1} \quad \framebox{2} \quad \framebox{3}
\item Produce an accurate sketch of the graph of a function.\hspace*{\fill} Grade: \framebox{1} \quad \framebox{2} \quad \framebox{3}
\end{enumerate}

\bigskip

Chapter 4: Applications of the derivative %
\begin{enumerate}
\item Solve word problems involving related rates of change.\hspace*{\fill} Grade: \framebox{1} \quad \framebox{2} \quad \framebox{3}
\item Solve word problems involving optimization.\hspace*{\fill} Grade: \framebox{1} \quad \framebox{2} \quad \framebox{3}
\item Use linear approximations to estimate function values.\hspace*{\fill} Grade: \framebox{1} \quad \framebox{2} \quad \framebox{3}
\item Compute the Taylor polynomial of a function to a specified degree.\hspace*{\fill} Grade: \framebox{1} \quad \framebox{2} \quad \framebox{3}
\item Quantify the error involved in a Taylor polynomial approximation.\hspace*{\fill} Grade: \framebox{1} \quad \framebox{2} \quad \framebox{3}
\end{enumerate}

\bigskip

Chapter 5: Integration %
\begin{enumerate} 
\item Compute antiderivatives and solve initial value problems.\hspace*{\fill} Grade: \framebox{1} \quad \framebox{2} \quad \framebox{3}
\item Understand and apply properties of definite integrals.\hspace*{\fill} Grade: \framebox{1} \quad \framebox{2} \quad \framebox{3}
\item Use a left- or right-endpoint Riemann sum to approximate area.\hspace*{\fill} Grade: \framebox{1} \quad \framebox{2} \quad \framebox{3}
\item Calculate a definite integral using the Riemann sum definition.\hspace*{\fill} Grade: \framebox{1} \quad \framebox{2} \quad \framebox{3}
\item Use Part I of the FTC to compute derivatives.\hspace*{\fill} Grade: \framebox{1} \quad \framebox{2} \quad \framebox{3}
\item Use Part II of the FTC to evaluate simple definite integrals.\hspace*{\fill} Grade: \framebox{1} \quad \framebox{2} \quad \framebox{3}
\item Use substitution to evaluate definite and indefinite integrals.\hspace*{\fill} Grade: \framebox{1} \quad \framebox{2} \quad \framebox{3}
\item Set up and evaluate a definite integral to compute area between curves.\hspace*{\fill} Grade: \framebox{1} \quad \framebox{2} \quad \framebox{3}
\end{enumerate}
%
 
\end{document}