\documentclass[10pt]{beamer}
\usetheme{Pittsburgh}
\usepackage[utf8]{inputenc}
\usepackage{amsmath}
\usepackage{amsfonts}
\usepackage{amssymb}
%\author{}
%\title{}
%\setbeamercovered{transparent} 
\setbeamertemplate{navigation symbols}{} 
%\logo{} 
%\institute{} 
%\date{} 
%\subject{} 
\newcommand{\di}{\displaystyle}
\begin{document}

%\begin{frame}
%\titlepage
%\end{frame}

%\begin{frame}
%\tableofcontents
%\end{frame}

\begin{frame}
\begin{itemize}
\item Standard 1.1: \begin{itemize}
\item Suppose $\di \lim_{x\to 2}f(x)=2$. What can you say about the value of $f(2)$? 
\item Sketch the graph of a function with the following features:\\ (a) $\di\lim_{x\to 1}f(x)$ exists, but $f(1)$ does not. (b) $f(2)$ is defined, but $\di\lim_{x\to 2}f(x)$ does not exist.
\end{itemize}
\item Standard 1.2: Let $f(x)=x^2-2x+4$. Using only the basic limit properties from Theorem 1.3.1 in the textbook, show that $\di \lim_{x\to a}f(x) = f(a)$ for any real number $a$. 
\item Standard 1.3: Evaluate the following limits:
\[
\lim_{x\to 1}\frac{x-\sqrt{x}}{x^3-1} \quad \text{ and } \quad \lim_{x\to -1}\left(\frac{2}{x^2-1}+\frac{1}{x+1}\right)
\]
\item Standard 1.4: Evaluate $\di\lim_{x\to 0}\frac{\tan^2(2x)}{x^2}$ and $\di \lim_{x\to 0}\frac{1-\cos(x)}{x^2}$
\item Standard 1.5: Let $f(x)=\begin{cases}\ln(x-1) & \text{ if } x\geq 2\\ 1-x^2 & \text{ if } x<2\end{cases}$.\\ Evaluate
$\di\lim_{x\to 2^-}f(x)$, $\di\lim_{x\to 2^+}f(x)$, $\di \lim_{x\to 2}f(x)$, and $f(2)$.
\end{itemize}
\end{frame}

\end{document}