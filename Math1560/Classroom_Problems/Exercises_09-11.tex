\documentclass[11pt]{beamer}
\usetheme{Pittsburgh}
\usepackage[utf8]{inputenc}
\usepackage{amsmath}
\usepackage{amsfonts}
\usepackage{amssymb}
%\author{}
%\title{}
%\setbeamercovered{transparent} 
\setbeamertemplate{navigation symbols}{} 
%\logo{} 
%\institute{} 
%\date{} 
%\subject{} 
\newcommand{\di}{\displaystyle}
\begin{document}

%\begin{frame}
%\titlepage
%\end{frame}

%\begin{frame}
%\tableofcontents
%\end{frame}

\begin{frame}
\begin{itemize}
\item Standard 1.1: The graph of a function $f$ can be seen at \href{https://www.geogebra.org/m/umxxz97b}{https://www.geogebra.org/m/umxxz97b}. Based on this graph, what can you say about $\di \lim_{x\to 1}f(x)$  and $\di\lim_{x\to 2}f(x)$?
Explain your answer.

\item Standard 1.2: Suppose you know that $\di \lim_{x\to 3}f(x)=3$ and $\di \lim_{x\to 3}g(x)=-2$. Determine the following limits if possible. 
\[
\lim_{x\to 3}(f(x)-4g(x))\quad \lim_{x\to 3}(f(x)g(x)) \quad \lim_{x\to 3}f(g(x))
\]
Justify your answer using properties of limits.
\item Standard 1.3: Evaluate the following limits:
\[
\lim_{x\to 3}\frac{x^2-5x+6}{x^2-9} \quad \text{ and } \quad \lim_{x\to 1}\frac{x-1}{\sqrt{x}-1}
\]
\end{itemize}
\end{frame}

\end{document}