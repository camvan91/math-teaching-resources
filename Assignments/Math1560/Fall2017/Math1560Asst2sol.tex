\documentclass[letterpaper,12pt]{article}

\usepackage{ucs}
\usepackage[utf8x]{inputenc}
\usepackage{amsmath}
%\usepackage{amsfonts}
%\usepackage{amssymb}
\usepackage[margin=1in]{geometry}
\usepackage{graphicx}
\usepackage[bitstream-charter]{mathdesign}
\usepackage[T1]{fontenc}

\newcommand{\len}[1]{\lVert #1\rVert}
\newcommand{\abs}[1]{\lvert #1\rvert}
\newcommand{\R}{\mathbb{R}}
\newcommand{\di}{\displaystyle}
\title{Math 1560 Assignment \#2 Solutions\\University of Lethbridge, Fall 2017}
\author{Sean Fitzpatrick}
\begin{document}
 \maketitle


\begin{enumerate}
\item Consider the function 
\[
f(x)=\begin{cases}x^3\sin\left(\frac{1}{x}\right), & \text{ if } x\neq 0\\0, &\text{ if } x=0\end{cases}.
\]

Show that $f$ is differentiable at $x=0$, and find $f'(0)$.

\medskip

\textbf{Solution:} By definition of the derivative, we have
\begin{align*}
f'(0) & = \lim_{h\to 0}\frac{f(h)-f(0)}{h}\\
& = \lim_{h\to 0}\frac{h^3\sin(\frac{1}{h})-0}{h} \tag{Note that $f(0)=0$}\\
& = \lim_{h\to 0}h^2\sin\left(\frac{1}{h}\right). \tag{Since $\frac{h^3}{h}=h^2$ for $h\neq 0$}
\end{align*}

Since the range of the sine function is $[-1,1]$, we have
\[
-1\leq \sin\left(\frac{1}{h}\right)\leq 1
\]
for any $h\neq 0$. Since $h^2>0$ for all $h\neq 0$, we can multiply across the inequality, giving us
\[
-h^2\leq h^2\sin\left(\frac{1}{h}\right)\leq h^2.
\]
Since $\di \lim_{h\to 0}(-h^2) = \lim_{h\to 0}(h^2) = 0$, it follows from the Squeeze Theorem that
\[
f'(0) = \lim_{h\to 0}h^2\sin\left(\frac{1}{h}\right) = 0.
\]

\pagebreak

\item Let $f$ and $g$ be differentiable functions, and let $h(x)=f(x)g(x)$. We know from the product rule that
\[
h'(x) = f'(x)g(x)+f(x)g'(x).
\]
\begin{enumerate}
\item Compute $h''(x)$ (in terms of $f$ and $g$ and their derivatives) and simplify.

\medskip

\textbf{Solution:} To compute $h''(x)$ we take the derivative of $h'(x)$:
\begin{align*}
h''(x) & = \frac{d}{dx}(h'(x)) = \frac{d}{dx}(f'(x)g(x)+f(x)g'(x))\\
& = (f''(x)g(x)+f'(x)g'(x)) + (f'(x)g'(x)+f(x)g''(x))\\
& = f''(x)g(x) +2f'(x)g'(x)+f(x)g''(x).
\end{align*}

\item Compute $h'''(x)$ (in terms of $f$ and $g$ and their derivatives) and simplify.

\medskip

\textbf{Solution:} To compute $h'''(x)$, we take the derivative of our result for $h''(x)$ above:
\begin{align*}
h'''(x) & = \frac{d}{dx}(h''(x))  = \frac{d}{dx}(f''(x)g(x) +2f'(x)g'(x)+f(x)g''(x))\\
 & = (f'''(x)g(x) +f''(x)g'(x))+2(f''(x)g'(x)+f'(x)g''(x))\\
 & \quad\quad\quad +(f'(x)g''(x)+f(x)g'''(x))\\
 & = f'''(x)g(x)+3f''(x)g'(x)+3f'(x)g''(x)+f(x)g'''(x).
\end{align*}

\item (Do not submit an answer to this part) Can you guess a general product rule formula for the $n^{\text{th}}$ derivative of $h(x)$?

\medskip

\textbf{Solution:} (In case you were curious, but not curious enough to look it up) If you know the binomial formula, you might notice a familiar pattern:
\begin{align*}
(a+b)^1 & = a + b \\
(a+b)^2 & = a^2+2ab+b^2 \\
(a+b)^3 & = a^3+3a^2b+3ab^2+b^3 \\
\vdots & \quad \quad \vdots 
\end{align*}
\begin{align*}
(fg)'(x) & = f'(x)g(x)+f(x)g'(x)\\
(fg)''(x) & = f''(x)g(x)+2f'(x)g'(x)+f(x)g''(x)\\
(fg)'''(x) &= f'''(x)g(x)+3f''(x)g'(x)+3f'(x)g''(x)+f(x)g'''(x)\\
\vdots & \quad \quad \vdots 
\end{align*}
Notice how the coefficients are the same in each case, for $n=1,2,3$: the binomial coefficients $\binom{n}{k}$ appearing in Pascal's Triangle. It turns out that this pattern does indeed continue:
\[
(a+b)^n = a^n+na^{n-1}b+\binom{n}{2}a^{n-2}b^2+\cdots + nab^{n-1}+b^n = \sum_{k=0}^n \binom{n}{k}a^{n-k}b^k,
\]
and
\begin{align*}
(fg)^{(n)}(x) &= f^{(n)}(x)g(x)+nf^{(n-1)}(x)g(x)+\binom{n}{k}f^{(n-2}(x)g''(x)+\cdots\\
& \hspace{2.3in} + nf'(x)g^{(n-1)}(x)+f(x)g^{(n)}(x)\\
& = \sum_{k=0}^n\binom{n}{k}f^{(n-k)}(x)g^{(k)}(x),
\end{align*}
where $f^{(n)}(x)$ denotes the $n^\textrm{th}$ derivative of $f$ (and $f^{(0)}(x) = f(x)$).

This result is often known as Leibniz's Rule or Leibniz's Identity.
\end{enumerate}

\bigskip

\item Two curves are said to be \textit{orthogonal} if, at each point of intersection, they meet at a right angle. Show that the ellipse $3x^2+2y^2=5$ and the curve $y^3=x^2$ are orthogonal.

Hint: The curves intersect at the points $(1,1)$ and $(-1,1)$.

\medskip

\textbf{Solution:} For the ellipse, we find
\begin{align*}
\frac{d}{dx}(3x^2+2y^2) & = \frac{d}{dx}(5)\\
6x+4y\frac{dy}{dx} & = 0
\end{align*}
using implicit differentiation. Solving for $\dfrac{dy}{dx}$, we find $\dfrac{dy}{dx} = -\dfrac{3x}{2y}$.

For the second curve, implicit differentiation gives us
\[
3y^2\frac{dy}{dx} = 2x, \,\text{ so } \, \dfrac{dy}{dx} = \dfrac{2x}{3y^2}.
\]



According to the hint, we need to check two points of intersection. At $(1,1)$, we put $x=1, y=1$ into the expression for $\dfrac{dy}{dx}$ for each curve. For the ellipse,
\[
m_1 = \left.\frac{dy}{dx}\right|_{\substack{x=1\\y=1}} = -\frac{3(1)}{2(1)} = -\frac{3}{2}.
\]
For the second curve, 
\[
m_2 = \left.\frac{dy}{dx}\right|_{\substack{x=1\\y=1}} = \frac{2(1)}{3(1)^2} = \frac{2}{3}.
\]

Since $m_1m_2 = -1$, the two curves intersect orthogonally at $(1,1)$. At the point of intersection $(-1,1)$, we put $x=-1$ and $y=1$, giving us
\[
m_1 = -\frac{3(-1)}{2(1)} = \frac{3}{2} \text{ and } m_2 = \frac{2(-1)}{3(1)^2} = -\frac{2}{3},
\]
and again we see that $m_1m_2=-1$, as required.
 \end{enumerate}

\end{document}
 
