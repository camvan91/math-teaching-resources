\documentclass[letterpaper,12pt]{article}

\usepackage{ucs}
\usepackage[utf8x]{inputenc}
\usepackage{amsmath}
\usepackage{amsfonts}
\usepackage{amssymb}
\usepackage[margin=1in]{geometry}
\newcommand{\di}{\displaystyle}
\newcommand{\abs}[1]{\lvert #1\rvert}
\newcommand{\R}{\mathbb{R}}
\newcommand{\C}{\mathbb{C}}
\newcommand{\F}{\mathbf{F}}
\newcommand{\dotp}{\boldsymbol{\cdot}}
\newcommand{\aaa}{\mathbf{a}}
\newcommand{\bbb}{\mathbf{b}}
\newcommand{\ccc}{\mathbf{c}}
\renewcommand{\r}{\mathbf{r}}

\title{Math 2580 Assignment \#6 Solutions\\University of Lethbridge, Spring 2016}
\author{Sean Fitzpatrick}
\begin{document}
 \maketitle
\begin{enumerate}
\item The \textit{First Theorem of Pappus} states that for any plane curve $C$, the surface area of the surface of revolution generated by revolving $C$ about an axis that does not intersect $C$ is $A = sc$, where $s$ is the length of $C$, and $c$ is the circumference of the circle generated by revolving the \textbf{centroid} of $C$ about the given axis.

Here, the centroid of a curve $C$ is the point $(\overline{x},\overline{y})$ given by
\[
 \overline{x} = \frac{1}{s}\int_C x\,ds \quad \text{ and } \quad \overline{y} = \frac{1}{s}\int_C y\,ds; \quad s = \int_C\,ds.
\]
Use the First Theorem of Pappus to find the surface area of the surface of revolution generated by revolving one arch of the cycloid $x=R(t-\sin t)$, $y=R(1-\cos t)$ about the $x$-axis.

\bigskip

{\bf Solution:} Since we're revolving about the $x$-axis, the distance from the centroid to the axis of rotation is given by $\overline{y}$, so the circumference of the circle generated by revolving the centroid about the $x$-axis is $2\pi\overline{y}$. The First Theorem of Pappus then gives us
\[
 A = sc = s(2\pi \overline{y}) = 2\pi s\left(\frac{1}{s}\int_C y\,ds\right) = 2\pi\int_C y\,ds.
\]
We know that $ds = \sqrt{x'(t)^2+y'(t)^2}\,dt$, and we have
\begin{align*}
 x(t) & = R(t-\sin t) & \Rightarrow  x'(t) & = R(1-\cos(t))\\
 y(t) & = R(1-\cos t) & \Rightarrow  y'(t) &= R\sin t,
\end{align*}
so
\begin{align*}
 \sqrt{x'(t)^2+y'(t)^2} & = \sqrt{R^2(1-2\cos t+\cos^2t)+R^2\sin^2t}\\
& = R\sqrt{2-2\cos t}\\
& = R\sqrt{4\sin^2(t/2)}\\
& = 2R\sin(t/2),
\end{align*}
using the identity $1-\cos t = 2\sin^2(t/2)$. It's not necessary to calculate the arc length $s$, since it cancels out above, but if you want to do it anyway, we have
\[
 s = \int_0^{2\pi}\sqrt{x'(t)^2+y'(t)^2}\,dt = \int_0^{2\pi}2R\sin(t/2)\,dt = 8R.
\]
(Notice that one arch of the cycloid corresponds to $0\leq t\leq 2\pi$.)

Now, we compute (noting that $y(t) = R(1-\cos t) = 2R\sin^2(t/2)$)
\begin{align*}
 A & = 2\pi\int_C y\,ds = 2\pi\int_0^{2\pi}y(t)\sqrt{x'(t)^2+y'(t)^2}\,dt\\
 & = 2\pi\int_0^{2\pi}(2R\sin^2(t/2))(2R\sin(t/2))\,dt\\
 & = 8\pi R^2\int_0^{2\pi}\sin^3(t/2)\,dt\\
 & = 8\pi R^2\int_0^{2\pi}\sin(t/2)(1-\cos^2(t/2))\,dt.
\end{align*}
At this point we use the substitution $u=\cos(t/2)$, so $du = -\frac{1}{2}\sin(t/2)\,dt$, and $u(0)=\cos(0)=1$, $u(2\pi) = \cos(\pi) = -1$, so
\[
 A = 8\pi R^2 \int_{-1}^1 (1-u^2)(2\,du) = 16\pi R^2 \left.\left(u-\frac{u^3}{3}\right)\right|_{-1}^1 = \frac{64\pi R^2}{3}.
\]




\item Suppose $\F$ is a vector field such that $\lVert \F(x,y,z)\rVert\leq M$ for all $(x,y,z)\in\R^3$, and $C$ is a curve with length $L$. Show that
\[
 \left\lvert \int_C \F\dotp d\mathbf{r}\right\rvert \leq ML.
\]

\textbf{Solution:} Let $\r(t)$, $t\in [a,b]$, be a choice of parameterization for $C$. With our assumption $\lVert \F\rVert \leq M$, we have
\begin{align*}
 \left\lvert \int_C \F\dotp d\mathbf{r}\right\rvert & = \left\lvert \int_a^b \F(\r(t))\dotp \r'(t)\,dt\right\rvert & &\\
 & \leq \int_a^b \lvert \F(\r(t))\dotp \r'(t)\rvert \,dt & &\text{Using the property $\left\lvert \int_a^b f(t)\,dt\right\rvert \leq \int_a^b \lvert f(t)\rvert\,dt$}\\
 & \leq \int_a^b \lVert \F(\r(t))\rVert \,\lVert \r'(t)\rVert \,dt & & \text{Using the Cauchy-Schwarz inequality}\\
 & \leq \int_a^b M\lVert \r'(t)\rVert\,dt = ML,
\end{align*}
as required.

\item  Let $\aaa$, $\bbb$, and $\ccc$ be any three constant, linearly independent vectors. (Linearly independent means that none of the vectors lies in the plane spanned by the other two, or equivalently, that none of the vectors can be written as a linear combination of the other two.) Let $\r = \langle x,y,z\rangle$, and let $E$ be the region in $\R^3$ given by the inequalities
\[
0\leq \aaa\dotp\r\leq \alpha,\quad 0\leq \bbb\dotp\r\leq \beta,\quad 0\leq \ccc\dotp\r\leq \gamma,
\]
where $\alpha,\, \beta$ and $\gamma$ are positive constants. Show that
\[
\iiint_E (\aaa\dotp\r)(\bbb\dotp\r)(\ccc\dotp\r)\,dV = \frac{(\alpha\beta\gamma)^2}{8\abs{\aaa\dotp(\bbb\times\ccc)}}.
\]
\textbf{Solution:} Let $\aaa = \langle a_1,a_2,a_3\rangle$, $\bbb = \langle b_1,b_2,b_3\rangle$, and $\ccc = \langle c_1,c_2,c_3\rangle$, and let
\begin{align*}
u &= \aaa\dotp\r = a_1x+a_2y+a_3z\\
v &= \bbb\dotp\r = b_1x+b_2y+b_3z\\
w &= \ccc\dotp\r = c_1x+c_2y+c_3z,
\end{align*}
which allows us to define the inverse transformation $T^{-1}(x,y,z) = (u,v,w)$. The Jacobian of this transformation is
\[
J_{T^{-1}}(x,y,z) = \det\begin{pmatrix}
a_1&a_2&a_3\\b_1&b_2&b_3\\c_1&c_2&c_3
\end{pmatrix} = \aaa\dotp(\bbb\times\ccc),
\]
and since the vectors $\aaa,\,\bbb,\,\ccc$ are linearly independent, we know that $\aaa\dotp(\bbb\times\ccc)\neq 0$.
Since solving for $x$, $y$ and $z$ in terms of $u$, $v$, and $w$ would take a lot of work, we simply use the fact that
\[
J_T(u,v,w) = \frac{1}{J_{T^{-1}}(T(u,v,w))} = \frac{1}{\aaa\dotp(\bbb\times\ccc)}.
\]
The given region $E$ is thus the transformation of the box $B$ given by $0\leq u\leq \alpha$, $0\leq v\leq \beta$, $0\leq w\leq\gamma$ under the transformation $T$, and thus
\begin{align*}
\iiint_E (\aaa\dotp\r)(\bbb\dotp\r)(\ccc\dotp\r)\,dV & = \iiint_B uvw \abs{J_T(u,v,w)}\,du\,dv\,dw\\
& = \int_0^\gamma\int_0^\beta\int_0^\alpha uvw \left|\frac{1}{\aaa\dotp(\bbb\times\ccc)}\right| \,du\,dv\,dw\\
& = \frac{1}{\abs{\aaa\dotp(\bbb\times\ccc)}}\int_0^\gamma\int_0^\beta \frac{\alpha^2}{2}vw\,dv\,dw\\
& = \frac{1}{\abs{\aaa\dotp(\bbb\times\ccc)}}\int_0^\gamma \frac{\alpha^2}{2}\frac{\beta^2}{2}w\,dw\\
& = \frac{\alpha^2\beta^2\gamma^2}{8\abs{\aaa\dotp{\bbb\times\ccc}}},
\end{align*}
as required.

\end{enumerate}



\end{document}
 
