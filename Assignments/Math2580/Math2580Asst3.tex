\documentclass[letterpaper,12pt]{article}

\usepackage{ucs}
\usepackage[utf8x]{inputenc}
\usepackage{amsmath}
\usepackage{amsfonts}
\usepackage{amssymb}
\usepackage[margin=1in]{geometry}

\newcommand{\abs}[1]{\lvert #1\rvert}
\newcommand{\R}{\mathbb{R}}
\newcommand{\C}{\mathbb{C}}
\newcommand{\dotp}{\boldsymbol{\cdot}}

\title{Math 2580 Assignment \#3\\University of Lethbridge, Spring 2016}
\author{Sean Fitzpatrick}
\begin{document}
 \maketitle

{\bf Due date:} Thursday, February 4th, by 3 pm.

\bigskip

Please provide solutions to the problems below, using the following guidelines:
\begin{itemize}
\item Your submitted assignment should be a {\bf good copy} -- figure out the problems first, and then write down organized solutions to each problem. 
\item You should include a {\bf cover page} with the following information: the course number and title, the assignment number, your name, and a list of any resources you used or people you worked with.
\item Since you have plenty of time to work on the problems, assignment solutions will be held to a higher standard than on a test. Your explanations should be clear enough that any of your classmates can understand your solutions.
\item Group work is permitted, but copying is not. If you're not sure what the difference is, feel free to ask. If you get help solving a problem, you should (a) make sure you completely understand the solution, and (b) re-write the solution for your good copy by yourself, in your own words.
\item Assignments can be submitted in class, or in the designated drop box on the 5th floor of University Hall, across from the Math Department office.
\item Late assignments will not be accepted without prior permission.

\end{itemize}
\newpage


\subsection*{Assigned problems}
\begin{enumerate}
 \item In class, I mentioned the fact that if we want to find the equation of the tangent line to a level curve $f(x,y)=c$ at a point $(a,b)$ on the curve (so $f(a,b)=c$), there are two ways to do it:
\begin{itemize}
 \item Using implicit differentiation, as in Calculus I: take the derivative of both sides with respect to $x$, assuming that the equation defines $y$ implicitly as a function of $x$ ($y=g(x)$), let's say.
 \item Using the gradient: since $\nabla f(a,b)$ is a normal vector for the tangent line, we have 
\begin{equation}\label{qe}
 0=\nabla f(a,b)\dotp \langle x-a, y-b\rangle = f_x(a,b)(x-a)+f_y(a,b)(y-b).
\end{equation}
\end{itemize}
\begin{enumerate}
 \item Verify that both above methods give the same equation for the tangent line to the curve $x^2y+xy^2=6$ at the point $(2,1)$.
 \item Confirm that the two methods are equivalent, as follows:

The Implicit Function Theorem for a function $f:\R^2\to\R$ states the following:
\begin{quotation}
 Let $f:D\subseteq \R^2\to \R$ be a continuously differentiable function. At any point $(a,b)$ such that $f_y(a,b)\neq 0$, the equation $f(x,y)=c$ defines $y$ implicitly as a function $g$ of $x$ for all $x$ in some interval\footnote{Don't worry too much about the ``in some interval'' part. The argument is as follows: since $f_y(x,y)$ is continuous, if $f_y(a,b)\neq 0$, then $f_y(x,y)\neq 0$ for all $(x,y)$ in some disk centred at $(a,b)$. (The function can't suddenly jump to zero.)} centred at $x=a$, and
\begin{equation}\label{eq}
 \frac{dy}{dx} = g'(x) = - \frac{f_x(x,y)}{f_y(x,y)}
\end{equation}
for all $x$ in this interval.
\end{quotation}
{\bf Assuming} that you can prove that the equation $f(x,y)=c$ defines $y$ as a function of $x$ for $x$ near $a$, if $f_y(a,b)\neq 0$, show that Equation \eqref{eq} is true.

\medskip

{\em Hint:} Using the Chain Rule, take the derivative with respect to $x$ of both sides of the equation $f(x,y)=c$. If you're finding it hard to see how the Chain Rule applies, consider the parametric curve $r(t) = (t,g(t))$, and calculate $\dfrac{d}{dt}(f(g(t)))$ using the Chain Rule. Then note that your choice of parametric curve defines $x=t$ and $y=g(t)$, and since $x=t$, the derivative with respect to $t$ is the same as the derivative with respect to $x$.

\medskip

In case it's not clear that Equation \eqref{eq} confirms that the two methods from part (a) are equivalent, note that using Calc I methods, the tangent line to the graph $y=g(x)$ at $a=x$ is given by $y=g(a) + g'(a)(x-a)$. On the other hand, if we solve Equation \eqref{qe} for $y$, we get $y = b-\dfrac{f_x(a,b)}{f_y(a,b)}(x-a)$. Comparing the two equations, since $g(a)=b$, our methods agree as long as we can show that $g'(a) = -\dfrac{f_x(a,b)}{f_y(a,b)}$, and that's what I'm asking you to show.
\end{enumerate}
\newpage

{\bf Note:} Take a minute to think about what happens if $f_y(a,b)=0$. Looking at your examples, it should be clear that at such a point $(a,b)$, the curve $f(x,y)=c$ will have a vertical tangent. (In general, if $f_y(a,b)=0$, then $\nabla f(a,b)$ is parallel to the $x$-axis, and if the normal vector is horizontal, the tangent line must be veritcal.) Is it clear why it's impossible to write $y$ as a function of $x$ near such points? So if $f_y(a,b)=0$, we can't define $y$ as a function of $x$ near $(a,b)$, but as long as $f_x(a,b)=\neq 0$, we could instead define $x$ as a function of $y$, and we could similarly show that $\dfrac{dx}{dy} = -\dfrac{f_y(a,b)}{f_x(a,b)}$.

But what if {\bf both} $f_x(a,b)=0$ and $f_y(a,b)=0$? Well, in this case, $\nabla f(a,b)$ is the zero vector -- not much use as a normal vector! Any such point where the gradient vector is zero is called a {\bf critical point}, and the value $c=f(a,b)$ is called a {\bf critical value}. Any value that is not a critical value is called a {\bf regular value}. What all of the above tells us is that if $r$ is a regular value, the level curve $f(x,y)=r$ will have a well-defined tangent line at every point: the curve is {\em smooth}. (If there was some point $(a,b)$ on the curve where $\nabla f(a,b)=\vec{0}$, then $r$ would be a critical value, not a regular value.)

So the ``nice'' level curves are the curves $f(x,y)=r$, where $r$ is a regular value. The curves $f(x,y)=c$, where $c$ is a critical value may not be so nice, or they might not even be a curve at all! Consider for example the functions $f(x,y)=x^2+y^2$ and $g(x,y)=x^2-y^2$. Both functions have exactly one critical point; namely $(0,0)$. (I'll leave this as an easy exercise.) The corresponding critical value for both functions is 0. We have the sets\footnote{That's right, I'm using preimage notation. I make no apologies for this.}
\[
 f^{-1}(0) = \{(x,y) | f(x,y)=0\} = \{ (x,y) | x^2+y^2 = 0\} = \{(0,0)\}
\]
and 
\[
 g^{-1}(0) = \{(x,y) | g(x,y)=0\} = \{ (x,y) | x^2-y^2=0\} = \{(x,y) | y=\pm x\}.
\]
In the first case, we don't even get a curve; we just get a single point. In the second case, instead of a curve, we get a pair of intersecting curves: the lines $y=x$ and $y=-x$. Here, the bad behaviour is that point of intersection at the origin. How to you find a tangent line when the curve is pointing in two directions at once?

\bigskip

As you work on the next problem, think about the analogous results for functions of three variables. If $F(x,y,z)=k$ defines a level surface, where $F:D\subseteq \R^3\to \R$ is continuously differentiable, we know that the normal vector at a point $(a,b,c)$ on the surface is given by $\nabla F(a,b,c)$. Note that if $F_z(a,b,c)=0$, then $\nabla F(a,b,c)$ is parallel to the $xy$-plane -- the normal vector is horizontal -- which means that the tangent plane is vertical. Near such points it wouldn't be reasonable to assume that the equation $F(x,y,z)=k$ defines $z$ implicitly as a function of $x$ and $y$.

However, as long as $F_z(a,b,c)\neq 0$, it's possible to prove that there exists a continuously differerntiable function $g(x,y)$ defined for all $(x,y)$ sufficiently close to $(a,b)$ such that $F(x,y,g(x,y))=k$.

 \newpage


\item Okay, have you recovered from the first problem? Good. Now consider a continuously differentiable function $F(x,y,z)$, and suppose $(a,b,c)$ is a point on the level surface $F(x,y,z)=k$. We discussed in class that one way to get the tangent plane to the surface at $(a,b,c)$ is to use the gradient: the vector $\nabla F(a,b,c)$ is normal to the surface at $(a,b,c)$, so
\[
 \nabla F(a,b,c) \dotp \langle x-a, y-b, z-c\rangle = 0
\]
gives the equation of the tangent plane. On the other hand, we could try generalizing the method of implicit differentiation above. Suppose that the equation $F(x,y,z)=k$ defines $z$ implicitly as a function of $x$ and $y$. That is, assume there exists a function $g:\R^2\to \R$ such that $z=g(x,y)$ satisfies
\[
 F(x,y,g(x,y))=k
\]
for all points $(x,y)$ near the point $(a,b)$.
\begin{enumerate}
 \item Using the Chain Rule, show that if $F_z(a,b,c)\neq 0$, then at the point $(a,b,c)$,
\[
 \frac{\partial z}{\partial x} = g_x(a,b) = -\frac{F_x(a,b,c)}{F_z(a,b,c)} \quad \text{ and } \quad \frac{\partial z}{\partial y} = g_y(a,b) = -\frac{F_y(a,b,c)}{F_z(a,b,c)}.
\]
 \item Suppose $F(x,y,z)=k$ implicitly defines $z=g(x,y)$ near a point $(a,b,c)$. Then near this point, we've expressed our level surface as a graph. It might not be possible to do this for the entire surface (there might, for example, be points where $F_z$ equals zero), but at least it works locally. This puts us in a position to calculate the normal vector to the surface at $(a,b,c)$ in two ways:
\begin{enumerate}
 \item Using the gradient vector $\nabla F(a,b,c)$, where we describe our surface via the equation $F(x,y,z)=k$.
 \item Using the result $\vec{n} = \langle g_x(a,b), g_y(a,b), -1\rangle$ that we obtained for graphs, where we describe our surface as the graph $z=g(x,y)$.
\end{enumerate}
Use your result from part (a) to show that these two vectors are parallel.
\end{enumerate}

\end{enumerate}



\end{document}
 
