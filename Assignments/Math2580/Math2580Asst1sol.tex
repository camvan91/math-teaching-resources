\documentclass[letterpaper,12pt]{article}

\usepackage{ucs}
\usepackage[utf8x]{inputenc}
\usepackage{amsmath}
\usepackage{amsfonts}
\usepackage{amssymb}
\usepackage[margin=1in]{geometry}

\newcommand{\abs}[1]{\lvert #1\rvert}
\newcommand{\R}{\mathbb{R}}
\newcommand{\C}{\mathbb{C}}
\title{Math 2580 Assignment \#1 Solutions\\University of Lethbridge, Spring 2016}
\author{Sean Fitzpatrick}
\begin{document}
 \maketitle


\begin{enumerate}
 \item Each of the equations below describes a quadric surface. Identify (as an ellipsoid, hyperboloid, etc.) and sketch each surface.
\begin{enumerate}
 \item $\dfrac{x^2}{4}+\dfrac{y^2}{9}+z^2=1$.
 
 \bigskip
 
 This is an ellipsoid centred at the origin, with $-2\leq x\leq 2$, $-3\leq y\leq 3$, and $-1\leq z\leq 1$. Anything that looks more or less like a rugby ball with the correct dimensions is acceptable.
 
 \item $x^2+z^2=1-2y^2$

\bigskip
 
 Rearranging gives the equation $x^2+2y^2+z^2=1$, so this is another ellipsoid with $-1\leq x,z\leq 1$ and $-1/\sqrt{2}\leq y\leq 1/\sqrt{2}$.
 
 
 \item $z+y^2=2x^2$.

\bigskip
 
 This equation can be rewritten as $z=2x^2-y^2$, so this is a hyperbolic paraboloid (saddle surface). These guys are hard to draw, so any reasonable attempt is acceptable. The 2 in front of the $x$ means it's squished a bit in the $x$ direction, but not by any amount that can reasonably be reflected in your sketch.
 
\end{enumerate}


\item Consider the function $f:\R^2\to \R$ defined by
\[
 f(x,y) = \begin{cases}
           \dfrac{xy(x^2-y^2)}{x^2+y^2}, & \text{ if } (x,y)\neq (0,0)\\ 0, & \text{ if } (x,y)=(0,0).
          \end{cases}
\]
\begin{enumerate}
 \item Compute $f_x$ and $f_y$ for $(x,y)\neq (0,0)$.
 
 \bigskip
 
 For $(x,y)\neq (0,0)$ we have
 \begin{align*}
 f_x(x,y) & = \frac{\partial}{\partial x}\left(\frac{x^3y-xy^3}{x^2+y^2}\right)\\
 & = \frac{(3x^2y-y^3)(x^2+y^2)-2x(x^3y-xy^3)}{(x^2+y^2)^2}\\
 & = \frac{x^4y+4x^2y^3-y^5}{(x^2+y^2)^2},
 \end{align*}
 and
 \begin{align*}
 f_y(x,y) & =  \frac{\partial}{\partial y}\left(\frac{x^3y-xy^3}{x^2+y^2}\right)\\
 & = \frac{(x^3-3xy^2)(x^2+y^2)-2y(x^3y-xy^3)}{(x^2+y^2)^2}\\
 & = \frac{x^5-4x^3y^2-xy^4}{(x^2+y^2)^2}.
 \end{align*}
 
 \item Show that $f_x(0,0)=f_y(0,0)=0$.
 
 \bigskip
 
 Using the limit definitions of $f_x$ and $f_y$, we have
\[
 f_x(0,0)  = \lim_{h\to 0}\frac{f(h,0)-f(0,0)}{h} = \lim_{h\to 0}\frac{0-0}{h}=0,
\]
and similarly, $f_y(0,0)=0$.

 \item Show that $f_x(0,y)=-y$ when $y\neq 0$. \label{a}
 
 \bigskip
 
 Plugging in $x=0$ to our solution from part (a), we have
 \[
 f_x(0,y) = \frac{-y^5}{(y^2)^2} = -y.
 \]
 
 \item What is $f_y(x,0)$ when $x\neq 0$? \label{b}
 
 \bigskip
 
 Using the same argument as in part \ref{a}, we have $f_y(x,0) = \dfrac{x^5}{x^4}=x$.
 
 \item Show that $f_{yx}(0,0)=1$ and $f_{xy}(0,0)=-1$. (You'll need to use limits again.)
 
 \bigskip
 
 Since $f_{yx}$ is the partial derivative of $f_y$ with respect to $x$, using our result from part \ref{b} we have
 \[
 f_{yx}(0,0) = \lim_{h\to 0}\frac{f_y(h,0)-f_y(0,0)}{h} = \lim_{h\to 0}\frac{h-0}{h} = 1.
 \]
 Similarly, $f_{xy}$ is the partial derivative of $f_x$ with respect to $y$, so using the result from \ref{a},
 \[
 f_{xy}(0,0) = \lim_{h\to 0}\frac{f_x(0,h)-f_x(0,0)}{h} = \lim_{h\to 0}\frac{-h-0}{h} = -1.
 \]
 
 \item Why does this not contradict the theorem about equality of mixed partials?
 
 \bigskip
 
 Clairaut's Theorem only guarantees equality of mixed second-order partial derivatives at a point if all second-order partial derivatives exist and are continuous on an open disc containing that point. Since $f_{xy}(0,0)\neq f_{yx}(0,0)$, it must be the case that these derivatives are not continuous at $(0,0)$. Checking this is a lot of work though: we'd have to compute $f_{xy}(x,y)$ and $f_{yx}(x,y)$ (and the other two second-order derivatives) for $(x,y)\neq (0,0)$ and show that either the limit of these functions does not exist as $(x,y)\to (0,0)$, or that the limits exist, but are not equal to $-1$ and $1$, respectively.
 
 If we believe Clairaut's Theorem to be true (and it is), then since the conclusion of the theorem failed in this case, it must be true that the hypothesis failed as well.
\end{enumerate}

 \end{enumerate}
\end{document}
 
