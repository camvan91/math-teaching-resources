\documentclass[letterpaper,12pt]{article}

\usepackage{ucs}
\usepackage[utf8x]{inputenc}
\usepackage{amsmath}
\usepackage{amsfonts}
\usepackage{amssymb}
\usepackage[margin=1in]{geometry}

\newcommand{\abs}[1]{\lvert #1\rvert}
\newcommand{\R}{\mathbb{R}}
\newcommand{\C}{\mathbb{C}}
\title{Math 2580 Assignment \#1\\University of Lethbridge, Spring 2016}
\author{Sean Fitzpatrick}
\begin{document}
 \maketitle

{\bf Due date:} Thursday, January 21, by 3 pm.

\bigskip

Please provide solutions to the problems below, using the following guidelines:
\begin{itemize}
\item Your submitted assignment should be a {\bf good copy} -- figure out the problems first, and then write down organized solutions to each problem. 
\item You should include a {\bf cover page} with the following information: the course number and title, the assignment number, your name, and a list of any resources you used or people you worked with.
\item Since you have plenty of time to work on the problems, assignment solutions will be held to a higher standard than on a test. Your explanations should be clear enough that any of your classmates can understand your solutions.
\item Group work is permitted, but copying is not. If you're not sure what the difference is, feel free to ask. If you get help solving a problem, you should (a) make sure you completely understand the solution, and (b) re-write the solution for your good copy by yourself, in your own words.
\item Assignments can be submitted in class, or in the designated drop box on the 5th floor of University Hall, across from the Math Department office.
\item Late assignments will not be accepted without prior permission.

\end{itemize}
\newpage
\subsection*{Assigned problems}
\begin{enumerate}
 \item Each of the equations below describes a quadric surface. Identify (as an ellipsoid, hyperboloid, etc.) and sketch each surface.
\begin{enumerate}
 \item $\dfrac{x^2}{4}+\dfrac{y^2}{9}+z^2=1$.
 \item $x^2+z^2=1-2y^2$
 \item $z+y^2=2x^2$.
\end{enumerate}
{\bf Note:} For each sketch, you will probably find it helpful to compute a few sections of each surface in planes parallel to the three coordinate planes. (That is, try sketching a few of the curves obtained by setting one of the variables equal to a constant.)

\item Consider the function $f:\R^2\to \R$ defined by
\[
 f(x,y) = \begin{cases}
           \dfrac{xy(x^2-y^2)}{x^2+y^2}, & \text{ if } (x,y)\neq (0,0)\\ 0, & \text{ if } (x,y)=(0,0).
          \end{cases}
\]
\begin{enumerate}
 \item Compute $f_x$ and $f_y$ for $(x,y)\neq (0,0)$.
 \item Show that $f_x(0,0)=f_y(0,0)=0$. (Hint: use the definitions. What is the value of $f(x,0)$ and $f(0,y)$?
 \item Show that $f_x(0,y)=-y$ when $y\neq 0$.
 \item What is $f_y(x,0)$ when $x\neq 0$?
 \item Show that $f_{yx}(0,0)=1$ and $f_{xy}(0,0)=-1$. (You'll need to use limits again.)
 \item Why does this not contradict the theorem about equality of mixed partials?
\end{enumerate}

 \end{enumerate}
\end{document}
 
