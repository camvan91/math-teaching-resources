\documentclass[letterpaper,12pt]{article}

\usepackage{ucs}
\usepackage[utf8x]{inputenc}
\usepackage{amsmath}
\usepackage{amsfonts}
\usepackage{amssymb}
\usepackage[margin=1in]{geometry}

\newcommand{\abs}[1]{\lvert #1\rvert}
\newcommand{\R}{\mathbb{R}}
\newcommand{\C}{\mathbb{C}}
\newcommand{\dotp}{\boldsymbol{\cdot}}

\title{Math 2580 Assignment \#4\\University of Lethbridge, Spring 2016}
\author{Sean Fitzpatrick}
\begin{document}
 \maketitle

{\bf Due date:} Thursday, February 25th, by 3 pm.

\bigskip

Please provide solutions to the problems below, using the following guidelines:
\begin{itemize}
\item Your submitted assignment should be a {\bf good copy} -- figure out the problems first, and then write down organized solutions to each problem. 
\item You should include a {\bf cover page} with the following information: the course number and title, the assignment number, your name, and a list of any resources you used or people you worked with.
\item Since you have plenty of time to work on the problems, assignment solutions will be held to a higher standard than on a test. Your explanations should be clear enough that any of your classmates can understand your solutions.
\item Group work is permitted, but copying is not. If you're not sure what the difference is, feel free to ask. If you get help solving a problem, you should (a) make sure you completely understand the solution, and (b) re-write the solution for your good copy by yourself, in your own words.
\item Assignments can be submitted in class, or in the designated drop box on the 5th floor of University Hall, across from the Math Department office.
\item Late assignments will not be accepted without prior permission.

\end{itemize}
\newpage


\subsection*{Assigned problems}
\begin{enumerate}
 \item Evaluate the integeral below, where $D=\{(x,y) | x^2+y^2\leq 1\}$ is the unit disc. Explain your result.
\[
 \iint_D (x^3e^{x^2}+y^{1/3}\sin(y^4)+3)\,dA.
\]
Hint: as regions go, the unit disc is about as symmetric as they get.

 \item The integral below expresses the integral of a function $f$ over a region $D$ as a sum of two iterated integrals. Sketch the region of integration, and express the integral as a single iterated integral with the order of integration reversed:
\[
 \iint_D f(x,y)\,dA = \int_0^1\int_1^{x^2+1}f(x,y)\,dy\,dx + \int_1^3\int_1^{\frac{1}{4}(x-3)^2+1}f(x,y)\,dy\,dx.
\]
 \item Prove that $\displaystyle 2\int_a^b\int_x^b f(x)f(y)\,dy\,dx = \left(\int_a^b f(x)\,dx\right)^2$.

{\em Hint:} Notice that $\left(\int_a^b f(x)\,dx\right)^2 = \iint_{[a,b]\times [a,b]}f(x)f(y)\,dA$.

 \item Prove the following Mean Value Theorem for double integrals: suppose $D\subseteq \R^2$ is an elementary region (Type 1 or Type 2), and that $f:D\to \R$ is continuous. Then there exists a point $(x_0,y_0)\in D$ such that 
\[
 \iint_Df(x,y)\,dA = f(x_0,y_0)A(D),
\]
where $A(D)$ denotes the area of $D$. \\
(Note that $\frac{1}{A(D)}\iint_D f(x,y)\,dA$ gives the average value of $f$ on $D$.)

You may use the following facts in your proof:
\begin{itemize}
 \item The Extreme Value Theorem holds in $\R^2$: if $D\subseteq \R^2$ is closed and bounded\footnote{A subset of $\R^n$ is {\em closed} if it contains its boundary. It is {\em bounded} if it can be contained within a disk of sufficiently large radius: it doesn't go off to infinity in any direction.} and $f:D\to\R$ is continuous, then there exist $m,M\in\R$ such that $m\leq f(x,y)\leq M$ for all $(x,y)\in D$; moreover, there exist points $(x_1,y_1), (x_2,y_2)\in D$ such that $f(x_1,y_1)=m$ and $f(x_2,y_2)=M$. (That is, $f$ attains its minimum and maximum values on $D$.)

 \item The Intermediate Value Theorem holds in $\R^2$: Suppose $D\subseteq \R^2$ is connected\footnote{A subset $D$ of $\R^2$ is {\em connected} if it is impossible to find two non-intersecting disks that both contain part of $D$. In other words, $D$ cannot be split into two pieces that do not touch. Note that in particular, every elementary region is connected.} and $f:D\to \R$ is continuous. Then if $f(x_1,y_1)=a$ for some $(x_1,y_1)\in D$ and $f(x_2,y_2)=b$ for some $(x_2,y_2)\in D$, and $c$ is any real number between $a$ and $b$, then there exists some point $(x_0,y_0)\in D$ such that $f(x_0,y_0)=c$.
\end{itemize}

\end{enumerate}



\end{document}
 
