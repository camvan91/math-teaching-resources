\documentclass[letterpaper,12pt]{article}

\usepackage{ucs}
\usepackage[utf8x]{inputenc}
\usepackage{amsmath}
\usepackage{amsfonts}
\usepackage{amssymb}
\usepackage[margin=1in]{geometry}

\newcommand{\abs}[1]{\lvert #1\rvert}
\newcommand{\R}{\mathbb{R}}
\newcommand{\C}{\mathbb{C}}
\title{Math 2580 Assignment \#2\\University of Lethbridge, Spring 2016}
\author{Sean Fitzpatrick}
\begin{document}
 \maketitle

{\bf Due date:} Thursday, January 28, by 3 pm.

\bigskip

Please provide solutions to the problems below, using the following guidelines:
\begin{itemize}
\item Your submitted assignment should be a {\bf good copy} -- figure out the problems first, and then write down organized solutions to each problem. 
\item You should include a {\bf cover page} with the following information: the course number and title, the assignment number, your name, and a list of any resources you used or people you worked with.
\item Since you have plenty of time to work on the problems, assignment solutions will be held to a higher standard than on a test. Your explanations should be clear enough that any of your classmates can understand your solutions.
\item Group work is permitted, but copying is not. If you're not sure what the difference is, feel free to ask. If you get help solving a problem, you should (a) make sure you completely understand the solution, and (b) re-write the solution for your good copy by yourself, in your own words.
\item Assignments can be submitted in class, or in the designated drop box on the 5th floor of University Hall, across from the Math Department office.
\item Late assignments will not be accepted without prior permission.

\end{itemize}
\newpage

\subsection*{Terminology}
With functions of several variables there is a hierarchy of ``well-behavedness''. For a function $f:\R^n\to\R^m$, continuity is defined as usual ($\lim_{\mathbf{x}\to\mathbf{a}}f(\mathbf{x})=f(\mathbf{a})$).  In class, we defined what it means for a function to be differentiable in terms of the existence of a linear approximation. 

The desired linear approximation at a point $\mathbf{a}\in\R^n$ is given by $L_{\mathbf{a}}(\mathbf{x}) = A(\mathbf{x}-\mathbf{a})+\mathbf{b}$, where $\mathbf{b}$ is a constant vector (point) in $\R^m$, and $A$ is an $m\times n$ matrix, called the {\bf Jacobian matrix} of $f$ at $\mathbf{a}$. The Jacobian matrix is often viewed as ``the'' derivative in higher dimensions, and it is denoted by
\[
 A = D_{\mathbf{a}}f \quad \text{ or } \quad A = \frac{\partial(f_1,\ldots, f_m)}{\partial (x_1,\ldots, x_n)}.
\]
The Jacobian matrix is an $m\times n$ matrix defined as follows: If $f=(f_1,\ldots, f_m)$, where each $f_i$ is a real-valued function of $x_1,\ldots, x_n$, then
\[
 D_{\mathbf{a}}f = \begin{bmatrix}
                    \dfrac{\partial f_1}{\partial x_1}(\mathbf{a}) & \dfrac{\partial f_1}{\partial x_2}(\mathbf{a}) & \cdots & \dfrac{\partial f_1}{\partial x_n}(\mathbf{a})\\
 & & & \\
                    \dfrac{\partial f_2}{\partial x_1}(\mathbf{a}) & \dfrac{\partial f_2}{\partial x_2}(\mathbf{a}) & \cdots & \dfrac{\partial f_2}{\partial x_n}(\mathbf{a})\\
 & & & \\
 \vdots & \vdots & \ddots & \vdots \\

 & & & \\
                    \dfrac{\partial f_m}{\partial x_1}(\mathbf{a}) & \dfrac{\partial f_m}{\partial x_2}(\mathbf{a}) & \cdots & \dfrac{\partial f_m}{\partial x_n}(\mathbf{a})\\
                   \end{bmatrix}
\]
Our funtion $f$ is then {\bf differentiable} at $\mathbf{a}$ if
\[
 \lim_{\mathbf{x}\to\mathbf{a}}\frac{\lVert f(\mathbf{x}) - L_{\mathbf{a}}(\mathbf{x})\rVert}{\lVert \mathbf{x}-\mathbf{a}\rVert} = 0,
\]
which means that the difference between $f$ and the linear approximation shrinks to zero as $\mathbf{x}$ approaches $\mathbf{a}$ (and that it does so faster than the difference between $\mathbf{x}$ and $\mathbf{a}$).

The following are true (see the handout on differentiability that I posted):
\begin{itemize}
 \item If $f$ is differentiable at $\mathbf{a}$, then $f$ is continuous at $\mathbf{a}$.
 \item If $f$ is differentiable at $\mathbf{a}$, then all partial derivatives of $f$ exist at $\mathbf{a}$.
\end{itemize}
Note however that if $f$ is not differentiable, $f$ might still be continuous, and $f$ might have partial derivatives, but neither of these properties implies the other. Now, while mere existence of partial derivatives at a point doesn't tell us much (it doesn't even guarantee continuity of $f$), it turns out that requiring the partial derivatives to be {\em continuous} is a much stronger requirement.

\pagebreak

\noindent{\bf Definition:} We say that a function $f:D\subseteq \R^n\to \R^m$ is {\bf continuously differentiable} at $\mathbf{a}$ if all partial derivatives of $f$ exist {\bf and} are continuous on an open neighbourhood\footnote{An {\bf open neighbourhood} of a point $\mathbf{a}\in\R^n$ is a set of the form $N_r(\mathbf{a}) = \{\mathbf{x}\in\R^n | \lVert\mathbf{x}-\mathbf{a}\rVert<r\}$.} of $\mathbf{a}$. 

\medskip

It is then a theorem (I'll post a proof on Moodle) that any continuously differentiable function is differentiable. So, while mere existence of partial derivatives is not enough to guarantee a good linear approximation, continuity of those partial derivatives is.

Continuously differentiable functions are sometimes referred to as $C^1$ functions, for brevity. This notation is part of a collection: a $C^0$ function is a function which is simply continuous. A $C^2$ function is one whose {\em second-order} partial derivatives exist and are continuous, a $C^3$ function has continuous third-order partial derivatives, and so on.

\subsection*{Assigned problems}
\begin{enumerate}
 \item Let $r:\R\to \R^3$ be a smooth\footnote{For us, a curve will be {\em smooth} if $r'(t)=\langle u'(t), v'(t), w'(t)\rangle$ exists and is {\bf non-zero} for all $t$.} curve given by $r(t)=(u(t),v(t),w(t))$, and let $f:\R^3\to\R^3$ be a continuously differentiable function given by
\[
 f(u,v,w) = (x(u,v,w), y(u,v,w), z(u,v,w)).
\]
 The composition $s(t) = (f\circ r) (t) = (x(r(t)), y(r(t)), z(r(t)))$ is then another curve in $\R^3$. Using the Chain Rule, show the following:
\begin{enumerate}
 \item If $r'(t)$ exists for all $t$, then $s'(t)$ exists for all $t$.
 
 \item If $\vec{v}$ is tangent to the curve $r(t)$ at a point $\mathbf{u}_0=(u_0,v_0,w_0) = r(t_0)$, then $D_{\mathbf{u}_0}f\vec{v}$ is tangent to the curve $s(t)$ at the point $\mathbf{x_0} = f(u_0,v_0,w_0) = s(t_0)$.

(Hint: If $\vec{v}$ is tangent to the curve $r(t)$ at $r(t_0)$, then it must be a scalar multiple of $r'(t_0)$.)

 \item {\bf Bonus:} In order to say that the curve $s(t)$ is ``smooth'', we would need to also guarantee that $s'(t)$ is never zero. What condition on $D_{\mathbf{x}}f$ will guarantee this? (Hint: if $\vec{v}$ is a non-zero vector, how can you guarantee that $A\vec{v}\neq 0$ for an $m\times n$ matrix $A$?)
\end{enumerate}

\item Let $r(t)=(2\cos(t), 3\sin(t))$ be a curve in the plane, and let $f:\R^2\to \R$ be the function $f(x,y) = x^2-4xy^3$. The curve
\[
 s(t) = (2\cos(t),3\sin(t), f(2\cos(t),3\sin(t)))
\]
is then a curve in $\R^3$ that lies on the surface $z=f(x,y)$.
\begin{enumerate}
 \item Explain why the claim above (that $s(t)$ defines a curve on the surface $z=f(x,y)$) is true.
 \item Show that the tangent vector to $s(t)$ when $t=0$ lies in the tangent plane to the surface $z=f(x,y)$ at the point $(2,0,4)$.
\end{enumerate}
Note: the general case for this example is at the end of Section 15.3 in the Marsden and Weinstein text.


\end{enumerate}



\end{document}
 
