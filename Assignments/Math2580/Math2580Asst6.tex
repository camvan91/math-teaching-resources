\documentclass[letterpaper,12pt]{article}

\usepackage{ucs}
\usepackage[utf8x]{inputenc}
\usepackage{amsmath}
\usepackage{amsfonts}
\usepackage{amssymb}
\usepackage[margin=1in]{geometry}
\newcommand{\di}{\displaystyle}
\newcommand{\abs}[1]{\lvert #1\rvert}
\newcommand{\R}{\mathbb{R}}
\newcommand{\C}{\mathbb{C}}
\newcommand{\F}{\mathbf{F}}
\newcommand{\dotp}{\boldsymbol{\cdot}}
\newcommand{\aaa}{\mathbf{a}}
\newcommand{\bbb}{\mathbf{b}}
\newcommand{\ccc}{\mathbf{c}}
\renewcommand{\r}{\mathbf{r}}

\title{Math 2580 Assignment \#6\\University of Lethbridge, Spring 2016}
\author{Sean Fitzpatrick}
\begin{document}
 \maketitle

{\bf Due date:} Thursday, March 31st, by 3 pm.

\bigskip

Please provide solutions to the problems below, using the following guidelines:
\begin{itemize}
\item Your submitted assignment should be a {\bf good copy} -- figure out the problems first, and then write down organized solutions to each problem. 
\item You should include a {\bf cover page} with the following information: the course number and title, the assignment number, your name, and a list of any resources you used or people you worked with.
\item Since you have plenty of time to work on the problems, assignment solutions will be held to a higher standard than on a test. Your explanations should be clear enough that any of your classmates can understand your solutions.
\item Group work is permitted, but copying is not. If you're not sure what the difference is, feel free to ask. If you get help solving a problem, you should (a) make sure you completely understand the solution, and (b) re-write the solution for your good copy by yourself, in your own words.
\item Assignments can be submitted in class, or in the designated drop box on the 5th floor of University Hall, across from the Math Department office.
\item Late assignments will not be accepted without prior permission.

\end{itemize}
\newpage


\subsection*{Assigned problems}
\begin{enumerate}
\item The \textit{First Theorem of Pappus} states that for any plane curve $C$, the surface area of the surface of revolution generated by revolving $C$ about an axis that does not intersect $C$ is $A = sc$, where $s$ is the length of $C$, and $c$ is the circumference of the circle generated by revolving the \textbf{centroid} of $C$ about the given axis.

Here, the centroid of a curve $C$ is the point $(\overline{x},\overline{y})$ given by
\[
 \overline{x} = \frac{1}{s}\int_C x\,ds \quad \text{ and } \quad \overline{y} = \frac{1}{s}\int_C y\,ds; \quad s = \int_C\,ds.
\]
Use the First Theorem of Pappus to find the surface area of the surface of revolution generated by revolving one arch of the cycloid $x=R(t-\sin t)$, $y=R(1-\cos t)$ about the $x$-axis.

\textit{Hint:} An arch of the cycloid begins and ends when $y(t)=0$. You will probably need the trig identity $1-\cos t = 2\sin^2(t/2)$ and a bit of perserverance (or Wolfram Alpha) to get through the integration. 

\textit{Hint \#2:} You can probably guess the value of $\overline{x}$ based on the symmetry of the cycloid.


\item Suppose $\F$ is a vector field such that $\lVert \F(x,y,z)\rVert\leq M$ for all $(x,y,z)\in\R^3$, and $C$ is a curve with length $L$. Show that
\[
 \left\lvert \int_C \F\dotp d\mathbf{r}\right\rvert \leq ML.
\]


\item  Let $\aaa$, $\bbb$, and $\ccc$ be any three constant, linearly independent vectors. (Linearly independent means that none of the vectors lies in the plane spanned by the other two, or equivalently, that none of the vectors can be written as a linear combination of the other two.) Let $\r = \langle x,y,z\rangle$, and let $E$ be the region in $\R^3$ given by the inequalities
\[
0\leq \aaa\dotp\r\leq \alpha,\quad 0\leq \bbb\dotp\r\leq \beta,\quad 0\leq \ccc\dotp\r\leq \gamma,
\]
where $\alpha,\, \beta$ and $\gamma$ are positive constants. Show that
\[
\iiint_E (\aaa\dotp\r)(\bbb\dotp\r)(\ccc\dotp\r)\,dV = \frac{(\alpha\beta\gamma)^2}{8\abs{\aaa\dotp(\bbb\times\ccc)}}.
\]
{\em Hint:} Note that $\aaa\dotp\r=0$ and $\aaa\dotp\r=\alpha$, etc. are equations of planes. Conclude that $E$ is a parallelepiped, and come up with an appropriate change of variables.

\end{enumerate}



\end{document}
 
