\documentclass[letterpaper,12pt]{article}

\usepackage{ucs}
\usepackage[utf8x]{inputenc}
\usepackage{amsmath}
%\usepackage{amsfonts}
%\usepackage{amssymb}
\usepackage[margin=1in]{geometry}

\usepackage[bitstream-charter]{mathdesign}
\usepackage[T1]{fontenc}

\newcommand{\len}[1]{\lVert #1\rVert}
\newcommand{\abs}[1]{\lvert #1\rvert}
\newcommand{\R}{\mathbb{R}}
\title{Math 1410 Assignment \#1\\University of Lethbridge, Spring 2017}
\author{Sean Fitzpatrick}
\begin{document}
 \maketitle

{\bf Due date:} {\bf Thursday}, January 26th, by 4 pm.

\bigskip

Please review the {\bf Guidelines for preparing your assignments} before submitting your work. You can find these guidelines, along with the required cover page, in the Assignments section on our Moodle site.



\subsection*{Assigned problems}
\begin{enumerate}
\item Prove the \textit{distributive property} for complex arithmetic. That is, prove that for any complex numbers $u, v, w$, we have
\[
 u(v+w) = uv+uw.
\]

\medskip

\textit{Reminder:} The phrase ``for any'' tells you that simply providing an example is not acceptable. You need to give a general argument that does not depend on any particular choices of values for your complex numbers.

\bigskip

\item Recall that the complex conjugate of $z\in\mathbb{C}$ is denoted by $\overline{z}$, and the modulus of $z$ is denoted by $\abs{z}$. Show that:
\begin{enumerate}
 \item $\abs{\overline{z}} = \abs{z}$
 \item $\abs{z} = \sqrt{z\overline{z}}$
 \item $\operatorname{Re}(z) = \dfrac{z+\overline{z}}{2}$ and $\operatorname{Im}(z) = \dfrac{z-\overline{z}}{2i}$, where $\operatorname{Re}(z)$ and $\operatorname{Im}(z)$ denote the real and imaginary parts of $z$, respectively.
\end{enumerate}

\item Convert $z=-1+\sqrt{3}i$ to polar form, and compute the value of $z^6 = (-1+\sqrt{3}i)^6$. Express your answer in rectangular form.

\item  Let $\vec{v} = \langle 3, -1, 4\rangle$ and $\vec{w} = \langle -2, 5, 1\rangle$ be two vectors in $\R^3$. Find the coordinates of:
\begin{enumerate}
 \item The point $P$, one half of the way from the tip of $\vec{v}$ to the tip of $\vec{w}$.
 \item The point $Q$, one third of the way from the tip of $\vec{v}+\vec{w}$ to the tip of $\vec{v}-\vec{w}$.
\end{enumerate}
(Assume all vectors are drawn with their tails at the origin.)
\end{enumerate}

\end{document}
 
