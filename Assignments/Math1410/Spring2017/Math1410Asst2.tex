\documentclass[letterpaper,12pt]{article}

\usepackage{ucs}
\usepackage[utf8x]{inputenc}
\usepackage{amsmath}
%\usepackage{amsfonts}
%\usepackage{amssymb}
\usepackage[margin=1in]{geometry}

\usepackage[bitstream-charter]{mathdesign}
\usepackage[T1]{fontenc}

\newcommand{\len}[1]{\lVert #1\rVert}
\newcommand{\abs}[1]{\lvert #1\rvert}
\newcommand{\R}{\mathbb{R}}
\newcommand{\dotp}{\boldsymbol{\cdot}}
\title{Math 1410 Assignment \#2\\University of Lethbridge, Spring 2017}
\author{Sean Fitzpatrick}
\begin{document}
 \maketitle

{\bf Due date:} {\bf Tuesday}, February 14th, by 4 pm.

\bigskip

Please review the {\bf Guidelines for preparing your assignments} before submitting your work. You can find these guidelines, along with the required cover page, in the Assignments section on our Moodle site.



\subsection*{Assigned problems}
\begin{enumerate}
\item Consider the triangle $\Delta PQR$ with vertices $P=(2,0,-3)$, $Q=(5,-2,1)$, and $R=(7,5,3)$.
\begin{enumerate}
 \item Show that $\Delta PQR$ is a right-angled triangle. (Hint: this is a question about dot products.)
 \item Compute the lengths of the three sides of $\Delta PQR$ and verify that the Pythagorean Theorem holds.
 \item Determine the equation of the plane containing $\Delta PQR$.
\end{enumerate}
\item Let $\vec{u}$ and $\vec{v}$ be any two vectors in $\R^3$.
\begin{enumerate}
 \item Show that $\len{\vec{u}+\vec{v}}^2+\len{\vec{u}-\vec{v}}^2 = 2(\len{\vec{u}}^2+\len{\vec{v}}^2)$.
 \item What does part (b) tell you about parallelograms?
\end{enumerate}
\noindent\textit{Note:} For parts (a) and (b) in Problem 2, it is \textbf{not} necessary (nor desirable) to write out $\vec{u}$ and $\vec{v}$ in terms of their components. Instead, work with the properties of the dot product given in Section 3.3 of the textbook.

\item  Let $\ell$ be a line through the origin in $\R^3$, and let $\vec{p}$ and $\vec{q}$ be the position vectors for any two points on $\ell$.
\begin{enumerate}
 \item Show that the point with position vector $\vec{p}+\vec{q}$ also lies on the line $\ell$.
 \item Show that for any scalar $c$, the point with position vector $c\vec{p}$ also lies on the line $\ell$.
 \item Repeat parts (a) and (b) with $\ell$ replaced by a \textit{plane} through the origin.
\end{enumerate}
\end{enumerate}

\end{document}
 
