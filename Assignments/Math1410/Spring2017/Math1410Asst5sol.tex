\documentclass[letterpaper,12pt]{article}

\usepackage{ucs}
\usepackage[utf8x]{inputenc}
\usepackage{amsmath}
%\usepackage{amsfonts}
%\usepackage{amssymb}
\usepackage[margin=1in]{geometry}

\usepackage[bitstream-charter]{mathdesign}
\usepackage[T1]{fontenc}

\newcommand{\len}[1]{\lVert #1\rVert}
\newcommand{\abs}[1]{\lvert #1\rvert}
\newcommand{\R}{\mathbb{R}}
\newcommand{\dotp}{\boldsymbol{\cdot}}
\newcommand{\bbm}{\begin{bmatrix}}
\newcommand{\ebm}{\end{bmatrix}}
\DeclareMathOperator{\proj}{proj}

\title{Math 1410 Assignment \#5 Solutions\\University of Lethbridge, Spring 2017}
\author{Sean Fitzpatrick}
\begin{document}
 \maketitle


\begin{enumerate}
\item Suppose $A$ and $B$ are $4\times 4$ matrices such that $\det(A) = 3$ and $\det(B)=-4$. Determine the values of
\begin{enumerate}
 \item $\det(A^2B) = (\det(A))^2(\det(B))=3^2(-4) = -36.$
 \item $\det(B^TBAB^{-1}) = \det(B^T)\det(B)\det(A)\det(B^{-1}) = \det(B)\det(B)\det(A)\left(\dfrac{1}{\det{B}}\right)=\det(B)\det(A) = -12.$
 \item $\det(2AB^{-1})=2^4\det(AB^{-1}) = 2^4\det(A)\det(B^{-1}) = 2^4\left(\dfrac{\det(A)}{\det(B)}\right) = 16\left(\dfrac{3}{-4}\right)=-12.$
\end{enumerate}

\medskip

\item Suppose $\det(AB)=0$. Must it be the case that $\det(A)=0$ or $\det(B)=0$? Prove this, or give a counterexample.

\bigskip

Yes. If this were not the case, then we'd have $\det(A)\neq 0$ and $\det(B)\neq 0$, in which case we'd know that both $A$ and $B$ are invertible. However, we know that if $A$ and $B$ are invertible, then so is $AB$, which would imply that $\det(AB)\neq 0$, contradicting our assumption that $\det(AB)=0$.

\medskip

\item We say that an $n\times n$ matrix $B$ is \textbf{similar} to an $n\times n$ matrix $A$ if $B=P^{-1}AP$ for some invertible matrix $P$, and write $B\sim A$.
\begin{enumerate}
 \item Show that if $B\sim A$, then $\operatorname{tr}(B)=\operatorname{tr}(A)$.

\medskip

Recall that $\operatorname{tr}(XY) = \operatorname{tr}(YX)$ for any $n\times n$ matrices $X$ and $Y$. If $B\sim A$, then $B=P^{-1}AP$ for some invertible matrix $P$. Therefore, (with $X=P^{-1}$ and $Y=AP$) we have
\[
 \operatorname{tr}(B) = \operatorname{tr}(P^{-1}AP) = \operatorname{tr}(APP^{-1}) = \operatorname{tr}(AI) = \operatorname{tr}(A).
\]

 \item Show that if $B\sim A$, then $\det(B)=\det(A)$.

We know that $\det(XY) = \det(X)\det(Y)$ for any $n\times n$ matrices $X$ and $Y$. If $B\sim A$, then $B=P^{-1}AP$ for some invertible matrix $P$, and
\[
 \det(B) = \det(P^{-1}AP) = \det(P^{-1})\det(A)\det(P) = \left(\frac{1}{\det(P)}\right)\det(A)\det(P) = \det(A).
\]

\medskip

 \item Suppose $A$ is similar to a matrix $D = \bbm x&0&0\\0&x&0\\0&0&y\ebm$, and we know that $\operatorname{tr}(A)=0$, and $\det(A)=16$. What are the values of $x$ and $y$?

\medskip

By parts (a) and (b), we know that $\operatorname{tr}(A) = \operatorname{tr}(D)$, and by direct computation we have $\operatorname{tr}(D) = x+x+y = 2x+y$. Thus, $2x+y = 0$. Similarly, $x^2y = \det(D) = \det(A) = 16$, so we have two equations:
\[
 2x+y = 0 \quad \text{ and } \quad x^2y=16.
\]
From the first equation we have $y=-2x$; substituting this into the second, we have $x^2y = -2x^3 = 16$, so $x^3=-8$, giving us $x=-2$. Since $y=-2x$, we have $y=4$.

\end{enumerate}

\medskip


\item Let $\operatorname{adj}(A)$ denote the adjugate matrix of an $n\times n$ matrix $A$. Show that 
\[\det(\operatorname{adj}(A)) = (\det(A))^{n-1}.\]

\medskip

We know from the proof of the adjugate formula for the inverse that $A(\operatorname{adj}(A)) = \det(A)I_n$. This is true for any $n\times n$ matrix $A$, but let us first assume that $A$ is invertible, so that $\det(A)\neq 0$. In this case, taking the determinant of both sides of the equation $A\operatorname{adj}(A)=\det(A)I_n$ gives us
\[
 \det(A)\det(\operatorname{adj}(A)) = \det(\det(A)I_n)=(\det(A))^n.
\]
(On the right-hand side, $\det(A)I_n$ is diagonal, with each diagonal entry equal to $\det(A)$, and the determinant is given by multiplying the diagonal entries.)

Assuming that $\det(A)\neq 0$, we divide both sides by $\det(A)$, giving $\det(\operatorname{adj}(A))=\det(A)^{n-1}$, as required.

Now, what if $\det(A)=0$? We claim that we must have $\det(\operatorname{adj}(A))=0$ as well. Notice that if $\det(A)=0$, we have $A\operatorname{adj}(A)=\det(A)I_n = \boldsymbol{0}_n$, since $\det(A)=0$. If $\det(\operatorname{adj}(A))\neq 0$, then $\operatorname{adj}(A)$ is invertible, and we would have
\[
 A = (A\operatorname{adj}(A))(\operatorname{adj}(A))^{-1} = \boldsymbol{0}_n(\operatorname{adj}(A))^{-1})=\boldsymbol{0}_n.
\]
But this is impossible, because if $A=\boldsymbol{0}_n$, then $\operatorname{adj}(A)=\boldsymbol{0}_n$ as well, and the zero matrix is not invertible.

Thus, if $\det(A)=0$, then $A$ is not invertible and then neither is $\operatorname{adj}(A)$, giving us 
\[\det(\operatorname{adj}(A)) = 0 = 0^{n-1}=\det(A)^{n-1}.\]

\end{enumerate}

\end{document}
 
