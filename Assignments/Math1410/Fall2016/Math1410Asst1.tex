\documentclass[letterpaper,12pt]{amsart}

\usepackage{ucs}
\usepackage[utf8x]{inputenc}
\usepackage{amsmath}
%\usepackage{amsfonts}
%\usepackage{amssymb}
\usepackage[margin=1in]{geometry}

\usepackage[bitstream-charter]{mathdesign}
\usepackage[T1]{fontenc}

\newcommand{\len}[1]{\lVert #1\rVert}
\newcommand{\abs}[1]{\lvert #1\rvert}
\newcommand{\R}{\mathbb{R}}
\title{Math 1410 Assignment \#1\\University of Lethbridge, Fall 2016}
\author{Sean Fitzpatrick}
\begin{document}
 \maketitle

{\bf Due date:} {\bf Thursday}, September 29th, by 5 pm.

\bigskip

Please review the {\bf Guidelines for preparing your assignments} before submitting your work. You can find these guidelines, along with the required cover page, in the Assignments section on our Moodle site.



\subsection*{Assigned problems}
\begin{enumerate}
\item Show that for \textbf{any} complex numbers $z$ and $w$, with $w\neq 0$, the complex modulus satisfies
\[
 \left\lvert\frac{z}{w}\right\rvert = \frac{\abs{z}}{\abs{w}}.
\]

\medskip

\textit{Hint:} See page 39 of the textbook for an outline of the argument you should use. On this same page you'll find the proof that $\abs{zw}=\abs{z}\cdot\abs{w}$, which might serve as a useful model for your own proof.

\bigskip

\item Compute the following complex roots, and plot them in the complex plane. \\(See Example 14 on page 47 of the text for guidance.)

\medskip

\begin{enumerate}
 \item Find the three cube roots of $z=-125$.
 \item Find the six 6th roots of $z=64$.

\end{enumerate}

\medskip

\textit{Note:} The answers for 2(b) are in the back of the book. I'm more interested in seeing you work through the process than I am in the final answers.

\bigskip

\item Let $\vec{v}$ and $\vec{w}$ be vectors in $\R^3$. In each case, either explain why the statement is true (in general), or give an example showing that it is false:

\medskip

\begin{enumerate}
 \item If $\len{\vec{v}-\vec{w}}=0$, then $\vec{v}=\vec{w}$.
 \item If $\vec{v}=-\vec{v}$, then $\vec{v}=\vec{0}$.
 \item If $\len{\vec{v}}=\len{\vec{w}}$, then $\vec{v}=\vec{w}$.
 \item If $\len{\vec{v}}=\len{\vec{w}}$, then $\vec{v}=\pm\vec{w}$.
 \item $\len{\vec{v}+\vec{w}} = \len{\vec{v}}+\len{\vec{w}}$.
\end{enumerate}

\bigskip

\item Consider the triangle in $\R^3$ with vertices (corners) at the points 
\[
P=(2,0,-3),\, Q=(5,-2,1),\, \text{ and } R=(7,5,3). 
\]
 Show that this is a right-angled triangle

\medskip

\begin{enumerate}
 \item Using dot products.
 \item Using the Pythagorean Theorem.
\end{enumerate}
\end{enumerate}

\end{document}
 
