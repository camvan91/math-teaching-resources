\documentclass[letterpaper,12pt]{amsart}

\usepackage{ucs}
\usepackage[utf8x]{inputenc}
\usepackage{amsmath}
%\usepackage{amsfonts}
%\usepackage{amssymb}
\usepackage[margin=1in]{geometry}
\usepackage{multicol}
\usepackage[bitstream-charter]{mathdesign}
\usepackage[T1]{fontenc}

\newcommand{\len}[1]{\lVert #1\rVert}
\newcommand{\abs}[1]{\lvert #1\rvert}
\newcommand{\R}{\mathbb{R}}
\newcommand{\bbm}{\begin{bmatrix}}
\newcommand{\ebm}{\end{bmatrix}}
                   
\title{Math 1410 Assignment \#4\\University of Lethbridge, Fall 2016}
\author{Sean Fitzpatrick}
\begin{document}
 \maketitle

{\bf Due date:} {\bf Thursday}, November 17th, by 4:30 pm.

\bigskip

Please review the {\bf Guidelines for preparing your assignments} before submitting your work. You can find these guidelines, along with the required cover page, in the Assignments section on our Moodle site.



\subsection*{Assigned problems}
\begin{enumerate}
\item Determine the null space and column space of the matrix
\[
 A = \bbm 1&-2&-1&3\\2&-4&1&0\\1&-2&2&-3\ebm
\]

\bigskip


\item Prove that if a system of linear equations has more than one solution, then it has\\infinitely many solutions.

\bigskip

\item For each statement below, either demonstrate that it is true, or give an example showing that it is false.

\medskip

{\bf Note:} For full marks, give your answers for five of the six statements below. For a small bonus, do all six.

\medskip

\begin{enumerate}
 \item For any $n\times n$ matrices $A$, $B$, and $C$, if $AB=AC$ and $A$ is invertible, then $B=C$.

\medskip

 \item If $A$ is an $n\times n$ matrix and $A\neq 0$, then $A$ is invertible.

\medskip

 \item If $A$ and $B$ are invertible $n\times n$ matrices, then $A+B$ is invertible.

\medskip

 \item If $A$ is an $n\times n$ matrix such that $A^2=A$ and $A\neq 0$, then $A$ is invertible. (Hint: your previous assignment provides examples of such matrices.)

\medskip

 \item If $A^4=I$, where $I$ is the $n\times n$ identity matrix, then $A$ is invertible.

\medskip

 \item If $A$ is an $n\times n$ matrix and $A^2$ is invertible, then $A$ is invertible.
\end{enumerate}

\bigskip

\bigskip

\item Consider the matrix $A = \bbm 2&-1\\0&3\ebm$.

\medskip

\begin{enumerate}
 \item Show that $A^2-5A+6I=0$.

\medskip

 \item Use part (a) to show that $A^{-1} = \dfrac{1}{6}(5I-A)$.
\end{enumerate}
\end{enumerate}

\bigskip

{\bf See over} for hints and suggestions.

\newpage

\begin{enumerate}
 \item Null space and column space are defined in Section 4.5 of the textbook. The methods for computing them are given in Section 6.1.\\
(Note also that there are similar problems in Online Homework \#8, and as always, I'm quite happy to show you how to do these during office hours.)

\bigskip

 \item The easiest way to do this is to start with a system written in the matrix form $A\vec{x}=\vec{b}$. If there is more than one solution to the system, then there are at least two solutions $\vec{x}_1$ and $\vec{x}_2$, with $\vec{x}_1\neq \vec{x}_2$. Now make use of two things: 1. What did we learn about $\vec{x}_1-\vec{x}_2$ in problem 3(c) of Assignment \#3? 2. The discussion on pages 241 and 242 of the textbook leading up to Key Idea 20.

\bigskip

 \item For statements that are false, you should be able to come up with a simple $2\times 2$ example to demonstrate this. In cases where you want to demonstrate that a matrix $A$ is invertible, recall that if you can find {\em any} matrix $B$ such that $AB=I$ (where $A$ and $B$ are both $n\times n$) then you can immediately conclude that $A$ is invertible, and that $B=A^{-1}$. This remark also applies to solving part (b) of Problem 4.

\end{enumerate}



\end{document}
 
