\documentclass[letterpaper,12pt]{article}

\usepackage{ucs}
\usepackage[utf8x]{inputenc}
\usepackage{amsmath}
\usepackage{amsfonts}
\usepackage{amssymb}
\usepackage[margin=1in]{geometry}
\usepackage{enumerate}

\newcommand{\abs}[1]{\lvert #1\rvert}
\newcommand{\len}[1]{\lVert #1\rVert}
\newcommand{\R}{\mathbb{R}}
\newcommand{\x}{\mathbf{x}}
\newcommand{\y}{\mathbf{y}}
\newcommand{\aaa}{\mathbf{a}}
\newcommand{\bbb}{\mathbf{b}}
\newcommand{\inter}[1]{\overset{\,\,\circ}{#1}}
\newcommand{\N}{\mathbb{N}}

\title{Math 4310 Assignment \#3 Solutions\\University of Lethbridge, Fall 2014}
\author{Sean Fitzpatrick}
\begin{document}
 \maketitle

\begin{enumerate}
\item For each pair of points $a,b\in\mathbb{R}^n$, prove that there is a topological equivalence $f:\R^n\to \R^n$ between $(\R^n,d)$ and itself (where $d$ is the Euclidean metric -- or whatever your favourite metric equivalent to the Euclidean metric happens to be) such that $f(a)=b$.

{\em Hint}: For $n=1$ you should be able to show that $f(x)=x+b-a$ does the job. For general $n\in\mathbb{N}$ you'll want to come up with something similar.

\bigskip

{\bf Solution}: Let $\aaa=(a_1,\ldots, a_n)$ and $\bbb=(b_1,\ldots, b_n)$ be the given points, and define $f:\R^n\to\R^n$ by
\[
f(x_1,\ldots, x_n) = (x_1+b_1-a_1,\ldots, x_n+b_n-a_n).
\]
Note that if we identify points in $\R^n$ (viewed as a metric space) with vectors (viewing $\R^n$ as a vector space with the standard basis) then $f$ is given simply by $f(\x) = \x+\bbb-\aaa$. Then we have $f(\aaa)=\bbb$, and $f$ is invertible, with inverse map $f^{-1}(\x) = \x-\bbb+\aaa$, so that $f$ must be a bijection. (It's clear that $f^{-1}$ is a well-defined function, so $f$ must be one-to-one, and $f$ is onto, since for any $\y\in\R^n$ we have $f(f^{-1}(\y))=\y$.)

Since $f$ and $f^{-1}$ are simply translations, it should be clear that they are continuous; to check this, note that for any $\epsilon>0$, taking $\delta=\epsilon$ gives
\[
\len{(\x-\bbb+\aaa)-(\y-\bbb+\aaa)} = \len{\x-\y}<\delta=\epsilon,
\]
so that $f$ is continuous, and an identical proof shows that $f^{-1}$ is continuous.

\item Let $X$ be the open interval $(-\pi/2,\pi/2)$, considered as a subspace of $\R$, with the usual metric. Prove that $X$ is topologically equivalent to all of $\R$ itself. (You might have to think back to your intro calculus material to come up with a function that gives a bijection between $X$ and $\R$.) Next, prove that any two open intervals, viewed as subspaces of $\R$ are topologically equivalent. Conclude that $\R$ is topologically equivalent to any open subinterval.

\bigskip

{\bf Solution}: Note that viewing any open interval as a subspace of the metric space $\R$ simply means that we keep the usual distance function on these spaces, so that continuity means the usual notion of continuity from real analysis. Consider the function $\tan:X=(-\pi/2,\pi/2)\to \R$. From basic calculus, we know that $\tan$ is one-to-one on $X$, has range $\R$, and is continuous. Moreover, its inverse is given by the continuous function $\arctan:\R\to X$, so that $X$ and $\R$ are homeomorphic. Given any two intervals $(a,b)$ and $(c,d)$, we can take the linear function
\[
f(x) = \left(\frac{d-c}{b-a}\right)(x-a)+c.
\]
We know that any linear (nonconstant) function is one-to-one and continuous, and we have $f(a)=c$ and $f(d)=b$, which tells us that $f$ is onto the interval $(c,d)$. The inverse of $f$ is $f^{-1}(x) = \left(\dfrac{b-a}{d-c}\right)(x-c)+a$, which is also a linear, and hence continuous, function. Taking one of the two intervals to be $X$, we conclude that any open interval is topologically equivalent to $\R$, since we can compose the map $(a,b)\to X$ with the map $\tan:X\to\R$. (Here we are using without proof the fact that topological equivalence is an equivalence relation, which was one of the textbook exercises.)

Finally, if we want to be thorough, we can note that the interval $(0,\infty)$ is homeomorphic to $\R$ via the homeomorphism given by the natural logarithm (with inverse the exponential function), and any interval of the form $(a,\infty)$ or $(-\infty, a)$ can be identified with $(0,\infty)$ by a simple translation and/or reflection (i.e. $x\mapsto x-a$ or $x\mapsto a-x)$.

\item Let $(Y,d')$ be a subspace of $(X,d)$. Let $(a_n)$ be a sequence of points in $Y$ such that $a_n\to a$ in $Y$ (i.e. $(a_n)$ converges in $Y$ to $a\in Y$.) Prove that $a_n\to a$ in $X$ as well.

{\em Hint}: recall that $d' = d|_{Y\times Y}$ for a subspace. Your proof shouldn't be very long. (Note however that the converse of this result is not true: for example, any sequence of rational numbers that converges to $\sqrt{2}$ cannot be considered convergent as a sequence in the metric space $\mathbb{Q}$.)

\bigskip

{\bf Solution}: Given such a sequence $(a_n)$ of points in $Y$ with limit $a\in Y$, we note that since $Y\subseteq X$, we can view the sequence as a sequence in $X$, and the metric remains unchanged, since $d'$ is simply the restriction to $Y$ of the metric $d$ on $X$. Since $(a_n)\to a$ in $Y$, given any $\epsilon>0$ we can find an $N\in\N$ such that for any $n\geq \N$ we have
\[
d(a_n,a) = d'(a_n,a) <\epsilon.
\]
Thus $a_n\to a$ in $X$.

\item Prove that if $f:X\to Y$ is a continuous map between metric spaces, and $(x_n)$ converges to $x$ in $X$, then $(f(x_n))$ converges to $f(x)$ in $Y$.

\bigskip

{\bf Solution}: Let $\epsilon>0$ be given. Since $f$ is continuous, there exists a $\delta>0$ such that whenever $d_X(x,a)<\delta$, we have $d_Y(f(x),f(a))<\epsilon$. Since $a_n\to a$ in $X$, there exists some $N\in\N$ such that whenever $n\geq N$, we have $d_X(x_n,x)<\delta$. Thus, since $f$ is continuous, whenever $n\geq N$, it follows that $d_Y(f(x_n),f(x))<\epsilon$.

\item Prove that any subset of a discrete topological space is simultaneously open and closed.

\bigskip

{\bf Solution}: By definition, every subset of a space $X$ with the discrete topology is open. Since $X\setminus A$ is also a subset of $X$ for any subset $A\subseteq X$, we know that $X\setminus A$ is open, and thus $A$ is closed.

\end{enumerate}
\end{document}
 
