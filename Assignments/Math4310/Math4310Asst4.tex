\documentclass[letterpaper,12pt]{article}

\usepackage{ucs}
\usepackage[utf8x]{inputenc}
\usepackage{amsmath}
\usepackage{amsfonts}
\usepackage{amssymb}
\usepackage[margin=1in]{geometry}
\usepackage{enumerate}

\newcommand{\abs}[1]{\lvert #1\rvert}
\newcommand{\len}[1]{\lVert #1\rVert}
\newcommand{\R}{\mathbb{R}}
\newcommand{\x}{\mathbf{x}}
\newcommand{\y}{\mathbf{y}}
\newcommand{\inter}[1]{\overset{\,\,\circ}{#1}}
\newcommand{\T}{\mathcal{T}}
\DeclareMathOperator{\Int}{Int}

\title{Math 4310 Assignment \#4\\University of Lethbridge, Fall 2014}
\author{Sean Fitzpatrick}
\begin{document}
 \maketitle

{\bf Due date:} Friday, October 3rd, by 5 pm.

\bigskip

Please submit solutions to the following problems (except where indicated). As usual, all the textbook exercises are recommended as practice problems if they don't appear as an assigned problem below.

\begin{enumerate}
\item For any subset $A$ of a topological space $(X,\T)$, prove that the sets $\Int(A)$, $\partial A$ and $\Int(A^c)$ are pairwise disjoint, and that $X = \Int(A)\cup \partial A\cup \Int(A^c)$. (Here, $\Int(A)$ denotes the interior of $A$, and $A^c$ denotes the complement of $A$.)

\item Let $X$ be an infinite set, and equip $X$ with the finite complement topology. (So that $A$ is open in $X$ if and only if $A^c$ is finite.) For a given set $A\subseteq X$, what is the closure $\overline{A}$ (a) when $A$ is finite, and (b) when $A$ is infinite?

\item Let $X=\R^n$ with the Euclidean topology, and let $B^n=\{(x_1,\ldots, x_n)\in\R^n\,|\, x_1^2+\cdots +x_n^2\leq 1\}$ be the unit ball centred at the origin. Prove that the boundary of $B$ is the sphere $S^{n-1}$, defined as the set of points in $\R^n$ such that $x_1^2+\cdots + \x_n^2 = 1$.

\item Let $U\subset X$ be an {\em open} subset of a topological space $(X,\T)$. Prove that a subset $V\subseteq U$ is open in $U$ (in the subspace topology) if and only if $V$ is open in $X$.

\item Let $Y$ be a subspace of $X$ and let $A$ be a subset of $Y$. Let $\Int_X(A)$ denote the interior of $A$, viewed as a subset of $X$, and let $\Int_Y(A)$ denote the interior of $A$ with respect to the subspace topology on $Y$. Prove that $\Int_X(A)\subseteq \Int_Y(A)$, and give an example where this inclusion is a proper inclusion.

\item In a topological space, each of the terms {\em open set, closed set, neighbourhood, closure of a set,  interior of a set, boundary of a set} may be characterized by any other one of these terms. Construct a table containing the thirty such possible definitions or theorems in which, for example, the entry in the row labelled interior and the colulmn labelled open set is the characterization of interior in terms of open sets (i.e., that the interior of $A$ is the union of all open sets contained in $A$), and so on.

{\bf Note}: I don't seriously expect you to submit that last problem. In fact, please don't. But it is something you might want to do for yourself.

\end{enumerate}
\end{document}
 
