\documentclass[letterpaper,12pt]{article}

\usepackage{ucs}
\usepackage[utf8x]{inputenc}
\usepackage{amsmath}
\usepackage{amsfonts}
\usepackage{amssymb}
\usepackage[margin=1in]{geometry}
\usepackage{enumerate}

\newcommand{\abs}[1]{\lvert #1\rvert}
\newcommand{\len}[1]{\lVert #1\rVert}
\newcommand{\R}{\mathbb{R}}
\newcommand{\x}{\mathbf{x}}
\newcommand{\y}{\mathbf{y}}
\newcommand{\inter}[1]{\overset{\,\,\circ}{#1}}

\title{Math 4310 Assignment \#3\\University of Lethbridge, Fall 2014}
\author{Sean Fitzpatrick}
\begin{document}
 \maketitle

{\bf Due date:} Friday, September 26, by 5 pm.

\bigskip

Please submit solutions to the following problems (except where indicated). As usual, all the textbook exercises are recommended as practice problems if they don't appear as an assigned problem below.

\begin{enumerate}
\item For each pair of points $a,b\in\mathbb{R}^n$, prove that there is a topological equivalence $f:\R^n\to \R^n$ between $(\R^n,d)$ and itself (where $d$ is the Euclidean metric -- or whatever your favourite metric equivalent to the Euclidean metric happens to be) such that $f(a)=b$.

{\em Hint}: For $n=1$ you should be able to show that $f(x)=x+b-a$ does the job. For general $n\in\mathbb{N}$ you'll want to come up with something similar.

\item Let $X$ be the open interval $(-\pi/2,\pi/2)$, considered as a subspace of $\R$, with the usual metric. Prove that $X$ is topologically equivalent to all of $\R$ itself. (You might have to think back to your intro calculus material to come up with a function that gives a bijection between $X$ and $\R$.) Next, prove that any two open intervals, viewed as subspaces of $\R$ are topologically equivalent. Conclude that $\R$ is topologically equivalent to any open subinterval.

\item Let $(Y,d')$ be a subspace of $(X,d)$. Let $(a_n)$ be a sequence of points in $Y$ such that $a_n\to a$ in $Y$ (i.e. $(a_n)$ converges in $Y$ to $a\in Y$.) Prove that $a_n\to a$ in $X$ as well.

{\em Hint}: recall that $d' = d|_{Y\times Y}$ for a subspace. Your proof shouldn't be very long. (Note however that the converse of this result is not true: for example, any sequence of rational numbers that converges to $\sqrt{2}$ cannot be considered convergent as a sequence in the metric space $\mathbb{Q}$.)
\item Prove that if $f:X\to Y$ is a continuous map between metric spaces, and $(x_n)$ converges to $x$ in $X$, then $(f(x_n))$ converges to $f(x)$ in $Y$.

\item Prove that any subset of a discrete topological space is simultaneously open and closed.


\end{enumerate}
\end{document}
 
