\documentclass[letterpaper,12pt]{article}

\usepackage{ucs}
\usepackage[utf8x]{inputenc}
\usepackage{amsmath}
\usepackage{amsfonts}
\usepackage{amssymb}
\usepackage{amsthm}
\usepackage[margin=1in]{geometry}

%\theoremstyle{theorem}
\newtheorem{theorem}{Theorem}
\newtheorem{lemma}{Lemma}
\newtheorem{problem}{Problem}
\theoremstyle{definition}
\newtheorem{definition}{Definition}
\newcommand{\abs}[1]{\lvert #1\rvert}
\newcommand{\R}{\mathbb{R}}
\newcommand{\C}{\mathbb{C}}
\newcommand{\N}{\mathbb{N}}

\title{Math 2000 Writing Assignment \#1}
\author{Sean Fitzpatrick}
\begin{document}
 \maketitle

{\bf Note: My original assignment instructions have gone missing. This is a summary of the questions being answered in the model solution.}

Recall that an integer greater than 1 is prime if its only positive divisors are 1 and itself. A positive integer greater than 1 is called composite if it is not prime. By a proper divisor, we mean a positive divisor that is not equal to the integer itself. A positive integer is said to be {\bf perfect} if it is the sum of its proper divisors.

\begin{enumerate}
 \item Show that 6 is a perfect number.
 \item Show that 6 is the only perfect number less than 10.
 \item Find another perfect number less than 30.
\end{enumerate}
By now you should have found the first two perfect numbers. The next is 496.

\begin{enumerate}\setcounter{enumi}{3}
 \item Check that 496 is perfect.
 \item Find five positive integers, each one being the product of all its proper divisors.
 \item Characterize all positive integers that are the product of their proper divisors.
\end{enumerate}
Now you are almost ready to prove the main theorem of this project. Steps 7-9 below will lead you through the proof.

\noindent{\bf Theorem} {\em There is only one positive integer that is both the sum and product of its proper divisors, and that number is 6.}


\begin{enumerate}\setcounter{enumi}{6}
 \item Let $p$ be a prime. Prove that $p^3$ is not perfect.
 \item Prove as many of the following as you need to, until you see the proof of the theorem:
\begin{enumerate}
 \item Prove that the only even number that is the sum and product of its proper divisors is 6.
 \item Prove that the only multiple of 3 that is the sum and product of its proper divisors is 6.
 \item Prove that there is no multiple of 5 with this property.
 \item Prove that there is no multiple of 7 with this property.
\end{enumerate}
 \item Prove the theorem.

\end{enumerate}




\end{document}
 
