\documentclass[letterpaper,12pt]{article}

\usepackage{ucs}
\usepackage[utf8x]{inputenc}
\usepackage{amsmath}
\usepackage{amsfonts}
\usepackage{amssymb}
\usepackage{amsthm}
\usepackage[margin=1in]{geometry}

%\theoremstyle{theorem}
\newtheorem{theorem}{Theorem}
\newtheorem{lemma}{Lemma}
\newtheorem{problem}{Problem}
\theoremstyle{definition}
\newtheorem{definition}{Definition}
\newcommand{\abs}[1]{\lvert #1\rvert}
\newcommand{\R}{\mathbb{R}}
\newcommand{\C}{\mathbb{C}}
\newcommand{\N}{\mathbb{N}}

\title{Math 2000 Writing Assignment \#1}
\author{Sean Fitzpatrick}
\begin{document}
 \maketitle

{\bf Due Date:} Tuesday, November 3\textsuperscript{rd}, by 4:30 pm.

\medskip

This is the first of two writing assignments designed to give you practice with explaining a mathematical result or concept, including the writing of proofs. The assignment is {\bf guided}, in the sense that you will be given a series of questions whose answers you will assemble into your finished assignment. The following guidelines apply to your assignment:
\begin{itemize}
 \item I have provided two options below, both chosen from the ``Explorations and Activities'' in the textbook. You should choose {\bf one} of these to submit as your assignment. (If you really don't like either option you're welcome to drop by during my office hours to propose an alternate activity.)
 \item Your solutions to the activity you choose should be written in complete sentences, using as little symbolic notation as necessary. (This means that you should avoid symbolic logic, or $\forall, \exists$, etc. but if you need to refer to an equation like $a^2+b^2=c^2$, you should write the equation, rather than trying to translate it into English.)
 \item For either option, the textbook breaks down your work into more easily digestible parts. (For example, you might have questions 14(a), (b), (c), etc.) Your completed project should {\bf not} contain these question numbers. Use them as a guideline for how you should organize your thoughts, but please organize your final draft in paragraph form.\footnote{For example, if you choose the first topic, you should probably write a short introduction explaining what it is you're going to talk about (Pythagorean triples) then give the definition of what these are, then some examples, etc. Pretend you're writing a section in a textbook.}
 \item In addition to answering the questions in the text, I will expect you to do a bit of follow-up work to add to the presentation. This could be as simple as going online and looking up a Wikipedia article on the topic. (Actually going to the library and finding a textbook reference will win you some respect, but no bonus points.) I'll suggest a follow-up activity for either topic below.
 \item Whatever references you use in your follow-up work need to be {\bf properly cited}. For example, if you find a proof online that you'd like to include, there is nothing wrong with quoting that proof verbatim, as long as you give credit to the source of the proof.
\end{itemize}

\section*{Topic \#1: Pythagorean Triples}
To complete this topic, you should answer Problems 13 and 14 from the Explorations and Activities in Section 1.2 (pp. 29-30), along with Problem 21 from the Explorations and Activities in Section 3.1 (p. 102).

You should {\bf supplement} your work by doing some further research surrounding the topic of Pythagorean Triples. For example, there are many proofs of the Pythagorean Theorem, and you could look up some of these and include them. You could also include some historical information (about Pythagorean Triples, or the theorem, or Pythagoras, etc.) or say something about a related topic. (Fermat's Last Theorem, perhaps?)

\section*{Topic \#2: Least Upper Bounds}
To complete this topic, you should answer Problems 14 and 15 from the Explorations and Activities in Section 2.4 (pp. 79-80). For your follow-up work, you need to find out what least upper bounds have to do with the definition of the real number system. (In particular you should provide the complete list of axioms for $\mathbb{R}$, including the {\em completeness axiom}, which involves least upper bounds.) You should then give an example (with proof) showing that the set $\mathbb{Q}$ of rational numbers does not satisfy the completeness axiom. (The subset $\{x\in \mathbb{Q} \, |\, x^2<2\}$ is a good one to consider. You should be able to find a proof online that this set is bounded above but has no least upper bound; as mentioned above you can quote a proof directly as long as you cite your source.)

For Topic \#2 I'll also include an opportunity for a 10\% bonus: explain to me how one can actually {\em construct} a set that satisfies all the axioms of the real number system. (You'll have to do some searching online for this one.)


\end{document}
 
