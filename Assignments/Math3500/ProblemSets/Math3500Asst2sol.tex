\documentclass[letterpaper,12pt]{article}

\usepackage{ucs}
\usepackage[utf8x]{inputenc}
\usepackage{amsmath}
\usepackage{amsfonts}
\usepackage{amssymb}
\usepackage[margin=1in]{geometry}
\DeclareMathOperator{\cl}{cl}

\newcommand{\R}{\mathbb{R}}
\newcommand{\N}{\mathbb{N}}
\newcommand{\F}{\mathcal{F}}
\newcommand{\abs}[1]{\lvert #1\rvert}
\newcommand{\len}[1]{\lVert #1\rVert}

\title{Math 3500 Assignment \#2 Solutions\\University of Lethbridge, Fall 2014}
\author{Sean Fitzpatrick}
\begin{document}
 \maketitle


\begin{enumerate}
 \item Let $S$ and $T$ be nonempty bounded subsets of $\R$. 
\begin{enumerate}
 \item Prove that if $S\subseteq T$, then $\inf T\leq \inf S\leq \sup S\leq \sup T$.

\bigskip

{\bf Solution}: Suppose that $S$ and $T$ are nonempty subsets of $\mathbb{R}$. If $a=\inf T$, then $a\leq s$ for all $s\in S$, since if $s\in S$, then $s\in T$ and $a$ is a lower bound for $T$. But this means that $\inf S\geq \inf T=a$, since $\inf S$ is the {\em greatest} lower bound. Similarly, the supremum of $T$ is an upper bound for $S$, since $S\subseteq T$, so $\sup S\leq \sup T$, since $\sup S$ is the least upper bound of $S$. Finally, since $S$ is nonempty, we can take any $s\in S$, and then (as we saw in class), we have $\inf S\leq s\leq \sup S$, by definition of the infimum and supremum. The result now follows by the transitivity of the order relation on $\mathbb{R}$.

 \item Prove that $\sup S\cup T = \max\{\sup S, \sup T\}$. \\(Do not assume that $S\subseteq T$ for part (b).)

\bigskip

{\bf Solution}: Let $a=\sup S$ and $b=\sup T$.
Let $c=max\{a,b\}$, so that $c \geq a$ and $c \geq b$.
If $x \in S \cup T,$ then 
 either $x \in S$, and $x \leq a \leq c$
 or $x \in T,$ and $s \leq b \leq c$.
Therefore $c$ is an upper bound for $S \cup T$.

If $d$ is an upper bound for $S \cup T$ then, since $S \subseteq S \cup T$ and $T \subseteq S \cup T$, $d$ is an upper bound for both $S$ and $T$. Thus $d \geq a$ and $d \geq b$, since $a$ and $b$ are the least upper bounds for $S$ and $T$, respectively. Thus $d\geq max\{a,b\} =c$. Since $d$ was an arbitrary upper bound, we can conclude that $c=\sup (S \cup T).$
\end{enumerate}
 \item Let $\mathcal{B}[a,b]$ denote the set of all bounded functions defined on the interval $[a,b]$. (That is, for each $f\in \mathcal{B}[a,b]$, there exist constants $k,l\in\R$ such that $k\leq f(x)\leq l$ for all $x\in [a,b]$.) The {\em norm} of a function $f\in \mathcal{B}[a,b]$ is defined by
\[
 \lVert f\rVert = \sup\{\abs{f(x)} : x\in [a,b]\}.
\]
Prove that $\lVert f+g\rVert \leq \lVert f\rVert +\lVert g\rVert$ for any $f,g\in\mathcal{B}[a,b]$.

\bigskip

{\bf Solution}: By the triangle inequality, for any $x\in [a,b]$ we have
\[
 \abs{f(x)+g(x)}\leq \abs{f(x)}+\abs{g(x)}\leq \len{f}+\len{g},
\]
since $\abs{f(x)}\leq \len{f}$ and $\abs{g(x)}\leq \len{g}$ for all $x\in [a,b]$. Thus, $\len{f}+\len{g}$ is an upper bound for $\{\abs{f(x)+ g(x)}\,|\, x\in[a,b]\}$. Since $\abs{f+g}$ is defined to be the {\em least} upper bound of this set, we have
\[
 \len{f+g}\leq \len{f}+\len{g},
\]
as required.

{\bf Note}: In this problem it's important to distinguish between the {\em function} $f$ and its {\em value} $f(x)$ at a particular $x\in [a,b]$. To establish that a particular fact holds for the function, you need to verify that it is true for {\em all} values of $x$.

\item Prove that if $A$ is any nonempty open subset of $\R$, then $A\cap \mathbb{Q}\neq \emptyset$.


\bigskip

{\bf Solution}: Suppose $A\subseteq \mathbb{R}$ is open, and $A \neq \emptyset$. (Again, note that $A$ being open does {\bf not} imply that $A$ is an interval!)
Then there exists and $a\in A$ and $\epsilon >0$ such that $N_{\epsilon}(a)=(a-\epsilon,a+\epsilon) \subseteq A.$
But since $\mathbb{Q}$ is dense in $\mathbb{R}$, we know that there exists a $q \in \mathbb{Q}$ with $a-\epsilon <q<a+\epsilon$.
Therefore $q \in N_{\epsilon}(a)\subseteq A,$ so $A \cap \mathbb{Q}\neq \emptyset.$



\item For any set $S\subseteq \R$, let $\overline{S}$ denote the intersection of all the closed sets containing $S$.
\begin{enumerate}
 \item Prove that $\overline{S}$ is a closed subset of $\R$.

\bigskip

{\bf Solution}: Let $\F = \{F\subseteq \R|S\subseteq F \text{ and } F \text{ is closed}\}$. We know that the intersection of any family of closed subsets is closed (e.g. via Corollary 3.4.11 in the text). Therefore, $\displaystyle \overline{S} = \bigcap_{F\in\F}F$ is closed.

 \item Prove that $\overline{S}$ is the {\em smallest} closed set containing $S$. That is, show that $S\subseteq \overline{S}$, and if $C$ is any closed set containing $S$, then $\overline{S}\subseteq C$.

\bigskip

{\bf Solution}: Let $C$ be any closed set containing $S$. Then $C\in\F$, and we know that for any collection of sets $\F$, the intersection $\bigcap_{F\in\F}F$ is a subset of each set in the collection. Thus, $\overline{S}\subseteq C$, as required.

 \item Prove that $\overline{S}$ is equal to the closure of $S$.

\bigskip

{\bf Solution}: Let $\cl S$ denote the closure of $S$. Since $\cl S$ is closed (e.g. via Theorem 3.4.17 in the text), and $S\subseteq \cl S$, we know that $\overline{S}\subseteq \cl S$, by part (b). Now, we need to show that $\cl S\subseteq \overline{S}$. If $x\in \cl S$, then either $x\in S$, in which case we have $x\in\overline{S}$, since $S\subseteq \overline{S}$, or $x$ is a limit point of $S$. If we know that $x\in F$ for all $F\in \F$, then we'd have $x\in\overline{S}$ and we'd be done. Thus, it suffices to prove the following lemma:

{\em Lemma}: If $x$ is a limit point of a set $S$, then $x\in F$ for any closed set $F$ with $S\subseteq F$.

Proof: Suppose $x$ is a limit point of $S$, and $S\subseteq F$, with $F$ closed. Suppose that $x\notin F$. Then $x\in F^c$, the complement of $F$, which is open, since $F$ is closed. Thus, there exists $\epsilon>0$ such that $N_\epsilon(x)\subseteq F^c$. But $F^c\subseteq S^c$, since $S\subseteq F$, which means that $N_\epsilon(x)\subseteq S^c$, or $N_\epsilon(x)\cap S = \emptyset$. Since this contradicts the assumption that $x$ is a limit point of $S$, it must be the case that $x\in F$.

 \item Prove that if $S$ is bounded, then $\overline{S}$ is bounded as well.

\bigskip

{\bf Solution}: If $S$ is bounded, then $S\subseteq [a,b]$ for some $a,b\in\R$. But then $[a,b]$ is a closed set containing $S$, so $[a,b]\in\F$ and thus $\overline{S}\subseteq [a,b]$ by part (b).
\end{enumerate}

\item The Nested Intervals Theorem (from the September 10th worksheet, and also mentioned on Piazza) states that if $\{A_n : n\in \mathbb{N}\}$ is a collection of closed bounded intervals (of the form $[a,b]$), and we have $A_{n+1}\subseteq A_n$ for all $n\in\mathbb{N}$, then the intersection $\bigcap A_n$ is nonempty.

Show that the intervals $A_n$ need to be {\bf both} closed and bounded by giving examples where the theorem fails (that is, where $\bigcap A_n =\emptyset$), if
\begin{enumerate}
 \item The intervals $A_n$ are closed, but not bounded.

\bigskip

{\bf Solution}: Consider the intervals $A_n = [n,\infty)$, for $n\in \N$. Each $A_n$ is closed, since $\partial A_n =\{n\}\subseteq A_n$, and the $A_n$ are not bounded. Moreover, we have that
\[
 \bigcap_{n=1}^\infty A_n = \emptyset,
\]
since for any $x\in \R$ there exists $N\in \N$ with $N>x$ (by the Archimedean property of $\R$), and thus $x\notin A_N$, so $x$ cannot be in the intersection. 

 \item The intervals $A_n$ are bounded, but not closed.

\bigskip

{\bf Solution}: This time we let $A_n = (0,1/n)$, for $n\in\N$. Each $A_n$ is bounded, since we have $A_n\subseteq [0,1]$ for all $n$, but none of the $A_n$ are closed, since $0\in\partial A_n$ for all $n$, but $0\notin A_n$. We then have that
\[
 \bigcap_{n=1}^\infty A_n = \emptyset,
\]
since any $x\in\R$ with $x\leq 0$ belongs to none of the $A_n$, and if $x>0$, then there exists $N\in \N$ such that $1/N<x$, by the Archimedean property of $\R$, and hence $x\notin A_N$, and thus $x\notin \bigcap A_N$.

{\bf Note}: Pointing out that $A_n$ is open is {\bf not} the same as saying that it's not closed! There are sets which are both open and closed (i.e. $\R$ and $\emptyset$), and many sets which are neither open nor closed (e.g. $[0,1)$).


\end{enumerate}

\item An important theorem regarding compact sets is that if $S\subseteq \R$ is compact, and $T$ is a closed subset of $S$, then $T$ is compact. Prove this fact using:
\begin{enumerate}
 \item The definition of compactness.

\bigskip

{\bf Solution:} We will use the same argument obtained during our class discussion: Let $S\subseteq \R$ be compact, and suppose $T\subseteq S$, with $T$ closed in $\R$. Let $\{G_\alpha\}_{\alpha\in I}$ be an open cover of $T$, for some index set $I$. We need to show that there are finitely many $\alpha_1,\ldots, \alpha_k\in I$ such that
\[
 T\subseteq G_{\alpha_1}\cup\cdots \cup G_{\alpha_k}.
\]
Since $T$ is closed, $\R\setminus T$ is open. Since $T\cup (\R\setminus T)=\R$ and $T\subseteq \bigcup G_{\alpha}$, it follows that $S\subseteq \left(\bigcup G_\alpha\right)\cup (R\setminus T)$, so that $\{G_\alpha\}_{\alpha\in I}\cup\{\R\setminus T\}$ is an open cover of $S$. Since $S$ is compact, this open cover must admit a finite subcover. Thus, we have
\[
 S\subseteq G_{\alpha_1}\cup \cdots G_{\alpha_k}\cup (\R\setminus T)
\]
for some $\alpha_1,\ldots, \alpha_k\in I$. Since $T\cap (\R\setminus T)=\emptyset$, we must have $T\subseteq G_{\alpha_1}\cup\cdots \cup G_{\alpha_k}$, which is what we needed to show.

 \item The Heine-Borel theorem.

\bigskip

{\bf Solution:} Let $S\subseteq \R$ be compact, and suppose $T\subseteq S$, with $T$ closed in $\R$. Since $S$ is compact, it is closed and bounded, by the Heine-Borel theorem. Since $T\subseteq S$, $T$ must be bounded as well. (Any upper bound for $S$ will be an upper bound for $T$, and likewise for lower bounds.) Since $T$ is also assumed to be closed, we must have that $T$ is compact, by the Heine-Borel theorem.
\end{enumerate}



\end{enumerate}

\end{document}
 
