\documentclass[letterpaper,12pt]{article}

\usepackage{ucs}
\usepackage[utf8x]{inputenc}
\usepackage{amsmath}
\usepackage{amsfonts}
\usepackage{amssymb}
\usepackage[margin=1in]{geometry}
%\usepackage{enumerate}

\newcommand{\R}{\mathbb{R}}
\newcommand{\N}{\mathbb{N}}
\renewcommand{\ss}{\subseteq}

\newcommand{\abs}[1]{\lvert #1\rvert}
\newcommand{\Abs}[1]{\left| #1\right|}

\title{Math 3500 Assignment \#4 Solutions\\University of Lethbridge, Fall 2014}
\author{Sean Fitzpatrick}
\begin{document}
 \maketitle

\begin{enumerate}
 \item Give an example of each of the following, or argue that such a request is impossible:
\begin{enumerate}
 \item A sequence that does not contain 0 or 1 as a term, but contains subsequences converging to both 0 and 1.

\bigskip

\noindent{\bf Solution}: One such example would be the sequence $(a_n)$ given by $a_n = \dfrac{1}{n}$, when $n$ is even, and $a_n = 1+\dfrac{1}{n}$ when $n$ is odd. (Note: you don't want to use $1/n$ when $n$ is odd or you'll end up with $a_1=1$. 


\bigskip

 \item A monotone sequence that diverges but has a convergent subsequence.

\bigskip

\noindent{\bf Solution}: This is impossible. Suppose $(a_n)$ diverges, but $(a_{n_k})$ is a convergent subsequence. Then we know that $(a_{n_k})$ is bounded, since this is true of every convergent sequence, so there exists some $M>0$ such that $\abs{a_{n-k}}\leq M$ for all $k\in\N$. Now, suppose $(a_n)$ is increasing. (If $(a_n)$ is decreasing the proof is similar.) For some $N\in\N$ we have $a_n\geq 0$ for all $n\geq N$, or else 0 is an upper bound for $(a_n)$ and $(a_n)$ would converge. But then for each $k\geq N$ we have $N\leq k\leq n_k$, and since $(a_n)$ is monotone, $0\leq a_k\leq a_{n_k}\leq M$, and again $(a_n)$ is bounded (by the maximum of $M$ and $a_1,\ldots, a_N$), and thus converges by the Monotone Convergence Theorem, contradicting the assumption that $(a_n)$ diverges.

\bigskip

 \item A sequence that contains subsequences converging to every point in the infinite set $\{1,1/2, 1/3, 1/4, \ldots\}$.

\bigskip

\noindent{\bf Solution}: Consider the sequence
\[
 1,1,1/2, 1, 1/2, 1/3, 1, 1/2, 1/3, 1/4, \ldots .
\]
Then for each $n\in\N$, the constant sequence $1/n, 1/n, 1/n,\ldots$ appears as a subsequence.

\bigskip

 \item An unbounded sequence with a convergent subsequence.

\bigskip

\noindent{\bf Solution}: One example is the sequence $(a_n)$ where $a_n = n$ when $n$ is even, and $a_n = 0$ when $n$ is odd.

\bigskip

\end{enumerate}
 \item Prove that $\displaystyle \lim_{x\to 2}\frac{2x+1}{x+2} = \frac{5}{4}$ using the $\epsilon-\delta$ definition of the limit.

\bigskip

\noindent{\bf Solution}: Let $\epsilon>0$ be given, and let $\delta = \min\{1,4\epsilon\}$. If $0<\abs{x-2}<\delta$, then in particular $0<\abs{x-2}<1$, which gives $-1<x-2<1$, so $3<x+2<5$, and therefore $\dfrac{3}{x+2}<1$. Thus, we have
\[
 \Abs{\frac{2x+1}{x+2}-\frac{5}{4}} = \Abs{\frac{8x+4-(5x+10)}{4(x+2)}} = \frac{3}{\abs{x+2}}\frac{\abs{x-2}}{4}\leq \frac{\abs{x-2}}{4}<\frac{\delta}{4}\leq \frac{4\epsilon}{4} = \epsilon. 
\]


\bigskip

 \item Suppose that $f_1$ and $f_2$ are functions for which $\lim_{x\to a^+}f_1(x) = L_1$ and $\lim_{x\to a^+}f_2(x) = L_2$ both exist.
\begin{enumerate}
 \item Show that if there exists an interval $(a,b)$ such that $f_1(x)\leq f_2(x)$ for all $x\in (a,b)$, then $L_1\leq L_2$. 

\bigskip

\noindent{\bf Solution}: We prove the contrapositive: if $L_1>L_2$, then there exists some $x>a$ for which $f_1(x)>f_2(x)$.

Noting that $L_1-L_2>0$, since $\lim_{x\to a^+}f_1(x) = L_1$ and $\lim_{x\to a^+}f_2(x) = L_2$, we can find some $\delta_1, \delta_2>0$ such that
\begin{align*}
 \text{If } a<x<a+\delta_1, &\text{ then } \abs{f_1(x)-L_1}<(L_1-L_2)/2, \text{ and }\\
 \text{if } a<x<a+\delta_2, &\text{ then } \abs{f_2(x)-L_2}<(L_1-L_2)/2.
\end{align*}
Let $\delta = \min\{\delta_1,\delta_2\}$, and suppose $x\in(a,a+\delta)$. Then
\[
 \abs{f_1(x)-L_1}<\frac{L_1-L_2}{2} \Rightarrow \frac{L_1+L_2}{2}<f_1(x)<\frac{3L_1-L_2}{2}
\]
and
\[
 \abs{f_2(x)-L_2}<\frac{L_1-L_2}{2} \Rightarrow \frac{3L_2-L_1}{2}<f_2(x)<\frac{L_1+L_2}{2}.
\]
Thus $f_2(x)<\dfrac{L_1+L_2}{2}<f_1(x)$, so $f_1(x)>f_2(x)$ for $x\in(a,a+\delta)$, which is what we wanted to show.

\bigskip

 \item Suppose that we in fact have that $f_1(x)<f_2(x)$ for all $x\in (a,b)$. Can we conclude that $L_1<L_2$?

\bigskip

\noindent{\bf Solution}: No. For example, if $f_1(x)=0$ and $f_2(x)=x$ for $x\in\R$, then $f_1(x)<f_2(x)$ for all $x\in (0,1)$, but \[\lim_{x\to 0^+}f_1(x)=0=\lim_{x\to 0^+}f_2(x).\]

\bigskip

\end{enumerate}
 \item Let $g:A\to \R$ be a given function and suppose that $f$ is a bounded function defined on $A$. (That is, there exists some constant $M\geq 0$ such that $\abs{f(x)}\leq M$ for all $x\in A$.) Let $a$ be a limit point of $A$, and show that if $\displaystyle \lim_{x\to a}g(x)=0$, then $\displaystyle \lim_{x\to a}(f(x)g(x))=0$ as well.

\bigskip

\noindent {\bf Solution}: Suppose that $\abs{f(x)}\leq M$ for all $x\in A$, for some $M>0$, and that $\lim_{x\to a}g(x)=0$. Given any $\epsilon>0$, there exists some $\delta>0$ such that if $x\in A$ and $0<\abs{x-a}<\delta$, then $\abs{g(x)}<\epsilon/M$, and thus
\[
 \abs{f(x)g(x)} = \abs{f(x)}\abs{f(x)}\leq M\abs{g(x)}<M(\epsilon/M)=\epsilon.
\]
Thus, $\lim_{x\to a}(f(x)g(x))=0$.


 %\item (The Contraction Mapping Theorem) Let $f$ be a function defined on all of $\R$, with the property that there exists some $c$ with $0<c<1$, such that for all $x, y\in \R$,
%\[
% \abs{f(x)-f(y)}\leq c\abs{x-y}.
%\]
%\begin{enumerate}
% \item Prove that $f$ is continuous on $\R$.
% \item Choose any point $x\in\R$, and consider the sequence $(x,f(x),f(f(x)),\ldots)$. (That is, the sequence is defined recursively by $x_1=x$ and $x_{n+1}) = f(x_n)$ for $n\geq 1$.) Show that this sequence is a Cauchy sequence.
% \item Since the sequence in part (b) is Cauchy, it converges. Let $y=\lim x_n$, and prove that $y$ is a {\em fixed point} of $f$. That is, prove that $f(y)=y$.
% \item Show that $y=\lim x_n$ is the {\bf unique} fixed point of $f$.
% \item Prove that if $z\in \R$ is any arbirary point, then the sequence $(z,f(z),f(f(z)),\ldots)$ still converges to $y$.
%\end{enumerate}



\end{enumerate}

\end{document}
 
