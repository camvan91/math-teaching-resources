\documentclass[letterpaper,12pt]{article}

\usepackage{ucs}
\usepackage[utf8x]{inputenc}
\usepackage{amsmath}
\usepackage{amsfonts}
\usepackage{amssymb}
\usepackage[margin=1in]{geometry}
%\usepackage{enumerate}

\newcommand{\R}{\mathbb{R}}
\newcommand{\N}{\mathbb{N}}
\renewcommand{\ss}{\subseteq}

\newcommand{\abs}[1]{\lvert #1\rvert}

\title{Math 3500 Assignment \#4\\University of Lethbridge, Fall 2014}
\author{Sean Fitzpatrick}
\begin{document}
 \maketitle

{\bf Due date}: Friday, October 17, by 6 pm.

\bigskip

{\bf Reminder}: You're free to discuss these problems, and you should ask for help with any you don't understand. (Both office hours and Piazza are available for this purpose, and there are evening help sessions Tuesday, Wednesday and Thursday.) But keep in mind that getting help should involve making sure that you {\em understand} the solution. The assignments count towards your course grade, but ultimately, the main purpose of the assignments is to improve your understanding of the material in order to prepare you for the final exam.

\begin{enumerate}
 \item Give an example of each of the following, or argue that such a request is impossible:
\begin{enumerate}
 \item A sequence that does not contain 0 or 1 as a term, but contains subsequences converging to both 0 and 1.
 \item A monotone sequence that diverges but has a convergent subsequence.
 \item A sequence that contains subsequences converging to every point in the infinite set $\{1,1/2, 1/3, 1/4, \ldots\}$.
 \item An unbounded sequence with a convergent subsequence.
\end{enumerate}
 \item Prove that $\displaystyle \lim_{x\to 2}\frac{2x+1}{x+2} = \frac{5}{4}$ using the $\epsilon-\delta$ definition of the limit.
 \item Suppose that $f_1$ and $f_2$ are functions for which $\lim_{x\to a^+}f_1(x) = L_1$ and $\lim_{x\to a^+}f_2(x) = L_2$ both exist.
\begin{enumerate}
 \item Show that if there exists an interval $(a,b)$ such that $f_1(x)\leq f_2(x)$ for all $x\in (a,b)$, then $L_1\leq L_2$. 
 \item Suppose that we in fact have that $f_1(x)<f_2(x)$ for all $x\in (a,b)$. Can we conclude that $L_1<L_2$?
\end{enumerate}
 \item Let $g:A\to \R$ be a given function and suppose that $f$ is a bounded function defined on $A$. (That is, there exists some constant $M\geq 0$ such that $\abs{f(x)}\leq M$ for all $x\in A$.) Let $a$ be a limit point of $A$, and show that if $\displaystyle \lim_{x\to a}g(x)=0$, then $\displaystyle \lim_{x\to a}(f(x)g(x))=0$ as well.
 %\item (The Contraction Mapping Theorem) Let $f$ be a function defined on all of $\R$, with the property that there exists some $c$ with $0<c<1$, such that for all $x, y\in \R$,
%\[
% \abs{f(x)-f(y)}\leq c\abs{x-y}.
%\]
%\begin{enumerate}
% \item Prove that $f$ is continuous on $\R$.
% \item Choose any point $x\in\R$, and consider the sequence $(x,f(x),f(f(x)),\ldots)$. (That is, the sequence is defined recursively by $x_1=x$ and $x_{n+1}) = f(x_n)$ for $n\geq 1$.) Show that this sequence is a Cauchy sequence.
% \item Since the sequence in part (b) is Cauchy, it converges. Let $y=\lim x_n$, and prove that $y$ is a {\em fixed point} of $f$. That is, prove that $f(y)=y$.
% \item Show that $y=\lim x_n$ is the {\bf unique} fixed point of $f$.
% \item Prove that if $z\in \R$ is any arbirary point, then the sequence $(z,f(z),f(f(z)),\ldots)$ still converges to $y$.
%\end{enumerate}



\end{enumerate}

\end{document}
 
