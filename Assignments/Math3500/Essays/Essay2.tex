\documentclass[letterpaper,12pt]{amsart}

\usepackage{amsmath}
\usepackage{amsfonts}
\usepackage{amssymb}
\usepackage[margin=1in]{geometry}
%\usepackage[dvips]{hyperref}
\usepackage{enumerate}

\newcommand{\R}{\mathbb{R}}
\newcommand{\Q}{\mathbb{Q}}

\title{Essay Assignment \#2\\Math 3500A Analysis I\\University of Lethbridge, Fall 2014}
\date{}

\begin{document}
\maketitle

 {\bf Due date:} Wednesday, the 3\textsuperscript{rd} of September, by 6:00 pm.

\bigskip

The second essay assignment is on a topic of your choice. The expectations in terms of presentation are the same as for the first essay (i.e. typed, or neatly hand-written). Your paper should be 5-6 pages in length. (There's some flexibility here, and of course the page count will vary if your writing it out by hand rather than typing. If you're unsure, check with me.)

{\bf Requirements}: 
\begin{itemize}
 \item The paper should be {\bf researched}. I will expect at least {\bf three references}, and at least one of these should be a textbook other than our course textbook. You should provide a list of the resources you used at the end of the paper.
 \item The paper should be {\bf relevant}. This means that your topic needs to be somehow related to the course material. For example, you could discuss an application, or some topic in analysis beyond what is covered in class.
\end{itemize}


{\bf Choosing a topic}: Please choose a topic no later than {\bf November 7th}. You will need time to work on your essay. If you're not sure what to write about, you have two options:
\begin{enumerate}
 \item Talk to me during office hours. I'll try to come up with some suggestions based on what you tell me you're interested in.
 \item Go to the library. There's a whole section of analysis textbooks. Any topic from any of these books is an option, provided that (a) it is not already part of the course, and (b) it's not too complicated (you need to understand it well enough to write about it in your own words).
\end{enumerate}

{\bf Applied topics}: choose some topic from the course, (or some part of analysis not in the course) and explain how it is used in another field, such as physics, chemistry, statistics, finance, etc.

{\bf Advanced topics}: some aspect of analysis that goes beyond the course. For example, you could discuss Taylor series, Fourier series, or sequences and series of functions in general; analysis in several variables, including topics like the implicit function theorem, vector calculus, differential forms, etc.; measure and integration; metric spaces, and so on. 

Note: since we're looking for a paper on analysis rather than calculus, some amount of rigour/proof is expected.

{\bf Education}: If you're in education, perhaps you could write about the ways in which studying higher mathematics can inform your approach to teaching at the grade school level, and maybe mentioning some insight from Analysis you found useful. (If you think there is in fact no benefit, another topic might be better...)

{\bf Working with partners}: I'm willing to accept an essay written by a pair of students. If you want to work with a partner, I'll ask for {\bf six} references and 7-9 pages, since there are two of you to share the work. For any such submissions, both students will receive the same grade -- I'm not going to police issues related to who did more work.

 





\end{document}
 
