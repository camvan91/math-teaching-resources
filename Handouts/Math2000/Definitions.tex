\documentclass[letterpaper,8pt,landscape]{article}

\usepackage{multicol}
\usepackage{amsmath}
\usepackage{amsfonts}
\usepackage{amssymb}
\usepackage[canadian]{babel}
\usepackage[margin=0.75in]{geometry}

\usepackage[dvips]{hyperref}

\newcommand{\R}[2]{#1\,R\,#2}

\title{Potentially useful facts and definitions}
\author{}

\begin{document}
\begin{center}
{\bf List of potentially useful facts and definitions} (you may remove this page)
\end{center}
\begin{multicols}{3}
\subsubsection*{Propositional logic}
Basic logical operations:\\
Negation: $\neg P$ (``not $P$'')\\
Conjunction: $P\wedge Q$ (``$P$ and $Q$'')\\
Disjunction: $P\vee Q$ (``$P$ or $Q$'')\\
Conditional: $P\to Q$ (``if $P$ then $Q$'')\\

\noindent Basic logical equivalences:\\
$P\to Q \equiv \neg P\vee Q$\\
$\neg(P\vee Q)\equiv \neg P\wedge\neg Q$\\
$\neg(P\wedge Q)\equiv \neg P\vee\neg Q$\\
$P\vee(Q\wedge R)\equiv (P\vee Q)\wedge (P\vee R)$\\
$P\wedge (Q\vee R)\equiv (P\wedge Q)\vee (P\wedge R)$\\
$P\vee \neg P\equiv T$, $P\wedge \neg P\equiv F$\\
$P\vee T \equiv T$, $P\wedge T \equiv P$, $P\vee F \equiv P$, $P\wedge F\equiv F$\\
$P\to Q \equiv \neg Q\to \neg P$\\

\noindent Quantifiers:\\
Universal (``for all''): $\forall x\in U, P(x)$\\
Existential (``there exists''): $\exists x\in U : P(x)$\\
Negtation: $\neg(\forall x\in U, P(x))\equiv \exists x\in U: \neg P(x)$\\
\phantom{Negation: }  $\neg (\exists x\in U: P(x)) \equiv \forall x\in U, \neg P(x)$

\subsubsection*{Sets and set operations}
{Membership}: $x\in A$ ($x$ belongs to $A$)\\
{Subset}: $A\subseteq B$, if $\forall x\in U, x\in A\to x\in B$.\\
{Equality}: $A=B$ if $A\subseteq B$ and $B\subseteq A$\\
{Empty set}: the set $\emptyset$ containing no elements.\\
{Power set}: $\mathcal{P}(A) = \{B\subseteq U : B\subseteq A\}$\\
{Union}: $A\cup B = \{x\in U : x\in A \vee x\in B\}$\\
{Intersection}: $A\cap B = \{x\in U : x\in A\wedge x\in B\}$\\
{Complement}: $A^c = \{x\in U : x\notin A\}$\\
{Set difference}: $A\setminus B = \{x\in A : x\notin B\}$\\
{Product}: $A\times B = \{(a,b) : a\in A \wedge b\in B\}$.\\

$\bigcup_{\alpha\in I}A_\alpha = \{x\in U \,|\, \exists \alpha\in I : x\in A_\alpha\}$

$\bigcap_{\alpha\in I}A_\alpha = \{x\in U \,|\, \forall \alpha \in I,\, x\in A_\alpha\}$\\

\noindent Basic set equalities:\\
$(A\cup B)^c = A^c\cap B^c$\\
$(A\cap B)^c = A^c\cup B^c$\\
$A\cup (B\cap C) = (A\cup B)\cap (A\cup C)$\\
$A\cap (B\cup C) = (A\cap B)\cup (A\cap C)$\\
$A\cup A^c = U$, $A\cap A^c = \emptyset$\\
$A\subseteq B$ if and only if $B^c\subseteq A^c$\\
$A\times (B\cup C) = (A\times B)\cup (A\times C)$\\
$A\times (B\cap C) = (A\times B)\cap (A\times C)$

\subsubsection*{Divisibility and congruence}
Divides: $m|n$ iff $\exists k\in\mathbb{Z}$ such that $n=mk$.\\
Congruence: $a\equiv b \pmod{n}$ iff $n|(a-b)$.\\
Division algorithm: $m=nq+r$, $r\in \{0,1,\ldots, n-1\}$

\subsubsection*{Functions}
$f:A\to B$ -- $\forall a\in A$ get {\em unique} $b=f(a)\in B$.\\
Domain: $A$ \hspace{16pt} Codomain: $B$\\
Range: $\mathrm{ran}(f) = \{f(a)\,|\,a\in A\}\subseteq B$\\
Composition: given $f:A\to B$ and $g:B\to C$ get $g\circ f:A\to C$, $(g\circ f)(a) = g(f(a))$.\\
One-to-one: for all $a,b\in A$, $f(a)=f(b) \to a=b$.\\
Onto: $\mathrm{ran}(f) = B$.\\
Bijection: $f$ is both one-to-one and onto.\\
Inverse: if $f:A\to B$ is a bijection, define $f^{-1}:B\to A$ by $f^{-1}(b)=a$ if and only if $f(a)=b$.\\
Cancellation laws: $\forall a\in A, f^{-1}(f(a))=a$, and $\forall b\in B, f(f^{-1}(b))=b$.\\
Image:  $f(C) = \{f(c)\,|\, c\in C\}$.\\
Preimage: $f^{-1}(D) = \{a\in A\,|\, f(a)\in D\}$.

\subsubsection*{Cardinality}
Equivalence: $A\approx B$, if $\exists$ a bijection $f:A\to B$\\
Finite sets: $A\approx\{1,2,\ldots, k\}$ for some $k\in\mathbb{N}$.\\
Infinite sets: any set that is not finite.\\
Cardinality: $|A| = k$ iff $A\approx \{1,2,\ldots, k\}$.\\
$A\approx B$ iff $|A|=|B|$.\\
Pigeonhole principle: if $|A|>|B|$, any $f:A\to B$ is not one-to-one.\\
If $A$ and $B$ are finite and $A\cap B = \emptyset$, then $|A\cup B| = |A|+|B|$.\\
If $A$ and $B$ are finite then $|A\times B| = |A|\cdot |B|$.\\
A set $A$ is {\bf countable} if there exists a one-to-one function $f:A\to\mathbb{N}$. (Bijection if $A$ infinite.)\\
The sets $\mathbb{N}, \mathbb{Z}$, and $\mathbb{Q}$ are all countable.\\
The set $\mathbb{R}$ of real numbers is {\bf uncountable}.

\subsubsection*{Mathematical induction}
Proof by induction: to prove a statement of the form $\forall n\in \mathbb{N}, P(n)$, show that $P(1)$ is true and that for $k\geq 1$, $P(k)\to P(k+1)$.\\
Strong induction: Instead of only assuming $P(k)$ is true, assume that $P(1), P(2), \ldots, P(k-1), P(k)$ are all true for some $k$, and use this to show $P(k+1)$ is true. Note that you may need more than one base case.

\subsubsection*{Equivalence relations}
Relation from $A$ to $B$: a subset $R\subseteq A\times B$.\\
If $(a,b)\in R$ we write $\R{a}{b}$.\\
Domain: $\{a\in A : \R{a}{b} \text{ for some } b \in B\}$\\
Range: $\{b\in A : \R{a}{b} \text{ for some } a\in A\}$\\
Reflexive: $\R{a}{a}$ for all $a\in A$\\
Symmetric: $\R{a}{b}\to \R{b}{a}$ for all $a,b,\in A$\\
Transitive: $\R{a}{b} \wedge \R{b}{c} \to \R{a}{c}$  for all $a,b,c\in A$.\\
Equivalence relation: reflexive, symmetric, and transitive.\\
Equivalence class: $[a] = \{b\in A \,|\, \R{b}{a}\}$.\\
An example of an equivalence relation on $\mathbb{Z}$ is congruence modulo $n$.\\
Given $n\in\mathbb{N}$, we define $\mathbb{Z}_n = \{[0],[1],\ldots,[n-1]\}$ to be the set of equivalence classes with respect to congruence modulo $n$: $[a] = \{b\in \mathbb{Z}\,|\, a\equiv b \pmod{n}\}$.\\
For any $[a],[b]\in\mathbb{Z}_n$, we define $[a]\oplus[b] = [a+b]$ and $[a]\odot[b] = [a\cdot b]$. These are the operations of {\bf modular arithmetic}.

\end{multicols}

\end{document}
