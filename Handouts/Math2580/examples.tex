\documentclass[12pt,letterpaper]{article}
\usepackage[latin1]{inputenc}
\usepackage[margin=1in]{geometry}
\usepackage{amsmath}
\usepackage{amsfonts}
\usepackage{amssymb}
\usepackage{amsthm}
\newtheorem{theorem}{Theorem}
\newtheorem{rem}[theorem]{Remark}
\newenvironment{remark}{\begin{rem}\rm}{\end{rem}}
\newtheorem{claim}[theorem]{Claim}
\newtheorem{proposition}[theorem]{Proposition}
\newtheorem{conjecture}[theorem]{Conjecture}
\newtheorem{lemma}[theorem]{Lemma}
\newtheorem{corollary}[theorem]{Corollary}
\newtheorem{eg}[theorem]{Example}
\newenvironment{example}{\begin{eg}\rm}{\end{eg}}
\newtheorem{definition}[theorem]{Definition}


\newcommand{\R}{\mathbb{R}}
\newcommand{\D}{\mathbf{D}}
\newcommand{\x}{\mathbf{x}}
\newcommand{\uu}{\mathbf{u}}
\newcommand{\y}{\mathbf{y}}
\newcommand{\h}{\mathbf{h}}
\newcommand{\aaa}{\mathbf{a}}
\renewcommand{\r}{\mathbf{r}}
\newcommand{\dotp}{\boldsymbol{\cdot}}
\newcommand{\len}[1]{\lVert #1\rVert}
\newcommand{\di}{\displaystyle}

\title{Two change of variables examples}
\author{Sean Fitzpatrick}
\begin{document}
\maketitle

Since I didn't have time to finish (or fully explain) the change of variables example at the end of Friday's lecture, I've written it up below so that you can go over the details. (This might be especially useful for those of you whose lunch plans conflict with the lecture time.) I'll throw in one more example as well (possibly because I can't remember which one I was doing in class - I'm pretty sure it was the first one).

\begin{enumerate}
\item Compute $\di \iint_E \left(\frac{y^2}{x^4}+\frac{x^2}{y^4}\right)\,dA$, where $E$ is bounded by $y=x^2$, $y=2x^2$, $x=y^2$, and $x=4y^2$.

\bigskip

\noindent {\bf Solution:} We need to find a region $D\subset \R^2$ and a transformation $T:D\to \R^2$ whose image is $E$. We use the fact that $T$ must map the boundary of $D$ to the boundary of $E$ as a guideline. In particular, note that since $T$ is $C^1$, it must map smooth curves to smooth curves by the chain rule. This tells us that the corners of $E$ must correspond to the corners of $D$, and in particular, that each of the four curves that make up the boundary of $E$ must come from four curves that make up the boundary of $D$. Since we would like the integral over $D$ to be as simple as possible, we try to find a transformation such that $D$ is a rectangle.

Since the sides of a rectangle in the $uv$-plane are given by either $u=\text{constant}$ or $v=\text{constant}$, we try to express the boundary of $E$ in terms of level curves $u(x,y)=c_1, c_2$ and $v(x,y)=d_1,d_2$. Let's look at the curves $y=x^2$ and $y=2x^2$. These both belong to the family of curves $y=cx^2$, or $\dfrac{y}{x^2}=c$, so we set $u(x,y) = \dfrac{y}{x^2}$. The region between these two parabolas is then given by $1\leq u\leq 2$, or $u\in [1,2]$. Similarly, the other two sides of the boundary of $E$, given by $x=y^2$ and $x=4y^2$ both belong to the family of curves $x=dy^2$, or $\dfrac{x}{y^2}=d$. This suggests that we take $v(x,y)=\dfrac{x}{y^2}$, with $1\leq v\leq 4$.

We have now determined a map $S:E\to D=[1,2]\times [1,4]$ given by
\[
S(x,y) = \left(\frac{y}{x^2}, \frac{x}{y^2}\right).
\]
This map is one-to-one and onto (check this), clearly $C^1$, and has Jacobian
\[
J_S(x,y) = \frac{\partial}{\partial x}\left(\frac{y}{x^2}\right)\frac{\partial}{\partial y}\left(\frac{x}{y^2}\right)-\frac{\partial}{\partial x}\left(\frac{x}{y^2}\right)\frac{\partial}{\partial y}\left(\frac{y}{x^2}\right)=\frac{3}{x^2y^2},
\]
which is defined and non-zero on all of $E$. This means that $S=T^{-1}$ for some transformation $T:D\to E$. We can now proceed to compute the integral via change of variables in one of two ways:
\begin{enumerate}
\item Directly, by solving for $x$ and $y$ in terms of $u$ and $v$, which will give us the transformation $T$.

\medskip

From $u=\dfrac{y}{x^2}$ we get $y=ux^2$, so $x=vy^2 = vu^2x^4$. Since $x\neq 0$ on $E$, this gives us $x^{-3} = u^2v$, so $x = u^{-2/3}v^{-1/3}$, and thus $y=ux^2 = u^{-1/3}v^{-2/3}$. The transformation $T$ is thus $T(u,v) = (u^{-2/3}v^{-1/3},u^{-1/3}v^{-2/3})$, and its Jacobian is given by
\[
J_T(u,v) = \frac{\partial}{\partial u}(u^{-2/3}v^{-1/3})\frac{\partial}{\partial v}(u^{-1/3}v^{-2/3})-\frac{\partial}{\partial u}(u^{-1/3}v^{-2/3})\frac{\partial}{\partial v}(u^{-2/3}v^{-1/3})=\frac{1}{3u^2v^2}.
\]
The integral is therefore
\begin{align*}
\iint_E\left(\frac{x^2}{y^4}+\frac{y^2}{x^4}\right)\,dA & = \iint_D\left(v^2+u^2\right)\left| \frac{1}{3u^2v^2}\right|\,du\,dv\\
& = \frac{1}{3}\int_1^4\int_1^2 \left(\frac{1}{u^2}+\frac{1}{v^2}\right)\,du\,dv\\
& = \frac{1}{3}\int_1^4\left(\frac{-1}{2}-\frac{-1}{1} +\frac{1}{v^2}\right)\, dv\\
& = \frac{1}{3}\left(\frac{1}{2}(4-1)-\frac{1}{4}+\frac{1}{1}\right)\\
& = \frac{3}{4}.
\end{align*}

\item Indirectly, using the fact that $J_T(u,v) = \dfrac{1}{J_{T^{-1}}(x(u,v),y(u,v))}$.

\medskip

From the above, we have that $J_{T^{-1}}(x,y) = \frac{3}{x^2y^2}$, so $J_T(u,v) = \frac{1}{3}(x(u,v))^2(y(u,v))^2$. From $u=\dfrac{y}{x^2}$ and $v=\dfrac{x}{y^2}$, we have $uv = \dfrac{xy}{x^2y^2} = \dfrac{1}{xy}$. Thus, $x^2y^2 = \dfrac{1}{u^2v^2}$, so $J_T(u,v) = \dfrac{1}{3u^2v^2}$ as before. From here we can proceed as in part (a).
\end{enumerate}

\bigskip

\item Compute $\di \iint_E xy \, dA$, where $E$ is bounded by $y=x$, $y=4x$, $y=1/x$, and $y=2/x$.

\bigskip

\noindent {\bf Solution:} We need to find a region $D\subset \R^2$ and a transformation $T:D\to \R^2$ whose image is $E$. Using the principle that $T$ must map the boundary of $D$ to the boundary of $E$ as above, we set $u=\dfrac{y}{x}$, so that $1\leq u\leq 4$ gives the region between $y=x$ and $y=4x$, and $v=xy$, so that $1\leq v\leq 2$ gives the region between $y=1/x$ and $y=2/x$. Thus the desired transformation is defined on the rectangle $D = [1,4]\times [1,2]$ and has an inverse given by $T^{-1}(x,y) = (y/x,xy)$.

This time I'll leave the direct method (solving for $x$ and $y$ in terms of $u$ and $v$) as an exercise (possibly because it's more or less an assignment problem) and use the interect method. The Jacobian of $T^{-1}$ is given by
\[
J_{T^{-1}}(x,y) = \det\begin{pmatrix}
\dfrac{\partial}{\partial x}\left(\dfrac{y}{x}\right)&\dfrac{\partial }{\partial y}\left(\dfrac{y}{x}\right)\\ & \\ \dfrac{\partial }{\partial x}(xy)& \dfrac{\partial }{\partial y}(xy)
\end{pmatrix} = \det\begin{pmatrix}
\dfrac{-y}{x^2}&\dfrac{1}{x}\\ & \\ y & x
\end{pmatrix} = \dfrac{-2y}{x}.
\]
The Jacobian of $T$ is thus $J_T(u,v) = \dfrac{1}{J_T^{-1}(x(u,v),y(u,v))} = -\dfrac{x(u,v)}{2y(u,v)} = -\dfrac{1}{2u}$, since $u=y/x$. The integral is thus
\begin{align*}
\iint_E xy\, dA & = \iint_D x(u,v)y(u,v)\lvert J_T(u,v)\rvert \,du\,dv\\
& = \int_1^2\int_1^4 v\left(\frac{1}{2u}\right)\,du\,dv\\
& = \int_1^2 \frac{v}{2}(\ln 4-\ln 1)\,dv\\
& = \frac{\ln 4}{4}(2^2-1^1) = \frac{3}{4}\ln 4=\frac{3}{2}\ln 2.
\end{align*}
\end{enumerate}
\end{document}
