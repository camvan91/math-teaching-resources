\documentclass[12pt]{article}
\usepackage{amsmath}
\usepackage{amssymb}
\usepackage[letterpaper,top=1.2in,bottom=1in,left=0.75in,right=0.75in,centering]{geometry}
%\usepackage{fancyhdr}
\usepackage{enumerate}
%\usepackage{lastpage}
\usepackage{multicol}
\usepackage{graphicx}

\reversemarginpar

%\pagestyle{fancy}
%\cfoot{}
%\lhead{Math 1560}\chead{Test \# 1}\rhead{May 18th, 2017}
%\rfoot{Total: 10 points}
%\chead{{\bf Name:}}
\newcommand{\points}[1]{\marginpar{\hspace{24pt}[#1]}}
\newcommand{\skipline}{\vspace{12pt}}
%\renewcommand{\headrulewidth}{0in}
\headheight 30pt

\newcommand{\di}{\displaystyle}
\newcommand{\abs}[1]{\lvert #1\rvert}
\newcommand{\len}[1]{\lVert #1\rVert}
\renewcommand{\i}{\mathbf{i}}
\renewcommand{\j}{\mathbf{j}}
\renewcommand{\k}{\mathbf{k}}
\newcommand{\R}{\mathbb{R}}
\newcommand{\aaa}{\mathbf{a}}
\newcommand{\bbb}{\mathbf{b}}
\newcommand{\ccc}{\mathbf{c}}
\newcommand{\dotp}{\boldsymbol{\cdot}}
\newcommand{\bbm}{\begin{bmatrix}}
\newcommand{\ebm}{\end{bmatrix}}                   
                  
\begin{document}


\author{Instructor: Sean Fitzpatrick}
\thispagestyle{empty}
\vglue1cm
\begin{center}
\emph{University of Lethbridge}\\
Department of Mathematics and Computer Science\\
{\bf MATH 2565 - Tutorial \#8}\\
Thursday, March 8
\end{center}
\skipline \skipline \skipline \noindent \skipline

\skipline
First Name:\underline{\hspace{348pt}}\\
\skipline

\vspace{1cm}

Last Name:\underline{\hspace{351pt}}
%Student Number:\underline{\hspace{322pt}}\\
%\skipline



\vspace*{\fill}

There's already plenty of practice on the following pages, but if you want more, you can always go to the textbook.





\newpage
%\thispagestyle{empty}





 \begin{enumerate}
\item Use the ratio or root test to determine whether the series is convergent or divergent. (If the test is inconclusive, or impractical, determine converge with another test.)
\begin{enumerate}
\item $\di\sum_{n=1}^\infty n\left(\frac{3}{5}\right)^n$

\vspace{1in}

\item $\di\sum_{n=1}^\infty \frac{(2n)!}{(n!)^2}$

\vspace{1in}

\item $\di\sum_{n=2}^\infty\left(\frac{1}{n}-\frac{1}{n^2}\right)^n$

\vspace{1in}

\item $\di\sum_{n=1}^\infty\frac{5^n+n^4}{7^n+n^2}$

\vspace{1in}

\item $\di\sum_{n=1}^\infty\left(1+\frac{1}{n}\right)^{n^2}$
\end{enumerate}
\newpage

\item Determine if the series converges conditionally, or absolutely, or not at all:
\begin{enumerate}
\item $\di\sum_{n=1}^\infty\frac{\sin(n\pi/3)}{1+n\sqrt{n}}$

\vspace{1in}

\item $\di\sum_{n=1}^\infty\frac{(-1)^n}{\sqrt{n}}$

\vspace{1in}

\item $\di\sum_{n=1}^\infty n\cos(\pi n)\sin(1/n)$
\end{enumerate}

\vspace{1.5in}

\item One can show that $\di \pi = \sum_{n=0}^\infty \frac{4(-1)^n}{2n+1}$. What is the least value of $N$ such that the partial sum $\di S_N=\sum_{n=1}^N \frac{4(-1)^n}{2n+1}$ approximates the value of $\pi$, correct to 3 decimal places.


\newpage

\item For each series below, indicate whether it converges or diverges. Also indicate which convergence test you used, and why.
\begin{enumerate}
\item $\di \sum_{n=1}^\infty me^{-m}$

\vspace{2in}

\item $\di \sum_{n=1}\infty \frac{2(n^2+2)^{2018}}{3(n^3+n+3)^{2018}}$

\vspace{2in}

\item $\di \sum_{k=1}^\infty \frac{(k!)^k}{k^{4k}}$
\end{enumerate}
\end{enumerate}
\end{document}