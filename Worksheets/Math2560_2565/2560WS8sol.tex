\documentclass[12pt]{article}
\usepackage{amsmath}
\usepackage{amssymb}
\usepackage[letterpaper,margin=0.85in,centering]{geometry}
\usepackage{fancyhdr}
\usepackage{enumerate}
\usepackage{lastpage}
\usepackage{multicol}
\usepackage{graphicx}

\reversemarginpar

\pagestyle{fancy}
\cfoot{}
\lhead{Math 2560}\chead{Worksheet \# 8 Solutions}\rhead{Thursday 17\textsuperscript{th} March, 2016}
%\rfoot{Total: 10 points}
%\chead{{\bf Name:}}
\newcommand{\points}[1]{\marginpar{\hspace{24pt}[#1]}}
\newcommand{\skipline}{\vspace{12pt}}
%\renewcommand{\headrulewidth}{0in}
\headheight 30pt

\newcommand{\di}{\displaystyle}
\newcommand{\abs}[1]{\lvert #1\rvert}
\newcommand{\len}[1]{\lVert #1\rVert}
\renewcommand{\i}{\mathbf{i}}
\renewcommand{\j}{\mathbf{j}}
\renewcommand{\k}{\mathbf{k}}
\newcommand{\R}{\mathbb{R}}
\newcommand{\aaa}{\mathbf{a}}
\newcommand{\bbb}{\mathbf{b}}
\newcommand{\ccc}{\mathbf{c}}
\newcommand{\dotp}{\boldsymbol{\cdot}}
\newcommand{\bbm}{\begin{bmatrix}}
\newcommand{\ebm}{\end{bmatrix}}                   
                  
\begin{document}


%\author{Instructor: Sean Fitzpatrick}
\thispagestyle{fancy}
%\noindent{{\bf Name and student number:}}
The problems on this worksheet are for in-class practice during tutorial. You are free to collaborate and to ask for help. They don't count for course credit, but it's a good idea to make sure you know how to do everything before you leave tutorial -- similar problems may show up on a test or assignment.

\begin{enumerate}
 \item Eliminate the parameter to obtain an equation for the curve involving only $x$ and $y$:
\begin{enumerate}
 \item $x=\sec t$, $y=\tan t$

\medskip

Since $\tan^2 t+1 = \sec^2 t$, we immediately get $y^2+1=x^2$, or $x^2-y^2=1$, which is the standard unit hyperbola.

 \item $x=4\sin t+1$, $y=3\cos t-2$ (Hint: first solve for $\cos t$ and $\sin t$.)

\medskip

We have $\sin t = \dfrac{x-1}{4}$ and $\cos t = \dfrac{y+2}{3}$, giving us the equation
\[
\dfrac{(x-1)^2}{4^2}+\dfrac{(y+2)^2}{3}=1, 
\]
which is an ellipse centred at $(1,-2)$, with semimajor axis of length 4, and semiminor axis of length 3.

 \item $x=\dfrac{1}{t+1}$, $y=\dfrac{3t+5}{t+1}$. (Hint: try doing long division on the expression for $y$.)

\medskip

It's possible to solve for $t$ in terms of $x$ in the first equation and substitute into the second, but it's faster to notice that
\[
 y = \dfrac{3t+5}{t+1} = 3+2\left(\frac{1}{t+1}\right) = 3+2x.
\]
Thus, we have the line $y=3+2x$. Note however that for the given parameterization, we have $x\neq 0$ for all $t$, so there is a hole in the line at the point $(0,3)$.

 \item $x=\cosh t, y=\sinh t$

\medskip

Thanks to the identity $\cosh^2t-\sinh^2t=1$, we immediately get the equation $x^2-y^2=1$ of the standard unit hyperbola. Note however that $\cosh t>0$ for all $t\in\R$, so this parameterization only gives us the right half of the hyperbola. (The left half is given by $x=-\cosh t, y=\sinh t$.)
\end{enumerate}
 
 \item Find the length of the parametric curve:
\begin{enumerate}
 \item $x=-3\sin(2t)$, $y=3\cos(2t)$, $t\in [0,\pi]$.

\bigskip

For a parametric curve we have $dx = x'(t)\,dt$ and $dy = y'(t)\,dt$, which gives $ds = \sqrt{x'(t)^2+y'(t)^2}\,dt$, and the arc length is given by $s = \int_0^\pi ds$ as usual.

In this case $x'(t) = -6\cos(2t)$ and $y'(t) = -6\sin(2t)$, so $x'(t)^2+y'(t)^2 = 36$. Thus, we have
\[
 s = \int_0^\pi \sqrt{36}\,dt = 6\pi.
\]

 \item $x=e^{t/10}\cos t, y=e^{t/10}\sin t$, $t\in [0,2\pi]$.

\bigskip

We use the same procedure as the previous problem. We have
\begin{align*}
 (x'(t))^2 & = \left(\frac{1}{10}e^{t/10}\cos t-e^{t/10}\sin t\right)^2 = e^{t/5}\left(\frac{1}{100}\cos^2 t - \frac{1}{5}\cos t\sin t + \sin^2t\right)\\
 (y'(t))^2 & = \left(\frac{1}{10}e^{t/10}\sin t+e^{t/10}\cos t\right)^2 = e^{t/5}\left(\frac{1}{100}\sin^2 t + \frac{1}{5}\cos t\sin t + \cos^2t\right),
\end{align*}
so $\sqrt{x'(t)^2+y'(t)^2} = \sqrt{e^{t/5}\left(\frac{1}{100}+1\right)} = \sqrt{101}\dfrac{e^{t/10}}{10}$. Thus,
\[
 s=\int_0^{2\pi} \sqrt{101}\frac{e^{t/10}}{10}\,dt = \sqrt{101}(e^{\pi/5}-1).
\]

\end{enumerate}
 \item Find the area enclosed by the astroid $x=\cos^3 t, y=\sin^3 t$, $t\in [0,2\pi]$. (There is some work involved here to evaluate the integral.)

\bigskip

For the given parameterization, the astroid is traced out in a counterclockwise direction. It follows that the area is given by
\begin{align*}
 A & = -\int_0^{2\pi} y\,dx = -\int_0^{2\pi} \sin^3 t(-3\cos^2t\sin t)\,dt\\
& = 3\int_0^{2\pi}\sin^4 t\cos^2 t\,dt\\
& = 3\int_0^{2\pi}\left(\frac{1-\cos(2t)}{2}\right)^2\left(\frac{1+\cos(2t)}{2}\right)\,dt\\
& = \frac{3}{8}\int_0^{2\pi}(1-\cos(2t)-\cos^2(2t)+\cos^3(2t))\,dt\\
& = \frac{3}{8}\int_0^{2\pi}\left(\frac{1}{2}-\frac{1}{2}\cos(4t)-\cos(2t)\sin^2(2t)\right)\,dt\\
& = \frac{3\pi}{8}.
\end{align*}

 \item Find the area enclosed by the loop of the ``teardrop'' curve $x=t(t^2-1), y=t^2-1$. (See Figure 5.34 in the text.)

\bigskip

We first note that $x = t(t-1)(t+1)$, so $x=0$ for $t=0,1,-1$, while $y=0$ for $t=1, -1$. It follows (referring to the figure) that the loop begins at $(0,0)$ when $t=-1$, and ends at $(0,0)$ when $t=1$. We check that $x>0$ for $-1<t<0$ and $x<0$ for $0<t<1$, which tells us that the loop is traversed in the clockwise direction. The area is thus given by
\begin{align*}
 A & = \int_{-1}^1 y\,dx = \int_{-1}^1 (t^2-1)(3t^2-1)\,dt \text{ (Note that $x(t)=t^3-t$, so $x'(t) = 3t^2-1$.)}\\
& = 2\int_0^1 (3t^4-4t^2+1)\,dt\\
& = 2\left(\frac{3}{5}-\frac{4}{3}+1\right) = \frac{8}{15}.
\end{align*}

 \item Verify that $x=Ce^{-t}+De^{2t}$ is a solution to $x''-x'-2x=0$.\label{a}

\bigskip

We have
\begin{align*}
 x(t) & = Ce^{-t}+De^{2t}\\
 x'(t) & = -Ce^{-t}+2De^{2t}\\
 x''(t) & = Ce^{-t}+4De^{2t},
\end{align*}
so $x''-x'-2x = e^{-t}(C-(-C)-2C)+e^{2t}(4D-2D-2D) = 0$, as required.

 \item Find the solution from Problem \ref{a} that satisfies $x(0)=3$ and $x'(0)=-2$.

\bigskip

Setting $x(0)=3$ gives us $C+D=3$. Setting $x'(0)=-2$ gives us $-C+2D=-2$. We have two equations in the unknowns $C$ and $D$, which can easily be solved to give us $C=\dfrac{8}{3}$ and $D=\dfrac{1}{3}$.

 \item Solve $y'=y^3$ when $y(0)=1$. (Hint: $\dfrac{1}{y'} = \dfrac{dx}{dy}$.)

\bigskip

There are two ways to solve this differential equation. The first follows the hint: 

We first note that $y(x)=0$ is a solution. If we assume that $y\neq 0$, we can write
\[
 \frac{1}{y'} = \frac{dx}{dy} = \frac{1}{y^3} = y^{-3}.
\]
Here we're assuming that $y=f(x)$, where $f$ has an inverse, so we can write $x=f^{-1}(y)$. (This may not be globally true, but it is true on any open interval that does not contain a critical point of $f$.)

If $\dfrac{dx}{dy} = y^{-3}$, then taking the antiderivative gives us $x = -\dfrac{1}{2y^2}+C$, so $y^2=\dfrac{1}{2C-2x}$. This leaves us with the problem of whether to take the positive or negative square root to solve for $y$, but the initial condition $y(0)=1>0$ tells us that we must take the postitive square root. Applying the initial condition gives us
\[
 1^1 = 1 = \frac{1}{2C},
\]
so $C=\frac{1}{2}$, and thus $y=\dfrac{1}{\sqrt{1-2x}}$.

The other approach is to treat the equation as a separable equation. From $\dfrac{dy}{dx}=y^3$ we have $\dfrac{dy}{y^3}=dx$, and integrating both sides gives us $-\dfrac{1}{y^2} = x+C$. The remainder of the solution is as above.

 \item Solve $\dfrac{dx}{dt} = x\sin(t)$ for $x(0)=1$.

\bigskip

We have a separable differential equation, which can be written as $\dfrac{dx}{x} = \sin t\,dt$. Integrating both sides gives us $\ln x = -\cos t+C$. We can solve now for $x$ as a function of $t$ but it's convenient to first apply the initial condition: when $t=0$ we have $x=1$, so
\[
 \ln (1) = 0 = -\cos(0)+C,
\]
which gives us $C=1$. Thus $\ln x = 1-\cos t$, so $x = e^{1-\cos t}$.
\end{enumerate}





\end{document}