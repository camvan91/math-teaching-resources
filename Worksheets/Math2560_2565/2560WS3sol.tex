\documentclass[12pt]{article}
\usepackage{amsmath}
\usepackage{amssymb}
\usepackage[letterpaper,margin=0.85in,centering]{geometry}
\usepackage{fancyhdr}
\usepackage{enumerate}
\usepackage{lastpage}
\usepackage{multicol}
\usepackage{graphicx}

\reversemarginpar

\pagestyle{fancy}
\cfoot{}
\lhead{Math 2560}\chead{Worksheet \# 3}\rhead{Thursday 4\textsuperscript{th} February, 2016}
%\rfoot{Total: 10 points}
%\chead{{\bf Name:}}
\newcommand{\points}[1]{\marginpar{\hspace{24pt}[#1]}}
\newcommand{\skipline}{\vspace{12pt}}
%\renewcommand{\headrulewidth}{0in}
\headheight 30pt

\newcommand{\di}{\displaystyle}
\newcommand{\abs}[1]{\lvert #1\rvert}
\newcommand{\len}[1]{\lVert #1\rVert}
\renewcommand{\i}{\mathbf{i}}
\renewcommand{\j}{\mathbf{j}}
\renewcommand{\k}{\mathbf{k}}
\newcommand{\R}{\mathbb{R}}
\newcommand{\aaa}{\mathbf{a}}
\newcommand{\bbb}{\mathbf{b}}
\newcommand{\ccc}{\mathbf{c}}
\newcommand{\dotp}{\boldsymbol{\cdot}}
\newcommand{\bbm}{\begin{bmatrix}}
\newcommand{\ebm}{\end{bmatrix}}                   
                  
\begin{document}


%\author{Instructor: Sean Fitzpatrick}
\thispagestyle{fancy}
%\noindent{{\bf Name and student number:}}
The problems on this worksheet are for in-class practice during tutorial. You are free to collaborate and to ask for help. They don't count for course credit, but it's a good idea to make sure you know how to do everything before you leave tutorial -- similar problems may show up on a test or assignment.

\bigskip

This week I've tried to make my best guess at what your test on Friday might look like.

\begin{enumerate}
 \item Evaluate the following ``immediate integrals'':

\begin{enumerate}
 \item $\di \int (2x+3)^4\,dx = \frac{1}{10}(2x+3)^5+C$

\bigskip

 \item $\di \int \frac{x^2+2x}{x^3+3x^2+5}\,dx = \frac{1}{3}\ln(x^3+3x^2+5)+C$

\bigskip

 \item $\di \int \tan^5(x)\sec^2(x)\,dx = \frac{1}{6}\tan^6(x)+C$

\bigskip

 \item $\di \int \frac{\ln(\sqrt{x+1})}{\sqrt{x+1}}\,dx = 2\sqrt{x+1}\ln\sqrt{x+1}-2\sqrt{x+1}+C$\\
(Okay, this one is only ``immediate'' if you remembered that $\int \ln u\,du = u\ln u-u+C$, which is obtained using integration by parts.)

\bigskip

 \item $\di \int \frac{1}{\sqrt{4-x^2}}\,dx = \sin^{-1}(x/2)+C$

\bigskip

 \item $\di \int \frac{x^3-4x^2}{\sqrt{x}}\,dx = \int(x^{5/2}-4x^{3/2})\,dx = \frac{2}{7}x^{7/2}-\frac{8}{5}x^{5/2}+C$

\bigskip

 \item $\di \int \frac{e^x+1}{e^x}\,dx = =\int (1-e^{-x})\,dx = x+e^{-x}+C$

\bigskip

 \item $\di \int \frac{\ln(x^3)}{x}\,dx = 3\int \frac{\ln(x)}{x}\,dx = 3(\ln(x))^2+C$ 

\bigskip

 \item $\di \int x(1-x^2)^5\,dx = -\frac{1}{12}(1-x^2)^6+C$

\bigskip

 \item $\di \int 3x^2\cos(x^3)e^{\sin(x^3)}\,dx = e^{\sin(x^3)}+C$
\end{enumerate}

\newpage

\item Evaluate the following integrals:
\begin{enumerate}
 \item $\di \int x\sec^2(x)\,dx = \int x d(\tan x) = x\tan x-\int \tan x\,dx = x\tan x+\ln\abs{\cos(x)}+C$


 \item $\di \int e^{\sqrt{x}}\,dx$

\medskip

First let $x=u^2$, so $dx=2u\,du$, giving us
\begin{align*}
 \int e^{\sqrt{x}}\,dx &= \int 2ue^u\,du = 2\int u d(e^u) = 2ue^2-2\int e^u\,du = 2ue^u-2e^u+C\\
& = 2\sqrt{x}e^{\sqrt{x}}-2e^{\sqrt{x}}+C.
\end{align*}



 \item $\di \int \cos(x)\cos(2x)\,dx = \int \cos(x)(1-2\sin^2x)\,dx = \sin(x)-\frac{2}{3}\sin^3(x)+C.$


 \item $\di \int \tan^5(x)\sec^4(x)\,dx = \int \tan^5(x)(1+\tan^2(x))\sec^2(x)\,dx = \frac{1}{6}\tan^6(x)+\frac{1}{8}\tan^8(x)+C$



 \item $\di \int \frac{8}{\sqrt{x^2+2}}\,dx$

\medskip

Letting $x=\sqrt{2}\tan\theta$, we have $\sqrt{x^2+2} = \sqrt{2\sec^2\theta} = \sqrt{2}\sec\theta$ and $dx = \sqrt{2}\sec^2\theta\,d\theta$, so
\[
 \int\frac{8}{\sqrt{x^2+2}}\,dx = \int \frac{8\sqrt{2}\sec^2\theta}{\sqrt{2}\sec\theta}\,d\theta = 8\ln\abs{\sec\theta+\tan\theta}+C = 8\ln\abs{x+\sqrt{x^2+2}}+C.
\]



 \item $\di \int \frac{\sqrt{5-x^2}}{x^2}\,dx$

\medskip

Letting $x=\sqrt{5}\sin\theta$, so $\sqrt{5-x^2} = \sqrt{5}\cos\theta$ and $dx = \sqrt{5}\cos\theta\,d\theta$, we have
\begin{align*}
 \int\frac{\sqrt{5-x^2}}{x^2}\,dx &= \int \frac{5\cos^2\theta}{5\sin^2\theta}\,d\theta = \int\cot^2\theta\,d\theta = \int (\csc^2\theta-1)\,d\theta\\
 & = -\cot\theta-\theta+C = -\frac{\sqrt{5-x^2}}{x}-\sin^{-1}\left(\frac{x}{\sqrt{5}}\right)+c
\end{align*}


 \item $\di \int \frac{16x^2-2x}{(x+3)(2x-1)(x-1)}\,dx$

We look for a partial fraction decomposition 
\[
\dfrac{16x^2-2x}{(x+3)(2x-1)(x-1)} = \dfrac{A}{x+3}+\dfrac{B}{2x-1}+\dfrac{C}{x-1}. 
\]


Multiplying both sides of this decomposition by $x+3$ gives us 
\[
 \dfrac{16x^2-2x}{(2x-1)(x-1)} = A +\dfrac{B(x+3)}{2x-1}+\dfrac{C(x+3)}{x-1}.
\]
 Plugging in $x=-3$ then gives
$A=\dfrac{75}{14}$.

Multiplying both sides of the decomposition by $2x-1$ gives 
\[
 \dfrac{16x^2-2x}{(x+3)(x-1)} = \dfrac{A(2x-1)}{x+3}+B+\dfrac{C(2x-1)}{x-1},
\]
 and plugging in $x=1/2$ gives $B = \dfrac{12}{7}$.

Multiplying both sides of the decomposition by $x-1$ gives 
\[
 \dfrac{16x^2-2x}{(x+3)(2x-1)} = \dfrac{A(x-1)}{x+3}+\dfrac{B(x-1)}{2x-1}+C,
\]
 and plugging in $x=1$ gives $C=\dfrac{7}{2}$.

Putting everything together, we get
\begin{align*}
 \int \frac{16x^2-2x}{(x+3)(2x-1)(x-1)}\,dx &= \frac{75}{14}\int\frac{1}{x+3}\,dx +\frac{12}{7}\int\frac{1}{2x-1}\,dx + \frac{7}{2}\int\frac{1}{x-1}\,dx\\
& = \frac{75}{14}\ln\abs{x+3}+\frac{6}{7}\ln\abs{2x-1}+\frac{7}{2}\ln\abs{x-1}+C.
\end{align*}


 \item $\di \int \frac{2x+1}{x^3+x}\,dx$

This time we look for a decomposition $\dfrac{2x+1}{x(x^2+1)} = \dfrac{A}{x}+\dfrac{Bx+C}{x^2+1}$. Getting a common denominator on the right-hand side, we have
\[
 \frac{2x+1}{x^3+x} = \frac{Ax^2+A+Bx^2+Cx}{x^3+x}.
\]
Comparing numerators, we have $0x^2+2x+1 = (A+B)x^2+Cx+A$. Constant terms must be equal, so $A=1$, Coefficients of $x$ must be equal, so $C=2$. Coefficients of $x^2$ must be equal, so $A+B=0$, giving $B=-A=-1$. Thus,
\[
 \int\frac{2x+1}{x^3+x}\,dx = \int \frac{1}{x}\,dx -\int\frac{x}{x^2+1}\,dx + 2\int \frac{1}{x^2+1}\,dx = \ln\abs{x}-\frac{1}{2}\ln(x^2+1)+2\tan^{-1}(x)+C.
\]


\end{enumerate}


These won't be on your test, but I thought I should give you a couple of practice problems involving improper integrals.

\bigskip


\item Evaluate the following improper integrals, or explain why they do not exist:
\begin{enumerate}
 \item $\di \int_0^4\frac{1}{\sqrt{x}}\,dx$

\bigskip

Since the integrand isn't defined at $x=0$, we have the improper integral
\[
 \int_0^4 x^{-1/2}\,dx= \lim_{a\to 0}\int_a^4x^{-1/2}\,dx = \lim_{a\to 0}(2(\sqrt{4}-\sqrt{a})) = 4.
\]


 \item $\di \int_1^\infty \frac{\ln(x)}{x^2}\,dx$

\bigskip

See Example 46 on Page 53 of the textbook.
\end{enumerate}

\end{enumerate}


\end{document}