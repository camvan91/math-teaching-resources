\documentclass[12pt]{article}
\usepackage{amsmath}
\usepackage{amssymb}
\usepackage[letterpaper,top=0.85in,bottom=1in,left=0.75in,right=0.75in,centering]{geometry}
%\usepackage{fancyhdr}
\usepackage{enumerate}
%\usepackage{lastpage}
\usepackage{multicol}
\usepackage{graphicx}
\usepackage{polynom}

\reversemarginpar

%\pagestyle{fancy}
%\cfoot{}
%\lhead{Math 1560}\chead{Test \# 1}\rhead{May 18th, 2017}
%\rfoot{Total: 10 points}
%\chead{{\bf Name:}}
\newcommand{\points}[1]{\marginpar{\hspace{24pt}[#1]}}
\newcommand{\skipline}{\vspace{12pt}}
%\renewcommand{\headrulewidth}{0in}
\headheight 30pt

\newcommand{\di}{\displaystyle}
\newcommand{\abs}[1]{\lvert #1\rvert}
\newcommand{\len}[1]{\lVert #1\rVert}
\renewcommand{\i}{\mathbf{i}}
\renewcommand{\j}{\mathbf{j}}
\renewcommand{\k}{\mathbf{k}}
\newcommand{\R}{\mathbb{R}}
\newcommand{\aaa}{\mathbf{a}}
\newcommand{\bbb}{\mathbf{b}}
\newcommand{\ccc}{\mathbf{c}}
\newcommand{\dotp}{\boldsymbol{\cdot}}
\newcommand{\bbm}{\begin{bmatrix}}
\newcommand{\ebm}{\end{bmatrix}}                   
                  
\begin{document}


\author{Instructor: Sean Fitzpatrick}
\thispagestyle{empty}
\vglue1cm
\begin{center}
{\bf MATH 2565 - Tutorial \#3 Solutions}\\

\end{center}

\textbf{Additional practice problems}:

\begin{enumerate}
 \item $\di \int \frac{x}{\sqrt{x^2-3}}\,dx$

\medskip

Letting $u=x^2-3$, we have $\frac{1}{2}du = x\,dx$, so
\[
 \int\frac{x}{\sqrt{x^2-3}}\,dx = \frac{1}{2}\int u^{-1/2}\,du = u^{1/2}+C = \sqrt{x^2-3}+C.
\]
Alternatively, you can let $x=\sqrt{3}\sec\theta$, so $dx = \sqrt{3}\sec\theta\tan\theta\,d\theta$ and $\sqrt{x^2-3} = \sqrt{3}\sqrt{\sec^2\theta-1} = \sqrt{3}\tan\theta$, and then
\[
 \int\frac{x}{\sqrt{x^2-3}}\,dx = \int \frac{\sqrt{3}\sec\theta}{\sqrt{3}\tan\theta}(\sqrt{3}\sec\theta\tan\theta)d\theta = \sqrt{3}\int\sec^2\theta\,d\theta = \sqrt{3}\tan\theta+C
\]
From our work above we see that $\sqrt{3}\tan\theta = \sqrt{x^2-3}$, and so we get the same answer as above.

If you want yet another option, try letting $x=\sqrt{3}\cosh(t)$, so $dx = \sqrt{3}\sinh(t)\,dt$ and $\sqrt{x^2-3} = \sqrt{3}\sqrt{\cosh^2(t)-1} = \sqrt{3}\sinh(t)$. With these substitutions, the integral becomes $\di \sqrt{3}\int\cosh(t)\,dt = \sqrt{3}\sinh(t)+C = \sqrt{x^2-3}$, as before. 


 \item $\di \int \frac{x^2}{\sqrt{x^2+4}}\,dx$
 
 \medskip
 
Seeing the pattern $x^2+a^2$, we make a tangent substitution: $x=2\tan\theta$, so $dx = 2\sec^2\theta\,d\theta$ and $\sqrt{x^2+4} = 2\sec\theta$, giving us
\begin{align*}
 \int\frac{x^2}{\sqrt{x^2+4}}\,dx & = \int\frac{4\tan^2\theta}{2\sec\theta}(2\sec^2\theta)\,d\theta\\
 & = 4\int \tan^2\theta\sec\theta\,d\theta = 4\int (\sec^2\theta-1)\sec\theta\,d\theta\\
 & = 4\int \sec^3\theta\,d\theta - 4\int \sec\theta\,d\theta.
\end{align*}
From class, you know that $\int \sec\theta\,d\theta = \ln\abs{\sec\theta+\tan\theta}+C$, and you know that
\[
 \int \sec^3\theta\,d\theta = \frac{1}{2}\sec(\theta)\tan(\theta)+\frac{1}{2}\ln\lvert \sec(\theta)+\tan(\theta)\rvert +C,
\]
so 
\begin{align*}
\di 4\int \sec^3\theta\,d\theta - 4\int \sec\theta\,d\theta &= 4\left(\frac{1}{2}\sec(\theta)\tan(\theta)+\frac{1}{2}\ln\lvert \sec(\theta)+\tan(\theta)\rvert-\ln\lvert \sec(\theta)+\tan(\theta)\rvert\right)+C\\
& = 2\sec\theta\tan\theta-2\ln\abs{\sec\theta+\tan\theta}+C 
\end{align*}
From the substitution work above, we know that $\tan\theta = \frac{x}{2}$, and that $\sec\theta = \frac{1}{2}\sqrt{x^2+4}$. Putting everything together, we get 
\begin{align*}
 \int \frac{x^2}{\sqrt{x^2+4}}\,dx &= 2\left(\left(\frac{1}{2}\sqrt{x^2+4}\right)\left(\frac{x}{2}\right) - \ln\left|\frac{x}{2}+\frac{1}{2}\sqrt{x^2+4}\right|\right)+C\\
 & = \frac{1}{2}x\sqrt{x^2+4}-2\ln\abs{x+\sqrt{x^2+4}}+C,
\end{align*}
where in the last line, I've used the fact that $\ln(u/2) = \ln(u) - \ln(2)$, and absorbed the constant $-\ln(2)$ into the constant of integration.

 
  \item $\di \int \frac{7x-2}{x^2+x}\,dx$
  
  \medskip
  
  Using partial fractions, if
\[
 \frac{7x-2}{x^2+x} = \frac{7x-2}{x(x+1)} = \frac{A}{x}+\frac{B}{x+1} = \frac{A(x+1)+Bx}{x(x+1)},
\]
then we must have $A(x+1)+Bx = 7x-2$. When $x=0$ we get $A=-2$, and when $x=-1$ we get $-B=-9$, so $B=9$. Thus, we have
\[
 \int\frac{7x-2}{x^2+x}\,dx = -2\int\frac{1}{x}\,dx + 9\int\frac{1}{x+1}\,dx = -2\ln\abs{x}+9\ln\abs{x+1}+C = \ln\left|\frac{(x+1)^9}{x^2}\right|+C.
\]

  \item $\di \int \frac{1}{x^3+2x^2+3x}\,dx$  
 
 \medskip
 
 Factoring the denominator, we have 
\[
 x^3+2x^2+3x = x(x^2+2x+3),
\]
where $x^2+2x+3 = (x+1)^2+2$ is an irreducible quadratic. Our partial fraction decomposition is thus
\[
 \frac{1}{x^3+2x^2+3x} = \frac{A}{x}+\frac{Bx+C}{x^2+2x+3} = \frac{A(x^2+2x+3)+(Bx+C)x}{x(x^2+2x+3)}.
\]
Equating numerators gives us $1 = A(x^2+2x+3)+(Bx+C)x$. Setting $x=0$ gives us $1=3A$, so $A=\frac{1}{3}$. Since $x^2+2x+3$ has no real roots, there isn't any $x$ value we can plug in to make the $A$ term vanish. Instead, we put $x=1$, giving us $1=\frac{1}{3}(1+2+3)+(B+C)(1)$, so $B+C=1-2 = -1$. Putting $x=-2$ gives us $1=\frac{1}{3}(4-4+3)+(-2B+C)(-2)$, so $4B-2C+1=1$, which simplifies to $2B-C=0$. (If you're wondering why I chose $x=-2$, it was so $x^2+2x+3$ would be a multiple of 3, allowing me to avoid fractions.)

We're left with the equations $B+C=-1$ and $2B-C=0$. Adding the two equations gives us $3B=-1$, so $B=-\frac{1}{3}$, and thus $C=2B = -\frac{2}{3}$, so 
\[
 \frac{1}{x^3+2x^2+x} = \frac{1}{3}\left(\frac{1}{x}-\frac{x+2}{x^2+2x+3}\right) = \frac{1}{3}\left(\frac{1}{x}-\frac{x+1}{x^2+2x+3}-\frac{1}{x^2+2x+3}\right).
\]
Why did we break up the second fraction into two pieces? Well, for the first piece, if we let $u=x^2+2x+3$, then $du = 2(x+1)\,dx$, so
\[
 \int \frac{x+1}{x^2+2x+3}\,dx = \frac{1}{2}\ln(x^2+2x+3)+C.
\]
For the second piece, writing $x^2+2x+3 = (x+1)^2+2$, we can let $x+1=\sqrt{2}\tan\theta$, so $dx = \sqrt{2}\sec^2\theta\,d\theta$ and $(x+1)^2+2 = 2\sec^2\theta$, so
\[
 \int\frac{1}{x^2+2x+3}\,dx = \frac{1}{\sqrt{2}}\tan^{-1}\left(\frac{x+2}{\sqrt{2}}\right)+C.
\]
Altogether, we have
\[
 \int\frac{1}{x^3+2x^2+x}\,dx = \frac{1}{3}\ln\abs{x}-\frac{1}{6}\ln(x^2+2x+3)-\frac{1}{3\sqrt{2}}\tan^{-1}\left(\frac{x+2}{\sqrt{2}}\right)+C.
\]

 
   \item $\di \int \frac{x+7}{(x+5)^2}\,dx$

\medskip

Again we use partial fractions. Because of the repeated root in the denominator, we write
\[
 \frac{x+7}{(x+5)^2}\,dx = \frac{A}{x+5}+\frac{B}{(x+5)^2} = \frac{A(x+5)+B}{(x+5)^2},
\]
and equating numerators gives us $x+7 = A(x+5)+B$. Putting $x=-5$ immediately gives us $B=2$, and plugging this back in, we have $x+7 = Ax+5A+2$, so we must have $A=1$. Thus,
\[
 \int\frac{x+7}{(x+5)^2}\,dx = \int\frac{1}{x+5}\,dx+2\int(x+5)^{-2}\,dx = \ln\abs{x+5}-2(x+5)^{-1}+C.
\]
   
   \item $\di \int \frac{9x^2+11x+7}{x(x+1)^2}\,dx$
   
   \medskip
   
   Our partial fraction decomposition in this case takes the form
\[
 \frac{9x^2+11x+7}{x(x+1)^2}\,dx = \frac{A}{x}+\frac{B}{x+1}+\frac{C}{(x+1)^2} = \frac{A(x+1)^2+Bx(x+1)+Cx}{x(x+1)^2},
\]
so $A(x+1)^2+Bx(x+1)+Cx = 9x^2+11x+7$. Putting $x=0$ gives us $A=7$ immediately, and putting $x=-1$ gives us $-C=9-11+7=5$, so $C=-5$. This leaves us with $7(x+1)^2+Bx(x+1)-5x=9x^2+11x+7$. To find $B$, we try $x=1$, which gives us $7(4)+2B-5=9+11+7$, so $2B = 27-19 = 8$, giving us $B=4$. Putting everything into the integral, we have
\[
 \int\frac{9x^2+11x+7}{x(x+1)^2}\,dx = \int\left(\frac{7}{x}+\frac{4}{x+1}-\frac{5}{(x+1)^2}\right)\,dx = 7\ln\abs{x}+4\ln\abs{x+1}+\frac{5}{x+1}+C.
\]


\end{enumerate}  

\newpage

\textbf{Assigned problems}:
 \begin{enumerate}


 \item $\di \int x^2\sqrt{1-x^2}\,dx$
 
\medskip

Letting $x=\sin\theta$, $dx=\cos\theta\,d\theta$ and $\sqrt{1-x^2}=\sqrt{1-\sin^2\theta} = \cos\theta$, so we get
\begin{align*}
 \int x^2\sqrt{1-x^2}\,dx &= \int \sin^2\theta \cos\theta (\cos\theta)\,d\theta = \int \sin^2\theta\cos^2\theta\,d\theta\\
& = \int \left(\frac{1-\cos(2\theta)}{2}\right)\left(\frac{1+\cos(2\theta)}{2}\right)\,d\theta\\
& = \frac{1}{4}\int (1-\cos^2(2\theta))\,d\theta = \frac{1}{4}\int \sin^2(2\theta)\,d\theta\\
& = \frac{1}{4}\int \left(\frac{1-\cos(4\theta)}{2}\right)d\theta\\
& = \frac{1}{8}\left(\theta -\frac{1}{4}\sin(4\theta)\right)+C\\
& = \frac{1}{8}\sin^{-1}(x)-\frac{1}{32}\sin(4\sin^{-1}x)+C.
\end{align*}
If you want to simplify that last term, note that $\sin \theta = x$ and $\cos \theta = \sqrt{1-x^2}$, and
\[
 \sin(4\theta) = 2\sin(2\theta)\cos(2\theta) = 4\sin(\theta)\cos(\theta)(\cos^2(\theta)-\sin^2(\theta)) = 4\sin(\theta)\cos^3(\theta)-4\sin^3(\theta)\cos(\theta),
\]
so
\[
 \frac{1}{32}\sin(4\sin^{-1}x) = \frac{1}{8}(x(1-x^2)^{3/2}-x^3(1-x^2)^{1/2}) = \frac{1}{8}x(1-2x^2)\sqrt{1-x^2}.
\]


 \item $\di \int \frac{1}{(x^2+4x+13)^2}\,dx$

\medskip

Completing the square, we have $x^2+4x+13 = x^2+4x+4+9 = (x+2)^2+3^2$, suggesting that we try letting $x+2=3\tan\theta$. This gives us $dx = 3\sec^2\theta\,d\theta$, and
\[
 x^2+4x+13 = (x+2)^2+3^3 = 3^2\tan^2\theta+3^2 = 3^2(\tan^2\theta+1) = 9\sec^2\theta.
\]
Substituting everything into the integral, we get
\begin{align*}
 \int\frac{1}{(x^2+4x+13)^2}\,dx &= \int \frac{1}{81\sec^4\theta}(3\sec^2\theta)\,d\theta\\
 & = \frac{1}{27}\int \cos^2\theta\,d\theta\\
 & = \frac{1}{54}\int (1+\cos(2\theta))\,d\theta\\
 & = \frac{\theta}{54} + \frac{1}{108}\sin(2\theta)+C\\
 & = \frac{\theta}{54} + \frac{1}{54}\sin\theta\cos\theta+C.
\end{align*}
To get everything back in terms of $x$, we note that $\tan\theta = \frac{x+2}{3}$. If we have a right-angled triangle with sides of length $x+2$ (opposite $\theta$) and 3 (adjacent $\theta$), then the hypotenuse has length $\sqrt{(x+2)^2+3^2} = \sqrt{x^2+4x+13}$, and we get $\sin\theta = \dfrac{x+2}{\sqrt{x^2+4x+13}}$ and $\cos\theta = \dfrac{3}{\sqrt{x^2+4x+13}}$. Plugging all of this in, we get the final answer
\[
 \int\frac{1}{(x^2+4x+13)^2}\,dx = \frac{1}{54}\tan^{-1}\left(\frac{x+2}{3}\right) + \frac{1}{18}\frac{x+2}{x^2+4x+13}.
\]


 \item $\di \int \frac{7x+7}{x^2+3x-10}\,dx$

\medskip

We look for a partial fraction decomposition
\[
 \frac{7x+7}{x^2+3x-10} = \frac{7x+7}{(x-2)(x+5)} = \frac{A}{x-2}+\frac{B}{x+5} = \frac{A(x+5)+B(x-2)}{(x-2)(x+5)}
\]
Since the denominators of the first and last terms of the above inequality are equal, the numerators must be equal as well:
\[
 7x+7 = A(x+5)+B(x-2).
\]
Since this equality holds for all values of $x$, it holds in particular when $x=2$ and $x=-5$. Putting $x=2$ gives us $7(2)+7 = A(7)+B(0)$, so $7A=21$ and thus $A=3$. Putting $x=-5$ gives us $7(-5)+7 - A(0)+B(-7)$, so $-7B = -28$, and thus $B=4$. Returning to the integral, we thus have
\begin{align*}
 \int \frac{7x+7}{x^2+3x-10}\,dx &= 3\int\frac{1}{x-2}\,dx + 4\int\frac{1}{x+5}\,dx\\
& = 3\ln\abs{x-2}+4\ln\abs{x+5}+C = \ln\abs{(x-2)^3(x+5)^4}+C.
\end{align*}




 


 \item $\di \int \frac{x^3}{x^2-x-20}\,dx$ (First do long division.)

\medskip

Since the degree of the numerator is not less than that of the denominator, we first perform long division:
\[
 \polylongdiv{x^3}{x^2-x-20}
\]
This tells us that we can write $\dfrac{x^3}{x^2-x-20} = x+1 + \dfrac{21x+20}{x^2-x-20}$, and it remains to perform a partial fraction decomposition on the last term:
\[
 \frac{21x+20}{(x-5)(x+4)} = \frac{A}{x-5}+\frac{B}{x+4} = \frac{A(x+4)+B(x-5)}{(x+4)(x-5)}, 
\]
giving us $21x+20 = A(x+4)+B(x-5)$. If $x=5$, we get $125 = 9A$, so $A = \frac{125}{9}$. If $x=-4$, we get $-64 = -9B$, so $B = \frac{64}{9}$. Thus, we have
\begin{align*}
 \int \frac{x^3}{x^2-x-20}\,dx & = \int\left(x+1 + \frac{125}{9(x-5)}+\frac{64}{9(x+4)}\right)\,dx\\
& = \frac{1}{2}x^2+x+\frac{125}{9}\ln\abs{x-5}+\frac{64}{9}\ln\abs{x+4}+C.
\end{align*}


 \item $\di \int \frac{2x^2+2x+1}{(x+1)(x^2+9)}\,dx$

\medskip

Once more with partial fractions: if
\[
 \frac{2x^2+2x+1}{(x+1)(x^2+9)}\,dx = \frac{A}{x+1}+\frac{Bx+C}{x^2+9} = \frac{A(x^2+9)+(Bx+C)(x+1)}{(x+1)(x^2+9)},
\]
then equating numerators gives us $2x^2+2x+1 = A(x^2+9)+(Bx+C)(x+1)$. Putting $x=-1$, we get $1=A(10)$, so $A=1/10$. Putting $x=0$, we get $1=9A+C$, so $C=1-9/10=1/10$. Finally, putting $x=1$ gives us $5=10A+2(B+C)$, so $2(B+C)=5-10(1/10)=4$, which simplifies to $B+C=2$. Since $C=1/10$, this gives us $B=19/10$. Therefore, we have
\begin{align*}
 \int\frac{2x^2+2x+1}{(x+1)(x^2+9)}\,dx &= \int\left(\frac{1}{10x}+\frac{19x}{10(x^2+9)}+\frac{1}{10(x^2+9)}\right)\,dx\\
& = \frac{1}{10}\ln\abs{x}+\frac{19}{20}\ln(x^2+9)+\frac{1}{30}\tan^{-1}\left(\frac{x}{3}\right)+C.
\end{align*}
 \item $\di \int \frac{1}{\sqrt{x}+\sqrt[3]{x}}\,dx$
 
 \medskip
 
 Here we see that we have both a square root and a cube root. The least common multiple of 2 and 3 being 6, we attempt the rationalizing substitution $x=u^6$, so $dx=6u^5\,du$, and $\sqrt{x}=\sqrt{u^6}=u^3$, while $\sqrt[3]{x}=\sqrt[3]{x^6}=u^2$.\footnote{In case you are concerned about the fact that $\sqrt{u^6}=\abs{u}^3$ in general (you probably weren't, but just in case): generally, to have a well-defined substitution, one must define $x=f(u)$ where $f$ is a one-to-one function. (When we do trig substitution, we officially are working with the restricted trig functions that are used when we define the inverse trig functions.) If the substitution $x=u^6$ is to be one-to-one, we implicitly have the restriction $u\geq 0$, even if we don't state it.}
 
 Making these substitutions, we find
 \[
 \int \frac{1}{\sqrt{x}+\sqrt[3]{x}}\,dx = \int\frac{6u^5}{u^3+u^2}\,du = 6\int\frac{u^3}{u+1}\,du.
 \]
 Using long division, we find that
 \[
 \frac{u^3}{u+1} = u^2-u+1-\frac{1}{u+1},
 \]
 so
 \begin{align*}
 \int \frac{1}{\sqrt{x}+\sqrt[3]{x}}\,dx & = 6\int\frac{u^3}{u+1}\,du\\
 & = 6\int\left(u^2-u+1-\frac{1}{u+1}\right)\,du\\
 & = 2u^3-3u^2+6u-6\ln(u+1)+C\\
 & = 2\sqrt{x}-3\sqrt[3]{x}+6\sqrt[6]{x}-6\ln(\sqrt[6]{x}+1)+C.
 \end{align*}
 \item $\di \int_0^{\pi/2}\frac{\cos(x)}{2-\cos(x)}\,dx$
 
 \medskip
 
 This one is borrowed from Dr. Kaminski's handout, and it's a bit of a workout, so hold onto your hats.
 
 We use the ``tangent half-angle'' substitution $t=\tan(x/2)$, which yields
 \[
 \cos(x) = \frac{1-t^2}{1+t^2} \quad \text{ and } \quad dx = \frac{2}{1+t^2}\,dt.
 \]
Notice that when $x=0$, $t=\tan(0/2)=0$, and when $x=\pi/2$, $t=\tan(\pi/4)=1$. Thus,
\[
\int_0^{\pi/2}\frac{\cos(x)}{2-\cos(x)}\,dx = \int_0^1 \frac{\frac{1-t^2}{1+t^2}}{2-\frac{1-t^2}{1+t^2}}\frac{2}{1+t^2}\,dt.
\]
Cleaning up this mess, we find
\[
\int_0^1 \frac{\frac{1-t^2}{1+t^2}}{2-\frac{1-t^2}{1+t^2}}\frac{2}{1+t^2}\,dt = \int_0^1\frac{2-2t^2}{(1+3t^2)(1+t^2)}\,dt.
\]
The remaining integral requires partial fractions, and it's lots of fun, because there are two irreducible quadratics. Writing
\[
\frac{2-2t^2}{(1+3t^2)(1+t^2)} = \frac{At+B}{1+3t^2}+\frac{Ct+D}{1+t^2},
\]
and then re-writing the right-hand side over a common denominator, we can equate numerators, giving us
\[
2-2t^2 = t^3(A+3C)+t^2(B+2D)+t(A+C)+(B+D).
\]
Comparing coefficients of odd powers, we find $A+3C=0$ and $A+C=0$, which is only possible if $A=C=0$. Comparing powers of even coefficients, we find $B+3D=-2$ and $B+D=2$. Solving these two equations gives us $B=4$ and $D=-2$.

Let's put in those values. We get
\[
\int_0^1\frac{2-2t^2}{(1+3t^2)(1+t^2)}\,dt = \int_0^1\left(\frac{4}{1+3t^2}-\frac{2}{1+t^2}\right)\,dt.
\]
Both of these terms produce arctangent integrals. The second is direct; the first, with a bit of work, produces
\[
\int\frac{1}{1+3t^2}\,dt = \frac{1}{\sqrt{3}}\arctan{\sqrt{3}t}+C.
\]
Now you're probably thinking about how you're going to substitute this back in terms of $x$ but worry not! We had the foresight to adjust the limits of integration when we substituted, so all that remains is to apply the Fundamental Theorem of Calculus:
\begin{align*}
\int_0^{\pi/2}\frac{\cos(x)}{2-\cos(x)}\,dx & = \int_0^1\left(\frac{4}{1+3t^2}-\frac{2}{1+t^2}\right)\,dt \tag{by all our work above}\\
& = \left.\left(\frac{4}{\sqrt{3}}\arctan(\sqrt{3}x)-2\arctan(x)\right)\right|_0^1\\
& = \frac{4}{\sqrt{3}}(\arctan(\sqrt{3})-\arctan(0))-2(\arctan(1)-\arctan(0))\\
& = \frac{4}{\sqrt{3}}\left(\frac{\pi}{3}\right)-2\left(\frac{\pi}{4}\right)\\
& = \pi\left(\frac{4}{3\sqrt{3}}-\frac12\right).
\end{align*}
 \end{enumerate}


\end{document}