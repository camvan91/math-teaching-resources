\documentclass[12pt]{article}
\usepackage{amsmath}
\usepackage{amssymb}
\usepackage[letterpaper,top=0.85in,bottom=1in,left=0.75in,right=0.75in,centering]{geometry}
%\usepackage{fancyhdr}
\usepackage{enumerate}
%\usepackage{lastpage}
\usepackage{multicol}
\usepackage{graphicx}

\reversemarginpar

%\pagestyle{fancy}
%\cfoot{}
%\lhead{Math 1560}\chead{Test \# 1}\rhead{May 18th, 2017}
%\rfoot{Total: 10 points}
%\chead{{\bf Name:}}
\newcommand{\points}[1]{\marginpar{\hspace{24pt}[#1]}}
\newcommand{\skipline}{\vspace{12pt}}
%\renewcommand{\headrulewidth}{0in}
\headheight 30pt

\newcommand{\di}{\displaystyle}
\newcommand{\abs}[1]{\lvert #1\rvert}
\newcommand{\len}[1]{\lVert #1\rVert}
\renewcommand{\i}{\mathbf{i}}
\renewcommand{\j}{\mathbf{j}}
\renewcommand{\k}{\mathbf{k}}
\newcommand{\R}{\mathbb{R}}
\newcommand{\aaa}{\mathbf{a}}
\newcommand{\bbb}{\mathbf{b}}
\newcommand{\ccc}{\mathbf{c}}
\newcommand{\dotp}{\boldsymbol{\cdot}}
\newcommand{\bbm}{\begin{bmatrix}}
\newcommand{\ebm}{\end{bmatrix}}                   
                  
\begin{document}


\author{Instructor: Sean Fitzpatrick}
\thispagestyle{empty}
\vglue1cm
\begin{center}
\emph{University of Lethbridge}\\
Department of Mathematics and Computer Science\\
{\bf MATH 2565 - Tutorial \#4}\\
Thursday, February 1
\end{center}
\skipline \skipline \skipline \noindent \skipline

\skipline
First Name:\underline{\hspace{348pt}}\\
\skipline

\vspace{1cm}

Last Name:\underline{\hspace{351pt}}
%Student Number:\underline{\hspace{322pt}}\\
%\skipline



\vspace{1cm}

\begin{quote}
Print your name clearly in the space above. 

\medskip

Complete the problems on the back of this page to the best of your ability. If there is a problem you especially desire feedback on, please indicate this. 

\medskip

It is recommended that you work out the details on scrap paper before writing your solutions on the worksheet.
\end{quote}

%\vspace{2cm}


Additional practice (don't include your solutions here):
\begin{enumerate}
\item Evaluate the improper integral, if possible:

\begin{multicols}{2}
\begin{enumerate}
 \item $\di \int_0^\infty x^2e^{-2x}\,dx$
 \item $\di \int_0^1 \frac{\sqrt{x}+1}{x}\,dx$
 \end{enumerate}  
\end{multicols}
 \item Determine if the improper integral converges or diverges:
 \begin{multicols}{2}
 \begin{enumerate}
  \item $\di \int_0^\infty \frac{1}{\sqrt{x^3+2x^2+5}}\,dx$
  \item $\di \int_1^\infty e^{-x}\ln(x)\,dx$
 \end{enumerate}
 \end{multicols}
 \item Determine the area bounded by the given curves:
 \begin{multicols}{2}
 \begin{enumerate}
 \item $y=\sqrt{x+2}$, $y=\dfrac{1}{x+1}$, $x=0$ and $x=2$.
 \item $y=2x^2+5x-3$ and $y=x^2+4x-1$.
  \item $y=x$ and $y=x^3$.
   \item $y=x^2+1$, $y=\frac{1}{4}(x-3)^2+1$, and $y=1$.
 \end{enumerate}
 \end{multicols}
\end{enumerate}

\newpage
%\thispagestyle{empty}

\vglue12pt


 \begin{enumerate}

\item Evaluate the improper integral, if possible:
\begin{multicols}{2}
\begin{enumerate}
\item $\di \int_1^\infty\frac{\ln(x)}{x^2}\,dx$



\item $\di \int_0^\infty\frac{1}{e^x+e^{-x}}\,dx$

\end{enumerate}
\end{multicols}
\vspace{3in}



\item Determine whether or not the following improper integrals converge or diverge. Use either direct comparison or limit comparison, as appropriate.
\begin{multicols}{2}
\begin{enumerate}
\item $\di \int_1^\infty \frac{1}{\sqrt{x^2+x}}\,dx$



\item $\di \int_0^1\frac{1}{\sqrt{x^2-x}}\,dx$



\end{enumerate}
\end{multicols}
\newpage

\vglue12pt

\item (For in-class discussion): how do you show that $\di\int_0^\infty \frac{x^n}{e^{x}}\,dx$ exists for any positive integer $n$?


\item Let $p(x)$ be any polynomial function. Does $\di \int_0^\infty \frac{p(x)}{e^x}\,dx$ converge or diverge? Justify your answer.

\vspace{1.75in}

\item Find the area between the given curves:
\begin{enumerate}
\item $y=\cos x$ and $y=\sin 2x$, between $x=0$ and $x=\pi/2$.

\vspace{2.5in}

  \item $y=x$, $y=5x$, and $y=6-x^2$, in the first quadrant. 
\end{enumerate}
\end{enumerate}
\end{document}