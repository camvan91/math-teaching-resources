\documentclass[12pt]{article}
\usepackage{amsmath}
\usepackage{amssymb}
\usepackage[letterpaper,margin=0.85in,centering]{geometry}
\usepackage{fancyhdr}
\usepackage{enumerate}
\usepackage{lastpage}
\usepackage{multicol}
\usepackage{graphicx}

\reversemarginpar

\pagestyle{fancy}
\cfoot{}
\lhead{Math 2560}\chead{Worksheet \# 10 Solutions}\rhead{Thursday 31\textsuperscript{st} March, 2016}
%\rfoot{Total: 10 points}
%\chead{{\bf Name:}}
\newcommand{\points}[1]{\marginpar{\hspace{24pt}[#1]}}
\newcommand{\skipline}{\vspace{12pt}}
%\renewcommand{\headrulewidth}{0in}
\headheight 30pt

\newcommand{\di}{\displaystyle}
\newcommand{\abs}[1]{\lvert #1\rvert}
\newcommand{\len}[1]{\lVert #1\rVert}
\renewcommand{\i}{\mathbf{i}}
\renewcommand{\j}{\mathbf{j}}
\renewcommand{\k}{\mathbf{k}}
\newcommand{\R}{\mathbb{R}}
\newcommand{\aaa}{\mathbf{a}}
\newcommand{\bbb}{\mathbf{b}}
\newcommand{\ccc}{\mathbf{c}}
\newcommand{\dotp}{\boldsymbol{\cdot}}
\newcommand{\bbm}{\begin{bmatrix}}
\newcommand{\ebm}{\end{bmatrix}}                   
                  
\begin{document}


%\author{Instructor: Sean Fitzpatrick}
\thispagestyle{fancy}
%\noindent{{\bf Name and student number:}}


\begin{enumerate}
 \item Calculate $\lim\limits_{n\to\infty}a_n$ to show that the series $\sum a_n$ diverges:

\begin{enumerate}
 \item $\di \sum_{n=1}^\infty \frac{3n^2}{n(n+2)}$
 
 \bigskip
 
 We have $a_n = \frac{3n^2}{n^2+2n}$, and
 \[
 \lim_{n\to\infty}\frac{3n^2}{n^2+2n} = \lim_{n\to\infty}\frac{3}{1+2/n} = \frac{3}{1+0} = 3\neq 0,
 \]
 so the series diverges.
 
 \item $\di \sum_{n=1}^\infty \frac{n!}{10^n}$

\bigskip

 Here $a_n = \frac{n!}{10^n}$, and intuitively we expect that $a_n\to \infty$ since $a_{n+1} = \dfrac{n+1}{10}a_n$, and for $n\geq 10$, $\dfrac{n+1}{10}>1$. One way to see this precisely is to notice that for $n>20$, we have
 \[
 a_n = a_{20}\left(\frac{21}{10}\right)\left(\frac{22}{20}\right)\cdots\left(\frac{n}{10}\right)>a_{20}(2^n)(2^n)\cdots (2^n) = a_{20}\cdot 2^{n-20}.
 \]
 Since $a_{20}$ is a constant and $\lim_{n\to \infty}2^{n-20} =\infty$, we see that $a_n\to \infty$, and since the sequence diverges, the series certainly does.
 

 \item $\di \sum_{n=0}^\infty \frac{2^n}{2^{n+1}+1}$
 
 \bigskip
 
 We have $a_n = \dfrac{2^n}{2^{n+1}+1} = \dfrac{1}{2+2^{-n}}$, so $\di\lim_{n\to\infty}a_n = \frac{1}{2+0} = \frac{1}{2}\neq 0$, and thus the series diverges.
 
\end{enumerate}

 \item Determine if the series diverges or converges. (Each series is a $p$-series, or geometric, or there is an argument involving basic properties of series. See Key Idea 17 on page 126 of the textbook for additional guidance.)

\begin{enumerate}
 \item $\di \sum_{n=1}^\infty\frac{1}{n^5}$
 
 \bigskip
 
This is a $p$-series with  $p=5>1$, so the series converges.
 
 \item $\di \sum_{n=1}^\infty \frac{\sqrt{n}+1}{n^2}$

\bigskip

We have
\[
\sum_{n=1}^\infty \frac{\sqrt{n}+1}{n^2} = \sum_{n=1}\infty \left(\frac{\sqrt{n}}{n^2}+\frac{1}{n^2}\right) = \sum_{n=1}^\infty \frac{1}{n^{3/2}}+\sum_{n=1}^\infty \frac{1}{n^2},
\]
giving us the sum of two $p$-series, with $p=3/2>1$ and $p=2>1$, respectively. Since both of these series converge, the original series converges.

 \item $\di \sum_{n=1}^\infty \frac{3^n}{5^n}$

\bigskip

We have 
\[
\sum_{n=1}^\infty \frac{3^n}{5^n} = \sum_{n=1}^\infty\left(\frac{3}{5}\right)^n,
\]
so this is a geometric series with $r=3/5<1$, which converges. Indeed, in this case we can even say what it converges to:
\[
\sum_{n=1}^\infty\left(\frac{3}{5}\right)^n = \frac{3}{5}\sum_{n=0}^\infty\left(\frac{3}{5}\right)^n = \frac{3}{5}\left(\frac{1}{1-3/5}\right) = \frac{3}{2}.
\]

 \item $\di \sum_{n=1}^\infty \frac{7^n}{6^n}$
 
 This is once again a geometric series, with $r=7/6>1$, so it diverges.
 
 \item $\di \sum_{n=1}^\infty \frac{10}{n!}$
 
 \bigskip
 
 We have $\di \sum_{n=1}^\infty \frac{10}{n!} = 10\sum_{n=1}^\infty \frac{1}{n!} = 10\left(\left(\sum_{n=0}^\infty\frac{1}{n!}\right)-1\right) = 10e-10.$
 
 Here, we've used the fact that $\sum_{n=0}^\infty\frac{1}{n!} = e$ (from Key Idea 17 in the text) and that
 \[
 \sum_{n=0}\infty \frac{1}{n!} = \frac{1}{0!}+\sum_{n=1}^\infty \frac{1}{n!} = 1+ \sum_{n=1}^\infty \frac{1}{n!}
 \]
 
 \item $\di \sum_{n=1}^\infty\left(\frac{1}{n!}+\frac{1}{n}\right)$
 
 \bigskip
 
 We can write the above series as the sum of two series, the second of which is the harmonic series, $\sum\frac{1}{n}$. Since we know that the harmonic series is divergent, the series diverges.
 
\end{enumerate}

 \item Determine if each series converges or diverges. If it converges, determine the value it converges to.

\begin{enumerate}
 \item $\di \sum_{n=0}^\infty \frac{1}{4^n}$. (Geometric)
 
 \bigskip
 
 This is geometric, with $r=1/4<1$, so the series converges to $\dfrac{1}{1-1/4} = \dfrac{4}{3}$.
 
 %\item $\di \sum_{n=1}^\infty (-1)^nn$.
 \item $\di \sum_{n=1}^\infty e^{-n}$. (Geometric?)

 \bigskip
 
 This is geometric, with $r = \dfrac{1}{e}<1$. (Notice that $r^n = \dfrac{1}{e^n} = e^{-n}$.) We know that 
\[
\sum_{n=0}^\infty e^{-n} = \dfrac{1}{1-1/e} = \dfrac{e}{e-1}
\]
using the formula for the sum of a geometric series. Since our series starts at $n=1$ instead of $n=0$, we have to subtract the value of $e^{-0} = 1$, giving us
\[
 \sum_{n=1}^\infty e^{-n} = \dfrac{e}{e-1}-1 = \dfrac{1}{e-1}.
\]

 
 \item $\di \sum_{n=1}^\infty \frac{1}{n(n+1)}$ (Telescoping)

\bigskip

Since $\dfrac{1}{n(n+1)} = \dfrac{1}{n}-\dfrac{1}{n+1}$, we see that the series is telescoping. The $N^{\textrm{th}}$ partial sum is
\[
s_N =  \left(1-\frac{1}{2}\right)+\left(\frac{1}{2}-\frac{1}{3}\right)+\left(\frac{1}{3}-\frac{1}{4}\right)+\cdots + \left(\frac{1}{N}-\frac{1}{N+1}\right) = 1-\frac{1}{N+1},
\]
so the series converges to $\di\lim_{N\to\infty}s_N = 1$.

 \item $\di \sum_{n=1}^\infty \ln\left(\frac{n}{n+1}\right)$ (Telescoping?)

Recall that $\ln\left(\frac{n}{n+1}\right) =\ln n - \ln (n+1)$ using the properties of logarithms, so the $N^{\textrm{th}}$ partial sum is
\[
s_N = (\ln(1)-\ln(2))+(\ln(2)-\ln(3))+\cdots + (\ln N-\ln (N+1)) = -\ln(N+1),
\]
so the series is telescoping, but it diverges, since $\di\lim_{N\to\infty}s_N = -\infty$.

 %\item $\di \sum_{n=1}^\infty \frac{2n+1}{n^2(n+1)^2}$
\end{enumerate}

\item Use the integral test to determine if the series converges:

 \begin{enumerate}
  \item $\di \sum_{n=2}^\infty\frac{1}{n\ln n}$

\bigskip

We compare to the integral
\[
\int_2^\infty\frac{1}{x\ln x}\,dx = \lim_{b\to\infty}\int_2^b\frac{1}{x\ln x}\,dx = \lim_{b\to\infty}(\ln(\ln b)-\ln(\ln 2)) = \infty
\]
which diverges, so the series diverges as well.

  \item $\di \sum_{n=2}^\infty\frac{1}{n(\ln n)^2}$
  
  \bigskip
  
We compare to the integral
\[
\int_2^\infty \frac{1}{x(\ln x)^2}\,dx = \lim_{b\to \infty}\left(\left.-\frac{1}{\ln x}\right|_2^b\right) = \frac{1}{\ln 2}.
\]
Since the improper integral converges, so does the series.
 \end{enumerate}

\item Use direct comparison to determine if the series converges:

\begin{enumerate}
 \item $\di \sum_{n=1}^\infty \frac{1}{4^n+n^2-n}$
 
 \bigskip
 
 Since $n^2\geq n$ for $n\geq 1$, we have $n^2-n\geq 0$, so $4^n+n^2-n\geq 4^n>0$, which shows that
 \[
 \frac{1}{4^n+n^2-n}\leq \frac{1}{4^n}
 \]
 for all $n\geq 1$. Since the series $\sum \dfrac{1}{4^n}$ converges (it's geometric with $r=1/4<1$), the original series converges as well, by the comparison test.
 
 \item $\di \sum_{n=1}^\infty \frac{1}{\sqrt{n}-2}$

\bigskip

 Since $\sqrt{n}-2<\sqrt{n}$, it follows that $\dfrac{1}{\sqrt{n}-2}>\dfrac{1}{\sqrt{n}}$, and since $\sum \dfrac{1}{\sqrt{n}}$ diverges ($p$-series with $p=1/2<1$), the original series diverges, by the comparison test.
 
 \item $\di \sum_{n=1}^\infty \frac{1}{n^2\ln n}$

\bigskip

 Recall that $\ln$ is an increasing function, and since $3>e$, we know that $\ln n\geq \ln 3\geq \ln e=1$ for all $n\geq 3$. It follows that $\dfrac{1}{n^2\ln n}\leq \dfrac{1}{n^2}$ for all $n\geq 3$, and since $\sum\dfrac{1}{n^2}$ converges ($p$-series with $p=2>1$), the original series converges, by the comparison test.
 
\end{enumerate}

\item Use the Limit Comparison Test to determine if the series converges. (Be sure to state what series you're using for comparison.)

 \begin{enumerate}
  \item $\di \sum_{n=1}^\infty \frac{1}{4^n-n^2}$

\bigskip

Notice that direct comparison with the geometric series $\sum 1/4^n$ doesn't work as easily as in the previous problem, since the terms in this series are \textit{larger} than those of the geometric series. But they're not ``too much'' larger, which is the right setting for the limit comparison test.

With $a_n = \dfrac{1}{4^n-n^2}$ and $b^n = \dfrac{1}{4^n}$, we have
\[
\lim_{n\to\infty}\frac{a_n}{b_n} = \lim_{n\to\infty}\frac{4^n}{4^n-n^2} = \lim_{n\to\infty}\frac{1}{1-n^2/4^n} = 1.
\]
Since the limit of $a_n/b_n$ is finite and nonzero, and $\sum b_n$ converges (geometric series with $r=1/4<1$), we can conclude that $\sum a_n$ converges as well, by the limit comparison test.

(In the above, we used the fact that $\lim_{n\to \infty}\dfrac{n^2}{4^n}=0$, which can be easily verified using l'Hospital's rule for the corresponding functions of $x$:
\[
\lim_{x\to\infty}\frac{x^2}{4^x} = \lim_{x\to \infty}\frac{2x}{4^x\ln 4} = \lim_{x\to\infty}\frac{2}{4^x(\ln 4)^2} = 0.
\]
If this was already clear to you since you're aware that exponential functions always go to infinity faster than any polynomial, there's no need to verify this limit.)
  \item $\di \sum_{n=1}^\infty \frac{1}{\sqrt{n^2+n}}$
  
  \bigskip
  
  We let $a_n = \dfrac{1}{\sqrt{n^2+n}}$ and take $b_n = \dfrac{1}{n}$. Then
  \[
  \lim_{n\to\infty}\frac{a_n}{b_n} = \frac{n}{\sqrt{n^2+n}} = \lim_{n\to\infty}\frac{1}{\sqrt{1+1/n}}=1.
  \]
  Again, we get a finite, nonzero limit, but since $\sum b_n$ diverges (harmonic series), we conclude that $\sum a_n$ diverges as well, by the limit comparison test.
  
  \item $\di \sum_{n=1}^\infty \frac{n+5}{n^3-5}$

Let $a_n = \dfrac{n+5}{n^3-5}$, and let $b_n=\frac{1}{n^2}$. Then
\[
\lim_{n\to\infty}\frac{a_n}{b^n} = \lim_{n\to\infty}\frac{n^2(n+5)}{n^3-5} = \lim_{n\to\infty}\frac{1+5/n}{1-5/n^3} = 1.
\]
Since the above limit is finite and nonzero, and since $\sum b_n$ converges ($p$-series with $p=2>1$), we know that $\sum a_n$ converges, by the limit comparison test.
 \end{enumerate}



\end{enumerate}





\end{document}