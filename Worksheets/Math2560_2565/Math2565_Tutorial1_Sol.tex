\documentclass[12pt]{article}
\usepackage{amsmath}
\usepackage{amssymb}
\usepackage[letterpaper,top=0.85in,bottom=1in,left=0.75in,right=0.75in,centering]{geometry}
%\usepackage{fancyhdr}
\usepackage{enumerate}
%\usepackage{lastpage}
\usepackage{multicol}
\usepackage{graphicx}

\reversemarginpar

%\pagestyle{fancy}
%\cfoot{}
%\lhead{Math 1560}\chead{Test \# 1}\rhead{May 18th, 2017}
%\rfoot{Total: 10 points}
%\chead{{\bf Name:}}
\newcommand{\points}[1]{\marginpar{\hspace{24pt}[#1]}}
\newcommand{\skipline}{\vspace{12pt}}
%\renewcommand{\headrulewidth}{0in}
\headheight 30pt

\newcommand{\di}{\displaystyle}
\newcommand{\abs}[1]{\lvert #1\rvert}
\newcommand{\len}[1]{\lVert #1\rVert}
\renewcommand{\i}{\mathbf{i}}
\renewcommand{\j}{\mathbf{j}}
\renewcommand{\k}{\mathbf{k}}
\newcommand{\R}{\mathbb{R}}
\newcommand{\aaa}{\mathbf{a}}
\newcommand{\bbb}{\mathbf{b}}
\newcommand{\ccc}{\mathbf{c}}
\newcommand{\dotp}{\boldsymbol{\cdot}}
\newcommand{\bbm}{\begin{bmatrix}}
\newcommand{\ebm}{\end{bmatrix}}                   
                  
\begin{document}


\author{Instructor: Sean Fitzpatrick}
\thispagestyle{empty}
\vglue1cm
\begin{center}
{\bf MATH 2565 - Tutorial \#1 Solutions}
\end{center}

Additional practice (don't include your solutions here):
\begin{enumerate}
 \item  $\di \int \frac{e^{\sqrt{x}}}{\sqrt{x}}\,dx$
 
 \bigskip
 
$\di \int \frac{e^{\sqrt{x}}}{\sqrt{x}}\,dx = 2\int e^u\,du = 2e^{\sqrt{x}}+C,$ using the $u$-substitution $u=\sqrt{x}$; $du = \dfrac{1}{2\sqrt{x}}\,dx$.
 
 \bigskip

 \item $\di \int x\sqrt{x-2}\,dx$. (Try this once using substitution, and again using integration by parts.)
 
 \bigskip
  
 If we let $u=x-2$, then $du=dx$ and $x=u+2$, so
\[
 \int x\sqrt{x-2}\,dx = \int (u+2)\sqrt{u}\,du = \int (u^{3/2}+2u^{1/2})\,du = \frac{2}{5}(x-2)^{5/2}+\frac{4}{3}(x-2)^{3/2}+C.
\]
If we use integration by parts with $u=x$ and $dv = \sqrt{x-2}\,dx$, then $du=dx$ and $v = \frac{2}{3}(x-2)^{3/2}$, so
\[
 \int x\sqrt{x-2}\,dx = \frac{2}{3}x(x-2)^{3/2}-\frac{2}{3}\int (x-2)^{3/2}\,dx = \frac{2}{3}x(x-2)^{3/2}-\frac{2}{3}\left(\frac{2}{5}\right)(x-2)^{5/2}+C.
\]
Note that the two answers appear to be different. Are they? (They'd better not be!)

\bigskip



 \item $\di \int e^{\ln x}\,dx$. (With a bit of work you can do this by substituting $u=\ln x$ and noting that $x=e^u$. Why is this a bad idea?)
 
 \bigskip
 
 Substitution is a bad idea here because $e^{\ln x} = x$, and you know how to do $\int x\,dx$.
\end{enumerate}  

\medskip


\newpage
%\thispagestyle{empty}
Evaluate the following integrals.
  \begin{enumerate}
  \item $\di \int_0^1 2x(1-x^2)^4\,dx$
  
  \bigskip
  
   $\di \int_0^1 2x(1-x^2)^4\,dx = -\int_1^0 u^4\,du = \int_0^1 u^4\,du = \left.\frac{u^5}{5}\right|^1_0 = \frac{1}{5}$, using the subsitution $u=1-x^2$, $du = -2x\,dx$, and noting that if $x=0$, then $u=1-0^2=1$, and if $x=1$, then $u=1-1^2=0$.
   
   \bigskip
   
  
 \item $\di \int \tan^2(x)\,dx$

 \bigskip
 
 $\di \int \tan^2(x)\sec^2(x)\,dx = \int u^2\,du = \frac{\tan^3(x)}{3}+C$, using the substitution $u=\tan(x)$; $du = \sec^2(x)\,dx$.
 
 \bigskip
 



 \item $\di \int x^3e^x\,dx$
 
 \bigskip
 
 This integral can be done using integration by parts directly, or by applying a reduction formula.  If we do it directly, we have
\begin{align*}
 \int x^3e^x\,dx &= x^3e^x - 3\int x^2e^x\,dx & & \text{using } u=x^3, du = 3x^2\,dx; dv = e^x\,dx, v = e^x\\
& = x^3e^x -3\left(x^2e^x - 2\int xe^x\,dx\right) & & \text{using } u=x^2, du = 2x\,dx; dv = e^x\,dx, v=e^x\\
& = x^3e^x-3x^2e^x+6\left(xe^x-\int e^x\, dx\right) & & \text{using } u=x, du = dx; dv = e^x\,dx, v=e^x\\
& = x^3e^x-3x^2e^x+6xe^x-6e^x+C.
\end{align*}
 
 As an additional exercise, see if you can come up with a general reduction formula for the integral $\di\int x^ne^x\,dx$
 
 \bigskip
 
 \item $\di \int e^{2x}\sin(3x)\,dx$
 
 \bigskip
 
 This integral requires integration by parts twice, and collecting terms after the second step. Taking $u=
\sin(3x)$ and $dv = e^{2x}\,dx$, we get
\begin{align*}
 \int \sin(3x)e^{2x}\,dx &= \frac{1}{2}e^{2x}\sin(3x)-\frac{3}{2}\int \cos(3x)e^{2x}\,dx\\
& = \frac{1}{2}e^{2x}\sin(3x)-\frac{3}{2}\left(\frac{1}{2}e^{2x}\cos(3x)-\frac{3}{2}\int (-\sin(3x))e^{2x}\right)\,dx\\
& = \frac{1}{2}e^{2x}\sin(3x)-\frac{3}{4}e^{2x}\cos(3x) - \frac{9}{4}\int \sin(3x)e^{2x}\,dx.
\end{align*}
Bringing the last integral over to the left-hand side, we have
\[
 \left(1+\frac{9}{4}\right)\int e^{2x}\sin(3x)\,dx = \frac{1}{2}e^{2x}\sin(3x)-\frac{3}{4}e^{2x}\cos(3x),
\]
so dividing by $1+\frac{9}{4} =\frac{13}{4}$ and adding the constant of integration, we find
\[
 \int e^{2x}\sin(3x)\,dx = e^{2x}\left(\frac{2}{13}\sin(3x)-\frac{3}{13}\cos(3x)\right)+C.
\]

\bigskip
 
 \item $\di \int \sec^5(x)\,dx$
 
 The integral for $\sec^3(x)$ was done in class, and this one's here just to drive home the point that odd powers are hard. We start out by writing $\sec^5(x) = \sec^3(x)\sec^2(x)$, and integrate by parts, with $u=\sec^3(x)$ (so $du = 3\sec^2(x)(\sec(x)\tan(x)\,dx = 3\sec^3(x)\tan(x)\,dx$), and $dv = \sec^2(x)\,dx$ (so $v=\tan(x)$). This gives
 \begin{align*}
 \int \sec^5(x)\,dx &= \tan(x)\sec^3(x) - \int \tan^2(x)\sec^3(x)\,dx\\
 & = \tan(x)\sec^3(x) - \int (\sec^2(x)-1)\sec^3(x)\,dx\\
 & = \tan(x)\sec^3(x) - \int \sec^5(x)\,dx +\int \sec^3(x)\,dx.
 \end{align*}
 At this point we see the reappearance of $\int\sec^5(x)\,dx$ on the right-hand side, with a minus sign, so we can move it over to the left, giving $2\int \sec^5(x)\,dx$. If we divide through by 2 and substitute in our answer for $\int \sec^3(x)\,dx$ above, we get
 \[
 \int \sec^5(x)\,dx = \frac{1}{2}\tan(x)\sec^3(x) + \frac{1}{4}\tan(x)\sec(x) +\frac{1}{4}\ln\abs{\tan(x)+\sec(x)}+C.
 \]
  

\item $\int \sec^6(x)\,dx$. 

Not on the worksheet, but I thought I'd include it to point out that even powers a much, much easier. We're raising the secant function to a higher power, which might make you think things will be harder, but for $\sec(x)$, even powers are easy, and odd powers are hard. Since $\sec^2(x) = \tan^2(x)+1$, we have
\begin{align*}
\int \sec^6(x)\,dx &= \int (\tan^2(x)+1)^2\sec^2(x)\,dx = \int (u^2+1)^2\,du \\
& = \int(u^4+2u^2+1)\,du= \frac{1}{5}\tan^5(x)+\frac{2}{3}\tan^3(x)+\tan(x)+C,
\end{align*}

using the $u$-substitution $u=\tan(x)$.

\end{enumerate}
  
\end{document}