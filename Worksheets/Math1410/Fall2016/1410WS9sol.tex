\documentclass[12pt]{article}
\usepackage{amsmath}
\usepackage{amssymb}
\usepackage[letterpaper,margin=0.85in,centering]{geometry}
\usepackage{fancyhdr}
\usepackage{enumerate}
\usepackage{lastpage}
\usepackage{multicol}
\usepackage{graphicx}

\reversemarginpar

\pagestyle{fancy}
\cfoot{}
\lhead{Math 1410}\chead{Worksheet \# 9 Solutions}\rhead{November 16/17, 2016}
%\rfoot{Total: 10 points}
%\chead{{\bf Name:}}
\newcommand{\points}[1]{\marginpar{\hspace{24pt}[#1]}}
\newcommand{\skipline}{\vspace{12pt}}
%\renewcommand{\headrulewidth}{0in}
\headheight 30pt

\newenvironment{amatrix}[1]{%
  \left[\begin{array}{@{}*{#1}{c}|c@{}}
}{%
  \end{array}\right]
}

\newcommand{\baam}{\left[\begin{array}{ccc|ccc}}
\newcommand{\eaam}{\end{array}\right]}

\newcommand{\di}{\displaystyle}
\newcommand{\abs}[1]{\lvert #1\rvert}
\newcommand{\len}[1]{\lVert #1\rVert}
\renewcommand{\i}{\mathbf{i}}
\renewcommand{\j}{\mathbf{j}}
\renewcommand{\k}{\mathbf{k}}
\newcommand{\R}{\mathbb{R}}
\newcommand{\aaa}{\mathbf{a}}
\newcommand{\bbb}{\mathbf{b}}
\newcommand{\ccc}{\mathbf{c}}
\newcommand{\dotp}{\boldsymbol{\cdot}}
\newcommand{\bbm}{\begin{bmatrix}}
\newcommand{\ebm}{\end{bmatrix}}                   
\newcommand{\bam}{\begin{amatrix}}
\newcommand{\eam}{\end{amatrix}}   
                 
\begin{document}

%\author{Instructor: Sean Fitzpatrick}
\thispagestyle{fancy}
%\noindent{{\bf Name and student number:}}
 \begin{enumerate}
\item Compute the inverse of the given matrix, if possible:
\begin{enumerate}
 \item $A = \bbm 1&-5&0\\-2&15&4\\4&-19&1\ebm$

\bigskip

Using the algorithm given in class, we have:
\begin{align*}
 \baam 1&-5&0&1&0&0\\-2&15&4&0&1&0\\4&-19&1&0&0&1\eaam \xrightarrow[R_3-4R_1\to R_3]{R_2+2R_1\to R_2} & \baam 1&-5&0&1&0&0\\0&5&4&2&1&0\\0&1&1&-4&0&1\eaam\\
\xrightarrow{R_2\leftrightarrow R_3} & \baam 1&-5&0&1&0&0\\0&1&1&-4&0&1\\0&5&4&2&1&0\eaam\\
\xrightarrow{R_3-5R_2\to R_3} & \baam 1&-5&0&1&0&0\\0&1&1&-4&0&1\\0&0&-1&22&1&-5\eaam\\
\xrightarrow{(-1)R_3\to R_3} & \baam 1&-5&0&1&0&0\\0&1&1&-4&0&1\\0&0&1&-22&-1&5\eaam\\
\xrightarrow{R_2-R_3\to R_2} & \baam 1&-5&0&1&0&0\\0&1&0&18&1&-4\\0&0&1&-22&-1&5\eaam\\
\xrightarrow{R_1+5R_2\to R_1} & \baam 1&0&0&91&5&-20\\0&1&0&18&1&-4\\0&0&1&-22&-1&5\eaam
\end{align*}
We thus see that the reduced row-echelon form of $A$ is equal to the identity matrix $I_3$, and we can conclude that the inverse of $A$ is given by 
\[
 A^{-1} = \bbm 91&5&-20\\18&1&-4\\-2&-1&5\ebm.
\]
To verify, we can check that 
\begin{align*}
 AA^{-1} &= \bbm 1&-5&0\\-2&15&4\\4&-19&1\ebm\bbm 91&5&-20\\18&1&-4\\-2&-1&5\ebm\\
& = \bbm 91-90+0 & 5-5+0 & 20-20+0\\-182+270-88&-10+15-4&40-60+20\\364-342-22&20-19-1&-80+76+5\ebm\\& = \bbm 1&0&0\\0&1&0\\0&0&1\ebm,
\end{align*}
as expected.


 \item $B = \bbm 2&3&4\\-3&6&9\\-1&9&13\ebm$

\bigskip

Proceeding as above, we have:
\begin{align*}
 \baam 2&3&4&1&0&0\\-3&6&9&0&1&0\\-1&9&13&0&0&1\eaam \xrightarrow{R_1\leftrightarrow R_3}& \baam -1&9&13&0&0&1\\-3&6&9&0&1&0\\2&3&4&1&0&0\eaam\\
\xrightarrow{(-1)R_1\to R_1} & \baam 1&-9&-13&0&0&-1\\-3&6&9&0&1&0\\2&3&4&1&0&0\eaam\\
\xrightarrow[R_3-2R_1\to R_3]{R_2+3R_1\to R_2} & \baam 1&-9&-13&0&0&-1\\0&-21&-30&0&1&2\\0&21&30&1&0&-3\eaam\\
\xrightarrow{R_3+R_2\to R_3} & \baam 1&-9&-13&0&0&-1\\0&-21&-30&0&1&2\\0&0&0&1&1&-1\eaam
\end{align*}
At this point the algorithm stops: we can already see that $\operatorname{rank}(B)=2<3$, since the row of zeros will prevent us from obtaining a leading one in the third column. Since $B$ cannot be carried to the identity matrix by elementary row operations, it is not invertible.
\end{enumerate}

\bigskip

\item Simplify the expression $A^2(B^{-1}A)^{-1}(AB)^{-1}B$.

\bigskip

Recalling from class that $(AB)^{-1} = B^{-1}A^{-1}$ and that $(B^{-1})^{-1}=B$, we have
\begin{align*}
 A^2(B^{-1}A)^{-1}(AB)^{-1}B & = A^2(A^{-1}(B^{-1})^{-1})(B^{-1}A^{-1})B\\
& = A(AA^{-1})(BB^{-1})(A^{-1}B)\\
& = A(I_n)(I_n)(A^{-1}B)\\
& = (AA^{-1})B = I_nB = B.
\end{align*}


\item Find an expression for $A^{-1}$, given that:
\begin{enumerate}
 \item $5A^3=I$

\bigskip

Since $5A^3=I$, it follows that $A(5A^2) = 5A(A^2) = 5A^3 = I$. Since $A(5A^2)=I$ and $A^{-1}$ is the unique matrix $X$ such that $AX=I$, we can conclude that $A^{-1}=5A^2$.

\medskip

 \item $A^2-2A+I=0$

\bigskip

From the given equation we have
\[
 I = 2A-A^2 =A(2I-A).
\]
As with the previous problem, since we have $A(2I-A)=I$, it follows that $A^{-1} = 2I-A$.

\medskip


 \item $A^2B$ is invertible, for some matrix $B$. (Give your answer in terms of $(A^2B)^{-1}$.)

\bigskip

Suppose that $A^2B$ is invertible, and let $C=(A^2B)^{-1}$. By definition of the inverse, we have $(A^2B)C=I$, and thus $A(ABC)=I$, and once again it follows that
\[
 A^{-1} = ABC = AB(A^2B)^{-1}.
\]

\end{enumerate}

\medskip


\item Write the matrix $A=\bbm 2&-4\\-4&9\ebm$ as a product of elementary matrices.

\bigskip

We look for a sequence of elementary row operations that carry our matrix to the identity. We have
\[
 \bbm 2&-4\\-4&9\ebm \xrightarrow{\frac{1}{2}R_1\to R_1}\bbm 1&-2\\-4&9\ebm \xrightarrow{R_2+4R_1\to R_2} \bbm 1&-2&0&1\ebm \xrightarrow{R_1+2R_2\to R_1} \bbm 1&0\\0&1\ebm.
\]
The elementary matrices corresponding to these row operations are
\[
 E_1 =\bbm \frac{1}{2}&0\\0&1\ebm\quad E_2 = \bbm 1&0\\4&1\ebm \quad E_3 = \bbm 1&2\\0&1\ebm.
\]
The inverses of these matrices are
\[
 E_1^{-1} = \bbm 2&0\\0&1\ebm \quad E_2^{-1} = \bbm 1&0\\-4&1\ebm \quad E_3^{-1} = \bbm 1&-2\\0&1\ebm.
\]
As discussed in class, we expect that $A^{-1} = E_3E_2E_1$ and $A = E_1^{-1}E_2^{-1}E_3^{-1}$, and indeed, we can verify that
\[
 E_1^{-1}E_2^{-1}E_3^{-1} = \bbm 2&0\\0&1\ebm\bbm 1&0\\-4&1\ebm\bbm 1&-2\\0&1\ebm = \bbm 2&-4\\-4&9\ebm = A,
\]
and
\[
 E_3E_2E_1 = \bbm 1&2\\0&1\ebm\bbm 1&0\\4&1\ebm\bbm \frac{1}{2}&0\\0&1\ebm = \bbm \frac{9}{2}&2\\2&1\ebm = A^{-1}.
\]


 \end{enumerate}
\end{document}