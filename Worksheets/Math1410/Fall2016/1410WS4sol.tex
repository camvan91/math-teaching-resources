\documentclass[12pt]{article}
\usepackage{amsmath}
\usepackage{amssymb}
\usepackage[letterpaper,margin=0.85in,centering]{geometry}
\usepackage{fancyhdr}
\usepackage{enumerate}
\usepackage{lastpage}
\usepackage{multicol}
\usepackage{graphicx}

\reversemarginpar

\pagestyle{fancy}
\cfoot{}
\lhead{Math 1410}\chead{Worksheet \# 4 Solutions}\rhead{October 5/6, 2016}
%\rfoot{Total: 10 points}
%\chead{{\bf Name:}}
\newcommand{\points}[1]{\marginpar{\hspace{24pt}[#1]}}
\newcommand{\skipline}{\vspace{12pt}}
%\renewcommand{\headrulewidth}{0in}
\headheight 30pt

\newcommand{\di}{\displaystyle}
\newcommand{\abs}[1]{\lvert #1\rvert}
\newcommand{\len}[1]{\lVert #1\rVert}
\renewcommand{\i}{\mathbf{i}}
\renewcommand{\j}{\mathbf{j}}
\renewcommand{\k}{\mathbf{k}}
\newcommand{\R}{\mathbb{R}}
\newcommand{\aaa}{\mathbf{a}}
\newcommand{\bbb}{\mathbf{b}}
\newcommand{\ccc}{\mathbf{c}}
\newcommand{\dotp}{\boldsymbol{\cdot}}
\newcommand{\bbm}{\begin{bmatrix}}
\newcommand{\ebm}{\end{bmatrix}}                   
                  
\begin{document}

%\author{Instructor: Sean Fitzpatrick}
\thispagestyle{fancy}
%\noindent{{\bf Name and student number:}}
 \begin{enumerate}
 
\item Find the distance between the skew lines 
\begin{align*}
 \ell_1(s) = \langle x,y,z\rangle & = \langle 1,2,1\rangle+s\langle 2,-1,1\rangle\\
 \ell_2(t) = \langle x,y,z\rangle & = \langle 3,3,3\rangle+t\langle 4,2,-1\rangle
\end{align*}
using the method of Example 56 in the text. \\

Letting $P_1 = (1,2,1)$ and $P_2=(3,3,3)$ be the given points on the lines $\ell_1$ and $\ell_2$, respectively, we have the vector 
\[
 \vec{w} = \overrightarrow{P_1P_2} = \langle 2,1,2\rangle,
\]
which has its tail on $\ell_1$ and head on $\ell_2$. To compute the distance from $\ell_1$ to $\ell_2$, we need to determine the component of $\vec{w}$ that is perpendicular to both lines. We are given the direction vectors $\vec{v}_1 = \langle 2,-1,1\rangle$ and $\vec{v}_2 = \langle 4,2,-1\rangle$ for the two lines; since these vectors are parallel to their respective lines, the cross product
\[
 \vec{n} = \vec{v}_1\times\vec{v}_2 = \begin{vmatrix}\hat{\imath}&\hat{\jmath}&\hat{k}\\2&-1&1\\4&2&-1\end{vmatrix} = \langle -1, 6, 8\rangle
\]
must be perpendicular to both lines. (The notation $\vec{n}$ is used to indicate the fact that the two skew lines lie in parallel planes, both of which have the normal vector $\vec{n}$. Finding the distance between the two lines is thus reduced to finding the distance between the corresponding parallel planes.)

The desired distance is now given by the length of the projection of $\vec{w}$ onto $\vec{n}$. We have
\[
 d = \len{\operatorname{proj}_{\vec{n}}\vec{w}} = \frac{\abs{\vec{n}\dotp\vec{w}}}{\len{\vec{n}}} = \frac{-2+6+16}{\sqrt{1^2+6^2+8^2}} = \frac{20}{\sqrt{101}}.
\]


\item Given the matrices $A = \bbm 2&-3\\1&5\ebm$ and $B = \bbm -7 & 2\\-1&4\ebm$, find and simplify the following matrices:
\begin{enumerate}
 \item $A+B = \bbm 2-7&-3+2\\1-1&5+4\ebm = \bbm -5&-1\\0&9\ebm$

\bigskip


 \item $4B-5A = 4\bbm -7 & 2\\-1&4\ebm - 5\bbm 2&-3\\1&5\ebm  = \bbm -28&8\\-4&16\ebm+\bbm-10&15\\-5&-25\ebm = \bbm -38&23\\-9&-9\ebm.$

 \item $2(A-B)-(A-2B)$ (Hint: you may want to first simplify the expression before plugging in values.)
\begin{align*}
 2(A-B)-(A-2B) &= (2A-2B)+(-A+2B)\\
& = (2A-A)+(-2B+2B)\\
& = A+\mathbf{0} = A = \bbm 2&-3\\1&5\ebm.
\end{align*}

\end{enumerate}
\newpage

 \item Prove that for any $m\times n$ matrices $A$ and $B$, and any scalar $k$, we have $k(A+B)=kA+kB$.

\bigskip

Let us denote $A$ and $B$ in terms of their entries as $A=[a_{ij}]$ and $B=[b_{ij}]$. Note that all matrices involved are of the same size ($m\times n$), so it suffices to show that each entry of $k(A+B)$ is equal to the corresponding entry of $kA+kB$. We have:
\begin{align*}
 k(A+B) & = k([a_{ij}]+[b_{ij}]) \tag*{Substituting notation}\\
& = k[a_{ij}+b_{ij}] \tag*{Definition of matrix addition}\\
& = [k(a_{ij}+b_{ij})] \tag*{Definition of scalar multiplication}\\
& = [ka_{ij} + kb_{ij}] \tag*{Distributive property for real numbers}\\
& = [ka_{ij}] + [kb_{ij}] \tag*{Definition of matrix addition}\\
& = k[a_{ij}]+k[b_{ij}] \tag*{Definition of scalar multiplication}\\
& = kA+kB \tag*{Substituting notation,}
\end{align*}
which is what we needed to show.


\end{enumerate}
\end{document}