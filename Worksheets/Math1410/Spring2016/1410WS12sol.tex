\documentclass[12pt]{article}
\usepackage{amsmath}
\usepackage{amssymb}
\usepackage[letterpaper,margin=0.75in,centering]{geometry}
\usepackage{fancyhdr}
\usepackage{enumerate}
\usepackage{lastpage}
\usepackage{multicol}
\usepackage{graphicx}

\reversemarginpar

\pagestyle{fancy}
\cfoot{}
\lhead{Math 1410}\chead{Worksheet \# 12 Solutions}\rhead{Tuesday, 12\textsuperscript{th} April, 2016}
%\rfoot{Total: 10 points}
%\chead{{\bf Name:}}
\newcommand{\points}[1]{\marginpar{\hspace{24pt}[#1]}}
\newcommand{\skipline}{\vspace{12pt}}
%\renewcommand{\headrulewidth}{0in}
\headheight 30pt

\newenvironment{amatrix}[1]{%
  \left[\begin{array}{@{}*{#1}{c}|c@{}}
}{%
  \end{array}\right]
}
\newenvironment{aamatrix}[1]{%
  \left[\begin{array}{@{}*{#1}{c}|*{#1}{c}@{}}
}{%
  \end{array}\right]
}

\newcommand{\di}{\displaystyle}
\newcommand{\abs}[1]{\lvert #1\rvert}
\newcommand{\len}[1]{\lVert #1\rVert}
\renewcommand{\i}{\mathbf{i}}
\renewcommand{\j}{\mathbf{j}}
\renewcommand{\k}{\mathbf{k}}
\newcommand{\R}{\mathbb{R}}
\newcommand{\aaa}{\mathbf{a}}
\newcommand{\bbb}{\mathbf{b}}
\newcommand{\ccc}{\mathbf{c}}
\newcommand{\dotp}{\boldsymbol{\cdot}}
\newcommand{\bbm}{\begin{bmatrix}}
\newcommand{\ebm}{\end{bmatrix}}       
\DeclareMathOperator{\proj}{proj}   
\newcommand{\bam}{\begin{amatrix}}
\newcommand{\eam}{\end{amatrix}}
\newcommand{\bvm}{\begin{vmatrix}}
\newcommand{\evm}{\end{vmatrix}}          
                  
\begin{document}

%\author{Instructor: Sean Fitzpatrick}
\thispagestyle{fancy}
%\noindent{{\bf Name and student number:}}

 \begin{enumerate}
  \item If $z=3-2i$ and $w = -5+4i$, compute:
  \begin{enumerate}
  \item $3z = 3(3-2i) = 9-6i$
  \item $z-2w = (3-2i)-2(-5+4i) = 3-2i+10-8i = 13-10i$
  \item $2w-3z = 2(-5+4i)-3(3-2i) = -10+8i-9+6i = -19+14i$
  \item $zw = (3-2i)(-5+4i) = -15+12i+10i-8i^2 = (-15+8)+i(12+10) = -7+22i$
  \item $\overline{z}=3+2i$ \\(The complex conjugate is defined by $\overline{x+iy}=x-iy$.)
  \item $\abs{w}=\sqrt{(-5+4i)(-5-4i)} = \sqrt{(-5)^2+4^2} = \sqrt{41}$ \\(The complex modulus (norm) is defined by $\abs{w} = \sqrt{w\overline{w}}$.)
  \item $\dfrac{z^2}{w} = \dfrac{(3-2i)(3-2i)}{-5+4i} = \dfrac{(5-12i)(-5-4i)}{(-5+4i)(-5-4i)} = \dfrac{-25-48+i(-20+60)}{(-5)^2+4^2} = -\dfrac{73}{41}+\dfrac{40}{41}i$
  \end{enumerate}
 
  
 \item Solve for $z$ in the following equations:
\begin{enumerate}
\item $z+(2-3i)=-5+4i$
\[
 z = -5+4i-(2-3i) = -7+7i
\]

\item $3z-2i = (2-i)(3+4i)$
\[
 3z=(10+5i)+2i=10+7i, \quad \text{ so } \quad z = \frac{10}{3}+i\frac{7}{3}.
\]

\item $2iz = 1+i$
\[
 z = -\frac{i}{2}(2iz) = -\frac{i}{2}(1+i) = \frac{1}{2}-\frac{i}{2}
\]

\item $(3+2i)z -1+3i = 4+i$
\[
 (3+2i)z = (4+i)-(-1+3i) = 5-2i, 
\]
so 
\[
13z = (3-2i)[(3+2i)z] = (3-2i)(5-2i) = 11-16i, 
\]
which gives $z = \dfrac{11}{13}-\dfrac{16}{13}i$.
\end{enumerate} 
\item Find the eigenvalues of the following matrices:
\[
A = \bbm 2&4\\-4&2\ebm \quad\quad B = \bbm 3&2+i\\2-i&7\ebm
\]
For $A$, we have
\[
 \det(A-xI) = \bvm 2-x&4\\-4&2-x\evm = (2-x)^2+16,
\]
so $(\lambda-2)^2 = -16$, giving $\lambda-2 = \pm\sqrt{-16} = \pm 4i$, so $\lambda=2\pm 4i$. (You can also expand the quadratic and use the quadratic formula.)

For the matrix $B$, we have
\begin{align*}
 \det(B-xI) &= \bvm 3-x&2+i\\2-i&7-x\evm = (3-x)(7-x)-(2+i)(2-i)\\
& = x^2-10x+21-5 = x^2-10x+16 = (x-2)(x-8),
\end{align*}
so $\lambda=2$ or $\lambda=8$.

\item Verify that $\bbm 1\\i\ebm$ and $\bbm i\\1\ebm$ are eigenvectors for the matrix $A$ in the previous problem, and that $\bbm 2+i\\-1\ebm$ and $\bbm 1\\2-i\ebm$ are eigenvectors for the matrix $B$ in the previous problem.

We have the following:
\[
 \bbm 2&4\\-4&2\ebm\bbm 1\\i\ebm = \bbm 2+4i\\-4+2i\ebm = (2+4i)\bbm 1\\i\ebm,
\]
so $\bbm 1\\i\ebm$ is an eigenvector of $A$ with eigenvalue $2+4i$.
\[
 \bbm 2&4\\-4&2\ebm\bbm i\\1\ebm = \bbm 2i+4\\-4i+2\ebm = (2-4i)\bbm i\\1\ebm,
\]
so $\bbm i\\1\ebm$ is an eigenvector of $A$ with eigenvalue  $2-4i$.
\[
 \bbm 3&2+i\\2-i&7\ebm\bbm 2+i\\-1\ebm = \bbm 6+3i-2-i\\(2-i)(2+i)-7\ebm = \bbm 4+2i\\-2\ebm = 2\bbm 2+i\\-1\ebm,
\]
so $\bbm 2+i\\-1\ebm$ is an eigenvector of $B$ with eigenvalue 2.
\[
 \bbm 3&2+i\\2-i&7\ebm\bbm 1\\2-i\ebm = \bbm 3+(2+i)(2-i)\\2-i+14-7i\ebm = \bbm 8\\16-8i\ebm = 8\bbm 1\\2-i\ebm,
\]
so $\bbm 1\\2-i\ebm$ is an eigenvector of $B$ with eigenvalue  8.


\item (Bonus superfun challenge problem) Let $Z=\bbm 0&1\\-1&0\ebm$.
\begin{enumerate}
\item Verify that $Z$ has eigenvalues $\pm i$ and eigenvectors $\vec{v}=\bbm i\\-1\ebm$ and $\vec{w}=\bbm -1\\i\ebm$.

\bigskip

We check that
\[
 Z\vec{v} = \bbm 0&1\\-1&0\ebm\bbm i\\-1\ebm = \bbm-1\\-i\ebm = i\bbm i\\-1\ebm,
\]
so $\vec{v}$ is an eigevector with eigenvalue $i$, and
\[
 Z\vec{w} = \bbm 0&1\\-1&0\ebm\bbm -1\\i\ebm = \bbm i\\1\ebm = -i\bbm -1\\i\ebm,
\]
so $\vec{w}$ is an eigenvector with eigenvalue $-i$.

\item Show that $\langle \vec{v},\vec{w}\rangle = 0$, where $\langle \vec{v},\vec{w}\rangle = \vec{v}\dotp \overline{\vec{w}}$ is the complex version of the dot product. (The notation $\overline{\vec{w}}$ means take the complex conjugate of each entry in $\vec{w}$.) 

\bigskip

We have
\[
 \langle \vec{v},\vec{w}\rangle = \vec{v}\dotp\overline{\vec{w}} = \bbm i\\-1\ebm \dotp \bbm -1\\-i\ebm = -i+i=0.
\]

\item A matrix $U$ is called \textbf{unitary} if $U^*U=I$, where $U^*=(\overline{U})^T$ is the \textit{Hermitian conjugate} of $U$, formed by taking the transpose of the complex conjugate of $U$.

Let $U = \dfrac{1}{\sqrt{2}}\bbm i&-1\\-1&i\ebm$. (Note that the columns of $U$ are eigenvectors of $Z$.) Show that $U$ is unitary and that $U^*ZU = \bbm i&0\\0&-i\ebm$.

\bigskip

We have
\[
 U^* = \dfrac{1}{\sqrt{2}}\bbm -i&-1\\-1&-i\ebm^T = \dfrac{1}{\sqrt{2}}\bbm -i&-1\\-1&-i\ebm,
\]
so
\[
 U^*U = \left(\dfrac{1}{\sqrt{2}}\bbm -i&-1\\-1&-i\ebm\right)\left(\dfrac{1}{\sqrt{2}}\bbm i&-1\\-1&i\ebm\right) = \frac{1}{2}\bbm 2&0\\0&2\ebm = I,
\]
so $U^*=U^{-1}$, showing that $U$ is unitary. Finally, we have
\begin{align*}
 U^*ZU &= \frac{1}{\sqrt{2}}\bbm -i&-1\\-1&-i\ebm\left(\bbm 0&1\\-1&0\ebm\right)\left(\frac{1}{\sqrt{2}}\bbm i&-1\\-1&i\ebm \right)\\
& = \frac{1}{2}\bbm -i&-1\\-1&-i\ebm\bbm -1&i\\-i&1\ebm\\
& = \frac{1}{2}\bbm 2i&0\\0&-2i\ebm = \bbm i&0\\0&-i\ebm,
\end{align*}
as required.
\item Compute $Z^{423}$.

\bigskip

Let $D = \bbm i&0\\0&-1\ebm$ be the diagonal matrix whose diagonal entries are the eigenvalues of $Z$. Then we have $U^*ZU=D$, and since $U^*=U^{-1}$, we can solve for $Z$, giving us $Z = UDU^*$. Now,
\[
 Z^{n} = (UDU^*)^n = (UDU^*)(UDU^*)(UDU^*)\cdots(UDU^*) = UD^nU^*,
\]
since all the $U^*U$ products in the interior are equal to the identity. Since $D$ is diagonal we can easily compute
\[
 D^{423} = \bbm i^{423}&0\\0&(-i)^{423}\ebm = \bbm i^{420}i^3&0\\0&(-i)^{420}(-i)^3\ebm = \bbm -i&0\\0&i\ebm,
\]
where we have used the fact that $i^4 = (i^2)^2 = (-1)^2=1$, so $i^{420} = (i^4)^{105} = 1^105 = 1$, and similarly $(-i)^{420}=1$. We therefore have
\[
 Z^{423} = UD^{423}U^* = \frac{1}{\sqrt{2}}\bbm i&-1\\-1&i\ebm\bbm -i&0\\0&i\ebm \left(\frac{1}{\sqrt{2}}\bbm -i&-1\\-1&-i\ebm \right)= \bbm 0&-1\\1&0\ebm.
\]

\end{enumerate}
 \end{enumerate}

\end{document}