\documentclass[12pt]{article}
\usepackage{amsmath}
\usepackage{amssymb}
\usepackage[letterpaper,margin=0.85in,centering]{geometry}
\usepackage{fancyhdr}
\usepackage{enumerate}
\usepackage{lastpage}
\usepackage{multicol}
\usepackage{graphicx}

\reversemarginpar

\pagestyle{fancy}
\cfoot{}
\lhead{Math 1410}\chead{Worksheet \# 5}\rhead{Tuesday, 9\textsuperscript{th} February, 2016}
%\rfoot{Total: 10 points}
%\chead{{\bf Name:}}
\newcommand{\points}[1]{\marginpar{\hspace{24pt}[#1]}}
\newcommand{\skipline}{\vspace{12pt}}
%\renewcommand{\headrulewidth}{0in}
\headheight 30pt

\newenvironment{amatrix}[1]{%
  \left[\begin{array}{@{}*{#1}{c}|c@{}}
}{%
  \end{array}\right]
}
\newenvironment{aamatrix}[1]{%
  \left[\begin{array}{@{}*{#1}{c}|cc}
}{%
  \end{array}\right]
}

\newcommand{\di}{\displaystyle}
\newcommand{\abs}[1]{\lvert #1\rvert}
\newcommand{\len}[1]{\lVert #1\rVert}
\renewcommand{\i}{\mathbf{i}}
\renewcommand{\j}{\mathbf{j}}
\renewcommand{\k}{\mathbf{k}}
\newcommand{\R}{\mathbb{R}}
\newcommand{\aaa}{\mathbf{a}}
\newcommand{\bbb}{\mathbf{b}}
\newcommand{\ccc}{\mathbf{c}}
\newcommand{\dotp}{\boldsymbol{\cdot}}
\newcommand{\bbm}{\begin{bmatrix}}
\newcommand{\ebm}{\end{bmatrix}}       
\DeclareMathOperator{\proj}{proj}   
\newcommand{\bam}{\begin{amatrix}}
\newcommand{\eam}{\end{amatrix}}
         
                  
\begin{document}
{\bf \large Name:} \hspace{2.5in} {\bf Tutorial time:}

\bigskip

{\bf Problem you want feedback on:}

\bigskip

%\author{Instructor: Sean Fitzpatrick}
\thispagestyle{fancy}
%\noindent{{\bf Name and student number:}}
Please complete all problems below.
 \begin{enumerate}
 \item When you put the symbol ``='' between two objects on the page, what are you saying about the relationship between those objects?

\bigskip

An expression such as $a=b$ tells the reader that the objects $a$ and $b$ are {\bf equal}. If an equal sign appears between two objectst that are not, in fact, equal, then you've written an incorrect statement, and will lose marks as a result. (Even if we ``knew what you meant''.)

\bigskip

 \item Each of the augmented matrices below is in reduced row-echelon form. For each matrix, indicate the following:

 \begin{enumerate}
  \item The {\em rank} of the augmented matrix.
  \item The number of variables in the corresponding system of equations.
  \item The number of parameters needed to write down the general solution.
  \item The general solution to the system, if any.

\begin{enumerate}
 \item $\di \bam{3} 1&0&-2&4\\0&1&3&-5\\0&0&0&0\eam$

\medskip

The rank is 2, and there are 3 variables (let's call them $x,y,z$), so there is $3-2=1$ parameter. 
The general solution is
\begin{align*}
 x&=4+2t\\
 y&=-5-3t\\
 z&=t,
\end{align*}
where $t$ can be any real number.
 \item $\di \bam{4} 1&-3&0&4&2\\0&0&1&-3&7\\0&0&0&0&0\eam$

\medskip

The rank is 2, and there are 4 variables (let's call them $w,x,y,z$), so there are $4-2=2$ parameters.
The general solution is
\begin{align*}
 w&=2+3s-4t\\
 x&=s\\
 y&=7+3t\\
 z=t,
\end{align*}
where $s$ and $t$ can be any real numbers.
 \item $\di \bam{3} 1&2&0&3\\0&0&1&4\\0&0&0&1\eam$

\medskip

The rank is 3, and there are 3 variables, but the third leading 1 appears in the constants column. There is therefore no solution.
 \item $\di \bam{4} 0&1&0&-1&4\\0&0&1&0&2\\0&0&0&0&0\eam$

\medskip

The rank is 2, and there are 4 variables ($w,x,y,z$), so there are $4-2=2$ parameters. 
The general solution is
\begin{align*}
 w&=s\\
 x&=4+t\\
 y=2\\
 z=t,
\end{align*}
where $s$ and $t$ can be any real numbers.
\end{enumerate}

 \end{enumerate}



\item Determine the value(s) of $a$ such that the system of equations given by the augmented matrix below has no solution, one solution, or infinitely many solutions, if possible.
\[
\begin{amatrix}{3}
a & 1 & 2 & 1 \\

2 & 1 & 7 & 3 \\

1 & 1 & 2 & 1 \\
\end{amatrix}
\]

\medskip

If we proceed with the standard Gaussian elimination algorithm, we first swap rows 1 and 3 to get a leading 1 in the upper left-hand corner, and then proceed to create zeros in the first column, as follows:
\[
 \bam{3} a& 1&2&1\\2&1&7&3\\1&1&2&1\eam \xrightarrow[]{R_1\leftrightarrow R_3} \bam{3} 1&1&2&1\\2&1&7&3\\a&1&2&1\eam \xrightarrow[R_3\to R_3-aR_1]{R_2\to R_2-2R_1} \bam{3}1&1&2&1\\0&-1&3&1\\0&1-a&2-2a&1-a\eam.
\]

At this point we notice that everything in the third row is a multiple of $1-a$. If $a=1$, then $1-a=0$, and we get a row of zeros. From here we get
\[
 \bam{3}1&1&2&1\\0&-1&3&1\\0&0&0&0\eam\xrightarrow{R_1\to R_1+R_2}\bam{3}1&0&5&2\\0&-1&3&1\\0&0&0&0\eam\xrightarrow{R_2\to -R_2}\bam{3}1&0&5&2\\0&1&-3&-1\\0&0&0&0\eam,
\]
so there are infinitely many solutions of the form $x=2-5t$, $y=-1+3t$, $z=t$, where $t$ can be any real number.

If $a\neq 1$, then $1-a\neq 0$ and we can divide the third row by $1-a$, giving us
\begin{align*}
 \bam{3}1&1&2&1\\0&-1&3&1\\0&1-a&2-2a&1-a\eam&\xrightarrow{R_3\to \frac{1}{1-a}R_3}\bam{3}1&1&2&1\\0&-1&3&1\\0&1&2&1\eam \xrightarrow{R_1\to R_1-R_3}\bam{3}1&0&0&0\\0&-1&3&1\\0&1&2&1\eam\\
&\xrightarrow{R_3\to R_3+R_2}\bam{3}1&1&2&1\\0&-1&3&1\\0&0&5&2\eam\xrightarrow[R_3\to \frac{1}{5}R_3]{R_2\to -R_2} \bam{3}1&0&0&0\\0&1&-3&-1\\0&0&1&\frac{2}{5}\eam,
\end{align*}
giving us the unique solution $x=0, y=\frac{1}{5}$, $z=\frac{2}{5}$.

\item Find the basic solutions to the homogeneous system of equations
\[
 \begin{array}{ccccccccc}
  x_1&-&2x_2&+&x_3&-&3x_4&=&0\\
 2x_1&+&x_2&-&4x_3&+&x_4&=&0\\
 3x_1&-&x_2&-&3x_3&-&2x_4&=&0
 \end{array}
\]
We set up the augmented matrix and apply row operations, as follows:
\begin{align*}
 \bam{4}1&-2&1&-3&0\\2&1&-4&1&0\\3&-1&-3&-2&0\eam &\xrightarrow[R_3\to R_3-3R_1]{R_2\to R_2-2R_1} \bam{4}1&-2&1&-3&0\\0&5&-6&7&0\\0&5&-6&7&0\eam\\
&\xrightarrow{R_3\to R_3-R_2}\bam{4}1&-2&1&-3&0\\0&5&-6&7&0\\0&0&0&0&0\eam\\
&\xrightarrow{R_2\to \frac{1}{5}R_2}\bam{4}1&-2&1&-3&0\\0&1&-\frac{6}{5}&\frac{7}{5}&0\\0&0&0&0&0\eam\\
&\xrightarrow{R_1\to R_1+2R_2}\bam{4}1&0&-\frac{7}{5}&-\frac{1}{5}\\0&1&-\frac{6}{5}&\frac{7}{5}&0\\0&0&0&0&0\eam.
\end{align*}
From here we can read off the general solution
\[
 \bbm x_1\\x_2\\x_3\\x_4\ebm = \bbm \frac{7}{5}s+\frac{1}{5}t\\ \frac{6}{5}s-\frac{7}{5}t\\s\\t\ebm = s\bbm \frac{7}{5}\\\frac{6}{5}\\1\\0\ebm+t\bbm \frac{1}{5}\\-\frac{7}{5}\\0\\1\ebm,
\]
so the basic solutions are $\vec{v} = \bbm \frac{7}{5}\\\frac{6}{5}\\1\\0\ebm$ and $\vec{w} = \bbm \frac{1}{5}\\-\frac{7}{5}\\0\\1\ebm$.

\medskip

{\em Note:} Many people prefer to pull out the scalar multiple of $\frac{1}{5}$ created to get the matrix into row-echelon form and use the vectors $\bbm 7\\6\\5\\0\ebm$ and $\bbm 1\\-7\\0\\5\ebm$ instead. Either answer is equivalent, since you can absorb the scalar multiple into the parameters.


 \end{enumerate}

\end{document}