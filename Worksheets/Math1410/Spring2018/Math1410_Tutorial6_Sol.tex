\documentclass[12pt]{article}
\usepackage{amsmath}
\usepackage{amssymb}
\usepackage[letterpaper,top=0.75in,bottom=1in,left=0.75in,right=0.75in,centering]{geometry}
%\usepackage{fancyhdr}
\usepackage{enumerate}
%\usepackage{lastpage}
\usepackage{multicol}
\usepackage{graphicx}

\reversemarginpar

%\pagestyle{fancy}
%\cfoot{}
%\lhead{Math 1560}\chead{Test \# 1}\rhead{May 18th, 2017}
%\rfoot{Total: 10 points}
%\chead{{\bf Name:}}
\newcommand{\points}[1]{\marginpar{\hspace{24pt}[#1]}}
\newcommand{\skipline}{\vspace{12pt}}
%\renewcommand{\headrulewidth}{0in}
\headheight 30pt

\newenvironment{amatrix}[1]{%
  \left[\begin{array}{@{}*{#1}{c}|c@{}}
}{%
  \end{array}\right]
}
\newcommand{\di}{\displaystyle}
\newcommand{\abs}[1]{\lvert #1\rvert}
\newcommand{\len}[1]{\lVert #1\rVert}
\renewcommand{\i}{\mathbf{i}}
\renewcommand{\j}{\mathbf{j}}
\renewcommand{\k}{\mathbf{k}}
\newcommand{\R}{\mathbb{R}}
\newcommand{\aaa}{\mathbf{a}}
\newcommand{\bbb}{\mathbf{b}}
\newcommand{\ccc}{\mathbf{c}}
\newcommand{\dotp}{\boldsymbol{\cdot}}
\newcommand{\bbm}{\begin{bmatrix}}
\newcommand{\ebm}{\end{bmatrix}}       
\DeclareMathOperator{\proj}{proj}      
\newcommand{\bam}{\begin{amatrix}}
\newcommand{\eam}{\end{amatrix}}   
                  
\begin{document}


\author{Instructor: Sean Fitzpatrick}
\thispagestyle{empty}
\vglue1cm
\begin{center}
{\bf MATH 1410 - Tutorial \#6 Solutions}
\end{center}

\textbf{Assigned problems:}

 \begin{enumerate}
\item For each system of equations below, write down the corresponding augmented matrix.
\begin{multicols}{2}
\begin{enumerate}
 \item $\arraycolsep=1pt\begin{array}{ccccccc}2x&-&3y&+&z&=&2\\[5pt] & &2y&-&5z&=&-3\\[5pt] -3x& & &+&2x&=&7\end{array}$

\medskip

 $\bam{3}2&-3&1&2\\0&2&-5&-3\\-3&0&2&7\eam$

 \item $\arraycolsep=1pt\begin{array}{ccccccccc}x_1&+&4x_2& & &-&7x_4&=&0\\[5pt]-3x_1&-&x_2&+&4x_3& & &=&2\\[5pt] & &2x_2&-&4x_3&+&x_4&=&-5\end{array}$
 
 \medskip
 
 $\bam{4}1&4&0&-7&0\\-3&-1&4&0&2\\0&2&-4&1&-5\eam$
 \end{enumerate}
\end{multicols}

\bigskip

\item For each augmented matrix below, write down a corresponding system of equations using whatever variables you prefer.
\begin{multicols}{2}
\begin{enumerate}
 \item $\bam{3}2&-1&0&4\\-3&4&1&-2\\0&2&3&-7\eam$

\medskip

 $\arraycolsep=1pt\begin{array}{ccccccc}
 2x&-&y& & &=&4\\-3x&+&4y&+&z&=&-2\\ & &2y&+&3z&=&-7\end{array}$

 \item $\bam{4}3&2&0&1&-5\\0&4&2&-7&2\eam$
 
\medskip

 $\arraycolsep1pt\begin{array}{ccccccccc}
 3x_1&+&2x_2& & &+&x_4&=&-5\\ & &4x_2&+&2x_3&-&7x_4&=&2\end{array}$
\end{enumerate}
\end{multicols}

\bigskip

\item State whether or not the given augmented matrix is in reduced row-echelon form (RREF), and if not, why.
\begin{multicols}{5}
$\bam{3}1&0&2&-1\\0&1&2&4\\0&0&0&0\eam$

RREF\columnbreak

$\bam{3}1&0&0&2\\0&1&1&0\\0&0&0&4\eam$

Not RREF: the 4 in row 3 should be a 1.\columnbreak

$\bam{3}1&2&0&3\\0&1&0&-4\\0&0&1&2\eam$

Not RREF due to the 2 in row 1 above the leading 1 in row 2.\columnbreak

$\bam{3}1&0&0&7\\0&2&0&3\\0&0&1&0\eam$

Not in RREF: the 2 in row 2 needs to be a 1.\columnbreak

$\bam{4}0&1&0&2&-3\\0&0&1&-3&4\\0&0&0&1&3\eam$

Not in RREF: There are two non-zero entries above the leading 1 in row 3.
\end{multicols}

\pagebreak

\item Suppose you want to perform Gaussian elimination on the augmented matrices below. \\For each matrix, what are the first two row operations you would perform, and why?

\begin{enumerate}
\item $\bam{3} 1&-4&2&0\\-2&4&1&6\\3&2&-1&1\eam$

$R_2+2R_1\to R_2$ and 
$R_3-3R_1\to R_3$, to create zeros in the first column below the leading 1.

\item $\bam{3} 2&4&-8&10\\-1&2&4&-5\\0&1&5&2\eam$

There are several reasonable options here. One is $\frac12 R_1\to R_1$ to get a leading one in the first row, then $R_2+R_1\to R_2$ to create a zero below it. Another option would be $R_1\leftrightarrow R_2$, followed by $-R_1\to R_1$, to get a leading one with minimal arithmetic. Another would be $R_1+R_2\to R_1$ to create a leading one in the first row, followed by $R_2-R_1\to R_2$ to create a zero below it.

\item $\bam{3} 3&2&-7&4\\1&2&-4&0\\0&-1&3&2\eam$

$R_1\leftrightarrow R_3$, to get a leading 1 in the first row without creating fractions, then $R_3-3R_1\to R_3$ to create a zero below the leading 1.
\end{enumerate}

\bigskip

\item For each matrix $A$ and $B$ below, write down the row operation that transforms $A$ into $B$.

\begin{enumerate}
\item $A = \bbm 3&-2&5\\2&8&-4\\1&-2&1\ebm$, $B = \bbm 3&-2&5\\1&4&-2\\1&-2&1\ebm$ \hspace{1cm} $\frac12 R_2\to R_2$

\bigskip

\item $A = \bbm 2&7&-3\\6&8&1\\1&12&-6\ebm$, $B = \bbm 2&7&-3\\0&-13&10\\1&12&-6\ebm$ \hspace{1cm} $R_2-3R_1\to R_2$

\bigskip

\item $A = \bbm 4&-2&3\\1&3&4\\-5&6&0\ebm$, $B = \bbm -5&6&0\\1&3&4\\4&-2&3\ebm$ \hspace{1cm} $R_1\leftrightarrow R_3$

\end{enumerate}


\pagebreak

\item Write down the augmented matrix of the following system, and then use Gaussian elimination to solve the system.
\[
 \arraycolsep=2pt \begin{array}{ccccccc}
                  x&+&2y&-&z&=&4\\-x&+&y&-&2z&=&-1\\ 2x&+&6y&-&3z&=&5
                  \end{array}
\]

We have the following augmented matrix and elimination steps. We begin by eliminating all the non-zero entries below our first leading one.
\[
\bam{3}
1&2&-1&4\\
-1&1&-2&-1\\
2&6&-3&5
\eam  \xrightarrow[]{R_2+R_1\to R_2}
\bam{3}
 1&2&-1&4\\
 0&3&-3&3\\
 2&6&-3&5
\eam
 \xrightarrow[]{R_3-2R_1\to R_3} 
\bam{3}
 1&2&-1&4\\
 0&3&-3&3\\
 0&2&-1&-3
\eam
\]
Next, we can get our second leading one by dividing by 3 in the second row. We can then use that leading one to eliminate the non-zero entry below it:
\[
\bam{3}
 1&2&-1&4\\
 0&3&-3&3\\
 0&2&-1&-3
\eam
\xrightarrow[]{\frac13 R_2\to R_2}
\bam{3}
 1&2&-1&4\\
 0&1&-1&1\\
 0&2&-1&-3
\eam 
\xrightarrow[]{R_3-2R_2\to R_3}
\bam{3}
 1&2&-1&4\\
 0&1&-1&1\\
 0&0&1&-5
\eam
\]
At this point we've reached row-echelon form, and we have the option of solving by back-substitution. Row 3 tells us that $z=-5$. Row 2 says $y-z=1$. Putting $z=-5$ into this equation, we get $y+5=1$, so $y=-4$. row 1 says $x+2y-z=4$. Putting $y=-4$ and $z=-5$, we get $x-8+5=4$, so $x=7$.



Alternatively, we can continue with the augmented matrix, performing the ``backward steps'' to reach RREF:

\noindent\hskip -60pt
\noindent\begin{minipage}{1.2\textwidth}
\[
\bam{3}
 1&2&-1&4\\
 0&1&-1&1\\
 0&0&1&-5
\eam  \xrightarrow[]{R_2+R_3\to R_2} 
\bam{3}
 1&2&-1&4\\
 0&1&0&-4\\
 0&0&1&-5
\eam
\xrightarrow[]{R_1+R_3\to R_1} 
\bam{3}
 1&2&0&-1\\
 0&1&0&-4\\
 0&0&1&-5
\eam
\xrightarrow[]{R_1-2R_2\to R_1}
\bam{3}
 1&0&0&7\\
 0&1&0&-4\\
 0&0&1&-5
\eam 
\]
\end{minipage}

From here we can directly read off the solution $x=7, y=-4, z=-5$.

Of course, we can also confirm that our solution works by plugging these values into each of our original equations: $7+2(-4)-(-5)=4$, $-7+(-4)+2(-5)=-1$, and $2(7)+6(-4)-3(-5) = 5$.

\bigskip


\item A system in variables $x,y,z$ has an augmented matrix with RREF $\di \bam{3} 1&0&-3&4\\0&1&2&6\eam$.

Write down the system of equations corresponding to this matrix. How would you describe the solution to the system?

(Hint: what geometric problem corresponds to a system of two equations in three variables?)

The first row corresponds to the equation $x-3z=4$, and the second to the equation $y+2z=6$. From Chapter 3, we recall that two equations in three variables represents the intersection of two planes, and we expect the solution to be a line. Indeed, we note that both equations can easily be solved, for $x$ and $y$ respectively, in terms of $z$. If we assign $z$ to a parameter $t$, then we have
\[
x=4+3t, y=6-2t, z=t,
\]
which represents the parametric equations for a line through the point $(4,6,0)$ in the direction of the vector $\langle 3,-2,1\rangle$.

 \end{enumerate}
  
Additional practice: (\textbf{do not submit}).
\begin{enumerate}
\item Use Gaussian elimination to find the reduced row-echelon form of the matrix:

\begin{enumerate}
\item \begin{align*}
\bbm 2&3&-1\\1&4&0\ebm \xrightarrow{R_1\leftrightarrow R_2}& \bbm 1&4&0\\2&3&-1\ebm \xrightarrow{R_2-2R_1}\bbm 1&4&0\\0&-5&-1\ebm
\xrightarrow{-\frac15 R_2\to R_2}\\ & \bbm 1&4&0\\0&1&\frac15\ebm \xrightarrow{R_1-4R_2\to R_1} \bbm 1&0&-\frac45\\0&1&\frac15\ebm
\end{align*}
\item $\bbm 4&8\\-2&-4\ebm\xrightarrow[-frac12 R_2\to R_2]{\frac14 R_1\to R_1}\bbm 1&2\\1&2\ebm \xrightarrow{R_2-R_1\to R_2} \bbm 1&2\\0&0\ebm$
\item 
\begin{align*}
\bbm 1&3&2&-1\\-2&1&3&4\\-1&4&5&3\ebm \xrightarrow[R_3+R_1\to R_3]{R_2+2R_1\to R_2} &\bbm 1&3&2&-1\\0&7&7&2\\0&7&7&2\ebm \xrightarrow{R_3-R_2\to R_2}\bbm 1&3&2&-1\\0&7&7&2\\0&0&0&0\ebm\xrightarrow{\frac17 R_2\to R_2}\\
&\bbm 1&3&2&-1\\0&1&1&\frac27\\0&0&0&0\ebm \xrightarrow{R_1-3R_2\to R_1}\bbm 1&0&-1&-\frac{13}{7}\\0&1&1&\frac27\\0&0&0&0\ebm
\end{align*}


\end{enumerate}

\item Solve the system of equations:

\begin{enumerate}
\item $\arraycolsep2pt\begin{array}{ccccc}
2x&-&3y&=&7\\-x&+&2y&=&2\end{array}$

We set up the corresponding augmented matrix and reduce:
\[
\bam{2}2&-3&7\\-1&2&2\eam \xrightarrow{R_1+R_2\to R_1} \bam{2} 1&-1&9\\-1&2&2\eam \xrightarrow{R_2+R_1\to R_2}\bam{2}1&-1&9\\0&1&11\eam.
\]
The second row gives us $y=11$, and the first gives $x-y=9$ Putting $y=11$ in this equation gives us $x=9+11=20$, so $x=20$, $y=11$ is the solution.

\item $\arraycolsep2pt\begin{array}{ccccccc}
x&-&2y&+&4z&=&2\\2x&-&3y&+&z&=&-2\\-x&+&2y&-&2z&=&6\end{array}$

We have the following augmented matrix and ``forward'' reduction steps:
\[
\bam{3}
1&-2&4&2\\
2&-3&1&-2\\
-1&2&-2&6
\eam \xrightarrow[R_3+R_1\to R_3]{R_2-2R_1\to R_2}
\bam{3}
1&-2&4&2\\
0&1&-7&-6\\
0&0&2&8
\eam \xrightarrow{\frac12 R_3\to R_3}
\bam{3} 
1&-2&4&2\\
0&1&-7&-6\\
0&0&1&4
\eam
\]
Our matrix is now in row-echelon form. If we proceed by back substitution, we have
\begin{align*}
z=4&\\
y-7z=-6,&\, \text{ so } y=-6+7(4)=22.\\
x-2y+4z=2,&\, \text{ so } x=2+2(22)-4(4)=30.
\end{align*}
Alternatively, we can continue with the ``backward'' reduction steps for our augmented matrix:
\[
\bam{3}
1&-2&4&2\\
0&1&-7&-6\\
0&0&1&4
\eam \xrightarrow[R_2+7R_3\to R_3]{R_1-4R_3\to R_1}
\bam{3}
1&-2&0&-14\\
0&1&0&22\\
0&0&1&4
\eam \xrightarrow{R_1+2R_2\to R_1}
\bam{3}
1&0&0&30\\
0&1&0&22\\
0&0&1&4\eam
\]
Our matrix is now in reduced row-echelon form, and we can read off the solution $x=30, y=22, z=4$.
\end{enumerate}

\end{enumerate}


\newpage
%\thispagestyle{empty}


\end{document}