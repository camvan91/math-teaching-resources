\documentclass[12pt]{article}
\usepackage{amsmath}
\usepackage{amssymb}
\usepackage[letterpaper,top=1in,bottom=1in,left=0.75in,right=0.75in,centering]{geometry}
%\usepackage{fancyhdr}
\usepackage{enumerate}
%\usepackage{lastpage}
\usepackage{multicol}
\usepackage{graphicx}

\reversemarginpar

%\pagestyle{fancy}
%\cfoot{}
%\lhead{Math 1560}\chead{Test \# 1}\rhead{May 18th, 2017}
%\rfoot{Total: 10 points}
%\chead{{\bf Name:}}
\newcommand{\points}[1]{\marginpar{\hspace{24pt}[#1]}}
\newcommand{\skipline}{\vspace{12pt}}
%\renewcommand{\headrulewidth}{0in}
\headheight 30pt

\newenvironment{amatrix}[1]{%
  \left[\begin{array}{@{}*{#1}{c}|c@{}}
}{%
  \end{array}\right]
}
\newcommand{\di}{\displaystyle}
\newcommand{\abs}[1]{\lvert #1\rvert}
\newcommand{\len}[1]{\lVert #1\rVert}
\renewcommand{\i}{\mathbf{i}}
\renewcommand{\j}{\mathbf{j}}
\renewcommand{\k}{\mathbf{k}}
\newcommand{\R}{\mathbb{R}}
\newcommand{\aaa}{\mathbf{a}}
\newcommand{\bbb}{\mathbf{b}}
\newcommand{\ccc}{\mathbf{c}}
\newcommand{\dotp}{\boldsymbol{\cdot}}
\newcommand{\bbm}{\begin{bmatrix}}
\newcommand{\ebm}{\end{bmatrix}}       
\DeclareMathOperator{\proj}{proj}      
\newcommand{\bam}{\begin{amatrix}}
\newcommand{\eam}{\end{amatrix}}   
                  
\begin{document}


\author{Instructor: Sean Fitzpatrick}
\thispagestyle{empty}
\vglue1cm
\begin{center}
{\bf MATH 1410 - Tutorial \#7 Solutions}
\end{center}


Additional practice: (\textbf{do not submit}).
\begin{enumerate}
\item Determine if the following sets of vectors are linearly independent:
\begin{enumerate}
\item $\left\{\bbm 3\\-2\\0\ebm, \bbm 1\\0\\-2\ebm, \bbm 0\\2\\-4\ebm\right\}$

Consider the equation $x\bbm 1\\0\\-2\ebm+y \bbm 0\\2\\-4\ebm+z\bbm 3\\-2\\0\ebm=\bbm 0\\0\\0\ebm$, which is equivalent to the matrix equation
\[
\bbm 1&0&3\\0&2&-2\\-2&-4&3\ebm\bbm x\\y\\z\ebm = \bbm 0\\0\\0\ebm.
\]
(We changed the order of the vectors for convenience, since now we have a leading 1 in the $(1,1)$-entry of our matrix.) We know that this equation can be solved by reducing the corresponding augmented matrix:
\[
\bam{3} 1&0&3&0\\0&2&-2&0\\-2&-4&3&0\eam \xrightarrow[\frac12 R_2\to R_2]{R_3+2R_1\to R_3}\bam{3}1&0&3&0\\0&1&-1&0\\0&-4&9\eam\\
\xrightarrow{R_3+4R_2\to R_3}\bam{3}1&0&3&0\\0&1&-1&0\\0&0&5&0\eam
\]
Dividing the last row by 5 puts us in row-echelon form, and since there is a leading 1 in each of the $x$, $y$, and $z$ columns, we know that the only solution is the unique solution $x=y=z=0$, and we can conclude that the vectors are linearly independent.

\item $\left\{\bbm 1\\-4\\2\ebm, \bbm 2\\-3\\1\ebm, \bbm -1\\-6\\4\ebm\right\}$

We look for solutions to the equation $x\bbm 1\\-4\\2\ebm+y \bbm 2\\-3\\1\ebm+z \bbm -1\\-6\\4\ebm = \bbm 0\\0\\0\ebm$. This corresponds to the system
\[\arraycolsep1pt
\begin{array}{ccccccc}
x&+&2y&-&z&=&0\\
-4x&-&3y&-&6z&=&0\\
2x&+&y&+&4z&=&0
\end{array},
\]
which we solve by reducing the corresponding augmented matrix, as usual:
\begin{align*}
\bam{3} 1&2&-1&0\\-4&-3&-6&0\\2&1&4&0\eam \xrightarrow[R_3-2R_1\to R_3]{R_2+4R_1\to R_2} &\bam{3} 1&2&-1&0\\0&5&-10&0\\0&-3&6&0\eam\\
\xrightarrow{\frac15 R_2\to R_2}&\bam{3} 1&2&-1&0\\0&1&-2&0\\0&-3&6&0\eam\\
\xrightarrow{R_3+3R_2\to R_3}&\bam{3} 1&2&-1&0\\0&1&-2&0\\0&0&0&0\eam\\
\xrightarrow{R_1-2R_2\to R_1}&\bam{3} 1&0&3&0\\0&1&-2&0\\0&0&0&0\eam
\end{align*}
This time, we see that non-trivial solutions exist: we can take $x=-3z$ and $y=2z$, where $z$ can be any real number. In particular, if $z=1$, we find
\[
-3\bbm 1\\-4\\2\ebm+2 \bbm 2\\-3\\1\ebm+ \bbm -1\\-6\\4\ebm = \bbm 0\\0\\0\ebm.
\]

\end{enumerate}

\item For each set of vectors in the previous problem, determine whether or not the vectors\\ $\vec{v} = \bbm 2\\-1\\3\ebm$ and $\vec{w} = \bbm -3\\2\\0\ebm$ belong to the span of the set.

The first set of vectors is linearly independent, and any three linearly independent vectors in $\R^3$ span all of $\R^3$, so both $\vec{v}$ and $\vec{w}$ belong to the span. (Notice that no matter what numbers we put to the right of the vertical bar in the augmented matrix, we will have a unique solution.)

I'll leave it as an exercise for you to determine what those unique solutions are.

Since the second set of vectors is not linearly independent, a more careful inspection is needed. The equations
\[
x\bbm 1\\-4\\2\ebm+y \bbm 2\\-3\\1\ebm+z \bbm -1\\-6\\4\ebm = \bbm 2\\-1\\3\ebm \quad \text{ and } \quad x\bbm 1\\-4\\2\ebm+y \bbm 2\\-3\\1\ebm+z \bbm -1\\-6\\4\ebm=\bbm -3\\2\\0\ebm
\]
correspond to matrix equations 
\[
\bbm 1&2&-1\\-4&-3&-6\\2&1&4\ebm\bbm x\\y\\z\ebm=\bbm 2\\-1\\3\ebm \quad \text{ and } \quad \bbm 1&2&-1\\-4&-3&-6\\2&1&4\ebm\bbm x\\y\\z\ebm = \bbm -3\\2\\0\ebm.
\]
Since the coefficient matrix is the same for both, we can solve both equations simultaneously by adding a second column to the right of the vertical bar in our augmented matrix. We have
\begin{align*}
\left[\begin{array}{ccc|c|c}
1&2&-1&2&-3\\
-4&-3&-6&-1&2\\
2&1&4&3&0
\end{array}\right] \xrightarrow[R_3-2R_1\to R_3]{R_2+4R_1\to R_2}
&\left[\begin{array}{ccc|c|c}
1&2&-1&2&-3\\
0&5&-10&7&-10\\
0&-3&6&-1&6
\end{array}\right]\\
\xrightarrow{\frac15 R_2\to R_2}
&\left[\begin{array}{ccc|c|c}
1&2&-1&2&-3\\
0&1&-2&7/5&-2\\
0&-3&6&-1&6
\end{array}\right]\\
\xrightarrow{R_3+3R_2\to R_3}
&\left[\begin{array}{ccc|c|c}
1&2&-1&2&-3\\
0&1&-2&7/5&-2\\
0&0&0&16/5&0
\end{array}\right]
\end{align*}
Looking at the first of the two constants columns, we see that there cannot be a solution using the vector $\vec{v}$ for the constants, so this vector is not in the span.

The zero at the bottom of the right-most column tells us that we will be able to solve using the vector $\vec{w}$ for the constants. Indeed, if we drop the $\vec{v}$ column and continue, we have
\[
\bam{3}1&2&-1&-3\\
0&1&-2&-2\\
0&0&0&0\eam \xrightarrow{R_1-2R_2\to R_1} \bam{3}1&0&3&1\\0&1&-2&-2\\0&0&0&0\eam.
\]
From here, we find the solution $x=1-3z$ and $y=-2+2z$, where $z$ is free. Putting $z=0$, we get the solution $x=1$, $y=-2$, $z=0$, and we can confirm that
\[\bbm 1\\-4\\2\ebm-2 \bbm 2\\-3\\1\ebm+0 \bbm -1\\-6\\4\ebm=\bbm -3\\2\\0\ebm.
\]
Note that if we put $z=t$ and input the general solution, we have
\begin{align*}
(1-3t)\bbm 1\\-4\\2\ebm+(-2+2t)\bbm 2\\-3\\1\ebm + t\bbm -1\\-6\\4\ebm & =  \bbm -3\\2\\0\ebm\\
\left(\bbm 1\\-4\\2\ebm-2 \bbm 2\\-3\\1\ebm\right)+t\left(-3\bbm 1\\-4\\2\ebm+2\bbm 2\\-3\\1\ebm + 1\bbm -1\\-6\\4\ebm\right) & = \bbm -3\\2\\0\ebm\\
\left(\bbm 1\\-4\\2\ebm-2 \bbm 2\\-3\\1\ebm\right)+t\bbm 0\\0\\0\ebm &= \bbm -3\\2\\0\ebm
\end{align*}
So we can make sense of why there are infinitely many solutions when there is a parameter: when testing for independence, we confirmed that 
$
-3\bbm 1\\-4\\2\ebm+2 \bbm 2\\-3\\1\ebm+ \bbm -1\\-6\\4\ebm = \bbm 0\\0\\0\ebm,
$
and this is precisely what is being multiplied by $t$. We can choose any value we want for $t$ because $t$ times the zero vector is still the zero vector, no matter what $t$ is!
\end{enumerate}


\newpage
%\thispagestyle{empty}

\textbf{Assigned problems}

 \begin{enumerate}
\item Let $\vec{u},\vec{v},\vec{w}$ be vectors in $\R^n$. Answer the following as precisely as possible:
\begin{enumerate}
\item What is a \textbf{linear combination} of $\vec{u}$, $\vec{v}$, and $\vec{w}$?

A \textbf{linear combination} of $\vec{u},\vec{v},\vec{w}$ is any expression of the form $a\vec{u}+b\vec{v}+c\vec{w}$, where $a,b,c$ are scalars.

\item What is the \textbf{span} of $\vec{u}$, $\vec{v}$, and $\vec{w}$?

The \textbf{span} of $\vec{u},\vec{v},\vec{w}$ is the set of all linear combinations that can be formed from these vectors. That is,
\[
\operatorname{span}\{\vec{u},\vec{v},\vec{w}\} = \{a\vec{u}+b\vec{v}+c\vec{w}\,|\, a,b,c\in\R\}.
\]

\item What does it mean to say that $\{\vec{u},\vec{v},\vec{w}\}$ is linearly independent?

The vectors $\vec{u},\vec{v},\vec{w}$ are \textbf{linearly independent} if the only solution to the equation
\[
x\vec{u}+y\vec{v}+z\vec{w}=\vec{0},
\]
for scalars $x,y,z$, is $x=y=z=0$.


\end{enumerate}
\item Find the basic solutions of the homogeneous system \hspace{24pt} $\arraycolsep1pt
\begin{array}{ccccccccc}
x&-&2y&+&z& & &=&0\\
-2x&+&y&-&3z&-&w&=&0\\
 & &3y&+&z&+&w&=&0
\end{array}
$.

We set up our augmented matrix and reduce:
\begin{align*}
\bam{4}
1&-2&1&0&0\\
-2&1&-3&-1&0\\
0&3&1&1&0
\eam \xrightarrow{R_2+2R_1\to R_2}& 
\bam{4}
1&-2&1&0&0\\
0&-3&-1&-1&0\\
0&3&1&1&0
\eam \xrightarrow{R_3+R_2\to R_3}
\bam{4}
1&-2&1&0&0\\
0&-3&-1&-1&0\\
0&0&0&0&0
\eam\\
\xrightarrow{-\frac13 R_2\to R_2}&
\bam{4}
1&-2&1&0&0\\
0&1&1/3&1/3&0\\
0&0&0&0&0
\eam \xrightarrow{R_1+2R_2\to R_1}
\bam{4}
1&0&5/3&2/3&0\\
0&1&1/3&1/3&0\\
0&0&0&0&0
\eam
\end{align*}
From here we see that $z$ and $w$ are free variables, while $x=-5/3z-2/3w$ and $y=-1/3z-1/3w$. In vector form, we have
\[
\bbm x\\y\\z\\w\ebm = \bbm -5/3z-2/3w\\-1/3z-1/3w\\z\\w\ebm = z\bbm-5/3\\-1/3\\1\\0\ebm + w\bbm -2/3\\-1/3\\0\\1\ebm,
\]
so our basic solutions are $\bbm-5/3\\-1/3\\1\\0\ebm$ and $\bbm -2/3\\-1/3\\0\\1\ebm$.

\newpage

\item The following problems can be answered without doing any computations (but be sure to justify your answer):
\begin{enumerate}
\item Is the set of vectors $\left\{\bbm 2\\-1\ebm, \bbm 1\\-3\ebm, \bbm 4\\7\ebm\right\}$ linearly independent?

They can't be, since this is a set of three vectors in $\R^2$, and at most two vectors in $\R^2$ can be independent.

\item Can the vectors $\left\{\bbm 1\\-1\\1\ebm, \bbm 2\\-2\\2\ebm, \bbm -3\\3\\-3\ebm\right\}$ span all of $\R^3$?

No, they can't, since all three vectors are parallel, so they only span a line through the origin.

\item Let $\vec{u}$ and $\vec{v}$ be vectors in $\R^3$. Will the set $\{\vec{u},\vec{v}, \vec{u}\times\vec{v}\}$ always be linearly independent? If not, what can go wrong?

If $\vec{u}$ and $\vec{v}$ are parallel (or if either one is zero), then $\vec{u}\times \vec{v}=\vec{0}$, and any set containing the zero vector is linearly dependent. However, if $\vec{u}$ and $\vec{v}$ are non-zero, non-parallel vectors, then they span a plane, and we know that $\vec{u}\times \vec{v}$ is a normal vector for that plane, so it can't possibly lie in the plane, and in this case the vectors are independent.

\end{enumerate}

\item Let $\vec{u} = \bbm 1\\-2\\0\ebm$ and $\vec{v} = \bbm 0\\-1\\3\ebm$. Determine a nonzero vector $\vec{w}$ such that the set $\{\vec{u},\vec{v},\vec{w}\}$ is:
\begin{enumerate}
\item Linearly dependent: Any linear combination of $\vec{u}$ and $\vec{v}$ will do; for example, $\vec{w}=\vec{v}$, or $\vec{w}=2\vec{u}$, or $\vec{w}=2\vec{u}-3\vec{v}$.



\item Linearly independent: Based on our answer for 3(c), taking $\vec{w}=\vec{u}\times \vec{v}$ should do the job. (Since $\vec{u}$ and $\vec{v}$ are not parallel, the set of all linear combinations $s\vec{u}+t\vec{v}$ is a plane. To get an independent set, we need to ensure that $\vec{w}$ is not a linear combination of $\vec{u}$ and $\vec{v}$, which means we can take any vector that does not lie in this plane. Since the cross product gives the normal vector, it certainly does not lie in the plane.)

Of course, this is a trick that is specialized to $\R^3$. What if we didn't have this trick available? Well, we would need to ensure that $\vec{w}$ is not in the span of $\vec{u}$ and $\vec{v}$, which means that we need to guarantee that there are no solutions to the equation $x\vec{u}+y\vec{v}=\vec{w}$. Writing $\vec{w}=\bbm a\\b\\c\ebm$ we set up and reduce the corresponding augmented matrix:
\[
\bam{2}1&0&a\\-2&-1&b\\0&3&c\eam \xrightarrow{R_2+2R_1\to R_2} \bam{2} 1&0&a\\0&-1&b+2a\\0&3&c\eam\xrightarrow{R_3+3R_2\to R_3}\bam{2} 1&0&a\\0&-1&b+2a\\0&0&c+3b+6a\eam
\]
From here, we see that the system will be inconsistent, and thus, $\{\vec{u},\vec{v},\vec{w}\}$ is independent, as long as $6a+3b+c\neq 0$.

(This, by the way, amounts to saying that $\vec{w}$ is not in the plane spanned by $\vec{u}$ and $\vec{v}$ --- if you compute $\vec{u}\times \vec{v}$ to get the normal vector, you'll find that $6x+3y+z=0$ is exactly the equation of the plane!)
\end{enumerate}
\newpage

\item Show that the vectors $\bbm 1\\2\\-1\ebm, \bbm 0\\1\\-2\ebm, \bbm 3\\4\\1\ebm$ are linearly dependent.

Linear dependence means that we can find a non-trivial solution to the vector equation
\[
x\bbm 1\\2\\-1\ebm+y \bbm 0\\1\\-2\ebm+z \bbm 3\\4\\1\ebm=\bbm 0\\0\\0\ebm.
\]
Combining the vectors on the left-hand side, we have
\[
\bbm x+0y+3z\\2x+y+4z\\-z-2y+z\ebm = \bbm 0\\0\\0\ebm,
\]
and equating components of our two vectors gives us a homogeneous system of three equations in three variables. We set up the augmented matrix for this system and reduce:
\[
\bam{3}1&0&3&0\\2&1&4&0\\-1&-2&1&0\eam \xrightarrow[R_3+R_1\to R_3]{R_2-2R_1\to R_2}\bam{3}1&0&3&0\\0&1&-2&0\\0&-2&4&0\eam \xrightarrow{R_3+2R_2\to R_3}\bam{3}1&0&3&0\\0&1&-2&0\\0&0&0&0\eam
\]
From here we can read off the general solution $x=-3t, y=2t, z=t$, where $t$ can be any real number. Since there is no restriction on $z=t$, we see that there do indeed exist non-trivial solutions, so the vectors are linearly dependent, as required.

For example, setting $t=1$ gives $x=-3, y=2, z=1$, and we can easily confirm that 
\[
-3\bbm 1\\2\\-1\ebm+2 \bbm 0\\1\\-2\ebm+ \bbm 3\\4\\1\ebm=\bbm 0\\0\\0\ebm.
\]

\item Challenge: Determine a condition on  $a$, $b$, and $c$ that guarantees $\bbm a\\b\\c\ebm$ belongs to the span of the vectors in the previous problem.

For a given $a,b,c$ we need to determine if there exist scalars $x,y,z$ such that
\[x\bbm 1\\2\\-1\ebm+y \bbm 0\\1\\-2\ebm+z \bbm 3\\4\\1\ebm=\bbm a\\b\\c\ebm.\]
This corresponds to a non-homogeneous system with the same left-hand sides as above, but $a,b,c$ replacing the zeros on the right. Since the left-hand sides are the same, we perform the same row operations on our augmented matrix as before:
\[
\bam{3}1&0&3&a\\2&1&4&b\\-1&-2&1&c\eam \xrightarrow[R_3+R_1\to R_3]{R_2-2R_1\to R_2}\bam{3}1&0&3&a\\0&1&-2&b-2a\\0&-2&4&c+a\eam \xrightarrow{R_3+2R_2\to R_3}\bam{3}1&0&3&a\\0&1&-2&b-2a\\0&0&0&c-3a+2b\eam
\]
Our system has no solution if the last entry in our matrix is non-zero; thus, the vector $\bbm a\\b\\c\ebm$ is in the span if and only if $c-3a+2b=0$.

(As one might expect, if we replace $a,b,c$ with $x,y,z$, we get the equation $3x-2y-z=0$, and this is indeed the equation of the plane spanned by the vectors $\vec{u}$ and $\vec{v}$.)
 \end{enumerate}
 
\end{document}