\documentclass[12pt]{article}
\usepackage{amsmath}
\usepackage{amssymb}
\usepackage[letterpaper,top=1in,bottom=1in,left=0.75in,right=0.75in,centering]{geometry}
%\usepackage{fancyhdr}
\usepackage{enumerate}
%\usepackage{lastpage}
\usepackage{multicol}
\usepackage{graphicx}

\reversemarginpar

%\pagestyle{fancy}
%\cfoot{}
%\lhead{Math 1560}\chead{Test \# 1}\rhead{May 18th, 2017}
%\rfoot{Total: 10 points}
%\chead{{\bf Name:}}
\newcommand{\points}[1]{\marginpar{\hspace{24pt}[#1]}}
\newcommand{\skipline}{\vspace{12pt}}
%\renewcommand{\headrulewidth}{0in}
\headheight 30pt

\newenvironment{amatrix}[1]{%
  \left[\begin{array}{@{}*{#1}{c}|c@{}}
}{%
  \end{array}\right]
}
\newcommand{\di}{\displaystyle}
\newcommand{\abs}[1]{\lvert #1\rvert}
\newcommand{\len}[1]{\lVert #1\rVert}
\renewcommand{\i}{\mathbf{i}}
\renewcommand{\j}{\mathbf{j}}
\renewcommand{\k}{\mathbf{k}}
\newcommand{\R}{\mathbb{R}}
\newcommand{\aaa}{\mathbf{a}}
\newcommand{\bbb}{\mathbf{b}}
\newcommand{\ccc}{\mathbf{c}}
\newcommand{\dotp}{\boldsymbol{\cdot}}
\newcommand{\bbm}{\begin{bmatrix}}
\newcommand{\ebm}{\end{bmatrix}}       
\newcommand{\bvm}{\begin{vmatrix}}
\newcommand{\evm}{\end{vmatrix}}      
\DeclareMathOperator{\proj}{proj}      
\newcommand{\bam}{\begin{amatrix}}
\newcommand{\eam}{\end{amatrix}}   
                  
\begin{document}


\author{Instructor: Sean Fitzpatrick}
\thispagestyle{empty}
\vglue1cm
\begin{center}
{\bf MATH 1410 - Tutorial \#11 Solutions}
\end{center}
 \begin{enumerate}
 
\item Let $A = \bbm 2&-2\\1&5\ebm$. Compute $A\vec{u}_i$ for $i=1,2,3,4$, where:
\[
\vec{u}_1 =  \bbm 1\\4\ebm, \vec{u}_2 = \bbm 4\\-2\ebm, \vec{u}_3 = \bbm 2\\-2\ebm, \vec{u}_4 = \bbm 3\\-3\ebm.
\]
Which of the above were eigenvectors? What are the eigenvalues of $A$?

We compute each product as follows:
\begin{align*}
A\vec{u}_1 & = \bbm 2&-2\\1&5\ebm\bbm 1\\4\ebm = \bbm -6\\21\ebm \neq \lambda \bbm 1\\4\ebm \text{ for any scalar } \lambda.\\
A\vec{u}_2 & = \bbm 2&-2\\1&5\ebm\bbm 4\\-2\ebm = \bbm 12\\-6\ebm = 3\bbm 4\\-2\ebm\\
A\vec{u}_3 & = \bbm 2&-2\\1&5\ebm\bbm 2&-2\ebm = \bbm 8\\-8\ebm = 4\bbm 2\\-2\ebm\\
A\vec{u}_4 & = \bbm 2&-2\\1&5\ebm\bbm 3&-3\ebm = \bbm 12\\-12\ebm = 4\bbm 3\\-3\ebm
\end{align*}
From the above, we can conclude that $\vec{u}_2,\vec{u}_3,\vec{u}_4$ are eigenvectors, and that $\lambda =3,4$ are eigenvalues. Since $A$ is $2\times 2$, these must be all the eigenvalues. (Note that $\vec{u}_3$ and $\vec{u}_4$ are parallel vectors.)

\item Verify that the matrix $Z = \bbm 3&1\\-2&1\ebm$ has eigenvalues $\lambda_1 = 2 + i$ and $\lambda_2=2-i$ with corresponding eigenvectors $\vec{x}_1 = \bbm 1+i\\-2\ebm$, $\vec{x}_2 = \bbm 1\\-1-i\ebm$.

We compute $Z\vec{x}_i$ and $\lambda_i\vec{x}_i$ for $i=1,2$ to confirm:
\begin{align*}
Z\vec{x}_1 & = \bbm 3&1\\-2&1\ebm\bbm 1+i\\-2\ebm = \bbm 3+3i-2\\-2-2i-2\ebm = \bbm 1+3i\\-4-2i\ebm\\
\lambda_1\vec{x}_1 &= (2+i)\bbm 1+i\\-2\ebm = \bbm 2-1+2i+i\\-4-2i\ebm = \bbm 1+3i\\-4-2i\ebm=Z\vec{x}_1\\
Z\vec{x}_2 & = \bbm 3&1\\-2&1\ebm\bbm 1\\-1-i\ebm = \bbm 3-1-i\\-2-1-i\ebm = \bbm 2-i\\-3-i\ebm\\
\lambda_2\vec{x}_2 &= (2-i)\bbm 1\\-1-i\ebm = \bbm 2-i\\-2-1-2i+i\ebm = \bbm 2-i\\-3-i\ebm = Z\vec{x}_2
\end{align*}
\newpage

\item The matrix $A = \bbm 3&2&1\\1&4&1\\1&2&3\ebm$ has characteristic polynomial $c_A(\lambda)=-(\lambda-2)^2(\lambda-6)$. Find the eiqenvalues of $A$, and the corresponding eigenvectors.

\bigskip

Since the eigenvalues of $A$ are the roots of the characteristic polynomial, we see that the eigenvalues are $\lambda_1=2$  (with multiplicity 2) and $\lambda_2=6$.

For $\lambda_1=2$, we get
\[
A-2I = \bbm 1&2&1\\1&2&1\\1&2&1\ebm \xrightarrow[R_3-R_1\to R_3]{R_2-R_1\to R_2} \bbm 1&2&1\\0&0&0\\0&0&0\ebm.
\]
Thus, for $\vec{v}=\bbm x\\y\\z\ebm$ to be a solution to $(A-2I)\vec{v}=\vec{0}$, we must have $x=-2y-z$, where $y$ and $z$ are free variables. Thus,
\[
\vec{v}=\bbm -2y-z\\y\\z\ebm = y\bbm -2\\1\\0\ebm + z\bbm -1\\0\\1\ebm,
\]
giving us two independent eigenvectors: $\vec{v}_1 = \bbm -2\\1\\0\ebm$ and $\vec{v}_2 = \bbm -1\\0\\1\ebm$.

\medskip

For $\lambda_2=6$, we get
\begin{align*}
A-6I = &\bbm -3&2&1\\1&-2&1\\1&2&-3\ebm \xrightarrow{R_1\leftrightarrow R_2} \bbm 1&-2&1\\-3&2&1\\1&2&-3\ebm\xrightarrow[R_3-R_1\to R_3]{R_2+3R_1\to R_3} \bbm 1&-2&1\\0&-4&4\\0&4&-4\ebm\\
\xrightarrow{R_3+R_2\to R_3}&\bbm 1&-2&1\\0&-4&4\\0&0&0\ebm \xrightarrow{-\frac14 R_2\to R_2}\bbm 1&-2&1\\0&1&-1\\0&0&0\ebm \xrightarrow{R_1+2R_2\to R_1}\bbm 1&0&-1\\0&1&-1\\0&0&0\ebm.
\end{align*}
If $\vec{w}=\bbm x\\y\\z\ebm$ is a solution to $(A-6I)\vec{w}=\vec{0}$, we must therefore have $x=z$ and $y=z$, so $\vec{w} = \bbm z\\z\\z\ebm = z\bbm 1\\1\\1\ebm$. This gives us the eigenvector $\vec{w} = \bbm 1\\1\\1\ebm$ corresponding to $\lambda_2=6$.
\newpage

\item Compute the eigenvalues of the matrix $A = \bbm 3&-1&2\\0&3&1\\0&4&3\ebm$

\bigskip

The characteristic polynomial is
\begin{align*}
c_A(\lambda) &= \bvm 3-\lambda & -1 &2\\0&3-\lambda &1\\0&4&3-\lambda \evm = (3-\lambda)\bvm 3-\lambda &1\\4&3-\lambda\evm\\
& = (3-\lambda)((3-\lambda)^2-4)=(3-\lambda)((3-\lambda)+2)((3-\lambda)-2)\\
& = (3-\lambda)(5-\lambda)(1-\lambda).
\end{align*}

The eigenvalues of $A$ are therefore $\lambda_1=1, \lambda_2=3, \lambda_3=5$.
\item Compute the eigenvalues and eigenvectors of the matrix $A = \bbm 1&4\\2&3\ebm$.

\bigskip

The characteristic polynomial is
\[
c_A(\lambda) = \bvm 1-\lambda&4\\2&3-\lambda\evm = (1-\lambda)(3-\lambda)-8=\lambda^2-4\lambda-5=(\lambda-5)(\lambda+1),
\]
so the eigenvalues are $\lambda_1=-1$ and $\lambda_2=5$.

For $\lambda_1=-1$, we get
\[
A-(-1)I = \bbm 2&4\\2&4\ebm \xrightarrow{\text{RREF}} \bbm 1&2\\0&0\ebm.
\]
For $\vec{v}=\bbm x\\y\ebm$ to solve $(A+I)\vec{v}=\vec{0}$ we must have $x=-2y$, so $\vec{v} = \bbm -2y\\y\ebm = -y\bbm 2\\-1\ebm$, giving us the eigenvector $\vec{v}_1=\bbm 2\\-1\ebm$ corresponding to $\lambda_1=-1$.

For $\lambda_2=5$, we get
\[
A-5I = \bbm -4&4\\2&-2\ebm \xrightarrow{\text{RREF}} \bbm 1&-1\\0&0\ebm.
\]
For $\vec{v} = \bbm x\\y\ebm$ to solve $(A-5I)\vec{v}=\vec{0}$ we must have $x=y$, so $\vec{v}=\bbm y\\y\ebm = y\bbm 1\\1\ebm$, giving us the eigenvector $\vec{v}_2=\bbm 1\\1\ebm$ corresponding to $\lambda_2=5$.


 \end{enumerate}
\end{document}