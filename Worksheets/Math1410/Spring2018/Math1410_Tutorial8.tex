\documentclass[12pt]{article}
\usepackage{amsmath}
\usepackage{amssymb}
\usepackage[letterpaper,top=1.2in,bottom=1in,left=0.75in,right=0.75in,centering]{geometry}
%\usepackage{fancyhdr}
\usepackage{enumerate}
%\usepackage{lastpage}
\usepackage{multicol}
\usepackage{graphicx}

\reversemarginpar

%\pagestyle{fancy}
%\cfoot{}
%\lhead{Math 1560}\chead{Test \# 1}\rhead{May 18th, 2017}
%\rfoot{Total: 10 points}
%\chead{{\bf Name:}}
\newcommand{\points}[1]{\marginpar{\hspace{24pt}[#1]}}
\newcommand{\skipline}{\vspace{12pt}}
%\renewcommand{\headrulewidth}{0in}
\headheight 30pt

\newenvironment{amatrix}[1]{%
  \left[\begin{array}{@{}*{#1}{c}|c@{}}
}{%
  \end{array}\right]
}
\newcommand{\di}{\displaystyle}
\newcommand{\abs}[1]{\lvert #1\rvert}
\newcommand{\len}[1]{\lVert #1\rVert}
\renewcommand{\i}{\mathbf{i}}
\renewcommand{\j}{\mathbf{j}}
\renewcommand{\k}{\mathbf{k}}
\newcommand{\R}{\mathbb{R}}
\newcommand{\aaa}{\mathbf{a}}
\newcommand{\bbb}{\mathbf{b}}
\newcommand{\ccc}{\mathbf{c}}
\newcommand{\dotp}{\boldsymbol{\cdot}}
\newcommand{\bbm}{\begin{bmatrix}}
\newcommand{\ebm}{\end{bmatrix}}       
\DeclareMathOperator{\proj}{proj}      
\newcommand{\bam}{\begin{amatrix}}
\newcommand{\eam}{\end{amatrix}}   
                  
\begin{document}


\author{Instructor: Sean Fitzpatrick}
\thispagestyle{empty}
\vglue1cm
\begin{center}
\emph{University of Lethbridge}\\
Department of Mathematics and Computer Science\\
{\bf MATH 1410 - Tutorial \#8}\\
Wednesday, March 14
\end{center}
\skipline \ \noindent \skipline

\skipline \ \noindent \skipline

Student \#1 :\underline{\hspace{348pt}}\\

\bigskip

\bigskip

Student \#2 :\underline{\hspace{348pt}}\\

\bigskip

\bigskip

Student \#3 :\underline{\hspace{348pt}}\\

\bigskip

\bigskip

Student \#4 :\underline{\hspace{348pt}}\\





\bigskip

\textbf{(Moodle ID not required.)}









\newpage
%\thispagestyle{empty}

 \begin{enumerate}
\item For each matrix $A$ and vector $\vec{b}$ below, solve the equation $A\vec{x}=\vec{b}$. Express your answer in terms of the vector $\vec{x}$.

If there are infinitely many solutions, give your answer in the form $\vec{x}=\vec{x}_p+\vec{x}_h$, where $\vec{x}_p$ is a particular solution, and $\vec{x}_h$ is the general solution to the homogeneous system $A\vec{x}=\vec{0}$. (Express $\vec{x}_h$ in terms of basic solutions.)
\begin{enumerate}
\item $A = \bbm 1&0&-4\\-2&1&4\\1&0&6\ebm$, $\vec{b}=\bbm 2\\-1\\5\ebm$.

\vspace{3in}

\item $A= \bbm 1&0&2&-4\\3&1&5&-7\\-2&-2&-2&-2\ebm$, $\vec{b}=\bbm 3\\2\\8\ebm$
\end{enumerate}

\newpage

\item Consider the matrices
\[
A = \bbm 2&-1&3\\5&4&-2\ebm, B = \bbm 4&-2\\5&1\ebm, C = \bbm 1&4\\-2&-1\\6&3\ebm.
\]
For each of the 9 possible products ($A^2, AB, AC, BA, B^2, BC, CA, CB, C^2$), compute the product, or state why it is undefined.

\newpage

\item Consider a system of equations, written in matrix form as $A\vec{x}=\vec{b}$. Prove that if there is more than one solution to the system (say, $\vec{x}_1$ and $\vec{x}_2$, with $\vec{x}_1\neq \vec{x}_2$), then there are infinitely many solutions.

\vspace{2.5in}

\item For which values of $k$ will the system \hspace{12pt}
$\arraycolsep1pt
\begin{array}{ccccccc}
x&+&y&+&kz&=&1\\
x&+&ky&+&z&=&1\\
kx&+&y&+&z&=&-2
\end{array}$
\hspace{12pt} have: 

(a) No solution? \hspace{12pt} (b) A unique solution? \hspace{12pt} (c) Infinitely many solutions?
 \end{enumerate}
 
\end{document}