\documentclass[12pt]{article}
\usepackage{amsmath}
\usepackage{amssymb}
\usepackage[letterpaper,margin=0.85in,centering]{geometry}
\usepackage{fancyhdr}
\usepackage{enumerate}
\usepackage{lastpage}
\usepackage{multicol}
\usepackage{graphicx}

\reversemarginpar

\pagestyle{fancy}
\cfoot{}
\lhead{Math 1560}\chead{Tutorial Assignment \# 10 Solutions}\rhead{June 13th, 2017}
%\rfoot{Total: 10 points}
%\chead{{\bf Name:}}
\newcommand{\points}[1]{\marginpar{\hspace{24pt}[#1]}}
\newcommand{\skipline}{\vspace{12pt}}
%\renewcommand{\headrulewidth}{0in}
\headheight 30pt

\newcommand{\di}{\displaystyle}
\newcommand{\abs}[1]{\lvert #1\rvert}
\newcommand{\len}[1]{\lVert #1\rVert}
\renewcommand{\i}{\mathbf{i}}
\renewcommand{\j}{\mathbf{j}}
\renewcommand{\k}{\mathbf{k}}
\newcommand{\R}{\mathbb{R}}
\newcommand{\aaa}{\mathbf{a}}
\newcommand{\bbb}{\mathbf{b}}
\newcommand{\ccc}{\mathbf{c}}
\newcommand{\dotp}{\boldsymbol{\cdot}}
\newcommand{\bbm}{\begin{bmatrix}}
\newcommand{\ebm}{\end{bmatrix}}                   
                  
\begin{document}

%\author{Instructor: Sean Fitzpatrick}
\thispagestyle{fancy}
%\noindent{{\bf Name and student number:}}

\begin{enumerate}
 \item Consider the integrals $\displaystyle \int_1^4 x^2\,dx$, $\displaystyle \int_1^4 2x\,dx$, and $\displaystyle \int_1^4 (x^2-2x)\,dx$
\begin{enumerate}
 \item Approximate the value of each integral using 6 rectangles, and left endpoints.

\medskip

With 6 rectangles, our partition has $\Delta x = \frac{1}{2}$, and is given by $P = \{1,1.5, 2, 2.5, 3, 3.5, 4\}$. We have
\[
 \int_1^4 x^2\,dx \approx \left(1^2+(1.5)^2+2^2+(2.5)^2+3^2+(3.5)^2\right)\Delta x = 17.375,
\]
and
\[
 \int_1^4 2x \,dx \approx\left(2(1)+2(1.5)+2(2)+2(2.5)+2(3)+2(3.5)\right)\Delta x = 13.5
\]
using left endpoints. If you used right endpoints instead, then we drop the $f(1)$ terms above and replace it by $f(4)$, giving $24.875$ for the first integral, and $16.5$ for the second.

Since $\int_1^4(x^2-2x)\,dx = \int_1^4 x^2\,dx - \int_1^4 2x\,dx$, we can approximate the last rectangle by subtracting our two approximations. Thus,
\[
 \int_1^4(x^2-2x)\,dx = 3.875
\]
using left endpoints, and $8.325$ using right endpoints.

\medskip

 \item Find an expression (in terms of $n$) for the value of each integral using $n$ rectangles, and right endpoints.

With $n$ rectangles, we have $\Delta x = \dfrac{4-1}{n} = \dfrac{3}{n}$, and $x_i = x_0+i\Delta x = 1+\dfrac{3i}{n}$.

For the first integral, using $c_i=x_{i}$, we have
\[
 f(x_{i}) = \left(1+\dfrac{3i}{n}\right)^2 = 1+\frac{6i}{n}+\frac{9i^2}{n^2}.
\]
We thus have
\[
 \int_1^4 x^2\,dx \approx \sum_{i=1}^n\frac{3}{n}\left(1+\frac{6i}{n}+\frac{9i^2}{n^2}\right).
\]
If we use the formulas $\di \sum_{i=1}^n 1 = n, \sum_{i=1}^n i = \frac{n(n+1)}{2}, \sum_{i=1}^n i^2 = \frac{n(n+1)(2n+1)}{6}$, we get
\begin{align*}
 \int_1^4 x^2\,dx &\approx \frac{3}{n}\sum_{i=1}^n 1+\frac{18}{n^2}\sum_{i=1}^n i + \frac{27}{n^3}\sum_{i=1}^n i^2\\
 &= \frac{3}{n}(n)+\frac{18}{n^2}\left(\frac{n(n+1)}{2}\right)+\frac{27}{n^3}\left(\frac{n(n+1)(2n+1)}{6}\right)\\
 &= 3+9\left(\frac{n+1}{n}\right)+\frac{9}{2}\left(\frac{n+1}{n}\right)\left(\frac{2n+1}{n}\right) = 21+\frac{27}{2n}+\frac{3}{2n^2}.
\end{align*}

For the second integral, we similarly have
\begin{align*}
 \int_1^4 2x\,dx & \approx \sum_{i=1}^n 2\left(1+\frac{3i}{n}\right)\frac{3}{n} = \sum_{i=1}^n \left(\frac{6}{n}+\frac{18i}{n^2}\right)\\
  & = \frac{6}{n}(n)+\frac{18}{n^2}\left(\frac{n(n+1)}{2}\right) = 6 +9\left(\frac{n+1}{n}\right) = 15+\frac{9}{n}.
\end{align*}
Since $\int_1^4(x^2-2x)\,dx = \int_1^4x^2\,dx-\int_1^4 2x\,dx$, we have
\[
 \int_1^4(x^2-2x)\,dx \approx 21+\frac{27}{2n}+\frac{3}{2n^2}-\left(15+\frac{9}{n}\right) = 6+\frac{9}{2n}+\frac{3}{2n^2}.
\]

\end{enumerate}


 \item Compute the derivatives of the following functions:
\begin{enumerate}
 \item $\di f(x) = \int_2^x\frac{2t^2}{t^3+4t}\,dt$

\medskip

By direct application of the Fundamental Theorem of Calculus, $f'(x)=\dfrac{2x^2}{x^3+4x}$.

\medskip

 \item $\di g(x) = \int_x^4\sin(t^2)\,dt$

\medskip

Since $g(x) = -\int_4^x\sin(t^2)\,dt$, the FTC gives us $g'(x) = -\sin(x^2)$.

\medskip

 \item $\di h(x) = \int_x^{\sin(x)} e^{t^2}\,dt$

\medskip

Using properties of integrals, we have 
\[
 h(x) = \int_0^{\sin(x)}e^{t^2}\,dt + \int_x^0e^{t^2}\,dt = \int_0^{\sin(x)}e^{t^2}\,dt -\int_0^xe^{t^2}\,dt.
\]
Using the FTC (plus the Chain Rule on the first term), we have
\[
 h'(x) = e^{\sin^2(x)}\cos(x)-e^{x^2}.
\]

\medskip

\end{enumerate}

 \item Evaluate the integral $\di \int_0^1\left(\frac{1}{1+x^2}-2x+5e^x\right)\,dx$

\medskip

We first determine that the function 
\[
 F(x) = \arctan(x)-x^2+5e^x
\]
is an antiderivative of the integrand $\frac{1}{1+x^2}-2x+5e^x$. It follows from the second part of the FTC that
\begin{align*}
 \int_0^1\left(\frac{1}{1+x^2}-2x+5e^x\right)\,dx &= F(1)-F(0) = \arctan(1)-1^2+5e^1 - (\arctan(0)-0^2+5e^0)\\& = \frac{\pi}{4}+5e-6.
\end{align*}

\end{enumerate}
\end{document}