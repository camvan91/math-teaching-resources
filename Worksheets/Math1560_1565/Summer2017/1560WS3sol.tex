\documentclass[12pt]{article}
\usepackage{amsmath}
\usepackage{amssymb}
\usepackage[letterpaper,margin=0.85in,centering]{geometry}
\usepackage{fancyhdr}
\usepackage{enumerate}
\usepackage{lastpage}
\usepackage{multicol}
\usepackage{graphicx}

\reversemarginpar

\pagestyle{fancy}
\cfoot{}
\lhead{Math 1560}\chead{Tutorial Assignment \# 3 Solutions}\rhead{May 18th, 2017}
%\rfoot{Total: 10 points}
%\chead{{\bf Name:}}
\newcommand{\points}[1]{\marginpar{\hspace{24pt}[#1]}}
\newcommand{\skipline}{\vspace{12pt}}
%\renewcommand{\headrulewidth}{0in}
\headheight 30pt

\newcommand{\di}{\displaystyle}
\newcommand{\abs}[1]{\lvert #1\rvert}
\newcommand{\len}[1]{\lVert #1\rVert}
\renewcommand{\i}{\mathbf{i}}
\renewcommand{\j}{\mathbf{j}}
\renewcommand{\k}{\mathbf{k}}
\newcommand{\R}{\mathbb{R}}
\newcommand{\aaa}{\mathbf{a}}
\newcommand{\bbb}{\mathbf{b}}
\newcommand{\ccc}{\mathbf{c}}
\newcommand{\dotp}{\boldsymbol{\cdot}}
\newcommand{\bbm}{\begin{bmatrix}}
\newcommand{\ebm}{\end{bmatrix}}                   
                  
\begin{document}


%\author{Instructor: Sean Fitzpatrick}
\thispagestyle{fancy}
%\noindent{{\bf Name and student number:}}

 \begin{enumerate}
 \item  The limit $\di \lim_{h\to 0}\frac{\dfrac{1}{2+h}-\dfrac{1}{2}}{h}$ represents the derivative of a function $f$ at some point $a$.
\begin{enumerate}
 \item Identify the function and the point. 

\medskip

Comparing to $f'(a) = \di \lim_{h\to 0}\frac{f(a+h)-f(a)}{h}$, we have $f(x)=\dfrac{1}{x}$ and $a=2$.

\bigskip


 \item Evaluate the limit.

\medskip

\begin{align*}
 \lim_{h\to 0}\frac{\dfrac{1}{2+h}-\dfrac{1}{2}}{h} & = \lim_{h\to 0}\frac{\dfrac{2-(2+h)}{(2+h)(2)}}{h}\\
 & = \lim_{h\to 0}\frac{1}{h}\cdot\frac{-h}{2(2+h)}\\
 & = \lim_{h\to 0}\frac{-1}{2(2+h)} = -\frac{1}{4}.
\end{align*}

 \end{enumerate}

\bigskip

 \item (\textbf{Bonus}) Using the definition of the derivative, show that for any differentiable function $f$ and constant $c$, we have $(c\cdot f)'(x) = c\cdot f'(x)$

\bigskip

Using the definition, we have:
\begin{align*}
 (c\cdot f)'(x) & = \lim_{h\to 0}\frac{(cf)(x+h)-(cf)(x)}{h}\\
& = \lim_{h\to 0} \frac{c(f(x+h))-c(f(x))}{h}\\
& = \lim_{h\to 0} \frac{c(f(x+h)-f(x))}{h}\\
& = c\lim_{h\to 0}\frac{f(x+h)-f(x)}{h} = c(f'(x)).
\end{align*}


\newpage

 \item Compute the derivatives of the following functions:
\begin{enumerate}
 \item $f(x) = 3x^3-2x^2+\sqrt{2}$

\bigskip

Using sum, constant, and power rules, we have
\[
 f'(x) = 3(3x^2)-2(2x)+0 = 9x^2-4x.
\]


 \item $g(x) = x^2\sin(x)$

\bigskip

Using the product rule (and the derivatives for $x^2$ and $\sin(x)$), we have
\begin{align*}
 g'(x) & = \left(\frac{d}{dx}(x^2)\right)\sin(x)+x^2\left(\frac{d}{dx}(\sin(x))\right)\\
  & = 2x\sin(x)+x^2\cos(x).
\end{align*}



 \item $h(x) = \dfrac{x^2+x}{2-3x}$

\bigskip

Using the quotient rule, we have

\begin{align*}
 h'(x) & = \frac{\left(\frac{d}{dx}(x^2+x)\right)(2-3x)-(x^2+x)\left(\frac{d}{dx}(2-3x)\right)}{(2-3x)^2}\\
 & = \frac{(2x+1)(2-3x)+3(x^2+x)}{(2-3x)^2}.
\end{align*}

\end{enumerate}
\end{enumerate}
\
\end{document}