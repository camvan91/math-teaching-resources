\documentclass[12pt]{article}
\usepackage{amsmath}
\usepackage{amssymb}
\usepackage[letterpaper,margin=0.85in,centering]{geometry}
\usepackage{fancyhdr}
\usepackage{enumerate}
\usepackage{lastpage}
\usepackage{multicol}
\usepackage{graphicx}

\reversemarginpar

\pagestyle{fancy}
\cfoot{}
\lhead{Math 1560}\chead{Tutorial Assignment \# 9}\rhead{June 8th, 2017}
%\rfoot{Total: 10 points}
%\chead{{\bf Name:}}
\newcommand{\points}[1]{\marginpar{\hspace{24pt}[#1]}}
\newcommand{\skipline}{\vspace{12pt}}
%\renewcommand{\headrulewidth}{0in}
\headheight 30pt

\newcommand{\di}{\displaystyle}
\newcommand{\abs}[1]{\lvert #1\rvert}
\newcommand{\len}[1]{\lVert #1\rVert}
\renewcommand{\i}{\mathbf{i}}
\renewcommand{\j}{\mathbf{j}}
\renewcommand{\k}{\mathbf{k}}
\newcommand{\R}{\mathbb{R}}
\newcommand{\aaa}{\mathbf{a}}
\newcommand{\bbb}{\mathbf{b}}
\newcommand{\ccc}{\mathbf{c}}
\newcommand{\dotp}{\boldsymbol{\cdot}}
\newcommand{\bbm}{\begin{bmatrix}}
\newcommand{\ebm}{\end{bmatrix}}                   
                  
\begin{document}
{\bf \large Name:} \hspace{2.5in} {\bf Tutorial time:}

\bigskip

\bigskip

%\author{Instructor: Sean Fitzpatrick}
\thispagestyle{fancy}
%\noindent{{\bf Name and student number:}}

\begin{enumerate}
 \item Determine the Maclaurin polynomial $p_5$ (degree 5 Taylor polynomial, about $x=0$) for the following functions:
\begin{enumerate}
 \item $f(x) = \tan(x)$

\bigskip

We compute the derivatives of $f$ at zero, as follows:
\[
 \begin{array}{ll}
  f(x)=\tan(x) & f(0)= 0\\
  f'(x)=\sec^2(x) & f'(0)=1\\
  f''(x)=2\sec^2(x)\tan(x)&f''(0)=0\\
  f^{(3)}(x)=4\sec^2(x)\tan^2(x)+2\sec^4(x) & f^{(3)}(0)=2\\
  f^{(4)}(x)=8\sec^2(x)\tan^3(x)+16\sec^4(x)\tan(x) & f^{(4)}(0)=0\\
  f^{(5)}(x)=16\sec^2(x)\tan^4(x)+88\sec^4(x)\tan^2(x)+16\sec^6(x) & f^{(5)}(0) = 16
 \end{array}
\]
Therefore,
\begin{align*}
 p_5(x)& = 0+1(x)+\frac{0}{2!}x^2+\frac{2}{3!}x^3+\frac{0}{4!}x^4+\frac{16}{5!}x^5\\
 & = x+\frac{1}{3}x^3+\frac{2}{15}x^5.
\end{align*}

\medskip

 \item $g(x) = e^x\sin(x)$

\bigskip

Our derivatives are given as follows:
\[
 \begin{array}{ll}
  g(x)=e^x\sin(x) & g(0) = 0\\
  g'(x)=e^x\sin(x)+e^x\cos(x) & g'(0)=1\\
  g''(x) = 2e^x\cos(x) & g''(0)=2\\
  g^{(3)}(x) = 2e^x\cos(x)-2e^x\sin(x) & g^{(3)}(0) = 2\\
  g^{(4)}(x) = -4e^x\sin(x) & g^{(4)}(0) = 0\\
  g^{(5)}(x) = -4e^x\sin(x)-4e^x\cos(x) & g^{(5)}(0) = -4
 \end{array}
\]
Our polynomial is therefore given by
\begin{align*}
 p_5(x) &= 0 + 1(x)+\frac{2}{2!}x^2+\frac{2}{3!}x^3+\frac{0}{4!}x^4-\frac{4}{5!}x^5\\
 & = x+x^2+\frac{1}{3}x^3-\frac{1}{30}x^5.
\end{align*}



\end{enumerate}
\newpage
 \item Determine the degree 4 Taylor polynomial for $f(x)=\cos(x)$ about $a=\pi/3$.


 \bigskip

We have
\begin{align*}
 f(\pi/3) & = \cos(\pi/3) = \frac{1}{2}\\
 f'(\pi/3) & = -\sin(\pi/3) = -\frac{\sqrt{3}}{2}\\
 f''(\pi/3) & = -\cos(\pi/3) = -\frac{1}{2}\\
 f'''(\pi/3) & = \sin(\pi/3) = \frac{\sqrt{3}}{2}\\
 f''''(\pi/3) & = \cos(\pi/3) = \frac{1}{2}.
\end{align*}
Our Taylor polynomial is therefore
\[
 p_4(x) = \frac{1}{2}-\frac{\sqrt{3}}{2}\left(x-\frac{\pi}{3}\right)-\frac{1}{4}\left(x-\frac{\pi}{3}\right)^2+\frac{\sqrt{3}}{12}\left(x-\frac{\pi}{3}\right)^3+\frac{1}{48}\left(x-\frac{\pi}{3}\right)^4.
\]


\bigskip

 \item Find a function $f(x)$ satisfying the given conditions:
\begin{enumerate}
 \item $f'(x) = \dfrac{1}{1+x^2}$, and $f(0)=\dfrac{\pi}{4}$.

\bigskip

We know that $\dfrac{d}{dx}(\arctan(x)) = \dfrac{1}{1+x^2}$, so we have $f(x) = \arctan(x)+C$ for some constant $C$. Since we must have $f(0)=0+C=\frac{\pi}{4}$, we have $C=\frac{\pi}{4}$, and thus
\[
 f(x) = \arctan(x)+\frac{\pi}{4}.
\]

\medskip


 \item $f''(x) = 6x+4$, $f(0)=3$, and $f'(0) = -2$.

\bigskip

 Since $f''(x)$ is the derivative of $f'(x)$, we find that $f'(x)$ is given by the antiderivative
\[
 f'(x) = 3x^2+4x+C,
\]
for some constant $C$. The requirement that $f'(0)=-2$ gives us $C=-2$, so $f'(x) = 3x^2+4x-2$. Taking the antiderivative again, we find
\[
 f(x) = x^3+2x^2-2x+D,
\]
for some constant $D$. Since we have to have $f(0)=3$, it follows that $D=3$, and thus $f(x) = x^3+2x^2-2x+3$.

\end{enumerate}
\end{enumerate}
\end{document}