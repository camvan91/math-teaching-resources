\documentclass[12pt]{article}
\usepackage{amsmath}
\usepackage{amssymb}
\usepackage[letterpaper,margin=0.85in,centering]{geometry}
\usepackage{fancyhdr}
\usepackage{enumerate}
\usepackage{lastpage}
\usepackage{multicol}
\usepackage{graphicx}

\reversemarginpar

\pagestyle{fancy}
\cfoot{}
\lhead{Math 1560}\chead{Tutorial Assignment \# 1}\rhead{May 10th, 2017}
%\rfoot{Total: 10 points}
%\chead{{\bf Name:}}
\newcommand{\points}[1]{\marginpar{\hspace{24pt}[#1]}}
\newcommand{\skipline}{\vspace{12pt}}
%\renewcommand{\headrulewidth}{0in}
\headheight 30pt

\newcommand{\di}{\displaystyle}
\newcommand{\abs}[1]{\lvert #1\rvert}
\newcommand{\len}[1]{\lVert #1\rVert}
\renewcommand{\i}{\mathbf{i}}
\renewcommand{\j}{\mathbf{j}}
\renewcommand{\k}{\mathbf{k}}
\newcommand{\R}{\mathbb{R}}
\newcommand{\aaa}{\mathbf{a}}
\newcommand{\bbb}{\mathbf{b}}
\newcommand{\ccc}{\mathbf{c}}
\newcommand{\dotp}{\boldsymbol{\cdot}}
\newcommand{\bbm}{\begin{bmatrix}}
\newcommand{\ebm}{\end{bmatrix}}                   
                  
\begin{document}
{\bf \large Name:} \hspace{2.5in} {\bf Tutorial time:}

\bigskip

\bigskip

%\author{Instructor: Sean Fitzpatrick}
\thispagestyle{fancy}
%\noindent{{\bf Name and student number:}}

 \begin{enumerate}
 \item  Solve the following inequalities:
\begin{enumerate}
 \item $x^2-2x\geq 15$

\vspace{2in}

 \item $1+\dfrac{3}{x+1}\leq\dfrac{4}{x}$
\end{enumerate}

\vspace{2.5in}

\item Give a one-sentence explanation (in words) why the following are true:
\begin{enumerate}
 \item $\di \lim_{x\to a} b = b$ for any real numbers $a$ and $b$.

\vspace{1in}

 \item $\di \lim_{x\to a} x = a$ for any real number $x$.
\end{enumerate}

\newpage

\item Using properties of limits and the facts given in Problem \#2, show that for any \textit{polynomial} $p(x)$, and any real number $a$, we have $\di \lim_{x\to a}p(x)=p(a)$.

\vspace{2.5in}

\item Evaluate each of the following limits, or explain it does not exist.
\begin{enumerate}
 \item $\di \lim_{x\to 3}\frac{x^2-9}{x^2-5x+6}$

\vspace{1.5in}

 \item $\di \lim_{x\to 2}\frac{x^2+4}{x-2}$

\vspace{1.5in}

 \item $\di \lim_{x\to 0}\frac{\sin(3x)}{\tan(5x)}$
\end{enumerate}


 \end{enumerate}
\end{document}