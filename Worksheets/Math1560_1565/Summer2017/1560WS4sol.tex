\documentclass[12pt]{article}
\usepackage{amsmath}
\usepackage{amssymb}
\usepackage[letterpaper,margin=0.85in,centering]{geometry}
\usepackage{fancyhdr}
\usepackage{enumerate}
\usepackage{lastpage}
\usepackage{multicol}
\usepackage{graphicx}

\reversemarginpar

\pagestyle{fancy}
\cfoot{}
\lhead{Math 1560}\chead{Tutorial Assignment \# 4 Solutions}\rhead{May 23rd, 2017}
%\rfoot{Total: 10 points}
%\chead{{\bf Name:}}
\newcommand{\points}[1]{\marginpar{\hspace{24pt}[#1]}}
\newcommand{\skipline}{\vspace{12pt}}
%\renewcommand{\headrulewidth}{0in}
\headheight 30pt

\newcommand{\di}{\displaystyle}
\newcommand{\abs}[1]{\lvert #1\rvert}
\newcommand{\len}[1]{\lVert #1\rVert}
\renewcommand{\i}{\mathbf{i}}
\renewcommand{\j}{\mathbf{j}}
\renewcommand{\k}{\mathbf{k}}
\newcommand{\R}{\mathbb{R}}
\newcommand{\aaa}{\mathbf{a}}
\newcommand{\bbb}{\mathbf{b}}
\newcommand{\ccc}{\mathbf{c}}
\newcommand{\dotp}{\boldsymbol{\cdot}}
\newcommand{\bbm}{\begin{bmatrix}}
\newcommand{\ebm}{\end{bmatrix}}                   
                  
\begin{document}

%\author{Instructor: Sean Fitzpatrick}
\thispagestyle{fancy}
%\noindent{{\bf Name and student number:}}

 \begin{enumerate}
 \item  Find an equation for the line tangent to the graph of the given function $f$ at the point $(a,f(a))$:
\begin{enumerate}
 \item $f(x) = (3x^2+2)\tan(x)$, $a=0$.

\bigskip

The prodcut rule gives us
\[
 f'(x) = 6x\tan(x)+(3x^2+2)\sec^2(x).
\]
Thus, 
\[
 f'(0) = 6(0)\tan(0)+(3(0^2)+2)\sec^2(0) = 0+2(1) = 2
\]
 is the slope of the tangent line at the point $(0,f(0)) = (0,0)$. The equation of the tangent line is therefore $y=2x$.

\bigskip

 \item $f(x) = \dfrac{x^2-2x+3}{x^2+4}$, $a=1$.

\bigskip

Using the quotient rule, we have
\[
 f'(x) = \frac{(2x-2)(x^2+4)-(x^2-2x+3)(2x)}{(x^2+4)^2}.
\]
When $x=1$, this gives us the slope 
\[
 f'(1) = \frac{(2-2)(5)-(1-2+3)(2)}{5^2} = \frac{4}{25}.
\]
 Since $f(1) = \frac{2}{5}$, we get the equation
\[
 y-\frac{2}{5} = \frac{4}{25}(x-1)
\]
for the tangent line. (Equivalently, this can be written as $4x-25y=-6$.)

\bigskip


 \item $f(x) = (x^4+2x)^5$, $a=-1$ 

\bigskip

Using the Chain rule, we find 
\[
 f'(x) = 5(x^4+2x)^4\frac{d}{dx}(x^4+2x) = 5(x^4+2x)^4(4x^3+2),
\]
 so 
\[
 f'(-1) = 5(1-2)^4(-4+2) = 5(1)(-2)=-10
\]
 is the slope of the tangent line, and when $x=-1$, $y=f(-1)=-1$, so our equation is $y+1=-10(x+1)$, or $y=-10x-11$.
\end{enumerate}
\newpage

\item Compute the derivative of $f(x)=\sin(2x)$:
\begin{enumerate}
 \item Using the Chain Rule.

\medskip

We find $f'(x) = \cos(2x)\dfrac{d}{dx}(2x) = 2\cos(2x)$.

\medskip

 \item Using the identity $\sin(2x)=2\sin(x)\cos(x)$.

\medskip

Using the constant and product rules,
\[
 f'(x) = 2\left(\frac{d}{dx}(\sin(x))(\cos(x))+\sin(x)\frac{d}{dx}(\cos(x))\right)  = 2(\cos^2(x)-\sin^2(x)).
\]

\bigskip

\end{enumerate}
Do your answers in parts (a) and (b) agree?

\bigskip

Yes, because $\cos(2x) = \cos^2(x)-\sin^2(x)$.

\bigskip


\item Given $f(x)=\tan(x)$, compute $f''(x)$ (also denoted $\dfrac{d^2}{dx^2}(\tan(x))$).

\bigskip

The first derivative is given by $f'(x)=\sec^2(x) = (\sec(x))^2$. Applying the Chain Rule to this expression, we find
\begin{align*}
 f''(x) & = \frac{d}{dx}(\sec(x))^2\\
 & = 2\sec(x)\frac{d}{dx}(\sec(x))\\
 & = 2\sec(x)(\sec(x)\tan(x))\\
 & = 2\sec^2(x)\tan(x).
\end{align*}

\bigskip

\item Discuss with your classmates, but don't hand in:

Determine values of $A$ and $B$ such that the derivative of 
\[
f(x) = \begin{cases}Ax^2+Bx+2, & \text{ if } x\leq 2,\\ Bx^2-A, & \text{ if } x>2\end{cases}
\]
is everywhere continuous. (Hint: note that if $f'(x)$ exists, $f(x)$ itself must be continuous.)

\bigskip

Note that continuity of $f(x)$ requires that $\di \lim_{x\to 2^-}f(x) = \lim_{x\to 2^+}f(x)$, which implies that $4A+2B+2=4B-A$.

For $x<2$, we have $f'(x) = 2Ax+B$, and for $x>2$, we have $f'(x)=2Bx$. If we want $f'(x)$ to be continuous at $2$, we need $\di \lim_{x\to 2^-}f'(x) = \lim_{x\to 2^+}f'(x)$, so we must have $4A+B = 4B$. This second equation suggests $4A=3B$, or $B=\frac{4}{3}A$. Plugging this into our earlier equation, we find
\[
 4A+2\left(\frac{4}{3}A\right) +2 = 4\left(\frac{4}{3}A\right)-A,
\]
 which simplifies to $A=-\frac{6}{7}$, and thus $B = -\frac{8}{7}$.

\end{enumerate}

\end{document}