\documentclass[12pt]{article}
\usepackage{amsmath}
\usepackage{amssymb}
\usepackage[letterpaper,top=0.85in,bottom=1in,left=0.75in,right=0.75in,centering]{geometry}
%\usepackage{fancyhdr}
\usepackage{enumerate}
%\usepackage{lastpage}
\usepackage{multicol}
\usepackage{graphicx}

\reversemarginpar

%\pagestyle{fancy}
%\cfoot{}
%\lhead{Math 1560}\chead{Test \# 1}\rhead{May 18th, 2017}
%\rfoot{Total: 10 points}
%\chead{{\bf Name:}}
\newcommand{\points}[1]{\marginpar{\hspace{24pt}[#1]}}
\newcommand{\skipline}{\vspace{12pt}}
%\renewcommand{\headrulewidth}{0in}
\headheight 30pt

\newcommand{\di}{\displaystyle}
\newcommand{\abs}[1]{\lvert #1\rvert}
\newcommand{\len}[1]{\lVert #1\rVert}
\renewcommand{\i}{\mathbf{i}}
\renewcommand{\j}{\mathbf{j}}
\renewcommand{\k}{\mathbf{k}}
\newcommand{\R}{\mathbb{R}}
\newcommand{\aaa}{\mathbf{a}}
\newcommand{\bbb}{\mathbf{b}}
\newcommand{\ccc}{\mathbf{c}}
\newcommand{\dotp}{\boldsymbol{\cdot}}
\newcommand{\bbm}{\begin{bmatrix}}
\newcommand{\ebm}{\end{bmatrix}}                   
                  
\begin{document}


\author{Instructor: Sean Fitzpatrick}
\thispagestyle{empty}
\vglue1cm
\begin{center}
\emph{University of Lethbridge}\\
Department of Mathematics and Computer Science\\
{\bf MATH 1560 - Tutorial \#5}\\
Monday, February 12
\end{center}
\skipline \ \noindent \skipline

Student \#1 :\underline{\hspace{348pt}}\\

\bigskip


Student \#2 :\underline{\hspace{348pt}}\\

\bigskip

Student \#3 :\underline{\hspace{348pt}}\\

\bigskip

Student \#4 :\underline{\hspace{348pt}}\\


\bigskip


\bigskip


\bigskip


Some additional practice (discuss the answers but don't write anything down):
\begin{enumerate}
\item Evaluate the limits:
\begin{multicols}{3}
\begin{enumerate}
\item $\di \lim_{x\to 0}\sqrt{4x^2-x+9}$
\item $\di \lim_{x\to \infty}\frac{8x^5+3x+5}{4+5x^3+2x^5}$
\item $\di \lim_{x\to 3}\frac{x^2-9}{x-3}$ 
\end{enumerate}
\end{multicols}

\item Compute the derivative:
\begin{multicols}{3}
\begin{enumerate}
\item $\di\frac{d}{dx}(x^4+4x+9)$
\item $\di \frac{d}{dx}\sqrt{x^4+4}$
\item $\di\frac{d}{dx}(5x^3e^x)$
\end{enumerate}
\end{multicols}

\item Evaluate the immediate integral:
\begin{multicols}{2}
\begin{enumerate}
\item $\di \int(4x^3+2x)\,dx$
\item $\di \int\frac{1}{x}\,dx$
\item $\di \int 7e^{7x}\sin(e^{7x})\,dx$
\item $\di \int \frac{\ln(x)}{x}\,dx$
\end{enumerate}
\end{multicols}
\end{enumerate}




\newpage
%\thispagestyle{empty}

\vglue12pt

Test problems: (note $\di \sum_{k=1}^n k^3=\frac{n^2(n+1)^2}{4}$)
  \begin{enumerate}
    \item Compute the limit:
    \begin{multicols}{2}
    \begin{enumerate}
    \item $\di \lim_{x\to 2^+}\frac{(x^2-4)^2}{x-2}$

    
    \item $\di \lim_{n\to\infty}\left(\frac{1^3}{n^4}+\frac{2^3}{n^4}+\cdots + \frac{n^3}{n^4}\right)$ 
    \end{enumerate}
\end{multicols}

    \vspace{2cm}

    \item Compute the derivative:
    \begin{multicols}{2}
    \begin{enumerate}
    \item $f(x) = \dfrac{\sin(x)}{e^x}$
    

    
    \item $\di g(x)= \tan(5x^2)$
    
  
    
    
    \end{enumerate}
    \end{multicols}
    
    \vspace{2cm}
    
    \item Compute $\dfrac{d}{dx}(x^x)$
    
    \vspace{2cm}
    
    \item Evaluate the integral:
    \begin{multicols}{2}
    \begin{enumerate}
    \item $\di\int 2x(x^2+4)^4\,dx$
    

    
    \item $\di\int(\cos(2x)-sec^2(x))\,dx$
    

    \end{enumerate}
    \end{multicols}
    
    \vspace{2cm}
    
    \item Compute $y'=\frac{dy}{dx}$, given:
    \begin{multicols}{2}
    \begin{enumerate}
    \item $x^2+y^2=25$, at the point $(-3,-4)$.
    

    
    \item $y^3+x^5y=2\ln(y)+\dfrac{x}{y}$ (don't solve for $y'$)
    \end{enumerate}
    \end{multicols}
    
    \vspace{2cm}
    
    \newpage
  \vglue12pt
  
    \item Given $f(x) = x^2e^x$, solve the equation $f'(x)=0$.
    
    \vspace{4cm}
    
    \item Given $f(x) = e^{-2x^2}$, solve the equation $f''(x)=0$.

 \vspace{3cm}
  
   \item Let $f(x) = x^3(x-4)^5$. Determine:
   \begin{enumerate}
   \item All values of $x$ such that $f'(x)=0$
   
   \vspace{3cm}
   
   \item The intervals on which $f$ is increasing or decreasing.
   
   \vspace{1.5cm}
   
   \item The coordinates of any local maxima or minima.
   \end{enumerate}
   

   
   
  \end{enumerate}
\end{document}