\documentclass[12pt]{article}
\usepackage{amsmath}
\usepackage{amssymb}
\usepackage[letterpaper,top=1.25in,bottom=1in,left=0.75in,right=0.75in,centering]{geometry}
%\usepackage{fancyhdr}
\usepackage{enumerate}
%\usepackage{lastpage}
\usepackage{multicol}
\usepackage{graphicx}

\reversemarginpar

%\pagestyle{fancy}
%\cfoot{}
%\lhead{Math 1560}\chead{Test \# 1}\rhead{May 18th, 2017}
%\rfoot{Total: 10 points}
%\chead{{\bf Name:}}
\newcommand{\points}[1]{\marginpar{\hspace{24pt}[#1]}}
\newcommand{\skipline}{\vspace{12pt}}
%\renewcommand{\headrulewidth}{0in}
\headheight 30pt

\newcommand{\di}{\displaystyle}
\newcommand{\abs}[1]{\lvert #1\rvert}
\newcommand{\len}[1]{\lVert #1\rVert}
\renewcommand{\i}{\mathbf{i}}
\renewcommand{\j}{\mathbf{j}}
\renewcommand{\k}{\mathbf{k}}
\newcommand{\R}{\mathbb{R}}
\newcommand{\aaa}{\mathbf{a}}
\newcommand{\bbb}{\mathbf{b}}
\newcommand{\ccc}{\mathbf{c}}
\newcommand{\dotp}{\boldsymbol{\cdot}}
\newcommand{\bbm}{\begin{bmatrix}}
\newcommand{\ebm}{\end{bmatrix}}                   
                  
\begin{document}


\author{Instructor: Sean Fitzpatrick}
\thispagestyle{empty}
\vglue1cm
\begin{center}
\emph{University of Lethbridge}\\
Department of Mathematics and Computer Science\\
{\bf MATH 1560 - Tutorial \#10}\\
Monday, March 26
\end{center}
\skipline \ \noindent \skipline

\vspace*{\fill}

\textbf{Intermediate Value Theorem (zero version):} Suppose a function $f$ is continuous on $[a,b]$, and either (a) $f(a)<0$ and $f(b)>0$, or (b) $f(a)>0$ and $f(b)<0$. Then there exists some real number $c\in (a,b)$ such that $f(c)=0$.

\bigskip

\textit{Extra fun:} Apply Newton's method to the equation $x^2-a=0$ (where $a>0$) to derive the formula
\[
a_{n+1} = \frac12\left(x_n+\frac{a}{x_n}\right).
\]
This formula represents the algorithm used by ancient Babylonians to compute $\sqrt{a}$.

\newpage
%\thispagestyle{empty}


\begin{enumerate}
\item Consider the \textit{Intermediate Value Theorem} (IVT), which is stated on the reverse of this page.
\begin{enumerate}
\item Use the IVT to show that the equation $3x^4-8x^3+2=0$ has a solution on the interval $[2,3]$.

\vspace{1in}

\item Use Newton's method to find the solution, correct to six decimal places.

\end{enumerate}

\vspace{2.5in}

\item Explain why Newton's Method doesn't work for finding a solution to the equation $x^3-3x+6-0$ if the initial approximation is $x_1=1$.

\vspace{1in}

\item Apply Newton's Method to the equation $1/x - a=0$ to derive the reciprocal algorithm $x_{n+1}=2x_n-ax_n^2$.

(This algorithm is used by computers to compute reciprocals without dividing.)
\end{enumerate}

\end{document}