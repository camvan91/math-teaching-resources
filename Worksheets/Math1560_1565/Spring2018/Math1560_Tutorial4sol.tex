\documentclass[12pt]{article}
\usepackage{amsmath}
\usepackage{amssymb}
\usepackage[letterpaper,top=0.85in,bottom=1in,left=0.75in,right=0.75in,centering]{geometry}
%\usepackage{fancyhdr}
\usepackage{enumerate}
%\usepackage{lastpage}
\usepackage{multicol}
\usepackage{graphicx}

\reversemarginpar

%\pagestyle{fancy}
%\cfoot{}
%\lhead{Math 1560}\chead{Test \# 1}\rhead{May 18th, 2017}
%\rfoot{Total: 10 points}
%\chead{{\bf Name:}}
\newcommand{\points}[1]{\marginpar{\hspace{24pt}[#1]}}
\newcommand{\skipline}{\vspace{12pt}}
%\renewcommand{\headrulewidth}{0in}
\headheight 30pt

\newcommand{\di}{\displaystyle}
\newcommand{\abs}[1]{\lvert #1\rvert}
\newcommand{\len}[1]{\lVert #1\rVert}
\renewcommand{\i}{\mathbf{i}}
\renewcommand{\j}{\mathbf{j}}
\renewcommand{\k}{\mathbf{k}}
\newcommand{\R}{\mathbb{R}}
\newcommand{\aaa}{\mathbf{a}}
\newcommand{\bbb}{\mathbf{b}}
\newcommand{\ccc}{\mathbf{c}}
\newcommand{\dotp}{\boldsymbol{\cdot}}
\newcommand{\bbm}{\begin{bmatrix}}
\newcommand{\ebm}{\end{bmatrix}}                   
                  
\begin{document}


\author{Instructor: Sean Fitzpatrick}
\thispagestyle{empty}
\vglue1cm
\begin{center}
{\bf MATH 1560 - Tutorial \#4 solutions}\\
Monday, February 5
\end{center}

\textbf{Additional practice problems:}
\begin{enumerate}
\item Compute the derivatives of the following functions using the chain rule:
\begin{enumerate}
\item $f(x) = \ln(\sqrt{x^4+3x^2})=\ln(x^2(x^2+3))^{1/2}=\frac12 (2\ln\abs{x}+\ln(x^2+3))$, so
\[
f'(x) = \frac{1}{x} + \frac{x}{x^2+3}.
\]
\item $\di g(x) = \sin(e^{x^3})$
\[
g'(x) = \cos(e^{x^3})\cdot e^{x^3} (3x^2).
\]
\item $\di h(x) = e^{\sin(\sqrt{x^2+1})}$ 
\[
h'(x) = e^{\sin(\sqrt{x^2+1})}\cdot \cos(\sqrt{x^2+1})\cdot\frac{x}{\sqrt{x^2+1}}.
\]
\end{enumerate}


\item Compute $\dfrac{dy}{dx}$ using implicit differentiation, if:

\begin{enumerate}
\item $(x^2+y^2)^2=x^2-y^2$

Taking the derivative of both sides with respect to $x$, we get
\[
2(x^2+y^2)(2x+2y\cdot y')=2x-2y\cdot y'.
\]
Gathering terms involving $y'$ on the left and factoring, 
\[
y'(4y(x^2+y^2)+2y)=2x-4x(x^2+y^2),
\]
so $\di y'=\frac{dy}{dx} = \frac{2x-4x(x^2+y^2)}{4y(x^2+y^2)+2y} = \frac{x-2x^3-2xy^2}{2x^2y+2y^3+y}$.

\item $\sqrt{xy}=x^2-3y$

Taking the derivative of both sides with respect to $x$, we get
\[
\frac{1}{2\sqrt{xy}}(y+x\cdot y')=2x-3y'.
\]
Gathering terms involving $y'$ on the left and factoring, 
\[
y'\left(\frac{x}{2\sqrt{xy}}+3\right)=2x-\frac{y}{2\sqrt{xy}},
\]
so $\di y' = \frac{dy}{dx} = \frac{2x-\frac{y}{2\sqrt{xy}}}{\frac{x}{2\sqrt{xy}}+3}$. We can clean this up a bit by multiplying top and bottom by $2\sqrt{xy}$, giving
\[
y' = \frac{4x\sqrt{xy}-y}{x+6\sqrt{xy}}.
\]

\item $x^3\sin(xy^2)=3$

Taking the derivative of both sides with respect to $x$, we get
\[
3x^2\sin(xy^2)+x^3\cos(xy^2)(y^2+2xy\cdot y')=0.
\]
Solving for $y'$, we find
\[
y' = \frac{-3x^2\sin(xy^2)-x^3y^2\cos(xy^2)}{2x^4y\cos(xy^2)}.
\]
\end{enumerate}


\item Compute $f'(x)$ using logarithmic differentiation, if:

\begin{enumerate}
\item $\di f(x)=(x^2+1)^x$

We have $\ln(f(x)) = \ln(x^2+1)^x = x\ln(x^2+1)$, so
\[
\frac{1}{f(x)}f'(x) = \ln(x^2+1)+\frac{2x^2}{x^2+1},
\]
giving us $\di f'(x) = (x^2+1)^x\left(\ln(x^2+1)+\frac{2x^2}{x^2+1}\right)$.

\item $\di f(x)=\sqrt{\frac{e^{x^3}(x-4)^7}{(x^2+1)^3\sin^3(x)}}$

We have
\begin{align*}
\ln(f(x)) &= \ln\left(\frac{e^{x^3}(x-4)^7}{(x^2+1)^3\sin^3(x)}\right)^{1/2}\\
& = \frac12\left(\ln(e^{x^3})+\ln(x-4)^7-\ln(x^2+1)^3-\ln(\sin^3(x))\right)\\
& = \frac12 x^3+\frac72 \ln(x-4)-\frac32 \ln(x^2+1)-\frac32\ln(\sin(x)).
\end{align*}
Thus
\[
\frac{1}{f(x)}f'(x) = \frac32 x^2+\frac{7}{2(x-4)}-\frac{6x}{3(x^2+1)}-\frac{3\cos(x)}{2\sin(x)}.
\]
Multiplying both sides of the above by $f(x)$, we get
\[
f'(x) = \sqrt{\frac{e^{x^3}(x-4)^7}{(x^2+1)^3\sin^3(x)}}\left(\frac32 x^2+\frac{7}{2(x-4)}-\frac{6x}{3(x^2+1)}-\frac{3\cos(x)}{2\sin(x)}\right).
\]
\end{enumerate}

\end{enumerate}



\newpage
%\thispagestyle{empty}
\textbf{Assigned problems:}


  \begin{enumerate}
    \item Compute the derivative of the following functions:
    \begin{enumerate}
    \item $f(x) = \tan(x^3+x)$
    
\[
f'(x) = \sec^2(x^3+x)\cdot (3x^2+1)
\]
    
    \item $\di g(x) = \sqrt{e^{x^4}+1}$
    
    \[
    g'(x) = \frac{1}{2\sqrt{e^{x^4}+1}}\cdot e^{x^4}\cdot 4x^3.
    \]
    \end{enumerate}

   
    \item Compute the \textbf{second} derivative of the following functions:
    \begin{enumerate}
    \item $f(x) = \tan(x^2)$
    
   We have $f'(x) = \sec^2(x^2)\cdot 2x$, and thus
   \[
   f''(x) = 2\sec^2(x^2)+2x(2\sec(x^2)\cdot \sec(x^2)\tan(x^2) \cdot 2x) = 2\sec^2(x^2) + 8x^2\sec^2(x^2)\tan(x^2).
   \]
    
    \item $\di g(x) = e^{\sec(x)}$
    
    Since $g'(x) = e^{\sec(x)}(\sec(x)\tan(x))$, we get
    \begin{align*}
    g''(x) &= e^{\sec(x)}(\sec(x)\tan(x))(\sec(x)\tan(x))+e^{\sec(x)}((\sec(x)\tan(x))\tan(x)+\sec(x)(\sec^2(x)))\\
    & = e^{\sec(x)}(\sec^2(x)\tan^2(x)+\sec(x)\tan^2(x)+\sec^3(x)).
    \end{align*}
    \end{enumerate}

  
   \item Compute the derivative of $f(x) = \ln\left(\sqrt[3]{\dfrac{x^4(x-4)^5}{e^{3x+1}(x^2+1)^7}}\right)$. (First use log properties)
  
Since
\begin{align*}
f(x) &= \ln\left(\sqrt[3]{\dfrac{x^4(x-4)^5}{e^{3x+1}(x^2+1)^7}}\right)=\ln\left(\dfrac{x^4(x-4)^5}{e^{3x+1}(x^2+1)^7}\right)^{1/3}=\frac{1}{3}\ln\left(\frac{x^4(x-4)^5}{e^{3x+1}(x^2+1)^7}\right)\\
&=\frac13\left(\ln(x^4)+\ln(x-4)^5-\ln(e^{3x+1})-\ln(x^2+1)^7\right)\\
& = \frac43\ln(x)+\frac53\ln(x-4)-\frac13(3x+1)-\frac73\ln(x^2+1),
\end{align*}
  we get
  \[
  f'(x) = \frac{4}{3x}+\frac{5}{3(x-4)}-1-\frac{14x}{3(x^2+1)}.
  \]  

 \textbf{Caution:} Although the final answer is correct, this solution glosses over a domain issue in the middle:
 
 \medskip
 
 When we write steps such as $\ln(x^4)=4\ln(x)$, we're not entirely telling the truth: the domain of the original function on the left is $(-\infty, 0)\cup (0,\infty)$ (everything but $x=0$), while the domain of the function on the right is $(0,\infty)$ -- so technically, these are not the same function. The correct equality is
 \[
 \ln (x^4)=4\ln\abs{x},
 \]
 since the absolute value takes care of the fact that the natural log is undefined for negative numbers, and $\abs{x}^4=x^4$.
 
 However, inclusion of the absolute value doesn't affect the derivative:
 \[
 \dfrac{d}{dx}\ln\abs{x} = \frac{1}{x},
 \]
  so even though there were domain mis-matches in the middle, we get the correct answer at the end. (If you're trying the online homework and it's marking you wrong, check to make sure you haven't missed an absolute value.)
    \item Use implicit differentiation to find the equation of the tangent line to the curve\\ $(x+y^3)^2=4x^2y$ at the point $(1,1)$.   
    
    Taking the derivative of both sides of the equation with respect to $x$, we have
    \[
    2(x+y^3)(1+3y^2\cdot y')=8xy+4x^2\cdot y'.
    \]
    At this point, since we're interested in the value of $y'$ \textit{at a particular point}, there are two ways to proceed:
    
    \begin{itemize}
    \item Option A: Solve for $y'$ and then put $x=1$ and $y=1$:
    
    \[
    6y^2(x+y^3)\cdot y'-4x^2\cdot y'=8xy-2(x^2+y^3),
    \]
    so
    \[
    y'=\frac{8xy-2x^2-2y^3}{6xy^2+6y^5-4x^2}.
    \]
    When $x=1$ and $y=1$, we find that the slope of our tangent line is given by
    \[
    m = \left.\frac{dy}{dx}\right|_{\substack{x=1\\y=1}} = \frac{8(1)(1)-2(1)^2-2(1)^3}{6(1)(1)^2+6(1)^5-4(1)^2}=\frac{4}{8}=\frac12.
    \]
    
    \item Option B: Put $x=1$ and $y=1$ and then solve for $y'$:
    \[
    2(1+1^3)(1+3(1)^2y')=8(1)(1)+4(1)^2y',
    \]
    so $4(1+3y')=8+4y'$. Dividing by 4, we get $1+3y'=2+y'$, and so $2y'=1$, or $y'=\dfrac12$.
    \end{itemize}
    Either way, we get the slope $m=\dfrac12$, and since the point $(1,1)$ is on the tangent line, the point-slope formula gives us
    \[
    y-1=\frac12(x-1)
    \]
    as the equation of the line.
  \end{enumerate}
\end{document}