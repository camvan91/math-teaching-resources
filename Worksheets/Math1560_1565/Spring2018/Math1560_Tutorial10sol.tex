\documentclass[12pt]{article}
\usepackage{amsmath}
\usepackage{amssymb}
\usepackage[letterpaper,top=1.25in,bottom=1in,left=0.75in,right=0.75in,centering]{geometry}
%\usepackage{fancyhdr}
\usepackage{enumerate}
%\usepackage{lastpage}
\usepackage{multicol}
\usepackage{graphicx}

\reversemarginpar

%\pagestyle{fancy}
%\cfoot{}
%\lhead{Math 1560}\chead{Test \# 1}\rhead{May 18th, 2017}
%\rfoot{Total: 10 points}
%\chead{{\bf Name:}}
\newcommand{\points}[1]{\marginpar{\hspace{24pt}[#1]}}
\newcommand{\skipline}{\vspace{12pt}}
%\renewcommand{\headrulewidth}{0in}
\headheight 30pt

\newcommand{\di}{\displaystyle}
\newcommand{\abs}[1]{\lvert #1\rvert}
\newcommand{\len}[1]{\lVert #1\rVert}
\renewcommand{\i}{\mathbf{i}}
\renewcommand{\j}{\mathbf{j}}
\renewcommand{\k}{\mathbf{k}}
\newcommand{\R}{\mathbb{R}}
\newcommand{\aaa}{\mathbf{a}}
\newcommand{\bbb}{\mathbf{b}}
\newcommand{\ccc}{\mathbf{c}}
\newcommand{\dotp}{\boldsymbol{\cdot}}
\newcommand{\bbm}{\begin{bmatrix}}
\newcommand{\ebm}{\end{bmatrix}}                   
                  
\begin{document}


\author{Instructor: Sean Fitzpatrick}
\thispagestyle{empty}
\vglue1cm
\begin{center}
{\bf MATH 1560 - Tutorial \#10 Solutions}
\end{center}

On the worksheet, you were provided with the following:

\textbf{Intermediate Value Theorem (zero version):} Suppose a function $f$ is continuous on $[a,b]$, and either (a) $f(a)<0$ and $f(b)>0$, or (b) $f(a)>0$ and $f(b)<0$. Then there exists some real number $c\in (a,b)$ such that $f(c)=0$.

\bigskip

\textit{Extra fun:} Apply Newton's method to the equation $x^2-a=0$ (where $a>0$) to derive the formula
\[
x_{n+1} = \frac12\left(x_n+\frac{a}{x_n}\right).
\]
This formula represents the algorithm used by ancient Babylonians to compute $\sqrt{a}$.


\bigskip

In Newton's Method, we set $f(x)=x^2-a$, so $f'(x)=2x$. We then get
\[
x_{n+1} = x_n-\frac{x_n^2-a}{2x_n} = \frac{2x_n^2-x_n^2+a}{2x_n} = \frac12\left(\frac{x_n^2+a}{x_n}\right) = \frac12\left(x_n+\frac{a}{x_n}\right),
\]
as required.
%\thispagestyle{empty}


\begin{enumerate}
\item Consider the \textit{Intermediate Value Theorem} (IVT), which is stated on the reverse of this page.
\begin{enumerate}
\item Use the IVT to show that the equation $3x^4-8x^3+2=0$ has a solution on the interval $[2,3]$.

We let $f(x)=3x^4-8x^3+2$, which is a continuous function, since it's a polynomial. We then find $f(2) = -14<0$ and $f(3)=29>0$. By the IVT as given on the worksheet, there must be some $c\in (2,3)$ such that $f(c)=0$.

\item Use Newton's method to find the solution, correct to six decimal places.

We know our solution must lie between 2 and 3 by part (a), so as our first guess, we go half-way between, and take $x_1=2.5$. We have
\[
x_{n+1}=x_n-\frac{f(x_n)}{f'(x_n)} = x_n-\frac{3x_n^4-8x_n^3+2}{12x_n^3-24x_n^2} = \frac{9x_n^4-16x_n^3-2}{12x_n^3-24x_n^2}.
\]
We now go to the calculator/computer. We get the following values:
\[
\begin{array}{c|cccccc}
n&1&2&3&4&5&6\\
x_n&2.500000&2.655000&2.630725&2.630021&2.630020&2.630020
\end{array}
\]
We see that, up to 6 decimal places, $x_6=x_5$, so our solution must be approximately $x_5=2.630020$.

\end{enumerate}



\item Explain why Newton's Method doesn't work for finding a solution to the equation $x^3-3x+6=0$ if the initial approximation is $x_1=1$.

Given $f(x)=x^3-3x$ we have $f'(x)=3x^2-3$, and $f'(1)=0$. We cannot use the Newton's method formula since we would have to divide by zero.

This makes sense since the formula computes the $x$-intercept of the tangent line at $(x_1,f(x_1))$, but in this case that would be the horizontal line $y=4$, which never intersects the $x$-axis.

\item Apply Newton's Method to the equation $1/x - a=0$ to derive the reciprocal algorithm $x_{n+1}=2x_n-ax_n^2$.

(This algorithm is used by computers to compute reciprocals without dividing.)

We set $f(x)=\dfrac1x-a$ and apply the Newton's Method formula:
\[
x_{n+1} = x_n-\frac{f(x_n)}{f'(x_n)}=x_n-\frac{1/x_n-a}{-1/x_n^2}=x_n+x_n^2(1/x_n-a)=x_n+x_n-ax_n^2=2x_n-ax_n^2,
\]
as required.
\end{enumerate}

\end{document}