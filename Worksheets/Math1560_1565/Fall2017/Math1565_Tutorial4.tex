\documentclass[12pt]{article}
\usepackage{amsmath}
\usepackage{amssymb}
\usepackage[letterpaper,top=1.35in,bottom=1.25in,left=0.75in,right=0.75in,centering]{geometry}
%\usepackage{fancyhdr}
\usepackage{enumerate}
%\usepackage{lastpage}
\usepackage{multicol}
\usepackage{graphicx}

%\reversemarginpar

%\pagestyle{fancy}
%\cfoot{}
%\lhead{Math 1560}\chead{Test \# 1}\rhead{May 18th, 2017}
%\rfoot{Total: 10 points}
%\chead{{\bf Name:}}
\newcommand{\points}[1]{\marginpar{\hspace{24pt}[#1]}}
\newcommand{\skipline}{\vspace{12pt}}
%\renewcommand{\headrulewidth}{0in}
\headheight 30pt

\newcommand{\di}{\displaystyle}
\newcommand{\abs}[1]{\lvert #1\rvert}
\newcommand{\len}[1]{\lVert #1\rVert}
\renewcommand{\i}{\mathbf{i}}
\renewcommand{\j}{\mathbf{j}}
\renewcommand{\k}{\mathbf{k}}
\newcommand{\R}{\mathbb{R}}
\newcommand{\aaa}{\mathbf{a}}
\newcommand{\bbb}{\mathbf{b}}
\newcommand{\ccc}{\mathbf{c}}
\newcommand{\dotp}{\boldsymbol{\cdot}}
\newcommand{\bbm}{\begin{bmatrix}}
\newcommand{\ebm}{\end{bmatrix}}                   
                  
\begin{document}


\author{Instructor: Sean Fitzpatrick}
\thispagestyle{empty}
\begin{center}
\emph{University of Lethbridge}\\
Department of Mathematics and Computer Science\\
{\bf MATH 1565 - Tutorial \#4}\\
\end{center}
%\skipline \skipline \skipline \noindent \skipline
%Last Name:\underline{\hspace{350pt}}\\
%\skipline
%First Name:\underline{\hspace{348pt}}\\
%\skipline
%Student Number:\underline{\hspace{322pt}}\\
%\skipline



\vspace{0.1in}

\vspace*{\fill}

\begin{quote}
Print your name and student number clearly in the space above. 

\medskip

Complete the problems on the back of this page to the best of your ability. If there is a problem you especially desire feedback on, please indicate this. 

\medskip

It is recommended that you work out the details on scrap paper before writing your solutions on this page.
\end{quote}
\newpage
\thispagestyle{empty}
\begin{enumerate}
 \item  Using \textbf{the definition of the derivative}, determine the equation of the line tangent to the curve $y=\dfrac{1}{\sqrt{x}}$ at the point $(4,\frac{1}{2})$.
 

\end{enumerate}
\end{document}