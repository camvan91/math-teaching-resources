\documentclass[12pt]{article}
\usepackage{amsmath}
\usepackage{amssymb}
\usepackage[letterpaper,top=0.85in,bottom=1in,left=0.75in,right=0.75in,centering]{geometry}
%\usepackage{fancyhdr}
\usepackage{enumerate}
%\usepackage{lastpage}
\usepackage{multicol}
\usepackage{graphicx}

\reversemarginpar

%\pagestyle{fancy}
%\cfoot{}
%\lhead{Math 1560}\chead{Test \# 1}\rhead{May 18th, 2017}
%\rfoot{Total: 10 points}
%\chead{{\bf Name:}}
\newcommand{\points}[1]{\marginpar{\hspace{24pt}[#1]}}
\newcommand{\skipline}{\vspace{12pt}}
%\renewcommand{\headrulewidth}{0in}
\headheight 30pt

\newcommand{\di}{\displaystyle}
\newcommand{\abs}[1]{\lvert #1\rvert}
\newcommand{\len}[1]{\lVert #1\rVert}
\renewcommand{\i}{\mathbf{i}}
\renewcommand{\j}{\mathbf{j}}
\renewcommand{\k}{\mathbf{k}}
\newcommand{\R}{\mathbb{R}}
\newcommand{\aaa}{\mathbf{a}}
\newcommand{\bbb}{\mathbf{b}}
\newcommand{\ccc}{\mathbf{c}}
\newcommand{\dotp}{\boldsymbol{\cdot}}
\newcommand{\bbm}{\begin{bmatrix}}
\newcommand{\ebm}{\end{bmatrix}}                   
                  
\begin{document}


\author{Instructor: Sean Fitzpatrick}
\thispagestyle{empty}
\vglue1cm
\begin{center}
%\emph{University of Lethbridge}\\
%Department of Mathematics and Computer Science\\
MATH 1565 - Tutorial \#11 Solutions
\end{center}
%\skipline \skipline \skipline \noindent \skipline
%Last Name:\underline{\hspace{350pt}}\\
%\skipline
%First Name:\underline{\hspace{348pt}}\\
%\skipline
%Student Number:\underline{\hspace{322pt}}\\
%\skipline



\begin{enumerate}
  \item Calculate the following Taylor polynomials:
  \begin{enumerate}
  \item For $f(x)=e^{x^2}$, degree 4, about $x=0$. \points{4}
  
  We have $f(0)=e^0=1$, and
  \begin{align*}
  f'(x) & = 2xe^{x^2} & f'(0) & = 0\\
  f''(x) &= 2e^{x^2}+4x^2e^{x^2} & f''(0) & = 2\\
  f^{(3)}(x) &= 12xe^{x^2}+8x^3e^{x^2} & f^{(3)}(0) & = 0\\
  f^{(4)}(x) &= 12e^{x^2}+48x^2e^{x^2} + 16x^4e^{x^2} & f^{(4)}(0) & = 12
  \end{align*}
  Thus,
  \[
  p_4(x) = f(0)+f'(0)x+\frac{f''(0)}{2!}x^2+\frac{f^{(3)}(0)}{3!}x^3+\frac{f^{(4)}(0)}{4!}x^4 = 1+x^2+\frac{1}{2}x^4.
  \]
  
  \item For $g(u)=e^u$, degree 2, about $u=0$. \points{2}
  
  Since $g(u)=g'(u)=g''(u)=e^u$, we have $g(0)=g'(0)=g''(0)=1$, and
  \[
  p_2(u) = g(0)+g'(0)u+\frac{g''(0)}{2!}u^2 = 1+u+\frac{1}{2}u^2.
  \]
  
  (What happens if you put $u=x^2$ in your answer for part (b)?)
  
  \medskip
  
  You didn't have to answer this part, but putting $u=x^2$ in your answer for (b) gives you the answer for (a), suggesting that the answer, for those of you who asked, is ``Yes, there is an easier way to do this.''
  
  \end{enumerate}
  \item Calculate the following antiderivatives:
  \begin{enumerate}
  \item The antiderivative $F$ of $f(x) = \dfrac{1}{1+x^2}$ such that $F(1) = \pi$. \points{3}
  
\medskip

We have $F(x) = \arctan(x)+C$ for some $C$. This gives us $\pi = F(1) = \arctan(1)+C=\frac{\pi}{4}+C$, so $C=\frac{3\pi}{4}$, and thus
\[
F(x) = \arctan(x)+\frac{3\pi}{4}.
\]
  
  \item $\di \int (x^3-3\sqrt{x}+4)\,dx = \frac{1}{4}x^4-2x^{3/2}+4x+C.$\points{3}
\end{enumerate}    
  \newpage
  
  \item Estimate the area under $f(x) = 4-3x^2$, for $0\leq x\leq 1$, using 3 rectangles and:
  \begin{enumerate}
  \item Left endpoints. \points{3}
  
 \medskip
 
 We have $\Delta x= \frac{1-0}{3} = \frac{1}{3}$, so our points are $x_0=0, x_1=\frac{1}{3}, x_2=\frac{2}{3}$, and $x_3 = 1$. Our left endpoints are $x_0, x_1, x_2$, so we have
 \[
 A\approx (f(0)+f(1/3)+f(2/3))(1/3) = (4+11/3+8/3)(1/3)=31/9 \approx 3.44.
 \]
  
  \item Right endpoints. \points {3}
  
  Using the data from above, our right endpoints are $x_1,x_2, x_3$, and
  \[
  A\approx (f(1/3)+f(2/3)+f(1))(1/3) = (11/3+8/3+1)(1/3) = 22/9\approx 2.44.
  \]
  
  \end{enumerate}

\item Given that 
\[
\int_1^4 f(x)\,dx = 4, \int_1^6 f(x)\,dx = 7, \int_1^4 g(x)\,dx = -3, \text{ and } \int_4^6 g(x)\,dx = 1,
\]
compute:
\begin{enumerate}
\item $\di \int_4^6 f(x)\,dx = \int_1^6f(x)\,dx - \int_1^4 f(x)\,dx = 7-4=3$\points{2}



\item $\di \int_1^6 (f(x)+g(x))\,dx$ \points{2}

Since $\di \int_1^6g(x)\,dx=\int_1^4g(x)\,dx+\int_4^6g(x)\,dx = -3+1=-2$, we have
\[
\int_1^6 (f(x)+g(x))\,dx = \int_1^6 f(x)\,dx+\int_1^6 g(x)\,dx = 7+(-2)=5.
\]
\end{enumerate}
\end{enumerate}
\end{document}