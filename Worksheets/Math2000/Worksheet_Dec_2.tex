\documentclass[letterpaper,12pt]{article}

%\usepackage{ucs}
%\usepackage[utf8x]{inputenc}
\usepackage{amsmath}
\usepackage{amsfonts}
\usepackage{amssymb}
%\usepackage[canadian]{babel}
\usepackage[margin=1in]{geometry}
\usepackage{multicol}
%\usepackage[dvips]{hyperref}
\newcommand{\divs}[2]{#1 \, | \, #2}
\renewcommand{\cong}[3]{#1 \equiv #2 \pmod{#3}}
\newcommand{\R}[2]{#1 \, R \, #2}
\renewcommand{\P}{\mathcal{P}}

%\author{Sean Fitzpatrick}
%\date{2015-09-15}
\title{Math 2000 Tutorial Worksheet}
\date{December 2nd, 2015}

\begin{document}
\maketitle

 \begin{enumerate}
  \item (Section 7.1 \#2) Let $A=\{a,b,c\}$ and let $R=\{(a,a), (a,c), (b,b), (b,c), (c,a), (c,b)\}$ define a relation on $A$. Are the following statements true or false? Explain.
\begin{enumerate}
 \item For each $x\in A$, $\R{x}{x}$.
 \item For every $x,y\in A$, if $\R{x}{y}$, then $\R{y}{x}$.
 \item For every $x,y,z\in A$, if $\R{x}{y}$ and $\R{y}{z}$, then $\R{x}{z}$.
 \item $R$ is a function from $A$ to $A$.
\end{enumerate}
 \item (Section 7.1 \#4) Let $U$ be a nonempty set, and let $R$ be the ``subset relation'' on $\mathcal{P}(U)$. That is,\label{Qa}
\[
 R = \{(S,T)\in \P(U)\times\P(U) | S\subseteq T\}.
\]
\begin{enumerate}
 \item Write the open sentence $(S,T)\in R$ using standard subset notation.
 \item What is the domain of the relation $R$?
 \item What is the range of the relation $R$?
 \item Is $R$ a function from $\P(U)$ to $\P(U)$? Explain.
\end{enumerate}
 \item (Section 7.1\#5) Repeat parts (b)-(d) of Problem \ref{Qa} for the ``element of'' relation
\[
 R = \{(x,S)\in U\times \P(U) | x\in S\}.
\]
\item (Section 7.1 \#6 and 7) Let $S = \{(x,y)\in \mathbb{R}\times \mathbb{R} | x^2+y^2=100\}$.
\begin{enumerate}
 \item Determine the set of all values of $x$ such that $(x,6)\in S$, and determine the set of all values of $x$ such that $(x,9)\in S$.
 \item Determine the domain and range of the relation $S$, and write each set using set builder notation.
 \item Is the relation on $S$ a function from $\mathbb{R}$ to $\mathbb{R}$? Explain.
 \item Since $S$ is a relation on $\mathbb{R}$, its elements can be graphed in the coordinate plane. Describe the graph of the relation $S$. Is the graph consistent with your answers in parts (a) - (c)? Explain.
 \item Repeat parts (a) - (d) for the relation $T=\{(x,y)\in \mathbb{R}\times\mathbb{R} | y=\sqrt{100-x^2}\}$. What is the connection between the relations $S$ and $T$?
\end{enumerate}
\item (Section 7.1 \#8) Determine the domain and range of each of the following relations on $\mathbb{R}$ and sketch the graph of each relation.
\begin{enumerate}
 \item $R= \{(x,y)\in \mathbb{R}\times \mathbb{R} | x^2+y^2=100\}$
 \item $S=\{(x,y)\in \mathbb{R}\times \mathbb{R} | y^2=x+10\}$
 \item $T=\{(x,y)\in \mathbb{R}\times \mathbb{R} | \lvert x\rvert + \lvert y\rvert = 10\}$
 \item $W=\{(x,y)\in \mathbb{R}\times \mathbb{R} | x^2=y^2\}$.
\end{enumerate}
 \item (Section 7.2 \#2) Let $A=\{a,b,c\}$. For each of the following, draw a directed graph that represents a relation on $A$ with the specified properties:
\begin{enumerate}
 \item A relation on $A$ that is symmetric but not transitive.
 \item A relation on $A$ that is transitive but not symmetric.
 \item A relation on $A$ that is symmetric and transitive but not reflexive.
 \item A relation on $A$ that is not reflexive on $A$, is not symmetric, and is not transitive.
 \item A relation on $A$, other than the identity relation, that is an equivalence relation on $A$.
\end{enumerate}
 \item (Section 7.2 \#5) A relation $R$ is defined on $\mathbb{Z}$ as follows: For all $a,b\in\mathbb{Z}$, $\R{a}{b}$ if and only if $\lvert a -b \rvert \leq 3$. Is $R$ an equivalence relation on $\mathbb{Z}$? If not, is $R$ reflexive? Symmetric? Transitive? Justify all conclusions.
 \item (Section 7.2 \#9) Define the relation $\sim$ on $\mathbb{Q}$ as follows: For $a,b\in\mathbb{Q}$, $a\sim b$ if and only if $a-b\in\mathbb{Z}$.
\begin{enumerate}
 \item Show that $\sim$ is an equivalence relation. (See Progress Check 7.9.)
 \item List four different elements of the set $C = \left\{x\in \mathbb{Q} \left| x\sim\dfrac{5}{7}\right.\right\}$.
 \item Use set builder notation, without using the symbol $\sim$, to specify the set $C$.
 \item use the roster method to specify the set $C$.
\end{enumerate}
 \item (Section 7.2 \#13) Let $\sim$ and $\approx$ be relations on $\mathbb{Z}$ defined as follows;
\begin{itemize}
 \item For $a,b\in\mathbb{Z}$, $a\sim b$ if and only if $2a+3b\equiv 0 \pmod{5}$.
 \item For $a,b\in\mathbb{Z}$, $a\approx b$ if and only if $a+3b\equiv 0 \pmod{5}$.
\end{itemize}
\begin{enumerate}
 \item Is $\sim$ an equivalence relation on $\mathbb{Z}$? If not, is this relation reflexive? Symmetric? Transitive?
 \item Is $\approx$ an equivalence relation on $\mathbb{Z}$? If not, is this relation reflexive? Symmetric? Transitive?
\end{enumerate}

 \end{enumerate}


\end{document}
 
