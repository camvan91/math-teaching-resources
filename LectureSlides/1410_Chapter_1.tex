 \documentclass{beamer}

\usepackage[utf8x]{inputenc}
\usepackage{default}
\usepackage{pgfpages}
\pgfpagesuselayout{resize to}[letterpaper,landscape,border shrink=5mm]

%\geometry{landscape,paper=letterpaper}
\beamertemplatenavigationsymbolsempty
\date{}
\title{Chapter 1: Systems of Linear Equations\\Math 1410\\University of Lethbridge\\Spring 2015}
\begin{document}
\begin{frame}
 \titlepage
\end{frame}
\begin{frame}\frametitle{Linear equations}
 \begin{definition}
  A \alert{linear equation} in $n$ variables $x_1, x_2, \ldots, x_n$ is an equation of the form
\[
 a_1x_1+a_2x_2+\cdots + a_nx_n = b,
\]
where $a_1,a_2,\ldots, a_n, b$ are constants (real numbers).
 \end{definition}

\bigskip

Examples:

\vspace{2in}

\end{frame}
\begin{frame}\frametitle{Systems of equations}
 A \alert{system of equations} (linear or otherwise) is a collection of one or more equations for which we want to find all common solutions (if any).
\begin{example}[A non-linear system]
 Solve the system
\begin{align*}
 x^2+y^2 &= 5\\
 x^2-y^2 &= 1
\end{align*}
\end{example}
Solutions to non-linear systems can be very complicated (and even impossible to solve exactly). For linear systems (which we will study) there are systematic methods for solving them.
\end{frame}
\begin{frame}\frametitle{A ``biological'' example}
 \begin{example}
A biologist wants to feed rats a diet consisting of fish and meal so that the rats get 30 grams of protein and 20 grams of carbohydrate every day. If fish consists of 70\% protien and 10\% carbohydrate, while meal consists of 30\% protein and 60\% carbohydrate, how much of each food is needed every day?
 \end{example}
\vspace{2in}

\end{frame}
\begin{frame}\frametitle{Geometric solutions}
 Linear equation in two dimensions:
\[
 ax+by = c
\]

\vspace{0.75in}

 Linear equation in three dimensions:
\[
 ax+by+cz=d
\]

\vspace{0.5in}

\end{frame}
\begin{frame}\frametitle{Algebraic solutions}
 A visual approach only works in two or three dimensions. (Realistically, it doesn't work that well in 3D either.)\\
\begin{example}
 Solve the system:
\begin{align*}
 2x-3y& = 7\\
 -x+4y& = 2
\end{align*}

\end{example}

\bigskip

 Some applied situations (economics, air traffic control) involve hundreds or even thousands of variables. In these cases, only algebraic (or numerical) methods will work.
\end{frame}

\end{document}
