\documentclass[12pt]{article}
\usepackage{amsmath}
\usepackage{amssymb}
\usepackage[letterpaper,margin=0.85in,centering]{geometry}
\usepackage{fancyhdr}
\usepackage{enumerate}
\usepackage{lastpage}
\usepackage{multicol}
\usepackage{graphicx}

\reversemarginpar

\pagestyle{fancy}
\cfoot{Page \thepage \ of \pageref{LastPage}}\rfoot{{\bf Total Points: 80}}
\chead{MATH 3500}\lhead{FINAL EXAM}\rhead{10th December, 2014}

\newcommand{\points}[1]{\marginpar{\hspace{24pt}[#1]}}
\newcommand{\skipline}{\vspace{12pt}}
%\renewcommand{\headrulewidth}{0in}
\headheight 30pt

\newcommand{\di}{\displaystyle}
\newcommand{\R}{\mathbb{R}}
\newcommand{\Q}{\mathbb{Q}}
\newcommand{\N}{\mathbb{N}}
\newcommand{\abs}[1]{\lvert #1\rvert}

\begin{document}

\author{Instructor: Sean Fitzpatrick}
\thispagestyle{plain}
\begin{center}
\emph{University of Lethbridge}\\
Department of Mathematics and Computer Science\\
10th December 2014, 9:00 am - 12:00 pm\\
{\bf MATH 3500 - FINAL EXAM}\\
\end{center}
\skipline \skipline \skipline \noindent \skipline
Last Name:\underline{\hspace{350pt}}\\
\skipline
First Name:\underline{\hspace{348pt}}\\
\skipline
Student Number:\underline{\hspace{322pt}}\\
\skipline

\vspace{0.5in}


\begin{quote}

 
 Record your answers below each question in the space provided.    Left-hand pages may be used as scrap paper for rough work.  If you want any work on the left-hand pages to be graded, please indicate so on the right-hand page.
 
 \bigskip
 
Partial credit will be awarded for partially correct work, so be sure to show your work, and include all necessary justifications needed to support your arguments. 

\bigskip

For full credit, you must complete {\bf eight} of the nine problems on the exam, but you can attempt all nine problems. (Yes, a score above 100\% is possible.)

\end{quote}


\vspace{0.25in}

For grader's use only:

\begin{table}[hbt]
\begin{center}
\begin{tabular}{|l|r|} \hline
Problem&Grade\\
\hline \hline
\cline{1-2} 1 & \enspace\enspace\enspace\enspace\enspace\enspace/10\\
\cline{1-2} 2 & \enspace\enspace\enspace\enspace\enspace\enspace/10\\
\cline{1-2} 3 & \enspace\enspace\enspace\enspace\enspace\enspace/10\\
\cline{1-2} 4 & \enspace\enspace\enspace\enspace\enspace\enspace/10\\
\cline{1-2} 5 & \enspace\enspace\enspace\enspace\enspace\enspace/10\\
\cline{1-2} 6 & \enspace\enspace\enspace\enspace\enspace\enspace/10\\
\cline{1-2} 7 & \enspace\enspace\enspace\enspace\enspace\enspace/10\\
\cline{1-2} 8 & \enspace\enspace\enspace\enspace\enspace\enspace/10\\
\cline{1-2} 9 & \enspace\enspace\enspace\enspace\enspace\enspace/10\\
\cline{1-2} Total & \enspace\enspace\enspace\enspace\enspace\enspace/80\\
\hline
\end{tabular}

\skipline

\skipline

\skipline

\end{center}
\end{table}
\newpage


\begin{enumerate}
\item Let $S$ and $T$ be nonempty subsets of $\R$ such that $s\leq t$ for all $s\in S$ and $t\in T$.
\begin{enumerate}
 \item Explain why $S$ is bounded above and $T$ is bounded below. \points{1}

\vspace{1.25in}

 \item Prove that $\sup S\leq \inf T$. \points{5}

\vspace{3.5in}

 \item Give an example of sets $S$ and $T$ as above with $S\cap T\neq \emptyset.$\points{2}

\vspace{1.5in}

 \item Give an example of sets $S$ and $T$ as above where $\sup S = \inf T$ but $S\cap T=\emptyset$. \points{2}

      \end{enumerate}
\newpage
\item \begin{enumerate}
       \item What is the set of all limit points of the set $A = \left\{\dfrac{1}{n}+\dfrac{1}{m} : n,m\in\N\right\}$? \points{4}\\ (A formal proof is not required.)

\vspace{3in}

       \item The closure $\overline{A}$ of a set $A$ is defined to be the union of $A$ and the set of limit points of $A$. \points{6} Prove that $\overline{A} = A\cup \partial A$, where $\partial A$ denotes the boundary of $A$.\\
(Hint: any limit point of $A$ either belongs to $A$, or it doesn't.)

      \end{enumerate}
\newpage

\item \begin{enumerate}
\item Prove that the union of two open sets is open. \points{4}

\vspace{3.25in}

\item Prove that the intersection of two closed sets is closed. \points{3}\\


\vspace{2.5in}

\item Prove that the intersection of two compact sets is compact.\points{3}

      \end{enumerate}

\newpage

\item Let $(x_n)$ be a sequence defined by $x_1=1$ and $x_{n+1} = \frac{1}{3}(x_n+1)$ for $n\geq 1$.
\begin{enumerate}
\item Find $x_2, x_3$, and $x_4$. \points{2}

\vspace{1.25in}

\item Use induction to prove $(x_n)$ is a decreasing sequence. \points{4}

\vspace{2.75in}

\item Argue that $(x_n)$ is a bounded sequence. (You can use induction, but it isn't necessary.) \points{2}

\vspace{2in}

\item Explain why $\lim x_n$ exists, and find $\lim x_n$. \points{2}

\end{enumerate}
\newpage

\item \begin{enumerate}
       \item Suppose that $f:[a,b]\to\R$ is continuous at $c\in [a,b]$, and $f(c)>0$. Prove that there exists an interval $I$ with $c\in I\subseteq [a,b]$ such that $f(x)>0$ for all $x\in I$.  \points{5}\\
(Hint: use the definition of continuity and choose $\epsilon>0$ wisely.)

\vspace{4in}

       \item Prove (using the definition) that $f(x)=1/x$ is uniformly continuous on $[a,\infty)$ for any $a>0$. \points{5} Why do we know that $f$ cannot be uniformly continuous on $(0,\infty)$?
      \end{enumerate}
\newpage

\item \begin{enumerate}
       \item Prove that if $f:[a,b]\to\R$ is continuous, and $f(x)>0$ for all $x\in [a,b]$, then $1/f$ is bounded on $[a,b]$. \points{5}

\vspace{4in}


       \item Prove that for any continuous function $f:[0,1]\to[0,1]$ there exists a point $x\in [0,1]$ such that $f(x)=x$. \points{5}



      \end{enumerate}
\newpage
\item \begin{enumerate}
       \item Suppose that $f$ is differentiable on $\R$, and $f(0)=0$, $f(1)=1$, and $f(2)=1$. \points{5} Show that $f'(x) = 1/2$ for some $x\in (0,2)$.

\vspace{3.5in}

       \item For which values of $a$ is the function $f(x) = \begin{cases} x^a\sin(1/x) &\text{ if } x\neq 0\\ 0 &\text{ if } x=0\end{cases}$\\ differentiable at $x=0$?\points{5}

      \end{enumerate}
\newpage


\newpage

 \item Let $\displaystyle L(x) = \int_1^x \frac{1}{t}\,dt$, for $x>0$.
\begin{enumerate}
 \item Calculate $L'(x)$. \points{2}

\vspace{1.5in}

 \item Prove that $L$ is an increasing function. \points{3}

\vspace{2in}

\end{enumerate}

 Since $L$ is increasing, it is one-to-one. Define a function $E:\R\to(0,\infty)$ by $E = L^{-1}$.
\begin{enumerate}
\setcounter{enumii}{2}
\item Prove that $E(0)=1$ and $E'(x) = E(x)$ for all $x\in\R$. \points{5}\\
(Hint: use the Chain Rule, and the fact that $L(E(x))=x$ for all $x\in\R$.)

\end{enumerate}
\newpage

\item Let $f$ and $g$ be bounded functions on $[a,b]$.
\begin{enumerate}
 \item Prove that for any partition $P$ of $[a,b]$, $U(f+g,P)\leq U(f,P)+U(g,P)$ \points{5}

\vspace{3.5in}

 \item Conclude that $U(f+g)\leq U(f)+U(g)$. \points{3} ($U(f) = \inf\{U(f,P) : P \text{ is a partition}\}$.)

\vspace{2in}

 \item Give an example where the above inequality is strict. \points{2}\\
(Hint: $f$ and $g$ won't be integrable, since in that case $U(f+g)=U(f)+U(g)$.)
\end{enumerate}


\end{enumerate}
\newpage


This page was left intentionally blank for rough work. Please do not remove.






\end{document}