\documentclass[12pt]{article}
\usepackage{amsmath}
\usepackage{amssymb}
\usepackage[letterpaper,margin=0.85in,centering]{geometry}
\usepackage{fancyhdr}
\usepackage{enumerate}
\usepackage{lastpage}
\usepackage{multicol}
\usepackage{graphicx}

\reversemarginpar

\pagestyle{fancy}
\cfoot{Page \thepage \ of \pageref{LastPage}}\rfoot{{\bf Total Points: 100}}
\chead{MATH 3500}\lhead{Practice Exam}\rhead{December, 2014}

\newcommand{\points}[1]{\marginpar{\hspace{24pt}[#1]}}
\newcommand{\skipline}{\vspace{12pt}}
%\renewcommand{\headrulewidth}{0in}
\headheight 30pt

\newcommand{\di}{\displaystyle}
\newcommand{\R}{\mathbb{R}}
\newcommand{\Q}{\mathbb{Q}}
\newcommand{\N}{\mathbb{N}}
\newcommand{\abs}[1]{\lvert #1\rvert}

\begin{document}

\author{Instructor: Sean Fitzpatrick}
\thispagestyle{plain}
\begin{center}
\emph{University of Lethbridge}\\
Department of Mathematics and Computer Science\\
December 2014\\
{\bf MATH 3500 - Practice Exam}\\
\end{center}
\skipline \skipline \skipline \noindent \skipline
Last Name:\underline{\hspace{350pt}}\\
\skipline
First Name:\underline{\hspace{348pt}}\\
\skipline
Student Number:\underline{\hspace{322pt}}\\
\skipline

\vspace{0.5in}


\begin{quote}

 
 {Record your answers below each question in the space provided.    Left-hand pages may be used as scrap paper for rough work.  If you want any work on the left-hand pages to be graded, please indicate so on the right-hand page.
 
 \bigskip
 
Partial credit will be awarded for partially correct work, so be sure to show your work, and include all necessary justifications needed to support your arguments. }

\end{quote}


\vspace{0.5in}

For grader's use only:

\begin{table}[hbt]
\begin{center}
\begin{tabular}{|l|r|} \hline
Problem&Grade\\
\hline \hline
\cline{1-2} 1 & \enspace\enspace\enspace\enspace\enspace\enspace/10\\
\cline{1-2} 2 & \enspace\enspace\enspace\enspace\enspace\enspace/10\\
\cline{1-2} 3 & \enspace\enspace\enspace\enspace\enspace\enspace/10\\
\cline{1-2} 4 & \enspace\enspace\enspace\enspace\enspace\enspace/10\\
\cline{1-2} 5 & \enspace\enspace\enspace\enspace\enspace\enspace/10\\
\cline{1-2} 6 & \enspace\enspace\enspace\enspace\enspace\enspace/10\\
\cline{1-2} 7 & \enspace\enspace\enspace\enspace\enspace\enspace/10\\
\cline{1-2} 8 & \enspace\enspace\enspace\enspace\enspace\enspace/10\\
\cline{1-2} 9 & \enspace\enspace\enspace\enspace\enspace\enspace/10\\
\cline{1-2} 10 & \enspace\enspace\enspace\enspace\enspace\enspace/10\\
\cline{1-2} Total & \enspace\enspace\enspace\enspace\enspace\enspace/100\\
\hline
\end{tabular}

\skipline

\skipline

\skipline

\end{center}
\end{table}
\newpage


\begin{enumerate}
\item \begin{enumerate}
       \item Find the supremum and infimum of the following sets, if they exist. If either one does not exist, explain why.\points{4}
\[
 A = \left\{(-1)^n\left(1+\frac{1}{n}\right):n\in\N\right\} \quad \quad B = \left\{\sin \frac{1}{x} : x\in (0,1]\right\}
\]

\vspace{3in}

\item Let $D\subseteq \R$ be non-empty and suppose $f:D\to\R$ and $g:D\to \R$ are bounded functions.
\begin{enumerate}
 \item Prove that $\sup[(f+g)(D)] \leq \sup[f(D)]+\sup[g(D)]$. \points{4}

\vspace{2.5in}

 \item Give an example to show that the above inequality may be strict. \points{2}
\end{enumerate}

      \end{enumerate}
\newpage
\item \begin{enumerate}
       \item For any non-empty subset $A\subseteq \R$, prove that $\partial A = \overline{A}\setminus A^\circ$, where $\partial A$ denotes the boundary of $A$, $A^\circ$ denotes the interior of $A$ and $\overline{A}$ denotes the closure of $A$. \points{5}

\vspace{3.5in}

       \item Give an example of a set $A\subseteq \R$ that has a limit point but does not contain a limit point. \points{2}

\vspace{1.5in}

       \item Prove that a closed subset of a compact set is compact. \points{3}
      \end{enumerate}
\newpage

\item Prove that a point $x$ is a limit point of a set $S\subseteq \R$ if and only if there exists a sequence $(x_n)$ of points in $S\setminus \{x\}$ such that $\lim x_n = x$. \points{10}

\newpage
\item\begin{enumerate}
\item Suppose $I_n = [a_n,b_n]$ is a nested sequence of intervals; that is, $I_{n+1}\subseteq I_n$ for all $n$. Prove that $\lim a_n$ and $\lim b_n$ both exist. \points{4}

\vspace{3in}

\item Prove that $\displaystyle \lim_{n\to \infty} \frac{n+1}{n} = 1$. \points{6}
\end{enumerate}
\newpage

\item \begin{enumerate}
       \item Define $f:\R\to\R$ by $\displaystyle f(x) = \begin{cases} x^2 + 6 & \text{ if } x\in \Q\\ 5x & \text{ if } x\notin \Q\end{cases}$. Prove that $f$ is continuous at $x=2$. \points{5}

\vspace{4in}

       \item Use the $\epsilon-\delta$ definition to prove that $f(x)=x^2+4$ is uniformly continuous on the interval $[1,3]$. \points{5}
      \end{enumerate}
\newpage

\item \begin{enumerate}
       \item Prove that any polynomial of odd degree has at least one real root. \points{6}

\vspace{4in}

       \item Prove that if $f$ is uniformly continuous on an interval $(a,b)$, then $f$ is bounded on $(a,b)$. \points{4}



      \end{enumerate}
\newpage
\item \begin{enumerate}
       \item Suppose that $f$ and $g$ are differentiable functions such that $f(a)=g(a)$ and $f'(x)< g'(x)$ for all $x\geq a$. Prove that $f(x)< g(x)$ for all $x\geq a$. \points{5}

\vspace{4in}

       \item Prove that if $n\geq 1$, then $(1+x)^n>1+nx$ for $x>0$. \points{5}
\end{enumerate}

\newpage
\item  \begin{enumerate}
       \item Let $f(x) = \begin{cases} x^2, \text{ if } x\geq 0\\ 0, \text{ if } x<0\end{cases}$. \points{5} Prove that $f$ is differentiable at $x=0$ and give a formula for $f'(x)$ for all $x\in\R$.

\vspace{3.5in}

       \item Prove that $f(x) = x^3+3x$ has exactly one root. \points{5}
      \end{enumerate}
\newpage

\item Suppose that $f$ is integrable on $[a,b]$ and that $[c,d]\subseteq [a,b]$. Prove that $f$ is integrable on $[c,d]$. \points{10}

\newpage

 \item Find a function $g$ such that
\begin{enumerate}
 \item  $\displaystyle\int_0^x tg(t)\,dt = x+x^2$ \points{3}

\vspace{2in}
 
\item  $\displaystyle\int_0^{x^2} tg(t) \,dt = x+x^2$ \points{3}

\vspace{2in}

\item Suppose that $f$ is a differentiable function with $f(0)=0$ and $0<f'\leq 1$. Prove that for all $x\geq 0$ we have \points{4}
\[
 \int_0^x f^3\leq \left(\int_0^x f\right)^2.
\]


\end{enumerate}


\end{enumerate}



\end{document}