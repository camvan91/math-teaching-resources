\documentclass[12pt]{article}
\usepackage{amsmath}
\usepackage{amssymb}
\usepackage[letterpaper,margin=0.85in,centering]{geometry}
\usepackage{fancyhdr}
\usepackage{enumerate}
\usepackage{lastpage}
\usepackage{multicol}
\usepackage{graphicx}

\reversemarginpar

\pagestyle{fancy}
\cfoot{Page \thepage \ of \pageref{LastPage}}\rfoot{{\bf Total Points: 60}}
\chead{MATH 2000B}\lhead{Midterm}\rhead{Tuesday, 3\textsuperscript{rd} March, 2015}

\newcommand{\points}[1]{\marginpar{\hspace{24pt}[#1]}}
\newcommand{\skipline}{\vspace{12pt}}
%\renewcommand{\headrulewidth}{0in}
\headheight 30pt

\newcommand{\di}{\displaystyle}
\newcommand{\R}{\mathbb{R}}
\newcommand{\N}{\mathbb{N}}
\newcommand{\Z}{\mathbb{Z}}

\begin{document}

\author{Instructor: Sean Fitzpatrick}
\thispagestyle{plain}
\begin{center}
\emph{University of Lethbridge}\\
Department of Mathematics and Computer Science\\
3\textsuperscript{rd} March, 2015, 3:05-4:20 pm\\
{\bf Math 2000B - Midterm}\\
\end{center}
\skipline \skipline \skipline \noindent \skipline
Last Name:\underline{\hspace{50pt}{\bf SOLUTIONS}\hspace{50pt}}\\
\skipline
First Name:\underline{\hspace{50pt}{\bf THE}\hspace{100pt}}\\
\skipline
Student Number:\underline{\hspace{322pt}}\\


\vspace{0.5in}


\begin{quote}
 Record your answers below each question in the space provided.    {\bf Left-hand pages may be used as scrap paper for rough work.}  If you want any work on the left-hand pages to be graded, please indicate so on the right-hand page.
 
 \bigskip
 
Partial credit will be awarded for partially correct work, so be sure to show your work, and include all necessary justifications needed to support your arguments. 

The value of each problem is indicated in the left-hand margins. The value of a problem does not always indicate the amount of work required to do the problem.

Outside aids, including, but not limited to, cheat sheets, smart phones, laptops, spy cameras, drones, and telepathic communication, are not permitted. You can keep a calculator with you if it makes you feel better.
\end{quote}


\vspace{0.5in}

For grader's use only:

\begin{table}[hbt]
\begin{center}
\begin{tabular}{|l|r|} \hline
Page&Grade\\
\hline \hline
\cline{1-2} 2 & \enspace\enspace\enspace\enspace\enspace\enspace/10\\
\cline{1-2} 3 & \enspace\enspace\enspace\enspace\enspace\enspace/10\\
\cline{1-2} 4 & \enspace\enspace\enspace\enspace\enspace\enspace/12\\
\cline{1-2} 5 & \enspace\enspace\enspace\enspace\enspace\enspace/12\\
\cline{1-2} 6 & \enspace\enspace\enspace\enspace\enspace\enspace/10\\
\cline{1-2} 7 & \enspace\enspace\enspace\enspace\enspace\enspace/6\\
\cline{1-2} Total & \enspace\enspace\enspace\enspace\enspace\enspace/60\\
\hline
\end{tabular}

\skipline

\skipline

\skipline


\end{center}
\end{table}
\newpage


\begin{enumerate}
\item For each of the following sentences, decide whether or not it is an assertion. If it is, indicate whether it is true or false, and why.
\begin{enumerate}
\item The number 4 is an even integer. \points{2}

\bigskip

This is an assertion, and it is true: $4=2(2)$.

\bigskip

\item For each integer $n$, $n^2-1$ is a prime number.\points{2}

\bigskip

This is an assertion, and it is false: for $n=2$ we get $2^2-1=8$, which is not prime.

\bigskip


\item There exists some $x\in\R$ such that $x+y=3$. \points{2}

\bigskip

This is not an assertion, since there is no quantifier for the variable $y$.

\bigskip
\end{enumerate}

\item For each conditional statement below, identify the hypothesis and conclusion, and indicate whether or not the statement is true or false.
\begin{enumerate}
\item If $3+5=10$, then $8<-3$. \points{2}

\bigskip

The hypothesis is $3+5=10$, and the conclusion is $8<-3$. Since the hypothesis is false, the conditional statement is true.

\bigskip

\item The fact that $14=2(7)$ implies that 14 is an odd integer. \points{2}

\bigskip

The hypothesis is $14=2(7)$, which is true, but the conclusion, that 14 is an odd integer, is false. Therefore the conditional statement is false.
\end{enumerate}
\newpage

\item For the following problems, you do {\bf not} need to show your work.
\begin{enumerate}
\item List four integers $n$ such that $n\equiv 3 \pmod{7}$. \points{2}

\bigskip

Any four integers such that $n-3$ is divisible by 7 will do; for example, 3, 10, 17, and 24.

\bigskip

\item Give an example of a tautology and an example of a contradiction. Your example can be specific or symbolic (using $P$, $Q$, etc.)\points{2}

\bigskip

For any statement $P$, we know that $P\vee \neg P$ is a tautology, and $P\wedge \neg P$ is a contradiction, by the Law of the Excluded Middle.

\medskip

Note: A tautology/contradiction must always be true/false regardless of the truth value of the statements it is comprised of. A single statement may itself be true or false, but simply knowing a statement to be true does not make it a tautology.

\bigskip

\item What is the contrapositive of the statement ``If $p$ is a prime number, then $p=2$ or $p$ is an odd number''?\points{2}

\bigskip

The contrapositive of a conditional statement $P\to Q$ is $\neg Q\to \neg P$, so we must have:

\medskip

If $p\neq 2$ and $p$ is even, then $p$ is not a prime number.

\bigskip

\item What does it mean to say that a subset of the integers is {\em inductive}? \points{2}

\bigskip

A subset $S\subseteq \mathbb{Z}$ is {\bf inductive} if for every $n\in\mathbb{Z}$, $n\in S \to n+1\in S$.

\bigskip

\item What is the negation of the statement ``For all $m\in\Z$, there exists some $n\in \Z$ such that $2m-5n=7$''? \points{2}

\bigskip

The statement may be written symbolically as $\forall m\in\Z, \exists n\in\Z : 2m-5n=7$. The negation is therefore
\[
 \exists m\in \Z \text{ such that } \forall n\in \Z, 2m-5n\neq 7.
\]



\end{enumerate}
\newpage

\item Prove the following logical equivalence: $(P\wedge Q)\to (R\vee S)\equiv (\neg R\wedge \neg S)\to (\neg P\vee \neg Q)$. \points{6}

\bigskip

By taking the contrapositive of the left-hand side and then applying one of de Morgan's Laws, we have
\[
 (P\wedge Q)\to (R\vee S)\equiv \neg(R\vee S)\to \neg (P\wedge Q) \equiv (\neg R\wedge \neg S)\to (\neg P\vee \neg Q).
\]
Alternatively, you could first eliminate the conditional, although it takes longer.:
\[
 (P\wedge Q)\to (R\vee S)\equiv \neg (P\wedge Q)\vee (R\vee S) \equiv \neg (\neg R\wedge \neg S)\vee (\neg P\vee \neg Q)\equiv (\neg R\wedge \neg S)\to (\neg P\vee \neg S).
\]
(The middle equivalence contains three or four steps: de Morgan's Laws (twice), double negation, and the commutative law for $\vee$.)

\bigskip

\item  Show that the syllogism $[(P\to Q)\wedge (Q\to R)]\to (P\to R)$ is a tautology. \points{6}

\bigskip

This is one example where it turns out that a truth table is the easier option:
\[
 \begin{array}{ccc|c|c|c|c|c}
  P&Q&R&P\to Q&Q\to R & (P\to Q)\wedge (Q\to R)& P\to R & [(P\to Q)\wedge (Q\to R)]\to (P\to R)\\
\hline
T&T&T&T&T&T&T&T\\
T&T&F&T&F&F&F&T\\
T&F&T&F&T&F&T&T\\
T&F&F&F&F&F&F&T\\
F&T&T&T&T&T&T&T\\
F&T&F&T&F&F&T&T\\
F&F&T&T&T&T&T&T\\
F&F&F&T&T&T&T&T
 \end{array}
\]
If you want to proceed using equivalences instead, the following will do the job (it's one of a few ways that will work):
\begin{align*}
 [(P\to Q)\wedge (Q\to R)]\to (P\to R)&\equiv \neg [(P\to Q)\wedge (Q\to R)]\vee (P\to R)\\
&\equiv \neg(P\to Q)\vee \neg (Q\to R)\vee (\neg P\vee R)\\
&\equiv [(P\wedge\neg Q)\vee\neg P]\vee [(Q\wedge \neg R)\vee R]\\
&\equiv [(P\vee \neg P)\wedge (\neg Q\vee \neg P)]\vee [(Q\vee R)\wedge (\neg R\vee R)]\\
&\equiv [T\wedge (\neg Q\vee \neg P)]\vee [(Q\vee R)\wedge T]\\
&\equiv (\neg Q)\vee \neg P)\vee (Q\vee R)\\
&\equiv (Q\vee \neg Q)\vee (\neg P\vee R)\\
&\equiv T\vee (\neg P\vee R)\equiv T.
\end{align*}

\newpage

\item Are the following propositions true or false? Justify your conclusion with a proof\footnote{Any true statements on this page can be proved using a direct proof of either the original statement or its contrapositive.} or counterexample.
\begin{enumerate}
\item For $a,b,c\in Z$, with $a\neq 0$, if $a|(bc)$, then $a|b$ or $a|c$.\points{4}

\bigskip

The proposition is false. For example, if we take $a=6, b=2$, and $c=3$, then $a\nmid b$ and $a\nmid c$, but $bc=6=6(1)$, so $a\,|\,(bc)$.

\bigskip

\item For any $n\in \N$, if $n^3$ is even, then $n$ is even. \points{4}

\bigskip

The proposition is true; we will prove this by proving the contrapositive. Suppose $n$ is odd. Then $n=2k+1$ for some integer $k$. Thus,
\[
 n^3 = (2k+1)^3 = 8k^3+12k^2+6k+1 = 2(4k^3+6k^2+3k)+1,
\]
which is of the form $n^3=2l+1$, where $l$ is the integer $4k^3+6k^2+3k$, and this shows that $n^3$ is odd, which completes the proof.

\bigskip

\item  For any $a,b\in\Z$, if $a\equiv 5\pmod{9}$ and $b\equiv 7\pmod{9}$, then $(a+b)\equiv 3\pmod{9}$.\points{4}

\bigskip

Suppose that $a\equiv 5\pmod{9}$ and $b\equiv 7\pmod{9}$. Then there exist integers $k$ and $l$ such that
\[
 a = 5+9k \quad\text{ and }\quad b = 7+9l.
\]
Thus, 
\[
 a+b - 3 = (5+9k)+(7+9l) -3 = (9k+9l+12) - 3 =9k+9l+9 = 9(k+l+1).
\]
Therefore, $9|(a+b-3)$, and it follows that $(a+b)\equiv 3\pmod{9}$.

\bigskip

Also acceptable: from the same initial assumption, conclude (using a theorem from class) that $a+b\equiv 5+7 \pmod{9}$. Then note that $5+7=12$ and $12-3$ is divisible by 9 to get your result.
\end{enumerate}
\newpage

\item Use proof by cases to prove the following proposition: For each $n\in\Z$, if $n\not\equiv 0\pmod{3}$, then $n^2\equiv 1\pmod{3}$. \points{5}\\
Hint: The division algorithm gives three possible remainders when $n$ is divided by 3.

\bigskip

Suppose that $n\not\equiv 0\pmod{3}$. Since the division algorithm gives possible remainders of $r=0,1,2$ when $n$ is divided by 3, and our assumption rules out the case $r=0$, we have two cases:

Case 1: $r=1$. In this case, $n\equiv 1\pmod{3}$, so $n^2\equiv 1^2\pmod{3}$, but $1^2=1$, and the result follows.

Case 2: $r=2$. If $n\equiv 2\pmod{3}$, then $n^2\equiv 2^2\pmod{3}$. Since $2^2=4$ and $4\equiv 1\pmod{3}$, we can again conclude that $n^2\equiv 1\pmod{3}$.

Since we have $n^2\equiv 1\pmod{3}$ in both cases where $n\not\equiv 0\pmod{3}$, we can conclude that if $n\not\equiv 0\mod{3}$, then $n^2\equiv 1\pmod{3}$.

\bigskip



\item Use proof by contradiction to prove the following proposition: For each $n\in \N$, $\sqrt{3n+2}$ is not a natural number. \points{5}\\
Hint: The two problems on this page are related.

\bigskip

Suppose for the sake of contradiction that the above is false. Then the negation must be true; that is, there must exist some $n\in\N$ such that $\sqrt{3n+2}=k$ is a natural number. But then
\[
 k^2 = 3n+2, \text{ and } 3n+2\equiv 2\pmod{3},
\]
which implies that $k^2 \equiv 2\pmod{3}$. But this is impossible, since we know that either $k\equiv 0\pmod{3}$ (in which case $k^2\equiv 0\pmod{3}$), or $k^2\equiv 1\pmod{3}$, from the problem above.

Thus, it must be the case that $\sqrt{3n+2}$ is not an integer for every $n\in\N$.
\newpage

\item Use mathematical induction to prove that for each natural number $n$, \points{6}
\[
1+3+5+\cdots + (2n-1) = n^2.
\]

\bigskip

For each $n\in \N$, let the predicate $P(n)$ represent the equation
\[
 1+3+\cdots + (2n-1) = n^2.
\]
When $n=1$ we note that $2(1)-1 = 1 = 1^2$, which shows that $P(1)$ is true, so the base case holds.

Let us now assume that $P(k)$ is true for some $k\geq 1$; thus, we assume that
\[
 1+3+\cdots (2k-1) = k^2.
\]
If we add $2(k+1)-1 = 2k+1$ to both sides of the above equation, we obtain
\begin{align*}
 1+3+\cdots + (2k-1)+(2(k+1)-1) & = k^2 + (2k+1)\\
& = (k+1)^2,
\end{align*}
and thus $P(k+1)$ is true. This shows that $P(k)\to P(k+1)$, and thus, we can conclude that $P(n)$ is valid for all $n\in\N$, by induction.

\end{enumerate}



\end{document}