\documentclass[12pt]{article}
\usepackage{amsmath}
\usepackage{amssymb}
\usepackage[letterpaper,margin=0.85in,centering]{geometry}
\usepackage{fancyhdr}
\usepackage{enumerate}
\usepackage{lastpage}
\usepackage{multicol}
\usepackage{graphicx}

\reversemarginpar

\pagestyle{fancy}
\cfoot{Page \thepage \ of \pageref{LastPage}}\rfoot{{\bf Total Points: 60}}
\chead{MATH 2000B}\lhead{Midterm}\rhead{Tuesday, 3\textsuperscript{rd} March, 2015}

\newcommand{\points}[1]{\marginpar{\hspace{24pt}[#1]}}
\newcommand{\skipline}{\vspace{12pt}}
%\renewcommand{\headrulewidth}{0in}
\headheight 30pt

\newcommand{\di}{\displaystyle}
\newcommand{\R}{\mathbb{R}}
\newcommand{\N}{\mathbb{N}}
\newcommand{\Z}{\mathbb{Z}}

\begin{document}

\author{Instructor: Sean Fitzpatrick}
\thispagestyle{plain}
\begin{center}
\emph{University of Lethbridge}\\
Department of Mathematics and Computer Science\\
3\textsuperscript{rd} March, 2015, 3:05-4:20 pm\\
{\bf Math 2000B - Midterm}\\
\end{center}
\skipline \skipline \skipline \noindent \skipline
Last Name:\underline{\hspace{350pt}}\\
\skipline
First Name:\underline{\hspace{348pt}}\\
\skipline
Student Number:\underline{\hspace{322pt}}\\


\vspace{0.5in}


\begin{quote}
 Record your answers below each question in the space provided.    {\bf Left-hand pages may be used as scrap paper for rough work.}  If you want any work on the left-hand pages to be graded, please indicate so on the right-hand page.
 
 \bigskip
 
Partial credit will be awarded for partially correct work, so be sure to show your work, and include all necessary justifications needed to support your arguments. 

The value of each problem is indicated in the left-hand margins. The value of a problem does not always indicate the amount of work required to do the problem.

Outside aids, including, but not limited to, cheat sheets, smart phones, laptops, spy cameras, drones, and telepathic communication, are not permitted. You can keep a calculator with you if it makes you feel better.
\end{quote}


\vspace{0.5in}

For grader's use only:

\begin{table}[hbt]
\begin{center}
\begin{tabular}{|l|r|} \hline
Page&Grade\\
\hline \hline
\cline{1-2} 2 & \enspace\enspace\enspace\enspace\enspace\enspace/10\\
\cline{1-2} 3 & \enspace\enspace\enspace\enspace\enspace\enspace/10\\
\cline{1-2} 4 & \enspace\enspace\enspace\enspace\enspace\enspace/12\\
\cline{1-2} 5 & \enspace\enspace\enspace\enspace\enspace\enspace/12\\
\cline{1-2} 6 & \enspace\enspace\enspace\enspace\enspace\enspace/10\\
\cline{1-2} 7 & \enspace\enspace\enspace\enspace\enspace\enspace/6\\
\cline{1-2} Total & \enspace\enspace\enspace\enspace\enspace\enspace/60\\
\hline
\end{tabular}

\skipline

\skipline

\skipline


\end{center}
\end{table}
\newpage


\begin{enumerate}
\item For each of the following sentences, decide whether or not it is an assertion. If it is, indicate whether it is true or false, and why.
\begin{enumerate}
\item The number 4 is an even integer. \points{2}

\vspace{1.25in}

\item For each integer $n$, $n^2-1$ is a prime number.\points{2}

\vspace{1.25in}

\item There exists some $x\in\R$ such that $x+y=3$. \points{2}

\vspace{1in}
\end{enumerate}

\item For each conditional statement below, identify the hypothesis and conclusion, and indicate whether or not the statement is true or false.
\begin{enumerate}
\item If $3+5=10$, then $8<-3$. \points{2}

\vspace{1.25in}

\item The fact that $14=2(7)$ implies that 14 is an odd integer. \points{2}
\end{enumerate}
\newpage

\item For the following problems, you do {\bf not} need to show your work.
\begin{enumerate}
\item List four integers $n$ such that $n\equiv 3 \pmod{7}$. \points{2}

\vspace{1.25in}

\item Give an example of a tautology and an example of a contradiction. Your example can be specific or symbolic (using $P$, $Q$, etc.)\points{2}

\vspace{1.25in}

\item What is the contrapositive of the statement ``If $p$ is a prime number, then $p=2$ or $p$ is an odd number''?\points{2}

\vspace{1in}

\item What does it mean to say that a subset of the integers is {\em inductive}? \points{2}

\vspace{1.25in}

\item What is the negation of the statement ``For all $m\in\Z$, there exists some $n\in \Z$ such that $2m-5n=7$''? \points{2}



\end{enumerate}
\newpage

\item Prove the following logical equivalence: $(P\wedge Q)\to (R\vee S)\equiv (\neg R\wedge \neg S)\to (\neg P\vee \neg Q)$. \points{6}

\vspace{3.5in}

\item  Show that the syllogism $[(P\to Q)\wedge (Q\to R)]\to (P\to R)$ is a tautology. \points{6}

\newpage

\item Are the following propositions true or false? Justify your conclusion with a proof\footnote{Any true statements on this page can be proved using a direct proof of either the original statement or its contrapositive.} or counterexample.
\begin{enumerate}
\item For $a,b,c\in Z$, with $a\neq 0$, if $a|(bc)$, then $a|b$ or $a|c$.\points{4}

\vspace{2in}

\item For any $n\in \N$, if $n^3$ is even, then $n$ is even. \points{4}

\vspace{2.5in}

\item  For any $a,b\in\Z$, if $a\equiv 5\pmod{9}$ and $b\equiv 7\pmod{9}$, then $(a+b)\equiv 3\pmod{9}$.\points{4}

\end{enumerate}
\newpage

\item Use proof by cases to prove the following proposition: For each $n\in\Z$, if $n\not\equiv 0\pmod{3}$, then $n^2\equiv 1\pmod{3}$. \points{5}\\
Hint: The division algorithm gives three possible remainders when $n$ is divided by 3.

\vspace{4in}


\item Use proof by contradiction to prove the following proposition: For each $n\in \N$, $\sqrt{3n+2}$ is not a natural number. \points{5}\\
Hint: The two problems on this page are related.

\newpage

\item Use mathematical induction to prove that for each natural number $n$, \points{6}
\[
1+3+5+\cdots + (2n-1) = n^2.
\]
\end{enumerate}



\end{document}