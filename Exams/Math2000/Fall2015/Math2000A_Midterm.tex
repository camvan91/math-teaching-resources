\documentclass[12pt]{article}
\usepackage{amsmath}
\usepackage{amssymb}
\usepackage[letterpaper,margin=0.85in,centering]{geometry}
\usepackage{fancyhdr}
\usepackage{enumerate}
\usepackage{lastpage}
\usepackage{multicol}
\usepackage{graphicx}

\reversemarginpar

\pagestyle{fancy}
\cfoot{Page \thepage \ of \pageref{LastPage}}\rfoot{{\bf Total Points: 60}}
\chead{MATH 2000A}\lhead{Midterm}\rhead{Thursday, 29\textsuperscript{th} October, 2015}

\newcommand{\points}[1]{\marginpar{\hspace{24pt}[#1]}}
\newcommand{\skipline}{\vspace{12pt}}
%\renewcommand{\headrulewidth}{0in}
\headheight 30pt

\newcommand{\di}{\displaystyle}
\newcommand{\R}{\mathbb{R}}
\newcommand{\N}{\mathbb{N}}
\newcommand{\Z}{\mathbb{Z}}

\begin{document}

\author{Instructor: Sean Fitzpatrick}
\thispagestyle{plain}
\begin{center}
\emph{University of Lethbridge}\\
Department of Mathematics and Computer Science\\
29\textsuperscript{th} March, 2015, 12:15 - 1:30 pm\\
{\bf Math 2000A - Midterm}\\
\end{center}
\skipline \skipline \skipline \noindent \skipline
Last Name:\underline{\hspace{350pt}}\\
\skipline
First Name:\underline{\hspace{348pt}}\\
\skipline
Student Number:\underline{\hspace{322pt}}\\


\vspace{0.5in}


\begin{quote}
 Record your answers below each question in the space provided.    {\bf Left-hand pages may be used as scrap paper for rough work.}  If you want any work on the left-hand pages to be graded, please indicate so on the right-hand page.
 
 \bigskip
 
Partial credit will be awarded for partially correct work, so be sure to show your work, and include all necessary justifications needed to support your arguments. 

The value of each problem is indicated in the left-hand margins. The value of a problem does not always indicate the amount of work required to do the problem.

Outside aids, including, but not limited to, cheat sheets, smart phones, laptops, spy cameras, drones, and telepathic communication, are not permitted. You can keep a calculator with you if it makes you feel better.
\end{quote}


\vspace{0.5in}

For grader's use only:

\begin{table}[hbt]
\begin{center}
\begin{tabular}{|l|r|} \hline
Page&Grade\\
\hline \hline
\cline{1-2} 2 & \enspace\enspace\enspace\enspace\enspace\enspace/10\\
\cline{1-2} 3 & \enspace\enspace\enspace\enspace\enspace\enspace/10\\
\cline{1-2} 4 & \enspace\enspace\enspace\enspace\enspace\enspace/8\\
\cline{1-2} 5 & \enspace\enspace\enspace\enspace\enspace\enspace/12\\
\cline{1-2} 6 & \enspace\enspace\enspace\enspace\enspace\enspace/10\\
\cline{1-2} 7 & \enspace\enspace\enspace\enspace\enspace\enspace/10\\
\cline{1-2} Total & \enspace\enspace\enspace\enspace\enspace\enspace/60\\
\hline
\end{tabular}

\skipline

\skipline

\skipline


\end{center}
\end{table}
\newpage


\begin{enumerate}
\item For each conditional statement below, identify (i) the hypothesis, (ii) the conclusion, and (iii) whether it is true or false.
\begin{enumerate}
 \item If $3+4=8$, then $42>15$. \points{2}

\vspace{1in}

 \item If $1+1=2$, then $5-3=7$. \points{2}

\vspace{1in}

 \item If $5\geq 5$, then $3-7>0$. \points{2}

\end{enumerate}

\vspace{1in}

\item For each predicate below, add (i) a universal set, and (ii) appropriate quantifier(s) such that the resulting quantified statement is true.\\
(For example, given the predicate $2x-4=6$, you could form the true statement\\ $(\exists x\in\mathbb{Z})(2x-4=6)$.)

\begin{enumerate}
 \item $x^2+1>0$. \points{2}

\vspace{1in}

 \item $2m-3n=4$. \points{2}

\end{enumerate}



\newpage

\item For each of the problems below, provide a definition or example, as requested.
\begin{enumerate}
\item Define the {\bf truth set} of a predicate $P(x)$. \points{2}

\vspace{1.5in}

\item Give an example of a tautology. \points{2}

\vspace{1.5in} 

\item Define what it means for an integer $a$ to be {\bf congruent} to an integer $b$, modulo $n$.\points{2}

\vspace{1.5in}

\item Give four examples of integers $a$ such that $a\equiv 3 \pmod{7}$. \points{2}

\vspace{1.5in}

\item What is the {\bf contrapositive} of a conditional statement $P\to Q$? \points{2}


\end{enumerate}
\newpage

\item A set of real numbers $A$ is defined to be {\em pronghornian}\footnote{Yes, I just made that up.} if for each $a\in A$ there exists some element $b\in A$ such that $a^2+b$ is even and $a^2\equiv b^3 \pmod{2015}$. Complete the following sentence:\points{3}

\medskip

A set of real numbers $A$ is {\bf not} pronghornian if...

\vspace{2.5in}

\item Prove the following logical equivalence using previously established logical equivalences: $(P\wedge Q)\to R\equiv (P\to R)\vee (Q\to R)$. \points{5}



\newpage

\item Determine whether the following statements are true or false. If a statement is true, give a {\bf direct} proof of the statement. If it is false, provide a counterexample to support your claim.
\begin{enumerate}
\item For any integers $a$ and $b$, if $a\equiv 3\pmod{5}$ and $b\equiv 2\pmod{5}$, then $ab\equiv 1\pmod{6}$. \points{4}

\vspace{2.5in}

\item For any integers $a$ and $b$, if $ab\equiv 0\pmod{12}$, then $a\equiv 0\pmod{12}$ or $b\equiv 0 \pmod{12}$. \points{4}

\vspace{2in}

\item For any integers $a$, $b$, and $c$, if $a|b$ and $a|c$, then $a|(2b-3c)$.\points{4}

\end{enumerate}
\newpage

\item Given that $x\in\R$ is {\bf irrational}, prove the following:   \points{5}
\begin{center}
 For every real number $y$, either $x+y$ is irrational, or $x-y$ is irrational.
\end{center}
{\bf Hint:} Use proof by contradition, and take care to correctly form the negation of the given statement.

\vspace{4in}

\item Use proof by cases to prove that for each integer $n$, $n^3\equiv n \pmod{3}$.\points{5}

\newpage

\item Use mathematical induction to prove that $4\,|\,(5^n-1)$ for each natural number $n$. \points{5}

\vspace{4in}

\item For which natural numbers $n$ is it true that $2^n>(n+1)^2$? Support your claim with a proof by induction. \points{5}
\end{enumerate}
\newpage
\subsection*{List of basic equivalences}
\begin{enumerate}
\item Equivalences involving conditional statements
\begin{enumerate}
 \item $P\to Q\equiv \neg P\vee Q$
 \item $\neg (P\to Q)\equiv P\wedge \neg Q$
 \item $P\to Q\equiv \neg Q\to \neg P$
\end{enumerate}
\item Commutative properties
\begin{enumerate}
\item $P \vee Q \equiv Q\vee P$
\item $P \wedge Q \equiv Q \wedge P$
\end{enumerate}
\item Associative properties
\begin{enumerate}
\item $(P\vee Q)\vee R \equiv P\vee (Q\vee R)$
\item $(P\wedge Q)\wedge R \equiv P\wedge (Q\wedge R)$
\end{enumerate}
\item Distributive properties
\begin{enumerate}
\item $P\vee (Q\wedge R) \equiv (P\vee Q)\wedge (P\vee R)$
\item $P\wedge (Q\vee R) \equiv (P\wedge Q)\vee (P\wedge R)$
\end{enumerate} 
\item Idempotent laws
\begin{enumerate}
\item $P \vee P \equiv P$
\item $P\wedge P \equiv P$
\end{enumerate}
\item De Morgan's Laws
\begin{enumerate}
\item $\neg (P\vee Q)\equiv \neg P \wedge \neg Q$
\item $\neg (P\wedge Q) \equiv \neg P \vee \neg Q$
\end{enumerate}
\item Law of the Excluded Middle
\begin{enumerate}
 \item $P\vee \neg P \equiv T$
 \item $P\wedge \neg P \equiv F$
\end{enumerate}
\item Effect of Tautologies and Contradictions
\begin{enumerate}
 \item $P\vee T \equiv T$
\item $P\wedge T \equiv P$
\item $P\vee F\equiv P$
\item $P\wedge F \equiv F$
\end{enumerate}
\end{enumerate}


\end{document}