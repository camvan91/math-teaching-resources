\documentclass[12pt]{article}
\usepackage{amsmath}
\usepackage{amssymb}
\usepackage[letterpaper,margin=0.85in,centering]{geometry}
\usepackage{fancyhdr}
\usepackage{enumerate}
\usepackage{lastpage}
\usepackage{multicol}
\usepackage{graphicx}

\reversemarginpar

\pagestyle{fancy}
\cfoot{Page \thepage \ of \pageref{LastPage}}\rfoot{{\bf Total Points: 118}}
\chead{MATH 1010}\lhead{PRACTICE EXAM}\rhead{Today, 2015}

\newcommand{\points}[1]{\marginpar{\hspace{24pt}[#1]}}
\newcommand{\skipline}{\vspace{12pt}}
%\renewcommand{\headrulewidth}{0in}
\headheight 30pt

\newcommand{\di}{\displaystyle}
\newcommand{\abs}[1]{\lvert #1\rvert}
\newcommand{\R}{\mathbb{R}}
\newcommand{\C}{\mathbb{C}}
\renewcommand{\P}{\mathcal{P}}
\DeclareMathOperator{\nul}{null}
\DeclareMathOperator{\range}{range}
\DeclareMathOperator{\spn}{span}
\newcommand{\len}[1]{\lVert #1\rVert}
\newcommand{\Q}{\mathbb{Q}}
\newcommand{\N}{\mathbb{N}}
\renewcommand{\L}{\mathcal{L}}
\newcommand{\dotp}{\boldsymbol{\cdot}}
\newcommand{\ds}{\displaystyle}

\begin{document}

\author{Instructor: Sean Fitzpatrick}
\thispagestyle{plain}
\begin{center}
\emph{University of Lethbridge}\\
Department of Mathematics and Computer Science\\
Today's Date, 2015, Current Time\\
{\bf MATH 1010 - PRACTICE EXAM}\\
\end{center}
\skipline \skipline \skipline \noindent \skipline
Last Name:\underline{\hspace{350pt}}\\
\skipline
First Name:\underline{\hspace{348pt}}\\
\skipline
Student Number:\underline{\hspace{322pt}}\\
\skipline

\vspace{0.5in}


\begin{quote}

 
 Record your answers below each question in the space provided.    Left-hand pages may be used as scrap paper for rough work.  If you want any work on the left-hand pages to be graded, please indicate so on the right-hand page.
 
 \bigskip
 
Partial credit will be awarded for partially correct work, so be sure to show your work, and include all necessary justifications needed to support your arguments. 

\bigskip

\end{quote}


\vspace{0.25in}

%For grader's use only:

%\begin{table}[hbt]
%\begin{center}
%\begin{tabular}{|l|r|} \hline
%Page&Grade\\
%\hline \hline
%\cline{1-2} 2 & \enspace\enspace\enspace\enspace\enspace\enspace/10\\
%\cline{1-2} 3 & \enspace\enspace\enspace\enspace\enspace\enspace/8\\
%\cline{1-2} 4 & \enspace\enspace\enspace\enspace\enspace\enspace/8\\
%\cline{1-2} 5 & \enspace\enspace\enspace\enspace\enspace\enspace/8\\
%\cline{1-2} 6 & \enspace\enspace\enspace\enspace\enspace\enspace/8\\
%\cline{1-2} 7 & \enspace\enspace\enspace\enspace\enspace\enspace/10\\
%\cline{1-2} 8 & \enspace\enspace\enspace\enspace\enspace\enspace/10\\
%\cline{1-2} 9 & \enspace\enspace\enspace\enspace\enspace\enspace/10\\
%\cline{1-2} 10 & \enspace\enspace\enspace\enspace\enspace\enspace/8\\
%\cline{1-2} 11 & \enspace\enspace\enspace\enspace\enspace\enspace/10\\
%\cline{1-2} 12 & \enspace\enspace\enspace\enspace\enspace\enspace/10\\
%\cline{1-2} Total & \enspace\enspace\enspace\enspace\enspace\enspace/100\\
%\hline
%\end{tabular}

\skipline

\skipline

\skipline

%\end{center}
%\end{table}
\newpage


\begin{enumerate}
 \item Evaluate the following limits:
\begin{enumerate}
\item $\ds \lim_{x\to 0}\frac{x^2+x}{x^3-4x^2+5x+43}$ \points{2}

\vspace{2in} 

\item $\ds \lim_{x\to 1}\frac{x^2-2x+1}{x-1}$ \points{2}

\vspace{2in}

 \item $\ds \lim_{x\to 0}\frac{\tan 5x}{x}$ \points{3}

\vspace{2in}

 \item $\ds \lim_{x\to -\infty}\frac{\sqrt{x^2+4x}}{3x+2}$ \points{3}
\end{enumerate}
\newpage

\item Determine the value of $a$ such that the function \points{4}
\[
 f(x) = \begin{cases}
         4x-3, & \text{ if } x\leq a\\
         x^2+2, & \text{ if } x>a
        \end{cases}
\]
is continuous.

\vspace{4in}

\item Using only the {\bf definition of the derivative}, compute $f'(2)$ if $f(x) = x^2+3$. \points{4}


\newpage

\item Compute the derivatives of the following functions:
\begin{enumerate}
 \item $f(x) = 4x^5-7x^3-4x^2+2015$ \points{2}

\vspace{1in}

 \item $g(x) = x^3\sin (x)$ \points{2}

\vspace{1in}

 \item $h(x) = \dfrac{x^3+x}{\sqrt{x}}$ \points{3}

\vspace{1.5in}

 \item $k(x) = \sin (2x)$ \points{2}

\vspace{1.5in}

 \item $F(x) = \dfrac{\tan x}{x\sec x}$ \points{3}
\end{enumerate}
\newpage

\item Find the global (absolute) maximum and minumum values of $f(x) = x+\dfrac{1}{x}$ on the interval $[1,5]$. \points{5}

\vspace{4in}

\item Find and classify the critical points of the function $g(x) = x^3+5x^2-11$. \points{5}

\newpage

\item Determine the function $f(x)$ such that $f'(x) = x^3-\sqrt{x}$ for all $x\geq 0$, and $f(0)=3$. \points{4}

\vspace{3in}

\item \begin{enumerate}
       \item Verify the trigonometric identity $\dfrac{1}{1+\sin (x)} = \sec^2(x)-\sec(x)\tan(x)$ \points{3}

\vspace{3in}

\item Find a function $f(x)$ such that $f'(x) = \dfrac{1}{1+\sin (x)}$ \points{3}
      \end{enumerate}
\newpage

\item Consider the function $f(x) = x^3-3x^2+4$.
\begin{enumerate}
 \item Given that $a=2$ is a zero of multiplicity 2, find the remaining real zero of $f$. \points{4}

\vspace{3in}

 \item Construct the sign diagram for $f$. \points{2}

\vspace{1in}

 \item Solve the inequality $x^3\geq 3x^2-4$. \points{2}

\vspace{1.5in}

 \item Give a rough sketch of the graph of $f$ based on your sign diagram from part (b). \points{2}
\end{enumerate}
\newpage

\item Consider again the function $f(x)=x^3-3x^2+4$ from the previous page.
\begin{enumerate}
 \item Calculate $f'(x)$. \points{2}

\vspace{1in}

 \item Construct the sign diagram for $f'(x)$. \points{2}

\vspace{1in}

 \item Calculate $f''(x)$, and construct the sign diagram for $f''(x)$. \points{3}

\vspace{1.5in}

 \item Use the information in parts (b) and (c) to determine the location of any local maxima, local minima, and inflection points in the graph of $f(x)$. Then use this information to produce a more accurate version of your sketch from the previous page. \points{5}
\end{enumerate}
\newpage

\item Consider the function $f(x) = \dfrac{x^2-1}{x^2-4}$.
\begin{enumerate}
 \item Construct the sign diagram for $f(x)$.  \points{3}

\vspace{2.5in}

 \item Determine any vertical and horizontal asymptotes for the graph $y=f(x)$. \points{3}

\vspace{2in}

 \item Draw a rough sketch of the graph of $f$ using the information in parts (a) and (b). \points{4}
\end{enumerate}
\newpage

\item Consider again the function $f(x) = \dfrac{x^2-1}{x^2-4}$.
\begin{enumerate}
 \item Calculate $f'(x)$. \points{3}

\vspace{2in}

 \item The function $f$ has one critical point. Using the sign diagram for $f'(x)$, find and classify this critical point, and find the corresponding critical value. \points{2}

\vspace{2in}

 \item Using your answer in (b), redraw your sketch from the previous page so that it accurately reflects the information above.\points{3}
\end{enumerate}
\newpage

\item Solve the following inequalities:
\begin{enumerate}
 \item $\lvert 2x-7\rvert \leq 8$. \points{3}

\vspace{3in}

 \item $\dfrac{2x+17}{x+1}>x+5$. \points{5}
\end{enumerate}
\newpage

\item Find the exact values of the trigonometric functions below at the given angles:
\begin{enumerate}
 \item $\cos (9\pi/4)$ \points{2}

\vspace{1.5in}

 \item $\sin (41\pi/6)$ \points{2}

\vspace{1.5in}

 \item $\cot (-23\pi/3)$ \points{3}

\vspace{2.5in}

 \item $\sin(\theta)$, if the angle $\theta$ lies in Quadrant IV and $\cos(\theta) = 12/13$. \points{3}
\end{enumerate}
\newpage

\item Suppose that $\theta$ is a Quadrant I angle. Show that $\cos(\theta/2) = \sqrt{\dfrac{1+\cos \theta}{2}}$. \points{4}

{\em Hint:} $\cos(\theta) = \cos(\theta/2+\theta/2)$.

\vspace{3in}

\item Use your result in part (a) to evaluate $\cos (\pi/8)$. \points{2}

\vspace{1.5in}

\item Show that for any angles $\alpha$ and $\beta$, \points{4}
\[
 \sin(\alpha+\beta)+\sin(\alpha-\beta) = 2\sin(\alpha)\cos(\beta).
\]



\end{enumerate}
\newpage

\begin{center}
 \textbf{Some true stuff from the course that you possibly didn't remember}
\end{center}
\begin{itemize}
 \item We say $x=a$ is a zero of {\bf multiplicity} $k$ for a polynomial $p(x)$ if $(x-a)^k$ is a factor of $p(x)$, but $(x-a)^{k+1}$ is not.
 \item The {\bf Factor Theorem} states that for a polynomial function $p(x)$, $p(a)=0$ if and only if $(x-a)$ is a factor of $p$.
 \item Values of $\sin\theta$ and $\cos\theta$ in the first quadrant:
\begin{align*}
 \sin 0 &= 0  \quad& \quad \cos 0 &= 1\\
 \sin \pi/6 &= 1/2   \quad& \quad\cos \pi/6 &= \sqrt{3}/2\\
 \sin \pi/4 &= \sqrt{2}/2  \quad &\quad \cos \pi/4& = \sqrt{2}/2\\
 \sin \pi/3 &= \sqrt{3}/2   \quad &\quad\cos \pi/3& = 1/2\\
 \sin \pi/2 &= 1  \quad &\quad \cos \pi/2 &= 0
\end{align*}

\item Fundamental identities:
\begin{enumerate}
 \item $\tan\theta = \dfrac{\sin\theta}{\cos\theta}$, $\cot\theta = \dfrac{\cos\theta}{\sin\theta}$, $\sec\theta = \dfrac{1}{\cos\theta}$, $\csc\theta = \dfrac{1}{\sin\theta}$
 \item $\cos^2\theta + \sin^2\theta =1$
 \item $\cos(\alpha + \beta) = \cos\alpha\cos\beta - \sin\alpha\sin\beta$
 \item $\cos(\alpha - \beta) = \cos\alpha\cos\beta + \sin\alpha\sin\beta$
 \item $\sin(\alpha + \beta) = \sin\alpha\cos\beta + \cos\alpha\sin\beta$
 \item $\sin(\alpha - \beta) = \sin\alpha\cos\beta - \cos\alpha\sin\beta$
\end{enumerate}
\item Obvious but occasionally forgotten facts that are sometimes useful in conjunction with some of the identities above:
\begin{enumerate}
 \item $2\theta = \theta + \theta$
 \item $\theta = \dfrac{\theta}{2}+\dfrac{\theta}{2}$
\end{enumerate}
 \item $\lim_{\theta\to 0}\dfrac{\sin\theta}{\theta} = 1$.
 \item $\ds f'(a) = \lim_{h\to 0}\frac{f(a+h)-f(a)}{h}$.
 \item $\dfrac{d}{dx}(x^n) = nx^{n-1}$, $\dfrac{d}{dx}(\sin x) = \cos x$, and $\dfrac{d}{dx}(\cos x) = -\sin x$.
 \item $(fg)' = f'g+fg'$ and $\left(\dfrac{f}{g}\right)' = \dfrac{f'g-fg'}{g^2}$.
 \item A {\bf critical point} for a function $f$ is a number $c$ such that $f'(c)=0$ (or doesn't exist); $f(c)$ is the corresponding critical value.
 \item If you spend all your time reading this page, you won't have time to complete the exam.
\end{itemize}




\end{document}